\documentclass{article}
%\documentclass[12pt]{article}
\pagestyle{empty}           % <======= surpress page numbers
\newcommand{\bld}[1]{\mbox{\boldmath $#1$}}
\newcommand{\utilde}[1]{\mbox{$#1$}\hspace{-.8em}\raisebox{-1ex}{$\sim$}}
\newcommand{\leftdef}[0]{\,\mbox{$\stackrel{\leftarrow}{=}$}\,}
\newcommand{\rightdef}[0]{\,\mbox{$\stackrel{\rightarrow}{=}$}\,}
\newcommand{\bra}[0]{<\!}
\newcommand{\braketend}[0]{\,|\,}
\newcommand{\braend}[0]{|\,}
\newcommand{\ketend}[0]{\,|}
\newcommand{\ket}[0]{\!>}
\newcommand{\ctext}[1]{\raise0.2ex\hbox{\textcircled{\scriptsize{#1}}}}
\newcommand{\maru}[1]{\ooalign{
\hfil\resizebox{.8\width}{\height}{\scriptsize{#1}}\hfil
\crcr
\raise0.2ex\hbox{\large$\bigcirc$}}}
\newcommand{\normord}[1]{%
  {:\mathrel{\mspace{1mu}#1\mspace{1mu}}:}%
}
\newcommand{\normalprod}[1]{%
  {:\mathrel{\mspace{1mu}#1\mspace{1mu}}:}%
}



%\oddsidemargin 10pt
%\evensidemargin 10pt
\setlength{\textheight}{42\baselineskip}
%\textheight 600pt
\topmargin -5mm
\footskip 25mm
 \renewcommand\footnoterule{
 %\vspace*{-3pt}%
 \vspace*{9pt}%
     \hrule width 2in height 0.4pt
     \vspace*{2.6pt}}
%\marginparsep 5cm
%\marginparwidth5cm

\newcounter{exercise}

\newcounter{problem}
\newcounter{refprob}
%080219 Problems are numbered by a new counter "problem".
%       One additional counter "refprob" is defined to refer
%       to a previous problem. Find an example in
%       set07/sec02F_3dim_x.tex
%\oddsidemargin10pt
\newcounter{exple}
\newcounter{refexple}
%100104 Examples are numbered by a new counter "exple".
%       One additional counter "refexple" is defined to refer
%       to a previous example. 

\usepackage{graphicx}
\usepackage{latexsym}
\usepackage{amssymb}
\usepackage{ulem}
\usepackage{cancel}
\usepackage{comment}
%\usepackage{multicol}
%\setlength{\columnseprule}{1pt}
%\usepackage{vwcol}
\usepackage{amsmath}
\usepackage{arydshln}  %dashed line in a matrix

\usepackage{feynmp}%%%{feynmp}

\usepackage{color}
\newcommand{\bl}[0]{\color{blue}}
\newcommand{\re}[0]{\color{red}}
\newcommand{\gr}[0]{\color{green}}
\newcommand{\cy}[0]{\color{cyan}}
\newcommand{\ma}[0]{\color{magenta}}
\newcommand{\ye}[0]{\color{yellow}}

%\include{hugenholtz}  % macros for hugenholtz diagrams

\usepackage{slashed}

\usepackage[vcentermath]{youngtab}


\begin{document}
\unitlength=1mm  %<<<---------------------------------------------- scale
\parbox{40mm}
{
\begin{fmffile}{f1}
	\begin{fmfgraph*}(40,20)
			%\fmfleft{i1,i2}
			\fmfbottom{i1,o1}
			%\fmfleft{i2}
			%\fmfright{o1,o2} 
			%\fmfright{o2} 
			\fmftop{i2,o2}
			\fmf{fermion}{i1,v1,v2,o1}
			\fmffreeze
			%\fmf{phantom}{i2,i1}
			%\fmf{phantom}{o2,o1}
			\fmf{photon}{i2,v1}
			\fmf{photon}{v2,o2}
	\end{fmfgraph*}
\end{fmffile}
}
\\
\parbox{40mm}
{
\begin{fmffile}{f2}
	\begin{fmfgraph*}(40,20)
 \fmfbottom{i1,d1,o1}
        \fmftop{i2,d2,o2}
        \fmf{fermion}{i1,v1,v2,o1}
        \fmf{fermion}{o2,v4,v3,i2}
        \fmf{photon,tension=0}{v1,v3}
        \fmf{photon,tension=0}{v2,v4}
	\end{fmfgraph*}
\end{fmffile}
}
\\
\parbox{40mm}
{
\begin{fmffile}{f3}
	\begin{fmfgraph*}(40,20)
 \fmfbottom{i1,d1,o1}
        \fmfright{o0,o2,o3}
        \fmf{fermion}{i1,v1,o1}
        \fmffreeze
        \fmf{fermion}{o2,v2,o3}
        \fmf{photon,tension=1.5}{v1,v2}	\end{fmfgraph*}
\end{fmffile}
}
\\
\parbox{40mm}
{
\begin{fmffile}{f4}
	\begin{fmfgraph*}(40,20)
    \fmfleft{i1,i2}
    \fmfright{o1,o2,o3,o4}
    \fmf{fermion,label=$q$,tension=2}{i1,v1}
    \fmf{fermion,label=$\bar{q}$,tension=2}{i2,v1}
    \fmf{boson,label=$Z$,tension=2}{v1,v2}
    \fmf{fermion,label=$\mu^-$}{v2,o4}
    \fmf{phantom}{v2,o1}
    \fmffreeze
    \fmf{phantom}{v2,v3}
    \fmf{phantom}{v2,v3,o1}
    \fmffreeze
    \fmf{fermion,label=$\mu^+$,label.side=right}{v2,o1}
    \fmf{boson,label=$\gamma^*$,label.side=left}{v3,v4}
    \fmf{fermion,label=$e^+$ (missed),label.side=left,label.dist=1}{v4,o2}
    \fmf{fermion,label=$e^-$,label.side=left}{v4,o3}
	\end{fmfgraph*}
\end{fmffile}
}
\end{document}

