\documentclass{article}
%\documentclass[12pt]{article}
\pagestyle{empty}           % <======= surpress page numbers
\newcommand{\bld}[1]{\mbox{\boldmath $#1$}}
\newcommand{\utilde}[1]{\mbox{$#1$}\hspace{-.8em}\raisebox{-1ex}{$\sim$}}
\newcommand{\leftdef}[0]{\,\mbox{$\stackrel{\leftarrow}{=}$}\,}
\newcommand{\rightdef}[0]{\,\mbox{$\stackrel{\rightarrow}{=}$}\,}
\newcommand{\bra}[0]{<\!}
\newcommand{\braketend}[0]{\,|\,}
\newcommand{\braend}[0]{|\,}
\newcommand{\ketend}[0]{\,|}
\newcommand{\ket}[0]{\!>}
\newcommand{\ctext}[1]{\raise0.2ex\hbox{\textcircled{\scriptsize{#1}}}}
\newcommand{\maru}[1]{\ooalign{
\hfil\resizebox{.8\width}{\height}{\scriptsize{#1}}\hfil
\crcr
\raise0.2ex\hbox{\large$\bigcirc$}}}
\newcommand{\normord}[1]{%
  {:\mathrel{\mspace{1mu}#1\mspace{1mu}}:}%
}
\newcommand{\normalprod}[1]{%
  {:\mathrel{\mspace{1mu}#1\mspace{1mu}}:}%
}



%\oddsidemargin 10pt
%\evensidemargin 10pt
\setlength{\textheight}{42\baselineskip}
%\textheight 600pt
\topmargin -5mm
\footskip 25mm
 \renewcommand\footnoterule{
 %\vspace*{-3pt}%
 \vspace*{9pt}%
     \hrule width 2in height 0.4pt
     \vspace*{2.6pt}}
%\marginparsep 5cm
%\marginparwidth5cm

\newcounter{exercise}

\newcounter{problem}
\newcounter{refprob}
%080219 Problems are numbered by a new counter "problem".
%       One additional counter "refprob" is defined to refer
%       to a previous problem. Find an example in
%       set07/sec02F_3dim_x.tex
%\oddsidemargin10pt
\newcounter{exple}
\newcounter{refexple}
%100104 Examples are numbered by a new counter "exple".
%       One additional counter "refexple" is defined to refer
%       to a previous example. 

\usepackage{graphicx}
\usepackage{latexsym}
\usepackage{amssymb}
\usepackage{ulem}
\usepackage{cancel}
\usepackage{comment}
%\usepackage{multicol}
%\setlength{\columnseprule}{1pt}
%\usepackage{vwcol}
\usepackage{amsmath}
\usepackage{arydshln}  %dashed line in a matrix

\usepackage{feynmp}%%%{feynmp}

\usepackage{color}
\newcommand{\bl}[0]{\color{blue}}
\newcommand{\re}[0]{\color{red}}
\newcommand{\gr}[0]{\color{green}}
\newcommand{\cy}[0]{\color{cyan}}
\newcommand{\ma}[0]{\color{magenta}}
\newcommand{\ye}[0]{\color{yellow}}

\include{hugenholtz}  % macros for hugenholtz diagrams

\usepackage{slashed}

\usepackage[vcentermath]{youngtab}


\begin{document}
\unitlength=1mm  %<<<---------------------------------------------- scale
\parbox{40mm}
{
\begin{fmffile}{NNbyPi1}
	\begin{fmfgraph*}(40,20)
			%\fmfpen{thick}
			\fmfleft{i1,i2}
			\fmfright{o1,o2} 
			%\fmfv{label=$p_b$, label.angle=90}{i1}
			\fmflabel{N}{i1}
			%OK \fmfiv{d.sh=circle,d.f=1,d.si=2thin}{c}
			%OK \fmfiv{l=pa,l.angle=120,label.distance=.2w}{c}
			\fmflabel{N}{i2}
			%\fmfv{label=$p_a$, label.angle=270}{i2}
			\fmflabel{N}{o1}
			\fmflabel{N}{o2}
			%\fmf{fermion,label=$p_b$, label.side=left, tension=2}{i1,v1}
			%\fmf{fermion, tension=2,label=$p_b$,label.dist=5mm}{i1,v1}
			\fmf{fermion, tension=2}{i1,v1}
			%\fmf{fermion,tension=2,label=none, label.side=left}{v1,o1}
			\fmf{fermion,tension=2}{v1,o1}
			%NG \fmfcmd{draw v1--o1 withpen pencircle scaled 1bp;}
			\fmf{fermion,tension=2}{i2,v2}
			%label.bot(btex $p_a$ etex,
			%NG \fmfcmd{label.bot("p_a",1/2[i2,v2])}
			\fmf{fermion,tension=2}{v2,o2}
			\fmf{dashes}{v1,v2}
			%\fmflabel{$-ig(2\pi)^4 \delta^4(\sum_{in} p_{in})$}{v1}
			%\fmflabel{$-ig(2\pi)^4 \delta^4(\sum_{in} p_{in})$}{v2}
			%
			%\fmfposition
			\fmffreeze
			\fmfipair{ain,avp}
			\fmfiset{ain}{vloc(__i2)}
			\fmfiset{avp}{vloc(__v2)}
			\fmfipath{pa}
			\fmfiset{pa}{vpath(__i2,__v2)}
			%NG \fmfcmd{label.bot("a",1/2pa)}
			%\fmfcmd{draw ain--avp}
			%\fmfipath{p}
			%\fmfiset{p}{vpath(__i2,__v2)}
			%NG \fmfcmd{label.bot("a",1/2[__i2,__v2])}
			%\fmflabel{left=0.5,label=$p_a$,label.side=left,label.dist=-0.05w}{i1}			
			%\fmflabel{left=0.5,label=$p_a$}{i1}			
	\end{fmfgraph*}
\end{fmffile}
}
\hspace{3mm}
$+$
\hspace{3mm}
\parbox{40mm}
{
\begin{fmffile}{NNbyPi2}
	\begin{fmfgraph*}(40,20)
			\fmfleft{i1,i2}
			\fmfright{o1,o2} 
			\fmflabel{$p_b$}{i1}
			\fmflabel{$p_a$}{i2}
			\fmflabel{$p_1$}{o1}
			\fmflabel{$p_2$}{o2}
			\fmf{fermion,tension=2}{i1,v1,o1}
			\fmf{fermion,tension=2}{i2,v2,o2}
			\fmf{dashes}{v1,v2}
	\end{fmfgraph*}
\end{fmffile}
}

\newpage


\begin{eqnarray}
\bra f \braend S - 1 \ketend i \ket
&=&
\frac{(-i g)^2}{2 (2\pi)^6} \int \frac{d^4 k}{(2\pi)^4} 
\frac{i}{k^2 - m^2 + i\epsilon}
\int d^4 x_1 d^4 x_2
\left\{
e^{i(p_1 + k - p_a)x_1}e^{i(p_2 - k - p_b)}
\right.
\nonumber\\
&&
+
e^{i(p_2 + k - p_a)x_1}e^{i(p_1 - k - p_b)}
+
e^{i(p_1 + k - p_b)x_1}e^{i(p_2 - k - p_1)}
\nonumber\\
&&
\left.
+
e^{i(p_2 + k - p_b)x_1}e^{i(p_1 - k - p_a)}
\right\}
\nonumber\\
&=&
\frac{i(-i g)^2}{(2\pi)^6} \int 
\frac{(2\pi)^4 d^4 k}{k^2 - m^2 + i\epsilon}
\left\{
\delta^4(p_1 + k - p_a)\delta^4(p_2 - k - p_b)
\right.
\nonumber\\
&&
\left.
\hspace{35mm}
+
\delta^4(p_2 + k - p_a)\delta^4(p_1 - k - p_b)
\right\}
\nonumber\\
&=&
\frac{i(-i g)^2}{(2\pi)^6} 
\left\{
\frac{1}{(p_1 - p_a)^2 - m^2}
+
\frac{1}{(p_2 - p_a)^2 - m^2}
\right\}
\nonumber\\
&&
\times (2\pi)^4 \delta^4(p_1 + p_2 - p_a - p_b)
\nonumber\\
&=&
\nonumber\\
&&
\unitlength=1mm  %<<<---------------------------------------------- scale
\parbox{40mm}
{
\begin{fmffile}{eqNNbyPi1}
	\begin{fmfgraph*}(40,20)
			\fmfleft{i1,i2}
			\fmfright{o1,o2} 
			%\fmfv{label=$p_b$, label.angle=90}{i1}
			\fmflabel{$p_b$}{i1}
			\fmflabel{$p_a$}{i2}
			%\fmfv{label=$p_a$, label.angle=270}{i2}
			\fmflabel{$p_2$}{o1}
			\fmflabel{$p_1$}{o2}
			\fmf{fermion,tension=2}{i1,v1,o1}
			\fmf{fermion,tension=2}{i2,v2,o2}
			\fmf{dashes}{v1,v2}
			%\fmflabel{$-ig(2\pi)^4 \delta^4(\sum_{in} p_{in})$}{v1}
			%\fmflabel{$-ig(2\pi)^4 \delta^4(\sum_{in} p_{in})$}{v2}
	\end{fmfgraph*}
\end{fmffile}
}
\hspace{3mm}
+
\hspace{3mm}
\parbox{40mm}
{
\begin{fmffile}{eqNNbyPi2}
	\begin{fmfgraph*}(40,20)
			\fmfleft{i1,i2}
			\fmfright{o1,o2} 
			\fmflabel{$p_b$}{i1}
			\fmflabel{$p_a$}{i2}
			\fmflabel{$p_1$}{o1}
			\fmflabel{$p_2$}{o2}
			\fmf{fermion,tension=2}{i1,v1,o1}
			\fmf{fermion,tension=2}{i2,v2,o2}
			\fmf{dashes}{v1,v2}
	\end{fmfgraph*}
\end{fmffile}
}
\end{eqnarray}


\end{document}

