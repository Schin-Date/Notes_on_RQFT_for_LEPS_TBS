\documentclass{article}
%\documentclass[12pt]{article}
\pagestyle{empty}           % <======= surpress page numbers
\newcommand{\bld}[1]{\mbox{\boldmath $#1$}}
\newcommand{\utilde}[1]{\mbox{$#1$}\hspace{-.8em}\raisebox{-1ex}{$\sim$}}
\newcommand{\leftdef}[0]{\,\mbox{$\stackrel{\leftarrow}{=}$}\,}
\newcommand{\rightdef}[0]{\,\mbox{$\stackrel{\rightarrow}{=}$}\,}
\newcommand{\bra}[0]{<\!}
\newcommand{\braketend}[0]{\,|\,}
\newcommand{\braend}[0]{|\,}
\newcommand{\ketend}[0]{\,|}
\newcommand{\ket}[0]{\!>}
\newcommand{\ctext}[1]{\raise0.2ex\hbox{\textcircled{\scriptsize{#1}}}}
\newcommand{\maru}[1]{\ooalign{
\hfil\resizebox{.8\width}{\height}{\scriptsize{#1}}\hfil
\crcr
\raise0.2ex\hbox{\large$\bigcirc$}}}
\newcommand{\normord}[1]{%
  {:\mathrel{\mspace{1mu}#1\mspace{1mu}}:}%
}
\newcommand{\normalprod}[1]{%
  {:\mathrel{\mspace{1mu}#1\mspace{1mu}}:}%
}



%\oddsidemargin 10pt
%\evensidemargin 10pt
\setlength{\textheight}{42\baselineskip}
%\textheight 600pt
\topmargin -5mm
\footskip 25mm
 \renewcommand\footnoterule{
 %\vspace*{-3pt}%
 \vspace*{9pt}%
     \hrule width 2in height 0.4pt
     \vspace*{2.6pt}}
%\marginparsep 5cm
%\marginparwidth5cm

\newcounter{exercise}

\newcounter{problem}
\newcounter{refprob}
%080219 Problems are numbered by a new counter "problem".
%       One additional counter "refprob" is defined to refer
%       to a previous problem. Find an example in
%       set07/sec02F_3dim_x.tex
%\oddsidemargin10pt
\newcounter{exple}
\newcounter{refexple}
%100104 Examples are numbered by a new counter "exple".
%       One additional counter "refexple" is defined to refer
%       to a previous example. 

\usepackage{graphicx}
\usepackage{latexsym}
\usepackage{amssymb}
\usepackage{ulem}
\usepackage{cancel}
\usepackage{comment}
%\usepackage{multicol}
%\setlength{\columnseprule}{1pt}
%\usepackage{vwcol}
\usepackage{amsmath}
\usepackage{arydshln}  %dashed line in a matrix

\usepackage{feynmp}%%%{feynmp}

\usepackage{color}
\newcommand{\bl}[0]{\color{blue}}
\newcommand{\re}[0]{\color{red}}
\newcommand{\gr}[0]{\color{green}}
\newcommand{\cy}[0]{\color{cyan}}
\newcommand{\ma}[0]{\color{magenta}}
\newcommand{\ye}[0]{\color{yellow}}

%\include{hugenholtz}  % macros for hugenholtz diagrams

\usepackage{slashed}

\usepackage[vcentermath]{youngtab}


\begin{document}
\unitlength=1mm  %<<<---------------------------------------------- scale
\parbox{40mm}
{
\begin{fmffile}{NNbarPiPi1}
	\begin{fmfgraph*}(40,20)
			%\fmfpen{thick}
			\fmfleft{i1,i2}
			\fmfright{o1,o2} 
			\fmflabel{$\overline{\mathrm N}$}{i1}
			\fmflabel{N}{i2}
			\fmflabel{$\pi$}{o1}
			\fmflabel{$\pi$}{o2}
			\fmfdotn{v}{2}
			\fmf{fermion, tension=2}{i2,v2}
			\fmf{dashes,tension=2}{v2,o2}
			\fmf{dashes,tension=2}{v1,o1}
			\fmf{fermion,tension=2}{v1,i1}
			\fmf{fermion}{v2,v1}
			%\fmfposition
			\fmffreeze
			%------------------------------------------------- duplicated vertexes and paths in immediate mode
			\fmfipair{upvin,upvmid,upvout}
			\fmfiset{upvin}{vloc(__i2)}
			\fmfiset{upvmid}{vloc(__v2)}
			\fmfiset{upvout}{vloc(__o2)}
			\fmfipath{uppthin}
			\fmfiset{uppthin}{vpath(__i2,__v2)}
			\fmfipath{uppthout}
			\fmfiset{uppthout}{vpath(__v2,__o2)}
			%-------------------------------------------------
			\fmfipair{dwvin,dwvmid,dwvout}
			\fmfiset{dwvin}{vloc(__i1)}
			\fmfiset{dwvmid}{vloc(__v1)}
			\fmfiset{dwvout}{vloc(__o1)}
			\fmfipath{dwpthin}
			\fmfiset{dwpthin}{vpath(__i1,__v1)}
			\fmfipath{dwpthout}
			\fmfiset{dwpthout}{vpath(__v1,__o1)}
			%-------------------------------------------------
			\fmfipath{updwpth}
			\fmfiset{updwpth}{vpath(__v1,__v2)}
			%------------------------------------------------- labels on paths
			%1 
			\fmfi{phantom,label=$p'$,label.side=right,label.dist=2mm}{updwpth}
			%------------------------------------------------- momentum arrows
			\fmfcmd{%
				style_def marrowa expr p = drawarrow 
					subpath (1/6, 3/6) of p shifted 6 down withpen pencircle scaled 0.4;
					label.bot(btex $p_N$ etex, point 0.3 of p shifted 6 down); enddef;
				   }
			\fmf{marrowa}{i2,v2}
			\fmfcmd{%
				style_def marrowone expr p = drawarrow 
					subpath (1/4, 3/4) of p shifted 6 down withpen pencircle scaled 0.4;
					label.bot(btex $k_1$ etex, point 0.5 of p shifted 6 down); enddef;
				   }
			\fmf{marrowone}{v2,o2}
			\fmfcmd{%
				style_def marrowb expr p = drawarrow 
					subpath (1/6, 3/6) of p shifted 6 up withpen pencircle scaled 0.4;
					label.top(btex $p_{Nbar}$ etex, point 0.3 of p shifted 6 up); enddef;
				   }
			\fmf{marrowb}{i1,v1}
			\fmfcmd{%
				style_def marrowtwo expr p = drawarrow 
					subpath (1/4, 3/4) of p shifted 6 up withpen pencircle scaled 0.4;
					label.top(btex $k_2$ etex, point 0.5 of p shifted 6 up); enddef;
				   }
			\fmf{marrowtwo}{v1,o1}
			%\fmfcmd{%
			%	style_def marrowk expr p = drawarrow 
			%		subpath (1/4, 3/4) of p shifted 6 left withpen pencircle scaled 0.4;
			%		label.lft(btex $p'$ etex, point 0.5 of p shifted 6 left); enddef;
			%	   }
			%\fmf{marrowk}{v1,v2}
			%------------------------------------------------- 			
			%\fmfipair{topmid,botmid}
			%\fmfiequ{topmid}{upvmid+(0,10)}
			%\fmfiv{}
			%\fmfipath{toppthv}{[upvmid,topmid]}
			%\fmfi{plane}{toppthv}
	\end{fmfgraph*}
\end{fmffile}
}
\hspace{3mm}
$+$
\hspace{3mm}
\parbox{40mm}
{
\begin{fmffile}{NNbarPiPi2}
	\begin{fmfgraph*}(40,20)
			%\fmfpen{thick}
			\fmfleft{i1,i2}
			\fmfright{o1,o2} 
			\fmflabel{$\overline{\mathrm N}$}{i1}
			\fmflabel{N}{i2}
			\fmflabel{$\pi$}{o1}
			\fmflabel{$\pi$}{o2}
			\fmfdotn{v}{2}
			\fmf{fermion, tension=2}{i2,v2}
			\fmf{dashes,tension=2}{v2,o2}
			\fmf{dashes,tension=2}{v1,o1}
			\fmf{fermion,tension=2}{v1,i1}
			\fmf{fermion}{v2,v1}
			%\fmfposition
			\fmffreeze
			%------------------------------------------------- duplicated vertexes and paths in immediate mode
			\fmfipair{upvin,upvmid,upvout}
			\fmfiset{upvin}{vloc(__i2)}
			\fmfiset{upvmid}{vloc(__v2)}
			\fmfiset{upvout}{vloc(__o2)}
			\fmfipath{uppthin}
			\fmfiset{uppthin}{vpath(__i2,__v2)}
			\fmfipath{uppthout}
			\fmfiset{uppthout}{vpath(__v2,__o2)}
			%-------------------------------------------------
			\fmfipair{dwvin,dwvmid,dwvout}
			\fmfiset{dwvin}{vloc(__i1)}
			\fmfiset{dwvmid}{vloc(__v1)}
			\fmfiset{dwvout}{vloc(__o1)}
			\fmfipath{dwpthin}
			\fmfiset{dwpthin}{vpath(__i1,__v1)}
			\fmfipath{dwpthout}
			\fmfiset{dwpthout}{vpath(__v1,__o1)}
			%-------------------------------------------------
			\fmfipath{updwpth}
			\fmfiset{updwpth}{vpath(__v1,__v2)}
			%------------------------------------------------- labels on paths
			%1 
			\fmfi{phantom,label=$p'$,label.side=right,label.dist=2mm}{updwpth}
			%------------------------------------------------- momentum arrows
			\fmfcmd{%
				style_def marrowa expr p = drawarrow 
					subpath (1/6, 3/6) of p shifted 6 down withpen pencircle scaled 0.4;
					label.bot(btex $p_N$ etex, point 0.3 of p shifted 6 down); enddef;
				   }
			\fmf{marrowa}{i2,v2}
			\fmfcmd{%
				style_def marrowone expr p = drawarrow 
					subpath (1/4, 3/4) of p shifted 6 down withpen pencircle scaled 0.4;
					label.bot(btex $k_2$ etex, point 0.5 of p shifted 6 down); enddef;
				   }
			\fmf{marrowone}{v2,o2}
			\fmfcmd{%
				style_def marrowb expr p = drawarrow 
					subpath (1/6, 3/6) of p shifted 6 up withpen pencircle scaled 0.4;
					label.top(btex $p_{Nbar}$ etex, point 0.3 of p shifted 6 up); enddef;
				   }
			\fmf{marrowb}{i1,v1}
			\fmfcmd{%
				style_def marrowtwo expr p = drawarrow 
					subpath (1/4, 3/4) of p shifted 6 up withpen pencircle scaled 0.4;
					label.top(btex $k_1$ etex, point 0.5 of p shifted 6 up); enddef;
				   }
			\fmf{marrowtwo}{v1,o1}
			%\fmfcmd{%
			%	style_def marrowk expr p = drawarrow 
			%		subpath (1/4, 3/4) of p shifted 6 left withpen pencircle scaled 0.4;
			%		label.lft(btex $p'$ etex, point 0.5 of p shifted 6 left); enddef;
			%	   }
			%\fmf{marrowk}{v2,v1}
	\end{fmfgraph*}
\end{fmffile}
}

\end{document}

