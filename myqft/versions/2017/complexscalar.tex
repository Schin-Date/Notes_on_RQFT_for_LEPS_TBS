\begin{equation}
{\cal L} = \partial_\mu \varphi^\dagger(x)\cdot \partial^\mu \varphi(x)
- m^2 \varphi^\dagger(x)\varphi(x)
\label{eqn:cmplscLagdens}
\end{equation}
\begin{equation}
\frac{\partial {\cal L}}{\partial(\partial_\mu \varphi)} = \partial^\mu \varphi^\dagger
\,,
\hspace{5mm}
\frac{\partial {\cal L}}{\partial(\partial_\mu \varphi^\dagger)} = \partial^\mu \varphi
\end{equation}
\begin{equation}
\left( \Box + m^2 \right) \varphi(x) = 0\,,
\hspace{5mm}
\left( \Box + m^2 \right) \varphi^\dagger(x) = 0\,,
%\hspace{5mm}
%\varphi(x)^* = \varphi(x)
\end{equation}
\begin{equation}
\pi^\dagger(x) = \frac{ \partial {\cal L}}{ \partial \dot{\varphi}(x) } = \dot{\varphi}^\dagger
%\pi_{}(x) = \partial_0 \varphi_{cl}(x)
%=
%\int \frac{d^3 \bld{k}}{\sqrt{(2\pi)^3}} (- i k^0 ) \left[
%a(\bld{k}) e^{-i k \cdot x} - b^\dagger(\bld{k})e^{i k \cdot x} \right]
\,,
\hspace{5mm}
\pi(x) = \frac{ \partial {\cal L}}{ \partial \dot{\varphi}^\dagger(x) } = \dot{\varphi}
\end{equation}
\begin{eqnarray}
\mathcal{H} 
&=&
\pi^\dagger \dot{\varphi} + \pi \dot{\varphi}^\dagger - \mathcal{L}
\nonumber\\
&=&
\pi^\dagger \pi + (\partial_i \varphi^\dagger) (\partial_i \varphi) + m \varphi^\dagger \varphi
\end{eqnarray}
%Classical solution:
%\begin{equation}
%\varphi_{cl}(x) = \int \frac{d^3 \bld{k}}{\sqrt{(2\pi)^3}} \left[
%a(\bld{k}) e^{-i k \cdot x} + b^*(\bld{k})e^{i k \cdot x} \right]
%\end{equation}
%where $k^0 = + \sqrt{\bld{k}^2 + m^2}$. 
%Solution
\begin{equation}
\begin{array}{l}
\displaystyle
\varphi_{}(x) = \int \frac{d^3 \bld{k}}{\sqrt{(2\pi)^3}2k^0} \left[
a(\bld{k}) e^{-i k \cdot x} + b^\dagger(\bld{k})e^{i k \cdot x} \right]
\\
\displaystyle
\varphi^\dagger_{}(x) = \int \frac{d^3 \bld{k}}{\sqrt{(2\pi)^3}2k^0} \left[
b(\bld{k})e^{- i k \cdot x}  + a^\dagger (\bld{k}) e^{i k \cdot x} \right]
\end{array}
\label{eqn:CompScFourier}
\end{equation}
Canonical quantization
\begin{equation}
\begin{array}{c}
[ a(\bld{k}), a^\dagger(\bld{k}')] = 
[ b(\bld{k}), b^\dagger(\bld{k}')] = 
2k^0\delta^3 (\bld{k} - \bld{k'} )\;,
\vspace{3mm}
\\
\mbox{other commutators} = 0
\label{eqn:complscalarcancomm}
\end{array}
\end{equation}
Four momentum
\begin{equation}
\hat{P}^\mu = \int \frac{d^3 \bld{k}}{2k^0} k^\mu \left(
a^\dagger(\bld{k}) a(\bld{k}) + b^\dagger(\bld{k}) b(\bld{k})
\right)
\label{eqn:KGcomplex_FourMomentum}
\end{equation}
$\hat{P}^0$ is the Hamiltonian.
Vacuum state is defined as $a(\bld{p}) \ketend 0 \ket = b(\bld{p}) \ketend 0 \ket =0$.
There are two kinds of single particle states $a^\dagger(\bld{p}) \ketend 0 \ket$ and
$b^\dagger(\bld{p}) \ketend 0 \ket$ both an eigenstate of $\hat{P}^\mu$.
So far, $a$ and $b$ are just particles independent of each other and having
the common mass $m$. Let us examine the electric charge given from
Eq. (\ref{eqn:NoetherElectricCharge}). We find
\begin{equation}
\hat{Q} = e \int \frac{d^3 \bld{k}}{2k^0}  \left(
a^\dagger(\bld{k}) a(\bld{k}) - b^\dagger(\bld{k}) b(\bld{k})
\right)
\label{eqn:KGcomplex_Charge}
\end{equation}
Thus they carry opposite electric charges.

We consider now inversion symmetries $U = P, C, T$. 
The vacuum is invariant under these inversions.
\begin{itemize}
\item
Space inversion:
$(\bld{x}, t) \mapsto (-\bld{x}, t)$ 
\begin{eqnarray}
P a(\bld{k}) P^{-1} = \pm a(-\bld{k})\,,
\hspace{3mm}
P b(\bld{k}) P^{-1} = \pm b(-\bld{k})\,,
\end{eqnarray}
$+$ for scalar and $-$ for pseudoscalar.
\item
Charge conjugation
\begin{eqnarray}
C a(\bld{k}) C^{-1} = \pm b(\bld{k})\,,
\hspace{3mm}
C b(\bld{k}) C^{-1} = \pm a(\bld{k})\,,
\end{eqnarray}
\item
Time reversal: $(\bld{x}, t) \mapsto (\bld{x}, -t)$ 
\begin{eqnarray}
T a(\bld{k}) T^{-1} = \pm a(-\bld{k})\,,
\hspace{3mm}
C b(\bld{k}) C^{-1} = \pm b(-\bld{k})\,,
\end{eqnarray}
and $T$ is antilinear so that
\begin{eqnarray}
T \varphi(\bld{x}, t) T^{-1} = \pm \varphi(\bld{x}, -t)
\end{eqnarray}
The sign is fixed through the invariance of interactions 
with other kind of fields which have definite signatures
under the $T$ transformation. Usually, $+$ for scalar and
$-$ for pseudoscalar.
\end{itemize}

\bigskip




%Propagator $ = \Delta_F(q)$

