{\cal L}agrangian density ($dim = E^4$) is given as a functional of field $\varphi(x)$
and its space time derivatives
\begin{equation}
	{\cal L} = {\cal L}[\varphi(x), \partial^\mu\varphi(x)]
\end{equation}
Euler-Lagrange Eq.
\begin{equation}
 \frac{ \partial {\cal L}}{ \partial {\varphi}(x) } - 
\partial_\mu \frac{ \partial {\cal L}}{ \partial (\partial_\mu {\varphi}(x) )} = 0
 \;\;
 \label{eqn:EulerLagrange}
\end{equation}
Canonical momentum field
\begin{equation}
\pi(x) = \frac{ \partial {\cal L}}{ \partial \dot{\varphi}(x) }
\label{eqn:conjmomentum}
\end{equation}
where $\dot{\varphi}(x) = \partial_0 \varphi(x) = \partial^0 \varphi(x)$.\\
${\cal H}$amiltonian density
\begin{equation}
{\cal H} = \pi(x) \dot{\varphi}(x) -  {\cal L}
\label{eqn:Hamiltoniandensity}
\end{equation}
By solving Eq. (\ref{eqn:conjmomentum}) for $\dot\varphi(x)$,
${\cal H}$ is a function solely of $\pi(x), \varphi(x)$ and $ \partial_i \varphi(x)$.
In the classical field theory, temporal developments of $\varphi(x)$ and $\pi(x)$
are given by Hamiltonian $H = \int \mbox{d}^3\bld{x} \; {\cal H}$
through the canonical equation of motion
\begin{equation}
\dot{\varphi}(x) = -i [\varphi(x), H ]\;, \; \; \; \; \; \dot{\pi}(x) = -i [\pi(x), H ]
\label{eqn:canonicalEOM}
\end{equation}
where $-i[\cdots]$ is the Poisson braket.
The Euler equation (\ref{eqn:EulerLagrange}) and 
the canonical equation (\ref{eqn:canonicalEOM})
are equivalent in the classical level.

\bigskip

Following the canonical quantization method, 
fields $\varphi(x)$ and $\pi(x')$ 
are postulated to satisfy equal time commutation relations
at a time $x^0 = x'^0 = t_0$ (which is called the time of quantization):
\begin{equation}
\begin{array}{l}
\hspace{8mm}
[ \hat{\varphi}(t_0, \bld{x}), \hat{\pi}(t_0, \bld{x'})] = i \delta^3 (\bld{x} - \bld{x'} )\;,
\\
\left[ \hat{\varphi}(t_0, \bld{x}), \hat{\varphi}(t_0, \bld{x'}) \right] = 0\;,\;\;\;\;
\left[ \hat{\pi}(t_0, \bld{x}), \hat{\pi}(t_0, \bld{x'}) \right] = 0
\end{array}
\label{eqn:eqtimecommrel}
\end{equation}
These relations define a set of field operators at time $x^0 = t_0$.
If the Hamiltonian $\hat{H}$ is provided,
time evolution of field operators as Heisenberg operators are given as
\begin{equation}
i \partial_t \, \hat{\varphi}_H(x) =  [\hat{\varphi}_H(x), \hat{H} ]\;, \; \; \; \; \; 
i \partial_t \, \hat{\pi}_H(x) = [\hat{\pi}_H(x), \hat{H} ]\,,
\label{eqn:HeisenbergEOM}
\end{equation}
where $[\cdots]$ denotes commutation relation.
Formal solutions to these equations are
written as 
\begin{equation}
\begin{array}{l}
\hat{\varphi}_H(x) = e^{i\hat{H}(t-t_0)} \hat{\varphi}(t_0, \bld{x}) e^{-i\hat{H}(t-t_0)} \\
\hat{\pi}_H(x) = e^{i\hat{H}(t-t_0)} \hat{\pi}(t_0, \bld{x}) e^{-i\hat{H}(t-t_0)} 
\end{array}
\label{eqn:Heisenbergformalsol}
\end{equation}
Substituting operators defined in 
%(\ref{eqn:HeisenbergEOM})(\ref{eqn:eqtimecommrel}) in 
Eq. (\ref{eqn:eqtimecommrel}) into
%(\ref{eqn:HeisenbergEOM})
Eq. (\ref{eqn:Heisenbergformalsol}),
we obtain a set of Heisenberg field operators at arbitrary time. They satisfy
equal time commutation relations at arbitrary time $t$:
\begin{equation}
\begin{array}{l}
\hspace{8mm}
[ \hat{\varphi}_H(t, \bld{x}), \hat{\pi}_H(t, \bld{x'})] = i \delta^3 (\bld{x} - \bld{x'} )\;,
\\
\left[ \hat{\varphi}_H(t, \bld{x}), \hat{\varphi}_H(t, \bld{x'}) \right] = 0\;,\;\;\;\;
\left[ \hat{\pi}_H(t, \bld{x}), \hat{\pi}_H(t, \bld{x'}) \right] = 0
\end{array}
\label{eqn:eqtimecommrelHeisenberg}
\end{equation}
In a situation that the Hamiltonian can be constructed by
Eq. (\ref{eqn:Hamiltoniandensity}) in the classical level,
we may obtain $\hat{H}$ by replacing fields by these Heisenberg operator fields
under certain care on the order of noncommuting operators.
The time $t$ in $\hat{H} = \int d^3 \bld{x} {\cal H}(t, \bld{x})$ can be chosen arbitrary since
$i \partial_t \hat{H} = [\hat{H}, \hat{H}] = 0$.
It is then shown\cite{ref:NIsh.1-2} that the set of the Heisenberg equations (\ref{eqn:HeisenbergEOM})
is equivalent with the set of Euler equation (\ref{eqn:EulerLagrange}) for the Heisenberg
field $\hat{\varphi}_H(x)$ and the definition of $\hat{\pi}_H(x)$ given by Eq.(\ref{eqn:conjmomentum}).
\footnote{
In the derivation in \cite{ref:NIsh.1-2}, a formula
\[
\left[ F(A, B, \dots), \pi \right]
=
(\partial F/ \partial A) [A, \pi] + (\partial F/ \partial B) [B, \pi] + \dots
\]
is employed.
} % footnote
Namely, they are equivalent in the quantum level. 
The validity of Eq. (\ref{eqn:eqtimecommrelHeisenberg}) shows 
that once the equal time commutation relation is set, it is valid
for arbitrary time
\footnote{
However, one may not equate time derivatives
of the both hand sides since it is valid at fixed times. }
, provided Hamiltonian exists.
In this sense, the quantization time is arbitral when
the canonical quantization method works in usual manner.

In summary, field equation (\ref{eqn:EulerLagrange}) can be read as
one for the Heisenberg field operator of which the operator nature is
set by the equal time commutation relation (\ref{eqn:eqtimecommrel})
at time $t_0$ that can be chosen arbitrary.

