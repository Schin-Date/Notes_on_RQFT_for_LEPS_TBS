\begin{eqnarray}
{\cal L} &=& \frac{1}{2} 
\left( \underline{\partial} \varphi(\underline{x}) \right)^2
-\frac{1}{2} m^2 \varphi^2(\underline{x})
\nonumber\\
&=& \frac{1}{2} 
\partial_\mu \varphi(\underline{x})\cdot \partial^\mu \varphi(\underline{x})
-\frac{1}{2} m^2 \varphi^2(\underline{x})
\label{eqn:RSCLagrangiandensity}
\end{eqnarray}
where a dot in the last equation indicates the former derivative acting only on the first $\varphi$.
Under a Lorentz transformation $\underline{x} \mapsto \underline{x}' = L\underline{x}$,
the field $\varphi(\underline{x})$ transforms as
\begin{equation}
\varphi(x) \mapsto \varphi'(x') = \varphi(x)\,.
\end{equation}
Here and hereafter, we omitt underlines on Lorentz vectors.
\begin{equation}
\begin{array}{c}
\displaystyle
\frac{\partial {\cal L}}{\partial(\partial_\mu \varphi)} = \partial^\mu \varphi
\\
\displaystyle
\pi(x) = \frac{ \partial {\cal L}}{ \partial \dot{\varphi}(x) } = \dot{\varphi}
%\pi_{}(x) = \partial_0 \varphi_{cl}(x)
%=
%\int \frac{d^3 \bld{k}}{\sqrt{(2\pi)^3}} (- i k^0 ) \left[
%a(\bld{k}) e^{-i k \cdot x} - b^\dagger(\bld{k})e^{i k \cdot x} \right]
\end{array}
\end{equation}
Euler-Lagrange equation
\begin{equation}
\left( \Box + m^2 \right) \varphi_{}(x) = 0
\hspace{10mm}
\mbox{Klein-Gordon}
\label{eqn:Klein-Gordon}
\end{equation}
where $\Box \leftdef \partial^2 = \partial_\mu \partial^\mu = \partial_0^2 - \bld{\partial}^2$.\\
Classical solution of the field equation (\ref{eqn:Klein-Gordon}) is written as
\footnote{%------------------------------- footnote >>
The following consideration lays behind:
\begin{eqnarray*}
\varphi(x)
&=&
\frac{1}{\sqrt{(2\pi)^3}}
\int 
d^4 k
\delta(k^2 - m^2)
\left[
\theta(k^0) + \theta(-k^0)
\right]
\tilde{\varphi} (k)
e^{-i k \cdot x}
\nonumber\\
&=&
\frac{1}{\sqrt{(2\pi)^3}}
\int 
d^4 k
\delta(k^2 - m^2)
\left[
\theta(k^0)
\tilde{\varphi} (k)
e^{-i k \cdot x}
 + \theta(k^0)
\tilde{\varphi} (-k)
e^{i k \cdot x}
\right]
\nonumber\\
&=&
\int 
\frac{d^3 \bld{k}}{\sqrt{(2\pi)^3}2 k^0}
\left[
\tilde{\varphi} (k)
e^{-i k \cdot x}
 + 
\tilde{\varphi} (-k)
e^{i k \cdot x}
\right]
\nonumber\\
&\rightdef&
\int 
\frac{d^3 \bld{k}}{\sqrt{(2\pi)^3}2 k^0}
\left[
a (\bld{k})
e^{-i k \cdot x}
 +
 a^*(\bld{k}) 
e^{i k \cdot x}
\right]
\end{eqnarray*}
}%------------------------------- footnote //
\begin{equation}
\varphi_{cl}(x) = \int \frac{d^3 \bld{k}}{\sqrt{(2\pi)^3}2k^0} \left[
a(\bld{k}) e^{-i k \cdot x} + a^*(\bld{k})e^{i k \cdot x} \right]
\label{eqn:KGclassicalsol}
\end{equation}
where $k^0 = + \sqrt{\bld{k}^2 + m^2}$. 
Canonical conjugate field reads,
\begin{equation}
\pi_{cl}(x) = \dot{\varphi}_{cl}(x)
=
\int \frac{d^3 \bld{k}}{\sqrt{(2\pi)^3}2k^0}  \left[
- i k^0 a(\bld{k}) e^{-i k \cdot x} + i k^0 a^*(\bld{k})e^{i k \cdot x} \right]
\label{eqn:KGclassicalpi}
\end{equation}
Hamiltonian density in the classical level is evaluated from Eq. (\ref{eqn:Hamiltoniandensity}) as
\begin{eqnarray}
{\cal H} &=& \pi(x) \dot{\varphi}(x) - \frac{1}{2} \left\{
\left( \dot{\varphi}(x) \right)^2 - 
\left( \bld{\partial} \varphi(x) \right)^2 \right\}
+ \frac{1}{2} m^2 \varphi^2(x)
\nonumber\\
&=&
\frac{1}{2} \left\{
\pi^2 (x) + \left( \bld{\partial} \varphi(x) \right)^2 + m^2 \varphi^2(x)
\right\}
\label{eqn:KGHamiltoniandens}
\end{eqnarray}
Canonical quantization
\begin{equation}
\begin{array}{c}
[ \varphi(t,\bld{x}), \pi(t, \bld{y})] = i \delta^3 (\bld{x} - \bld{y} )\,,
\vspace{2mm}
\\
\left[ \varphi(t,\bld{x}), \varphi(t,\bld{y}) \right] = 0\,,
\;\;\;\;\left[ \pi(t,\bld{x}), \pi(t,\bld{y}) \right] = 0\,.
\end{array}
\label{eqn:cancomm_eqtime_RS}
\end{equation}
The fields $\varphi(x)$ and $\pi(x)$ are settled as operators at a time $t$.
As Heisenberg operators,
they obey the Heisenberg equation (\ref{eqn:HeisenbergEOM}) with quantum Hamiltonian
%$\hat{H} = \int d^3 \bld{x} \hat{{\cal H}}(x)$ with $\hat{{\cal H}}(x)$ given as
%the normal product of the $r.h.s$ of Eq. (\ref{eqn:KGHamiltoniandens})
% fields replaced by 
given by
\begin{equation}
\hat{H} = \int d^3 \bld{x} \hat{{\cal H}}(x)\,,
\hspace{3mm}
\hat{{\cal H}}(x) = 
\normord{ \frac{1}{2} \left\{
\pi^2 (x) + \left( \bld{\partial} \varphi(x) \right)^2 + m^2 \varphi^2(x)
\right\}
}\,,
\label{eqn:KG_Hamiltonian}
\end{equation}
where $\normord{\dots}$ denotes the normal product.
Since we have the Hamiltonian,
the Heisenberg equation (\ref{eqn:HeisenbergEOM}) is equivalent to the
quantum level Euler equation (\ref{eqn:Klein-Gordon}) and its solution 
and the canonical conjugate field are written as
\begin{equation}
\varphi_{}(x) = \int \frac{d^3 \bld{k}}{\sqrt{(2\pi)^3}2k^0} \left[
a(\bld{k}) e^{-i k \cdot x} + a^\dagger(\bld{k})e^{i k \cdot x} \right]\,,
\label{eqn:RS_fieldexpansion}
\end{equation}
\begin{equation}
\pi_{}(x) 
=
\frac{-i}{2}
\int \frac{d^3 \bld{k}}{\sqrt{(2\pi)^3}}  \left[
a(\bld{k}) e^{-i k \cdot x} - a^\dagger(\bld{k})e^{i k \cdot x} \right]\,,
\label{eqn:RS_pifieldexpansion}
\end{equation}
They have the same form as Eqs. (\ref{eqn:KGclassicalsol}) and (\ref{eqn:KGclassicalpi})
but now the coefficients $a(\bld{k})$ and $a^\dagger(\bld{k})$ are
quantum operators defined by the equal-time commutation relations (\ref{eqn:cancomm_eqtime_RS})
at time $x^0 = t$.
The requirements of Eq. (\ref{eqn:cancomm_eqtime_RS}) is equivalent with
requiring
\begin{equation}
\begin{array}{c}
[ a(\bld{k}), a^\dagger(\bld{k}')] = 2k^0 \delta^3 (\bld{k} - \bld{k'} )\;,
\vspace{2mm}
\\
\left[ a(\bld{k}), a(\bld{k}') \right] = 0\;,\;\;\;\;\left[ a^\dagger(\bld{k}), a^\dagger(\bld{k}') \right] = 0
\end{array}
\label{eqn:cancomm_RS}
\end{equation}

\verb/-----------.-----------.-----------.-----------.-----------/\\
\vspace{-3mm}
{\small
\begin{center}
Addendum: Proof of (\ref{eqn:cancomm_eqtime_RS}) $\Leftrightarrow$ (\ref{eqn:cancomm_RS})
\end{center}
Proof of the necessity of Eq. (\ref{eqn:cancomm_RS}) is straightforward,
We show the sufficiency
of Eq. (\ref{eqn:cancomm_eqtime_RS}) in the following.
We may write Eqs. (\ref{eqn:RS_fieldexpansion}) and (\ref{eqn:RS_pifieldexpansion})
as
\begin{equation}
\begin{array}{l}
\displaystyle
\varphi(x) = \int \frac{d^3\bld{k}}{\sqrt{(2\pi)^3}} Q_{\bld{k}} (t) e^{i \bld{k} \cdot \bld{x}}\,,
\\
\displaystyle
\pi_{}(x)  = \int \frac{d^3\bld{k}}{\sqrt{(2\pi)^3}} P_{\bld{k}} (t) e^{- i \bld{k} \cdot \bld{x}}\,,
\end{array}
\end{equation}
with
\begin{equation}
Q_{\bld{k}}(t)
=
\frac{1}{2k^0} \left[
a(\bld{k}) e^{-i k^0 t} + a^\dagger(-\bld{k}) e^{i k^0 t} \right]
\label{eqn:RSC_Qk_def}
\end{equation}
\begin{equation}
P_{\bld{k}}(t)
=
\frac{i}{2} \left[
a^\dagger(\bld{k}) e^{i k^0 t} - a(-\bld{k}) e^{- i k^0 t} \right]
\label{eqn:RSC_Pk_def}
\end{equation}
Relationships $Q^\dagger_{\bld{k}} = Q_{-\bld{k}}$ and
$P^\dagger_{\bld{k}} = P_{-\bld{k}}$ ensure 
that $\varphi$ and $\pi$ are real.
From the linear independence of Fourier components, we have
\begin{equation}
\begin{array}{l}
0 = [\varphi(t, \bld{x}), \varphi(t, \bld{y})]
\;\Longleftrightarrow\;
[Q_{\bld{k}}(t), Q_{\bld{k'}}(t)] = 0\,,
\vspace{2mm}
\\
0 = [\pi(t, \bld{x}), \pi(t, \bld{y})]
\;\Longleftrightarrow\;
[P_{\bld{k}}(t), P_{\bld{k'}}(t)] = 0\,,
\end{array}
\label{eqn:RSC_QQcomm}
\end{equation}
and
\begin{eqnarray}
i\delta^3(\bld{x} - \bld{y})
&=&
\int \frac{d^3 \bld{k} d^3 \bld{k}'}{(2\pi)^3}
i \delta^3 (\bld{k} - \bld{k}')
e^{i \bld{k} \cdot \bld{x} - i \bld{k}' \bld{y}}
\nonumber\\
&=&
[\varphi(t, \bld{x}), \pi(t, \bld{y}) ]
\nonumber\\
&=&
\int \frac{d^3 \bld{k} d^3 \bld{k}'}{(2\pi)^3} [Q_{\bld{k}}, P_{\bld{k}'}]
e^{i \bld{k} \cdot \bld{x} - i \bld{k}' \cdot \bld{y}}
\nonumber\\
&\Longleftrightarrow&    
\nonumber\\
\left[ Q_{\bld{k}}, P_{\bld{k}'} \right] &=& i \delta^3 (\bld{k} - \bld{k}')
\label{eqn:RSC_QPcomm}
\end{eqnarray}
Eqs. (\ref{eqn:RSC_Qk_def}) and (\ref{eqn:RSC_Pk_def})
reads,
\begin{equation}
a(\bld{k}) = \left( k^0 Q_{\bld{k}} (t) + i P_{\bld{k}}^\dagger (t) \right)
e^{i k^0 t}\,,
\label{eqn:RSC_a_by_QP} 
\end{equation}
\begin{equation}
a^\dagger(\bld{k}) = \left( k^0 Q_{\bld{k}}^\dagger (t) - i P_{\bld{k}} (t) \right)
e^{- i k^0 t}\,,
\label{eqn:RSC_a_by_QP} 
\end{equation}
and Eq. (\ref{eqn:cancomm_RS}) follows
from Eqs. (\ref{eqn:RSC_QQcomm}) and (\ref{eqn:RSC_QPcomm}).
}\\
\verb/-----------.-----------.-----------.-----------.-----------/\\


%=================================

Notice that $dim [\varphi] = E^1$ as can be seen from Eq. (\ref{eqn:RSCLagrangiandensity})
and $dim [\pi] = E^2$. Thus they are physical quantities  quite different from
those in the case of the Schr\"odinger field theory.\\

We are now making use of a fact that operators in Eq. (\ref{eqn:cancomm_RS}) satisfiy
the condition of the bosonic creation-annihilation operators, (\ref{eqn:Nbodycreanncomm}), 
for the case of continuous eigenvalues.
The total number operator can be defined as
\begin{equation}
\hat{N} = \int \frac{d^3 \bld{k}}{2k^0} a^\dagger(\bld{k}) a(\bld{k})
\label{eqn:KG_totalNumberOp}
\end{equation}
It holds that
\begin{equation}
\begin{array}{l}
\displaystyle
\hat{N} a(\bld{k}) = 
\int \frac{d^3 \bld{k}'}{2k^{0'}} \left\{
a(\bld{k}) a^\dagger(\bld{k}') - 2 k^0 \delta^3(\bld{k} - \bld{k}') \right\} a(\bld{k}')\\
\hspace{12mm} 
= a(\bld{k}) (\hat{N} - 1)\,,
\\
\hat{N} a^\dagger(\bld{k}) 
= a^\dagger(\bld{k}) (\hat{N} + 1)
\end{array}
\end{equation}
The Hamiltonian (\ref{eqn:KG_Hamiltonian}) reads from 
Eqs. (\ref{eqn:RS_fieldexpansion}) and (\ref{eqn:RS_pifieldexpansion}) as,
\begin{equation}
\hat{H} = \int \frac{d^3 \bld{k}}{2k^0} k^0 a^\dagger(\bld{k}) a(\bld{k})
\label{eqn:KG_Hamiltonian_aadagg}
\end{equation}
□Total momentum comes here
%-------------------------------------------
\footnote{
Applying  Noether's theorem to the invariance under space-time translations,
we have an expression for the conserved energy-momentum vector as
\begin{equation*}
P^\mu = \int T^{0 \mu} (x) d^3 \bld{x}
\end{equation*}
where the conserved Noether current is given as
\begin{equation*}
T^{\mu}_{\;\nu} (x) =
\frac{\partial {\cal L}}{\partial (\partial_\mu \varphi(x))}
\partial_\nu \varphi(x) 
- g^\mu_\nu \,{\cal L}
\end{equation*}
\begin{equation*}
\hat{P}^\mu =  \int \frac{d^3 \bld{k}}{2k^0} k^\mu a^\dagger(\bld{k}) a(\bld{k})
\end{equation*}
} % footnote end
\\
We have
\begin{equation}
[ \hat{P}^\mu, \hat{N}] = 0
\label{eqn:commutingHandN}
\end{equation}
and this relationship establishes particle interpretation.
Namely a state $a^\dagger({\bld{k}})\ketend 0 \ket$ is interpreted as
one particle eigenstate of the momentum associated with an eigenvalue $\bld{k}$.
\begin{equation}
\hat{\bld{P}} \hat{a}^\dagger(\bld{k}) \ketend 0 \ket
=
\bld{k} \hat{a}^\dagger(\bld{k}) \ketend 0 \ket
\end{equation}
so that we may write
\begin{equation}
\hat{a}^\dagger(\bld{k}) \ketend 0 \ket = \ketend \bld{k} \ket
\end{equation}
State normalization
\[
\bra \bld{p} \braend \bld{p}' \ket
= \bra 0 \braend [a(\bld{p}), a^\dagger(\bld{p}')] \ketend 0 \ket
= 2 k^0 \delta^3 (\bld{p} - \bld{p}')
\]
The Hamiltonian (\ref{eqn:KG_Hamiltonian_aadagg}) is diagonalized
by creation-annihilation operators:
\begin{equation}
[ \hat{H}, \hat{a}^\dagger(\bld{k}) ] = k^0 \hat{a}^\dagger(\bld{k})
\,, \hspace{3mm}
[ \hat{H}, \hat{a}(\bld{k}) ] = - k^0 \hat{a}(\bld{k})
\label{eqn:KGcommHanda}
\end{equation}
so that
\begin{equation}
\hat{H} \hat{a}^\dagger(\bld{k}) \ketend 0 \ket
=
k^0 \hat{a}^\dagger(\bld{k}) \ketend 0 \ket
\end{equation}
Wave function:\\
For
\begin{equation*}
\ketend \Psi^{(1)} \ket
\leftdef
\int \frac{d^3 \bld{k}}{2k^0}  \Psi^{(1)} (\bld{k}) a^\dagger(\bld{k}) \ketend 0 \ket\,,
\end{equation*}
\begin{equation*}
\hat{N} \ketend \Psi^{(1)} \ket = \ketend \Psi^{(1)} \ket\,,
\end{equation*}
\begin{equation*}
\bra \bld{k} \braend \Psi^{(1)} \ket
=  \Psi^{(1)} (\bld{k})
\end{equation*}
The state is normalized through a relationship
\begin{eqnarray*}
\bra \Psi^{(1)} \braend \Psi^{(1)} \ket
=
\int \frac{d^3 \bld{k}}{2k^0}  |\Psi^{(1)}(\bld{k})|^2
\end{eqnarray*}

A state of $n$ particles at momenta $\bld{k}_1$, $\bld{k}_2, \dots \bld{k}_n$,  
is written as
\begin{eqnarray*}
\ketend \bld{k}_1 \cdots \bld{k}_n \ket
=
\frac{1}{\sqrt{n!}}
a^{\dagger}(\bld{k}_1) \cdots a^{\dagger}(\bld{k}_n)
\ketend 0 \ket
\end{eqnarray*}
Since $a^{\dagger}(\bld{k}_1), \dots a^{\dagger}(\bld{k}_n)$
commute among themselves, the state is symmetric under
change of orders of momenta.
We have
\footnote{%----------------------------------------------footnote >>
Useful relationships
\begin{eqnarray*}
\left[
a, a_1^\dagger \cdots a_n^\dagger
\right]
=
\sum_{i=1}^n
a_1^\dagger \cdots [a, a_i^\dagger] \cdots a_n^\dagger
\end{eqnarray*}
\begin{eqnarray*}
\left[
a_1 \cdots a_n, a^\dagger
\right]
=
\sum_{i=1}^n
a_1 \cdots [a_i, a^\dagger] \cdots a_n
\end{eqnarray*}

}%----------------------------------------------footnote //
\begin{eqnarray*}
\hat{N} \ketend \bld{k}_1 \cdots \bld{k}_n \ket
&=&
\frac{1}{\sqrt{n!}}
 \int \frac{d^3 \bld{k}}{2k^0} a^\dagger(\bld{k}) 
 \left[ a(\bld{k}), 
a^{\dagger}(\bld{k}_1) \cdots a^{\dagger}(\bld{k}_n)
\right]
\ketend 0 \ket
 \\
&=&
\frac{1}{\sqrt{n!}}
 \int \frac{d^3 \bld{k}}{2k^0} a^\dagger(\bld{k}) 
 \left(
 \left[ a(\bld{k}), 
a^{\dagger}(\bld{k}_1)
\right]
a^{\dagger}(\bld{k}_2) \cdots a^{\dagger}(\bld{k}_n)
\right.
\\
&&+
\left.
a^{\dagger}(\bld{k}_1)
 \left[ a(\bld{k}), 
a^{\dagger}(\bld{k}_2) \cdots a^{\dagger}(\bld{k}_n)
\right]
\right)
\ketend 0 \ket
 \\
&=&
\cdots
 \\
&=&
n \ketend \bld{k}_1 \cdots \bld{k}_n \ket
\end{eqnarray*}
It is normalized as
\begin{eqnarray}
&&\bra \bld{k}_1 \cdots \bld{k}_n 
\ketend \bld{k}'_1 \cdots \bld{k}'_n \ket
\nonumber\\
&=&
\frac{1}{n!}
\sum_{i'_1 = 1}^n
\bra 0 \braend
a_2 \cdots a_{n} a_{1'} \cdots [a_1, a_{i'_1}^\dagger ] \cdots a_{n'}^\dagger
\ketend 0 \ket
\nonumber\\
&=&
\frac{1}{n!}
\sum_{i'_1 = 1}^n
\sum_{i'_2 \neq i'_i}^n
\bra 0 \braend
a_3 \cdots a_{n} a_{1'} \cdots \cancel{a_{i'_1}} \cdots \cancel{a_{i'_2}} \cdots a_{n'}^\dagger
\ketend 0 \ket 
\left[ a_1, a_{i'_1}^\dagger \right]
\left[ a_2, a_{i'_2}^\dagger \right]
\nonumber\\
&=&
\cdots
\nonumber\\
&=&
\frac{1}{n!}
\sum_{i'_1 \dots i'_n = perm(1\dots n)}^{n! \mbox{ terms}}
\prod_{l}^n
2 k_l^0 \delta^3 (\bld{k}_l - \bld{k}_{i'_l})
\label{eqn:nscalarnorm}
\end{eqnarray}
We may construct a state of $n$ scalar particles described by a wave function
$\Psi^{(N)}(\bld{k}_1, \dots, \bld{k}_n)$ as
\begin{eqnarray}
\ketend \Psi^{(N)} \ket
=
\int \prod_i^N 
\frac{d^3 \bld{k}_i}{2 k_i^0}
 \Psi^{(N)} (\bld{k}_1,\dots,\bld{k}_N) 
 \ketend
 \bld{k}_1,\cdots,\bld{k}_N
\ket
\label{eqn:NscalarState}
\end{eqnarray}
Since $n$ particle momentum satate is symmetric under exchange of momenta, 
we may presuppose the function $\Psi^{(N)}(\bld{k}_1, \dots, \bld{k}_n)$ is also
symmetric. Then we have
\begin{eqnarray*}
\bra  \bld{k}_1 \cdots \bld{k}_n \braend
\Psi^{(N)} \ket
= 
 \Psi^{(N)} (\bld{k}_1,\dots,\bld{k}_N) 
 \end{eqnarray*}
When the number of particles is fixed, normalization of the state  is written as
\begin{eqnarray*}
\| \ketend \Psi^{(N)} \ket \|^2
=
\int \prod_i^N 
\frac{d^3 \bld{k}_i}{2 k_i^0}
\| \Psi^{(N)} (\bld{k}_1,\dots,\bld{k}_N) \|^2
=
1
\end{eqnarray*}
The most general state 
in the Fock space
%>>>>>>>>>>>>>>>>>>>>>>>>>comment
%\begin{comment}
may be written as
\begin{equation*}
\ketend \Psi^{} \ket
=
\sum_N
\ketend \Psi^{(N)} \ket
\end{equation*}
In this case, $\| \ketend \Psi^{(N)} \ket \|^2$
gives the probability to find the system 
with $n$ particles.
%\end{comment}
%<<<<<<<<<<<<<<<<<<<<<<<<<< comment //

Particle states composed like this are these in the Heisenberg picture.
If we choose $e^{-i k^0 t_0} \hat{a}(\bld{k})$ and $e^{i k^0 t_0} \hat{a}^\dagger(\bld{k})$
as initial values of Heisenberg operators $\hat{a}_H(\bld{k}, t)$ and $\hat{a}_H^\dagger(\bld{k}, t)$
respectively at $t = t_0$, we have from Eq. (\ref{eqn:Heisenbergformalsol}) that
\footnote{
From Eq. (\ref{eqn:KGcommHanda}), we have
\begin{equation*}
H a^\dagger (\bld{k}) = a^\dagger (H + k^0)
\,,\hspace{3mm}
H^2 a^\dagger (\bld{k}) = a^\dagger (H + k^0)^2\,,\dots
\end{equation*}
so that
\begin{eqnarray*}
e^{i H (t-t_0)} a^\dagger (\bld{k})
&=&
\sum_n^\infty \frac{i^n}{n!} (t-t_0)^n H^n a^\dagger (\bld{k})
\\
&=&
a^\dagger (\bld{k}) \sum_n^\infty \frac{i^n}{n!} (t-t_0)^n (H + k^0)^n 
\\
&=&
a^\dagger (\bld{k})  e^{i (H + k^0) (t-t_0)}
\end{eqnarray*}
} % footnote end
\begin{equation}
a_H(\bld{k}, t) = e^{-i k^0 t} a (\bld{k})\,,
\hspace{3mm}
a_H^\dagger(\bld{k}, t) = e^{i k^0 t} a^\dagger (\bld{k})
\end{equation}
We introduce state vectors in the Schr\"odinger picture by
\begin{equation}
{}_S\bra \Psi(t) \braend {\cal O} \ketend \Psi(t) \ket_S
=
\bra \Psi \braend {\cal O}_H(t) \ketend \Psi \ket
\end{equation}
State vector in the Schr\"odinger picture obeys 
\begin{equation}
i \partial_t \ketend \Psi(t) \ket_S = \hat{H} \ketend \Psi(t) \ket_S
\end{equation}

\bigskip

%<<<<<<<<<<<<<<<<<<<<<<<<<<<<<<<<<<<<<<
\noindent
%\begin{equation}
%\left( \Box + m^2 \right) \varphi(x) = 0
%\end{equation}
Propagator
\begin{equation}
\left( \Box + m^2 \right) \Delta_F(x) = \delta^4(x) 
\end{equation}
\begin{eqnarray}
\Delta_F(q) 
&=&
i \int d^4x e^{iqx} 
\bra 0 \braend \mbox{T}\varphi(x) \varphi(0) 
\ketend 0 \ket
\nonumber\\
&=&
\frac{-1}{q^2 - m^2 + i\epsilon}\,,
\end{eqnarray}
where $\epsilon$ is infinitesimal positive number.
T-product
\begin{equation}
T[\varphi(x) \varphi(y)] =
\theta(x^0 - y^0)\varphi(x) \varphi(y)
+
\theta(y^0 - x^0)\varphi(y) \varphi(x)
\end{equation}
One will find
\[
\left( \Box + m^2 \right) T[\varphi(x) \varphi(0)] = -i \delta^4(x)
\]
\begin{equation}
\Delta_F(x) = \int \frac{d^4 q}{(2\pi)^4} e^{-iqx} \Delta_F(q)
\end{equation}