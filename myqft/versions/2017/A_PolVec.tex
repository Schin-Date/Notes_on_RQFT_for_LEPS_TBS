Polarization vectors $\epsilon_\mu^{(\lambda)} (\bld{p})$
for a free Klein-Goldon vector field 
%without any constraint
are chosen as 4 linearly independent basis:
\begin{eqnarray}
\left\{
\begin{array}{l}
g^{\mu \nu }{\epsilon_\mu^{(\lambda)}}^*   \epsilon_\nu^{(\lambda')}
=
g^{\lambda \lambda'}
\hspace{5mm}
\mbox{(orthonormal)}
\vspace{2mm}
\\
g_{\lambda \lambda'} \epsilon_\mu^{(\lambda)} {\epsilon_\nu^{(\lambda')}}^* 
= g_{\mu \nu}
\hspace{5mm}
\mbox{(complete)}
\end{array}
\right.
\label{eqn:APolOrthComp}
\end{eqnarray}
An obvious choice is $\epsilon^{(\lambda)\mu} = \epsilon^{(\lambda)*\mu} = e^{(\lambda)\mu}$
where
\begin{eqnarray}
e^{(0)\mu} =
\left(
\begin{array}{c}
1\\ \bld{0}
\end{array}
\right)^\mu
\,,
\hspace{2mm}
e^{(i)\mu} =
\left(
\begin{array}{c}
0\\ \bld{e}^{(i)}
\end{array}
\right)^\mu
\,
i = 1, 2, 3
\label{eqn:APolBasis}
\end{eqnarray}
and $\bld{e}^{(i)}$ are 3 orthnormal spacial vectors.
The choice of $\bld{e}^{(i)}$ may depend on the direction of
the momentum $\bld{p}$.
In particular, we may choose $\bld{e}^{(3)} = \hat{\bld{p}} \equiv \bld{p} / |\bld{p}|$  and
$\bld{e}_2 = \hat{\bld{p}} \times \bld{e}_1$ with
$\bld{e}_1$ being an arbitrary chosen spacial unit vector perpendicular 
to $\hat{\bld{p}}$. Regarding with the choice of transverse polarization basis,
it is also common to employ
\begin{eqnarray}
e^{(\pm)} = \frac{1}{\sqrt{2}} 
\left(
e^{(1)} \pm i e^{(2)}
\right)
\label{eqn:APolCircPol}
\end{eqnarray}
to represent the two circular polarizations.
The problem of indefinite norm of the massless Klein-Gordon field
is avoided by suppressing excitations in the time direction in terms of
subsidiary conditions, which usually take a form of gauge conditions.
In a consistent scheme, the longitudinal excitation is also dropped 
and only transverse modes, which satisfy $p \cdot \epsilon^{(T)} = 0$,
 are survived as physical excitations.

In cases of massive fields,
excitations of time components are excluded from
dynamical degrees of freedom by virtue of
the Lorentz condition, which is the unique Lorentz invariant
restriction on the field. It is therefore essential to have
polarization vectors which satisfy
$\epsilon^{(\uparrow)} \cdot p = 0$.
Since there exists a rest frame for each momentum 
$p = (p^0, \bld{p}) = m (\gamma, \gamma \bld{\beta})$,
we may write
\begin{eqnarray}
\epsilon^{(\lambda)\mu} (\bld{p}) 
=
\Lambda^\mu_{\;\;\nu} (\bld{\beta})\, e^{(\lambda) \mu}
\,,
\hspace{3mm}
\Lambda^\mu_{\;\;\nu}
=
\left(
\begin{array}{cc}
\gamma & \gamma \bld{\beta} \cdot
\\
\gamma \bld{\beta} &
1 + \hat{\bld{\beta}}(\gamma - 1) \hat{\bld{\beta}} \cdot
\end{array}
\right)
\,,
\label{eqn:ApolVecMassive}
\end{eqnarray}
for which
we surely have above mentinoed relationship.
Also, it is straightforward to confirm relationships in Eq. (\ref{eqn:APolOrthComp})
and
\begin{eqnarray}
g_{ij} \epsilon^{(i)\mu} \epsilon^{(j)*\nu}
&=&
g^{\mu \nu} - g^{00} \epsilon^{(0)\mu} \epsilon^{(0)*\nu}
\nonumber\\
&=&
g^{\mu \nu} -
\frac{p^\mu p^\nu}{m^2}
\label{eqn:APolMassSpatialComp}
\end{eqnarray}