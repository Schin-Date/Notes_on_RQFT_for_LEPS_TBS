In place of Eq. (\ref{eqn:SchrFieldprimitive}), if we define
\begin{equation}
\hat{\Psi}(t, \bld{x}) = \sum_l \hat{a}_l e^{-i \epsilon_l t}\varphi_l (\bld{x})\,,
\label{eqn:SchWaveOp}
\end{equation}
this operator satisfies the single particle Schr\"odinger equation:
\begin{equation*}
\hspace{40mm}
i \partial_t \hat{\Psi}(t, \bld{x}) = {H} \hat{\Psi}(t, \bld{x}) \,
\hspace{40mm}
(\widehat{\ref{eqn:SchroedingerEq}})
\end{equation*}
We already know that any number of Schr\"odinger particles can be
generated by $\hat{\varphi}(\bld{x}) = \hat{\Psi}(0, \bld{x})$.
Comparing Eq. (\ref{eqn:SchWaveOp}) with
Eq. (\ref{eqn:Psiexpansion}), we observe 
$\hat{\Psi}(t, \bld{x})$ is obtained by replacing 
c-number amplitude $\Psi_l$ in ${\Psi}(t, \bld{x})$
by annihilation operator $\hat{a}_l$.
The field operator  $\hat{\Psi}$ satisfies the following
equaltime commutation relations:
\begin{equation}
\begin{array}{l}
[\hat{\Psi}(t, \bld{x}), \hat{\Psi}^\dagger(t, \bld{x'})]
=
\delta^3 (\bld{x} - \bld{x}')
\vspace{1mm}\\ \relax
[\hat{\Psi}(t, \bld{x}), \hat{\Psi}(t, \bld{x'}) ]
=
[\hat{\Psi}^\dagger(t, \bld{x}), \hat{\Psi}^\dagger(t, \bld{x'})]
= 0
\end{array}
\label{eqn:SchFieldcommlutationRel}
\end{equation}
If we require Eq. (\ref{eqn:SchFieldcommlutationRel}), then Eq. (\ref{eqn:Nbodycreanncomm})
follows.
Schr\"odinger equation ($\widehat{\ref{eqn:SchroedingerEq}}$) follows from a
{\cal L}agrangian density
\begin{equation}
{\cal L} = i \Psi^* \partial_t \Psi + \frac{1}{2m} \Psi^* \bld{\partial}^2 \Psi - V \Psi^* \Psi
\label{eqn:SchroedingerFieldLagrangian}
\end{equation}
Here we note $dim [{\cal L}] = E^4$ and $dim [\Psi] = E^{3/2}$.
Canonical momentum field is given as
\begin{equation}
\Pi \leftdef \frac{\partial \cal L}{\partial \dot{\Psi}} = i \Psi^*\,,
\end{equation}
which has the same dimension as $\Psi$.
The set (\ref{eqn:SchFieldcommlutationRel}) is equivalent with
\begin{equation}
\begin{array}{l}
[\hat{\Psi}(t, \bld{x}), \hat{\Pi}(t, \bld{x'})]
=
\delta^3 (\bld{x} - \bld{x}')
\vspace{1mm}\\ \relax
[\hat{\Psi}(t, \bld{x}), \hat{\Psi}(t, \bld{x'}) ]
=
[\hat{\Pi}(t, \bld{x}), \hat{\Pi}(t, \bld{x'})]
= 0
\end{array}
\label{eqn:SchFieldCanonicalcommlutationRel}
\end{equation}
This is the canonical commutation relation.
We may reverse our argument starting from Eq. (\ref{eqn:SchroedingerFieldLagrangian}),
writing down the "field equation" ($\widehat{\ref{eqn:SchroedingerEq}}$) and
requiring the equaltime canonical commutation relation (\ref{eqn:SchFieldCanonicalcommlutationRel}).
This procedure is called the 2nd quantizaton.
%-------------------------------------------------------------------- comment block below
\begin{comment}
%Operators acting in a Hilbert space. Physical quantities are hermitian operators.
%Born's probability interpretation is employed.\\
\begin{eqnarray}
\mbox{Schr\"odinger eq.} \hspace{24mm} 
&\hspace{-5mm}&
\hspace{-20mm}
i \partial_t \Psi(t, \bld{x}) = {H} \Psi(t, \bld{x}) \,,
%i \partilal_t \Psi (t, \bld{x}) = H \Psi (t, \bld{x})
\hspace{3mm}
H = - \frac{\bld{\partial}^2}{2m} + V(\bld{x}) = H^\dagger
\label{eqn:SchroedingerEq}
\\
\mbox{Eigenstates of } H \hspace{22mm} 
&\hspace{-5mm}&
\hspace{-20mm}
H \varphi_l (\bld{x} ) = \epsilon_l \varphi_l (\bld{x} )\,,
\hspace{3mm}
\{ \varphi_l (\bld{x} ) \} \mbox{: orthogonal, complete}
\label{eqn:energyEigenSt}
\\
\mbox{Solution of the Schr\"odinger eq.}
&\hspace{-5mm}&
\Psi(t, \bld{x})
=
\sum_l \Psi_l e^{-\epsilon_l t} \varphi_l (\bld{x})
\label{eqn:Psiexpansion}
\end{eqnarray}
%-------------------------------------------------------------------
%
Bras and kets
\begin{equation}
\Psi(t, \bld{x}) = \bra \bld{x} \braend \Psi(t) \ket\,,
\hspace{5mm}
\varphi_l(\bld{x})  = \bra \bld{x} \braend \epsilon_l \ket\,, 
\end{equation}
\begin{equation*}
\hspace{21mm}
i \partial_t  \ketend \Psi(t) \ket = \hat{H} \ketend \Psi(t) \ket\,,
\hspace{5mm}
\hat{H} = \frac{\hat{\bld{p}}^2}{2 m} + V(\bld{x}) = \hat{H}^\dagger
\hspace{20mm}
\bra \ref{eqn:SchroedingerEq} \ket
\end{equation*}
\begin{equation*}
\hspace{13mm}
\hat{H} 
\ketend \epsilon_l \ket
= 
\epsilon_l \ketend \epsilon_l \ket\,,
\hspace{3mm}
\bra \epsilon_l \ketend \epsilon_{l'} \ket = 0 \mbox{ if } l\neq l'\,,
\hspace{3mm}
\sum_l \ketend \epsilon_{l} \ket \bra \epsilon_{l} \braend = 1
\hspace{10mm}
\bra \ref{eqn:energyEigenSt} \ket
\end{equation*}
\begin{equation*}
\hspace{32mm}
\ketend \Psi(t) \ket =
\sum_l \bra \epsilon_l \braend   \Psi(0) \ket
e^{-\epsilon_l t} \ketend \epsilon_l \ket
\hspace{30mm}
\bra \ref{eqn:Psiexpansion} \ket
\end{equation*}
%-------------------------------------------------------------------
%
1st quantization
\begin{equation}
[\hat{x}_i, \hat{p}_j] = i \delta_{ij}
\end{equation}
Eigen states
\begin{equation}
\begin{array}{l}
\hat{\bld{x}} \ketend \bld{x} \ket = \bld{x} \ketend \bld{x} \ket
\\
\hat{\bld{p}} \ketend \bld{p} \ket = \bld{p} \ketend \bld{p} \ket
\end{array}
\end{equation}
Conventional ("half relativistic") normalization
\begin{equation}
\bra \bld{x} \braend \bld{x}' \ket = \delta^3(\bld{x}-\bld{x}')\,,
\hspace{5mm}
\int \ketend \bld{x} \ket d^3 \bld{x} \bra \bld{x} \braend = 1
\end{equation}
\begin{equation}
\bra \bld{p} \braend \bld{p}' \ket = 2E \delta^3(\bld{p}-\bld{p}')\,,
\hspace{5mm}
\int \ketend \bld{p} \ket \frac{d^3 \bld{p}}{2E} \bra \bld{p} \braend = 1
\label{eqn:norm_p_state_rel}
\end{equation}
Coordinate representation of the momentum operator
\begin{equation}
\bra \bld{x} \braend \hat{\bld{p}}
=
\frac{1}{i} \bld{\partial} \bra \bld{x} \braend
\end{equation}
and of the momentum eigenstate
\begin{equation}
\bra \bld{x} \braend \bld{p} \ket
=
\sqrt{\frac{2E}{(2\pi)^3}}\, e^{i \bld{p} \cdot \bld{x}}
\end{equation}
Requirement for the normalization factor is understood by employing
the first equation of (\ref{eqn:norm_p_state_rel}) in the $l.h.s$ of
$\nolinebreak{\bra \bld{p} \braend \bld{p}' \ket}= $
$\nolinebreak{\int \bra \bld{p} \braend \bld{x} \ket d^3\bld{x} \bra \bld{x} \ketend \bld{p}' \ket}$
and remembering a formula
\[
\int {d^3 \bld{x}}\, e^{\pm i \bld{p} \cdot \bld{x}}= (2\pi)^3 \delta^3 (\bld{p})
\]
\end{comment}