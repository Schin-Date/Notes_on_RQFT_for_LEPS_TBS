
\indent

So far, we are evaluating S-matrix elements making use of Dyson's formula (\ref{eqn:SmatrixPertSerCov})
where interaction Hamiltonian density in the interaction picture is denoted just as $\mathcal{H}_{int}$.
This is however a shorthand for $(\mathcal{H}_{int})_I$ 
if we faithfully follow notations used in Eqs. (\ref{eqn:SmatrixPertSer}) and (\ref{eqn:defHintI}).
This shorthand had been allowed since $(\mathcal{H}_{int})_I$ 
could be obtained just by replacing Heisenberg operator fields involved in $\mathcal{H}_{int} = -\mathcal{L}_{int}$
by ones in the interaction picture.
However, this blessed situation relies on a fact that
we have been considering only $\mathcal{L}_{int}$
that does not involve derivatives of fields.
Let us consider
 a theory with field $\varphi (x)$ described by
$\mathcal{L} = \mathcal{L}_0 (\varphi, \dot{\varphi}) + \mathcal{L}_{int}$, where
we omit spatial derivatives of $\varphi$ included in $\mathcal{L}$.
When $\mathcal{L}_{int}$ does not involve $\dot{\varphi}$, 
the conjugate momentum is given by
$\pi = \partial \mathcal{L}_0 / \partial \dot{\varphi}$ and the Hamiltonian density reads
$\mathcal{H} = \pi \dot{\varphi} - \mathcal{L} = \mathcal{H}_0 - \mathcal{L}_{int}$, where
$\mathcal{H}_0 = \dot{\varphi} \partial \mathcal{L}_0 / \partial \dot{\varphi} - \mathcal{L}_0$.
Thus we may write $\mathcal{H}_{int} = -\mathcal{L}_{int} (\varphi)$ and
understand that $(\mathcal{H}_{int})_I =  -\mathcal{L}_{int} (\varphi_I)$.
This is the situation we have been engaged so far.
For derivative couplings at play, we have $\mathcal{L}_{int} = \mathcal{L}_{int} (\varphi, \dot{\varphi})$
and it turns out
\footnote{%------------------------------- footnote >>
See addendum 1 below which follows ref. \cite{ref:NIsh.}.
}%------------------------------- footnote \\
that
$(\mathcal{H}_{int})_I = -\mathcal{L}_{int} (\varphi_I, \dot{\varphi}_I)
+ (1/2)\left( \partial \mathcal{L}_{int} / \partial \dot{\varphi}_I \right)^2$ 
for systems with $\partial \mathcal{L}_0 / \partial \dot{\varphi} = \dot{\varphi}$
and $\mathcal{L}_{int}$ depending only linearly on $\dot{\varphi}$.
The appearance of the second term in $(\mathcal{H}_{int})_I$ is rather messy.


There is another problem caused by the presence of derivative couplings.
In evaluating S-matrix, we will meet vacuum expectation value of
the T-product of derivatives of fields. 
It turns out
\footnote{%------------------------------- footnote >>
See addendum 2 below.
}%------------------------------- footnote \\
 that
\begin{eqnarray}
\bra 0 \braend T\left[
\partial_\mu \varphi(x) \partial_\nu \varphi(y) \right]
\ketend 0 \ket
&=&
\partial_{x^\mu}
\partial_{y^\nu}
\bra 0 \braend T\left[
\varphi(x) \varphi(y) \right]
\ketend 0 \ket
\nonumber\\
&&
-
i \delta_{\mu 0} \delta_{\nu 0}
\delta^4(x - y)
\label{eqn:DerivCoupTprod}
\end{eqnarray}
The second term in the $r.h.s$ is, again, rather messy.
It turns out, however, that the above mentioned two problems cancel each other
%and the result is a quite simple prescription.
resulting in a  simple prescription.
Namely, we may keep to write 
$\mathcal{H}_{int} = -\mathcal{L}_{int}(\varphi_I, \partial_\mu \varphi_I)$ 
in Eq. (\ref{eqn:SmatrixPertSerCov})
with a compensation that we must replace T products by
T${}^*$ products defined by
\begin{eqnarray}
T^*\left[
\partial_\mu \varphi(x) \partial_\nu \varphi(y) \right]
&\equiv&
\partial_{x^\mu}
\partial_{y^\nu}
T\left[
\varphi(x) \varphi(y) \right]
\end{eqnarray}

\bigskip

\verb/-----------.-----------.-----------.-----------.-----------/\\
\vspace{-3mm}
{\small
\begin{center}
Addendum1. Effect of derivative couplings on the form of $(\mathcal{H}_{int})_I$
\end{center}

Field operators $\varphi_I (x)$ in the interaction picture follow free field equations 
corresponding to a Lagrangian density $\mathcal{L}_{0}(\varphi_I (x), \dot{\varphi_I} (x))$.
The momentum field conjugate to $\varphi_I (x)$ is given as
$\pi_I (x) = \partial \mathcal{L}_0 / \partial \dot{\varphi_I} (x)$
and the free Hamiltonian density in this picture
$(\mathcal{H}_0)_I = \pi_I \dot{\varphi_I} -  \mathcal{L}_{0}(\varphi_I (x), \dot{\varphi_I} (x))$
govern the time evolution of $\varphi_I (x)$.
As we have discussed in \ref{sec:InteractingFields2},
operators in the interaction picture $\mathcal{O}_I$ are related with
ones in the Heisenberg picture, $\mathcal{O}_H$, by an unitary time evolution 
operator $U$ as $\mathcal{O}_I = U \mathcal{O}_H U^{-1}$.
%\begin{eqnarray}
%\mathcal{O}_I = U \mathcal{O}_H U^{-1}
%\end{eqnarray}

Writing the full Lagrangian density as 
$\mathcal{L} = \mathcal{L}_0 + \mathcal{L}_{int} (\varphi (x), \dot{\varphi} (x))$,
the conjugate momentum is given by
$\pi = \partial \mathcal{L}_0 / \partial \dot{\varphi} + \partial \mathcal{L}_{int} / \partial \dot{\varphi}$.
The second term contributes when derivative couplings are involved in $\mathcal{L}_{int}$.
For simplicity, we consider the case 
such like as the scalar field
where
$\partial \mathcal{L}_0 / \partial \dot{\varphi} = \dot{\varphi}$
so that $\pi = \dot{\varphi} + \partial \mathcal{L}_{int} / \partial \dot{\varphi}$
and $\pi_I = \dot{\varphi}_I$.
The full Hamiltonian density is defined by
$\mathcal{H} = \pi \dot{\varphi} - \mathcal{L}(\varphi (x), \dot{\varphi} (x))$
and it is written in terms of Heisenberg fields.
Converting $\mathcal{H}$ into the interaction picture as
$\mathcal{H}_{I} = U \mathcal{H} U^{-1}$,
one meets an object 
$U^{-1} \dot{\varphi} U \rightdef (\dot{\varphi})_I$
 which is different from
$\dot{\varphi}_I = \partial_0 ({\varphi}_I)$. 
We write
\begin{eqnarray}
(\dot{\varphi})_I
&=&
\pi_I - U \frac{\partial \mathcal{L}_{int}}{\partial \dot{\varphi}} (\varphi , \dot{\varphi} ) U^{-1}
\nonumber\\
&=&
\dot{\varphi}_I -  \frac{\partial \mathcal{L}_{int}}{\partial \dot{\varphi}_I} 
(\varphi_I , (\dot{\varphi})_I ) 
\nonumber\\
&=&
\dot{\varphi}_I -  \frac{\partial \mathcal{L}_{int}}{\partial \dot{\varphi}_I} 
(\varphi_I , \dot{\varphi}_I ) 
+
\Delta_2
\end{eqnarray}
where $\Delta_2$ represents terms of the second and higher orders of $\mathcal{L}_{int}$
and it appears only when $\mathcal{L}_{int}$ involves quadratic and higher orders of $\dot{\varphi}$.
The second term in $\mathcal{H}$ is converted as
\begin{eqnarray}
&&\mathcal{L} (\varphi_I , 
\dot{\varphi}_I -  \frac{\partial \mathcal{L}_{int}}{\partial \dot{\varphi}_I} 
+
\Delta_2
 )
\nonumber\\
&&=
\mathcal{L} (\varphi_I , \dot{\varphi}_I)
+
\frac{\partial \mathcal{L}}
{\partial \dot{\varphi}_{I \alpha} }
\left\{
 -  \frac{\partial \mathcal{L}_{int}}{\partial \dot{\varphi}_I} 
+
\Delta_2
\right\}_\alpha
 \nonumber\\
&& \hspace{10mm}
+
\frac{1}{2}
\frac{\partial^2 \mathcal{L}}
{\partial \dot{\varphi}_{I \alpha} \partial \dot{\varphi}_{I \beta}}
\left\{
 -  \frac{\partial \mathcal{L}_{int}}{\partial \dot{\varphi}_I} 
+
\Delta_2
\right\}_\alpha
\left\{
 -  \frac{\partial \mathcal{L}_{int}}{\partial \dot{\varphi}_I} 
+
\Delta_2
\right\}_\beta
\label{eqn:DerCoupFullLagIntPic1}
\end{eqnarray}
where an extra suffix $\alpha$ or $\beta$ is added on the field to
correspond to cases of fields with multiple components. With our assumption
on the form of $\mathcal{L}_0$, we have
\begin{eqnarray}
\frac{\partial \mathcal{L}}
{\partial \dot{\varphi}_{I \alpha}}
=
\pi_I +
\frac{\partial \mathcal{L}_{int}}
{\partial \dot{\varphi}_{I \alpha}}
\,,
\hspace{5mm}
\frac{\partial^2 \mathcal{L}}
{\partial \dot{\varphi}_{I \alpha} \partial \dot{\varphi}_{I \beta}}
=
\delta^{\alpha \beta} + 
\frac{\partial^2 \mathcal{L}_{int}}
{\partial \dot{\varphi}_{I \alpha} \partial \dot{\varphi}_{I \beta}}
\end{eqnarray}
Then we write, 
\begin{eqnarray}
\mathcal{H}_I 
&=&
\pi_I (\dot{\varphi})_I -
\mathcal{L} (\varphi_I , (\dot{\varphi})_I )
\nonumber\\
&=&
\pi_I
\left\{
\dot{\varphi}_I
-
\frac{\partial \mathcal{L}_{int}}
{\partial \dot{\varphi}_{I }}
+
\Delta_2
\right\}
- \mathcal{L} (\varphi_I , \dot{\varphi}_I)
-
\left\{
\pi_I +
\frac{\partial \mathcal{L}_{int}}
{\partial \dot{\varphi}_{I }}
\right\}_\alpha
\left\{
 -  \frac{\partial \mathcal{L}_{int}}{\partial \dot{\varphi}_I} 
+
\Delta_2
\right\}_\alpha
\nonumber\\
&&
- \frac{1}{2}
\left\{
 -  \frac{\partial \mathcal{L}_{int}}{\partial \dot{\varphi}_I} 
+
\Delta_2
\right\}^2
-
\frac{\partial^2 \mathcal{L}_{int}}
{\partial \dot{\varphi}_{I}^2  } \times
\mathcal{O}(\mathcal{L}_{int}^2)
\nonumber\\
&=&
\pi_I \dot{\varphi}_I
- \mathcal{L} (\varphi_I , \dot{\varphi}_I)
-
\frac{\partial \mathcal{L}_{int}}
{\partial \dot{\varphi}_{I \alpha}}
\left\{
 -  \frac{\partial \mathcal{L}_{int}}{\partial \dot{\varphi}_I} 
+
\Delta_2
\right\}_\alpha
\nonumber\\
&&
- \frac{1}{2}
\left\{
 -  \frac{\partial \mathcal{L}_{int}}{\partial \dot{\varphi}_I} 
 +
\Delta_2
\right\}^2
-
\frac{\partial^2 \mathcal{L}_{int}}
{\partial \dot{\varphi}_{I}^2  } \times
\mathcal{O}(\mathcal{L}_{int}^2)
\nonumber\\
&=&
(\mathcal{H}_0)_I  - \mathcal{L}_{int} (\varphi_I , \dot{\varphi}_I)
+\frac{1}{2} \left(
 \frac{\partial \mathcal{L}_{int}}{\partial \dot{\varphi}_I} 
 \right)^2
 - 
 \left\{
 \frac{(\Delta_2)^2}{2}
 +
\frac{\partial^2 \mathcal{L}_{int}}
{\partial \dot{\varphi}_{I}^2  } \times
\mathcal{O}(\mathcal{L}_{int}^2)
\right\}
 \nonumber\\
\end{eqnarray}
Thus, we confirm a statement on the form of $(\mathcal{H}_{int})_I$ as it is mentioned in the text.


\vspace{7mm}
\begin{center}
Addendum2. Derivation of Eq. (\ref{eqn:DerivCoupTprod})
\end{center}
%\\
From the definition of T product in Eq. (\ref{eqn:TProdDef}), we have
\begin{eqnarray}
\partial_{x^\mu} T[\varphi(x) \varphi(y)]
=
T[\partial_{\mu} \varphi(x) \cdot \varphi(y)]
+
\delta(x^0 - y^0) [\varphi(x), \varphi(y)]
\label{eqn:DerivCoupSingleDeriv}
\end{eqnarray}
The second term in the $r.h.s.$ vanishes in the scheme of canonical quantization.
Thus, single derivative does not make any trouble. However,
\begin{eqnarray}
\partial_{y^\nu} T[\partial_{\mu} \varphi(x) \cdot \varphi(y)]
&=&
T[\partial_{\mu} \varphi(x)  \partial_{\nu} \varphi(y)]
+
\delta_{\nu 0} \delta(x^0 - y^0) [\varphi(y), \partial_{\mu} \varphi(x)]
\nonumber\\
&=&
T[\partial_{\mu} \varphi(x)  \partial_{\nu} \varphi(y)]
+
i \delta_{\mu 0}\delta_{\nu 0} \delta^4(x - y) 
\end{eqnarray}
and we obtain Eq. (\ref{eqn:DerivCoupTprod})
by substituting the relationship (\ref{eqn:DerivCoupSingleDeriv})
in the $l.h.s$ and
taking vacuum expectation values of the both hand sides.
} %------------------------------------ small //
\\
\verb/-----------.-----------.-----------.-----------.-----------/\\

\bigskip

Let us consider the Dirac Yukawa theory discussed in \ref{sec:DiracYukawa}
with a derivative coupling 
$\mathcal{L}_{int} = -g  \overline{\psi} \mathcal{O}^\mu \psi \partial_\mu \phi$,
where $\mathcal{O}^\mu$ denotes a Dirac matrix such as $\gamma^\mu$ for
vector current and $\gamma^5 \gamma^\mu$ for pseudo vector current.
For $NN \to NN$ scattering amplitude, we set the same initial and final states as
in Eq. (\ref{eqn:IniFinNNtoNN}).
The second order S-matrix reads
\begin{eqnarray}
S^{(2)}_{fi}
&=&
\frac{(-ig)^2}{2} \int d^4 x_{1'} d^4 x_{2'}
\bra N_2 N_1 \braend
T^*[
\normalprod{ \overline{\psi}_{1'} \mathcal{O}^\mu \psi_{1'} \partial_{x_{1'}^\mu}\phi_{1'}}
\normalprod{ \overline{\psi}_{2'} \mathcal{O}^\nu \psi_{2'} \partial_{x_{2'}^\nu}\phi_{2'}}
]
\ketend N_a N_b \ket
\nonumber\\
&=&
\frac{(-ig)^2}{2} \int d^4 x_{1'} d^4 x_{2'}
\bra 0 \braend T^*[
\partial_{x_{1'}^\mu}\phi_{1'}
\partial_{x_{2'}^\nu}\phi_{2'}
]
\ketend 0 \ket
\nonumber\\
&&
\hspace{30mm}
\bra N_2 N_1 \braend
T[
\normalprod{ \overline{\psi}_{1'} \mathcal{O}^\mu \psi_{1'} } 
\normalprod{ \overline{\psi}_{2'} \mathcal{O}^\nu \psi_{2'} }
]
\ketend N_a N_b \ket
\nonumber\\
%\label{eqn:DiracNNNNS2formula}
\end{eqnarray}
The last factor in the second equation is 
nothing but one in Eq. (\ref{eqn:DiracNNNNsand})
with addional Dirac matrices $\mathcal{O}^\mu$ and $\mathcal{O}^\nu$ 
between bilinear forms of Dirac spinors and the computation of this factor
goes along quite similar way as that through 
Eq. (\ref{eqn:DirNNNNsandIntermed}) to (\ref{eqn:DirNNNNabbamp}).
For the factor of the vacuum expectation value, we write
\begin{eqnarray}
\bra 0 \braend T^*[
\partial_{1'\mu}\phi_{1'}
\partial_{2'\nu}\phi_{2'}
]
\ketend 0 \ket
&=&
\bra 0 \braend 
\partial_{1'\mu}
\partial_{2'\nu}
T[
\phi_{1'}
\phi_{2'}
]
\ketend 0 \ket
\nonumber\\
&=&
\partial_{1'\mu}
\partial_{2'\nu}
\Delta_F(1' - 2')
\nonumber\\
&=&
i \int
\frac{d^4 k}{(2\pi)^4}
\frac{k_\mu k_\nu}{k^2 - \mu^2 + i\epsilon}
e^{- ik (x_{1'} - x_{2'})}
\nonumber\\
\end{eqnarray}
where we have used Eq. (\ref{eqn:FeynmanPropDefSc}).
Combining them, we would write
\begin{eqnarray}
T^{(2)}_{fi}
&=&
\frac{(-ig)^2}{(2\pi)^6}
\int d^4 k
\frac{k_\mu k_\nu}{k^2 - \mu^2 + i\epsilon}
\left\{
\delta^4(p_1 - p_a - k)
\left[
\bar{u}^{(1)} \mathcal{O}^\mu u^{(a)} \right]
\left[ \bar{u}^{(2)} \mathcal{O}^\nu u^{(b)} \right]
\right.
\nonumber\\
&&
\hspace{35mm}
-
\left.
\delta^4(p_2 - p_a - k)
\left[
\bar{u}^{(2)} \mathcal{O}^\mu u^{(a)} \right]
\left[
 \bar{u}^{(1)} \mathcal{O}^\nu u^{(b)} \right]
\right\}
\nonumber\\
\end{eqnarray}
Take your time to compare this formula with Eqs. (\ref{eqn:DirYukS2Feynm}, \ref{eqn:DirYukT2Feynm}).
 and also with Eqs. (\ref{eqn:MollerS2}, \ref{eqn:MollerT2}).
