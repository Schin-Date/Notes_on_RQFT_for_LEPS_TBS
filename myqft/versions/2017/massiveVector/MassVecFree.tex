\noindent
neutral or charged Spin 1 particle
$\left( \Box + m^2 \right) \varphi_\mu(x) = 0$


\subsubsection{Lagrangian}
For the complex field, terms which can appear in Lagrangian density are 
$\varphi^*_\mu \varphi^\mu$, 
$\partial_\mu \varphi^*_\nu \cdot \partial^\mu \varphi^\nu$, 
$\partial_\mu \varphi^*_\nu \cdot \partial^\nu \varphi^\mu$
and
$\partial^\mu \varphi^*_\mu \cdot \partial_\nu \varphi^\mu$.
We want Klein-Gordon equation as  the EOM.

Proca (1936) proposed a Lagrangian density which leads to a positive definite Hamiltonian density:
\begin{eqnarray}
{\cal L}_{\mbox{\scriptsize Pr}} &=&
-\frac{1}{2} F^*_{\mu\nu} F^{\mu\nu}
+ m^2 \varphi_\mu^* \varphi^\mu
\label{eqn:ProcaLagdens}
\\
&&
(
=
- \partial_\mu \varphi^*_\nu
F^{\mu \nu}
=
- F^*_{\mu \nu}
\partial^\mu \varphi^\nu
)
\nonumber
\,,
\end{eqnarray}
where
\begin{eqnarray}
F_{\mu \nu} \equiv
\partial_\mu \varphi_\nu
-
\partial_\nu \varphi_\mu
\end{eqnarray}
dim[$\varphi$] $=$ dim[$m$] $=$ $E$.
For the case of real field, remove conjugation symbol ${}^*$ and divide each term by 2.
\begin{eqnarray}
\frac{\partial \mathcal{L}}
{\partial (\partial_\mu \varphi_\nu^*)}
=
-F^{\mu \nu}
\,,
\hspace{5mm}
\frac{\partial \mathcal{L}}
{\partial (\partial_\mu \varphi_\nu)}
=
-F^{*\mu \nu}
\end{eqnarray}
Euler-Lagrange equation reads
\begin{eqnarray}
\partial_\mu F^{\mu \nu} = - m^2 \varphi^\nu
\,,
\hspace{5mm}
\partial^\mu F^*_{\mu \nu} = - m^2 \varphi_\nu^*
\label{eqn:massRealProcaEOM}
\end{eqnarray}
Taking divergence in both hand sides, we get the Lorentz condition
\begin{eqnarray}
\partial_\nu \varphi^\nu = 0\,,
\hspace{5mm}
\partial^\nu \varphi^*_\nu = 0
\label{eqn:massVProcaCond}
\end{eqnarray}
when $m^2 \neq 0$.
This condition suppresses irrelevant degrees of freedom in $\varphi_\mu$ and $\varphi^*_\mu$
by one for each and these fields come to describe spin 1 fields correctly.
Owing to this condition, field equations in Eq. (\ref{eqn:massRealProcaEOM})
become a pair of Klein-Goldon equations.
%$\partial_\mu F^{\mu \nu} = \square \varphi^\nu$ and 
% $\partial^\mu F^*_{\mu \nu} = \square \varphi^*_\nu$ so that 
Conjugate momenta are given as
\begin{eqnarray}
\varphi^*_\mu
\longleftrightarrow
\pi^\mu 
 \equiv \frac{
\partial {\cal L} }
{\partial \dot{\varphi}_\mu^*}
= F^{\mu 0}
\,,
\hspace{5mm}
\varphi_\mu
\longleftrightarrow
\pi^{* \mu} 
\equiv \frac{
\partial {\cal L} }
{\partial \dot{\varphi}_\mu}
= F^{*\mu 0}
\label{eqn:ProcaConjMom}
\end{eqnarray}
One observes that $\pi^0 = \pi^{*0} = 0$ as it was happen in the electromagnetism.
In this case, however, one can make use of Eqs. (\ref{eqn:massRealProcaEOM}) and
(\ref{eqn:ProcaConjMom}) to solve $\varphi^0$ and $\varphi^{*0}$ 
in terms of other fields as
\footnote{%------------------------------- footnote >>
Time derivatives of fields are expressed in terms of $\pi$ and $\pi^*$ from
Eq. (\ref{eqn:ProcaConjMom}) as
\begin{eqnarray*}
\partial_0 \varphi^i = - \pi^i - \partial_i \varphi^0
\,,
\hspace{5mm}
\partial_0 \varphi^{*i} = - \pi^{*i} - \partial_i \varphi^{*0}
\end{eqnarray*}
The Hamiltonian density is then written as
\begin{eqnarray*}
\mathcal{H}
&=&
\pi^\mu \dot{\varphi}^*_\mu + \pi^*_\mu \dot{\varphi}^\mu - \mathcal{L}
\nonumber\\
&=&
|\bld{\pi}|^2
+
\frac{|\bld{\partial}\cdot \bld{\pi}|^2}{m^2}
+
|\bld{\partial} \times \bld{\varphi}|^2
+
m^2 |\bld{\varphi}|^2
+ 
\mbox{total derivative}
\end{eqnarray*}
which is obviously positive definite.
}%------------------------------- footnote //
\begin{eqnarray}
\varphi^0 
=
- \frac{\partial_\mu \pi^\mu}{m^2}
\,,
\hspace{5mm}
\varphi^{*0 }
=
- \frac{\partial_\mu \pi^{*\mu}}{m^2}
\end{eqnarray}

For the Lagrangian density (\ref{eqn:ProcaLagdens}) can be written as
\begin{eqnarray}
{\cal L}_{\mbox{\scriptsize Pr}} &=&
- \partial_\mu \varphi_\nu^* \cdot \partial^\mu \varphi^\nu
+ m^2 \varphi_\mu^* \varphi^\mu
%\nonumber\\
%&&
- \varphi_\nu^*\partial^\nu 
%\cancel{
\partial_\mu \varphi^\mu
%}
+ \partial_\mu \{ \varphi_\nu^*\partial^\nu  \varphi^\mu \}
\,,
\end{eqnarray}
we can also employ
\begin{eqnarray}
\mathcal{L}_{\mbox{\scriptsize KG}}
&=&
- \partial_\mu \varphi_\nu^* \cdot \partial^\mu \varphi^\nu 
 + m^2 |\varphi|^2
\end{eqnarray}
as our Lagrangian density
together with the Lorentz condition (\ref{eqn:massVProcaCond})
as a subsidiary condition.
In this case, we have Klein-Gordon equations as EOM's and
non-vanishing $\pi^0$ as we have already seen in the electromagnetism.

In the following, however, we employ Stueckelberg's Lagrangian given by
\begin{eqnarray}
\mathcal{L}_{\mbox{\scriptsize St}}
&=&
\mathcal{L}_{\mbox{\scriptsize Pr}}
- \frac{1}{\alpha}
(\partial \cdot \varphi^*)
(\partial \cdot \varphi)
\end{eqnarray}
which includes a "gauge fixing term"
although the gauge invariance is already violated by 
the mass term in $\mathcal{L}_{\mbox{\scriptsize Pr}}$.
The involvement of this term is said \cite{ref:Hioki} necessary
for the theory to be renormalizable.
It reads
\begin{eqnarray}
\frac{\partial \mathcal{L}}
{\partial (\partial_\mu \varphi_\nu^*)}
=
-F^{\mu \nu}
- \frac{1}{\alpha}g^{\mu \nu}
(\partial\cdot \varphi)
\,
\hspace{5mm}
\frac{\partial \mathcal{L}}
{\partial (\partial_\mu \varphi_\nu)}
=
-F^{*\mu \nu}
- \frac{1}{\alpha}g^{\mu \nu}
(\partial\cdot \varphi^*)
\,.
\nonumber\\
\label{eqn:StueckelbergConjExpress}
\end{eqnarray}
Field equations become
\footnote{%------------------------------- footnote >>
Taking the divergence of the field equation (\ref{eqn:StueckelbergEOM}), we have
\begin{eqnarray}
\frac{1}{\alpha}
\left[
\square + \alpha m^2
\right]
(\partial\cdot \varphi)
= 0
\,,
\hspace{5mm}
\mbox{(c.c.)}
\,.
\label{eqn:StueckelScalarComp}
\end{eqnarray}
Namely, the field $(\partial\cdot \varphi)$ is a Klein-Gordon field
with mass squared $\alpha m^2$.
We assume $\alpha > 0$ in the following.

It would be also noted that the field
\begin{eqnarray}
\varphi^T_\mu =
\left(
g_{\mu \nu}
+
\frac{1}{\alpha m^2}\partial_\mu \partial_\nu
\right)
\varphi^\nu
\end{eqnarray}
extracts the component of spin 1 from a mixture $\varphi^\nu$
composed of spin 1 and spin 0 components. This tansverse
component satisfies $\partial \cdot \varphi^T = 0$ by virtue of
Eq. (\ref{eqn:StueckelScalarComp}).
}%------------------------------- footnote //
\begin{eqnarray}
\begin{array}{l}
\displaystyle
(\square + m^2) \varphi^\mu
+
\left(
\frac{1}{\alpha} -1 
\right)
\partial^\mu
(\partial \cdot \varphi)
=
0
\vspace{2mm}
\\
\displaystyle
(\square + m^2) \varphi^{*\mu}
+
\left(
\frac{1}{\alpha} -1 
\right)
\partial^\mu
(\partial \cdot \varphi^*)
=
0
\end{array}
\label{eqn:StueckelbergEOM}
\end{eqnarray}
and conjugate momenta are now written as
\begin{eqnarray}
\begin{array}{l}
\displaystyle
\varphi^*_\mu
\longleftrightarrow
\pi^\mu 
 \equiv \frac{
\partial {\cal L} }
{\partial \dot{\varphi}_\mu^*}
= F^{\mu 0}
- \delta^{\mu 0}\frac{1}{\alpha}
(\partial \cdot \varphi)
\vspace{2mm}
\\
\displaystyle
\varphi_\mu
\longleftrightarrow
\pi^{* \mu} 
\equiv \frac{
\partial {\cal L} }
{\partial \dot{\varphi}_\mu}
= F^{*\mu 0}
- \delta^{\mu 0}\frac{1}{\alpha}
(\partial \cdot \varphi^*)
\end{array}
\label{eqn:StueckelConjMom}
\end{eqnarray}
Though fields $\varphi^\mu$ and $\varphi^{*\mu}$ are independent of each other,
expression for one is obvious from another and we omit to write expressions for the later 
in the following.
The solution to the Stueckelberg's field equation (\ref{eqn:StueckelbergEOM})
is writen as
\begin{eqnarray}
\varphi_\mu(x)
&=&
\int
\frac{d^3 \bld{p}}{\sqrt{(2\pi)^3}}
\sum_{\lambda=0}^3
\frac{1}{2p^0_{(\lambda)}}
\left[
a_\lambda (\bld{p})
\,
\tilde{\epsilon}_\mu^{(\lambda)} (\bld{p})
e^{-i p_{(\lambda)} x}
\right.
\nonumber\\
&&
\left.
\hspace{30mm}
+
b_\lambda^\dagger (\bld{p})
\,
\tilde{\epsilon}_\mu^{(\lambda)*} (\bld{p})
e^{i p_{(\lambda)} x}
\right]
\nonumber\\
&=&
\int
\frac{d^3 \bld{p}}{\sqrt{(2\pi)^3}}
\left(
\sum_{\lambda=1}^3
\frac{1}{2p^0}
\left[
a_\lambda 
\,
\tilde{\epsilon}_\mu^{(\lambda)} 
e^{-i p x}
+
b_\lambda^\dagger 
\,
\tilde{\epsilon}_\mu^{(\lambda)*} 
e^{i p x}
\right]
\right.
\nonumber\\
&&
\left.
\hspace{23mm}
+
\frac{1}{2p_\alpha^0}
\left[
a_0
\,
\tilde{\epsilon}_\mu^{(0)} 
e^{-i p_\alpha x}
+
b_0^\dagger 
\,
\tilde{\epsilon}_\mu^{(0)*} 
e^{i p_\alpha x}
\right]
\right)
\nonumber\\
\label{eqn:MassVStueckSol}
\end{eqnarray}
where
\begin{eqnarray}
p_{(\lambda)}
=
\left\{
\begin{array}{l}
p
\equiv
(p^0, \bld{p})
\,,
\hspace{3mm}
p^0
\equiv
\sqrt{\bld{p}^2 + m^2}
\,,
\hspace{3mm}
p^2 = m^2
\hspace{5mm}
\mbox{for }
\lambda = 1, 2, 3
\\
p_\alpha
\equiv
(p^0_\alpha, \bld{p})
\,,
\hspace{3mm}
p^0_\alpha
\equiv
\sqrt{\bld{p}^2 + \alpha m^2}
\,,
\hspace{3mm}
p^2_\alpha = \alpha m^2
\hspace{5mm}
\mbox{for }
\lambda = 0
\end{array}
\right.
\label{eqn:StuckelExpressMomenta}
\end{eqnarray}
and polarization vectors $\tilde{\epsilon}^{(\lambda)}$ are written as
\begin{eqnarray}
\left\{
\begin{array}{l}
\tilde{\epsilon}^{(0)}_\mu = p_{\alpha \mu} / m
\vspace{2mm}
\\
\tilde{\epsilon}^{(i)}_\mu = \Lambda^\mu_{\;\;\nu} (\bld{\beta}) e^{(i)\mu}
\,,
\hspace{3mm}
i = 1, 2, 3
\end{array}
\right.
\label{eqn:StueckelPolVecs}
\end{eqnarray}
For notations of the second equation, we refer to Appendix \ref{sec:PolarizationVector}.

\bigskip

\verb/-----------.-----------.-----------.-----------.-----------/\\
\vspace{-3mm}
{\small
\begin{center}
Addendum: Solution of the Stueckelberg's field equation
\end{center}
Let us confirm that the expression (\ref{eqn:MassVStueckSol}) satisfies the field equation
(\ref{eqn:StueckelbergEOM}). First, we prepare components:
\begin{eqnarray}
\partial \cdot \varphi
&=&
-i
\int
\frac{d^3 \bld{p}}{\sqrt{(2\pi)^3}}
\sum_{\lambda=0}^3
\frac{1}{2p^0_{(\lambda)}}
\left[
a_\lambda 
\,
(\tilde{\epsilon}^{(\lambda)} \cdot p_{(\lambda)})
e^{-i p_{(\lambda)} x}
%\right.
%\nonumber\\
%&&
%\left.
%\hspace{30mm}
-
b_\lambda^\dagger 
\,
(\tilde{\epsilon}^{(\lambda)*} \cdot p_{(\lambda)})
e^{i p_{(\lambda)} x}
\right]
\nonumber\\
&=&
-i
\int
\frac{d^3 \bld{p}}{\sqrt{(2\pi)^3}}
\left(
\sum_{\lambda=1}^3
\frac{1}{2p^0}
\left[
a_\lambda 
\,
(\tilde{\epsilon}^{(\lambda)} \cdot p)
e^{-i p x}
-
b_\lambda^\dagger 
\,
(\tilde{\epsilon}^{(\lambda)*} \cdot p)
e^{i p x}
\right]
\right.
\nonumber\\
&&
\left.
\hspace{23mm}
+
\frac{1}{2p_\alpha^0}
\left[
a_0
\,
(\tilde{\epsilon}^{(0)} \cdot p_{\alpha})
e^{-i p_\alpha x}
-
b_0^\dagger 
\,
(\tilde{\epsilon}^{(0)*} \cdot p_{\alpha})
e^{i p_\alpha x}
\right]
\right)
\nonumber\\
\label{eqn:MassVStueckDphi}
\end{eqnarray}
%---------------------------------------------------------
\begin{eqnarray}
\partial^\mu (\partial \cdot \varphi)
&=&
-
\int
\frac{d^3 \bld{p}}{\sqrt{(2\pi)^3}}
\sum_{\lambda=0}^3
\frac{p_{(\lambda)}^\mu}{2p^0_{(\lambda)}}
\left[
a_\lambda 
\,
(\tilde{\epsilon}^{(\lambda)} \cdot p_{(\lambda)})
e^{-i p_{(\lambda)} x}
+
b_\lambda^\dagger 
\,
(\tilde{\epsilon}^{(\lambda)*} \cdot p_{(\lambda)})
e^{i p_{(\lambda)} x}
\right]
\nonumber\\
\label{eqn:MassVStueckDmuDphi}
\end{eqnarray}
%---------------------------------------------------------
\begin{eqnarray}
\square (\partial \cdot \varphi)
&=&
i
\int
\frac{d^3 \bld{p}}{\sqrt{(2\pi)^3}}
\sum_{\lambda=0}^3
\frac{p_{(\lambda)}^2}{2p^0_{(\lambda)}}
\left[
a_\lambda 
\,
(\tilde{\epsilon}^{(\lambda)} \cdot p_{(\lambda)})
e^{-i p_{(\lambda)} x}
-
b_\lambda^\dagger 
\,
(\tilde{\epsilon}^{(\lambda)*} \cdot p_{(\lambda)})
e^{i p_{(\lambda)} x}
\right]
\nonumber\\
&=&
i
\int
\frac{d^3 \bld{p}}{\sqrt{(2\pi)^3}}
\left(
\sum_{\lambda=1}^3
\frac{m^2}{2p^0}
\left[
a_\lambda 
\,
(\tilde{\epsilon}^{(\lambda)} \cdot p)
e^{-i p x}
-
b_\lambda^\dagger 
\,
(\tilde{\epsilon}^{(\lambda)*} \cdot p)
e^{i p x}
\right]
\right.
\nonumber\\
&&
\left.
\hspace{23mm}
+
\frac{\alpha m^2}{2p_\alpha^0}
\left[
a_0
\,
(\tilde{\epsilon}^{(0)} \cdot p_{\alpha})
e^{-i p_\alpha x}
-
b_0^\dagger 
\,
(\tilde{\epsilon}^{(0)*} \cdot p_{\alpha})
e^{i p_\alpha x}
\right]
\right)
\nonumber\\
\label{eqn:MassVStueckDalembDphi}
\end{eqnarray}
%---------------------------------------------------------
Second, we evaluate the $l.h.s.$ of Eq. (\ref{eqn:StueckelbergEOM}):
\begin{eqnarray}
&&(\square + m^2) \varphi_\mu
+
\left(
\frac{1}{\alpha} -1 
\right)
\partial_\mu
(\partial \cdot \varphi)
\nonumber\\
&&=
-
\int
\frac{d^3 \bld{p}}{\sqrt{(2\pi)^3}}
\sum_{\lambda=0}^3
\frac{1}{2p^0_{(\lambda)}}
\left[
a_\lambda 
\left\{
( p_{(\lambda)}^2 - m^2) \tilde{\epsilon}^{(\lambda)}_\mu
+ 
(\frac{1}{\alpha} - 1) p_{(\lambda)\mu}
(\tilde{\epsilon}^{(\lambda)} \cdot p_{(\lambda)})
\right\}
e^{-i p_{(\lambda)} x}
\right.
\nonumber\\
&&
\left.
\hspace{33mm}
+
b_\lambda^\dagger 
\left\{
( p_{(\lambda)}^2 - m^2) \tilde{\epsilon}^{(\lambda)*}_\mu
+ 
(\frac{1}{\alpha} - 1) p_{(\lambda)\mu}
(\tilde{\epsilon}^{(\lambda)*} \cdot p_{(\lambda)})
\right\}
e^{i p_{(\lambda)} x}
\right]
\nonumber\\
&&=
-
\int
\frac{d^3 \bld{p}}{\sqrt{(2\pi)^3}}
\left(
\sum_{l=1}^3
\frac{1}{2p^0}
\left[
a_l
(\frac{1}{\alpha} - 1) p_\mu
(\tilde{\epsilon}^{(l)} \cdot p)
e^{-i p x}
+
b_l^\dagger 
(\frac{1}{\alpha} - 1) p_\mu
(\tilde{\epsilon}^{(l)*} \cdot p)
e^{i p x}
\right]
\right.
\nonumber\\
&&
\hspace{27mm}
+
\frac{1}{2p^0_\alpha}
\left[
a_0 (\alpha -1)
\left\{
m^2 
\tilde{\epsilon}^{(0)} _\mu
-
\frac{1}{\alpha}
p_{\alpha \mu}
(\tilde{\epsilon}^{(0)} \cdot p_\alpha)
\right\}
e^{-i p_\alpha x}
\right.
\nonumber\\
&&
\left.
\left.
\hspace{35mm}
+
b_0^\dagger (\alpha -1)
\left\{
m^2 
\tilde{\epsilon}^{(0)*} _\mu
-
\frac{1}{\alpha}
p_{\alpha \mu}
(\tilde{\epsilon}^{(0)*} \cdot p_\alpha)
\right\}
e^{i p_\alpha x}
\right]
\right)
\label{eqn:StueckelFeqLHS}
\end{eqnarray}
In the first line of the last expression, we have from Eq. (\ref{eqn:StueckelPolVecs}) that
$(\tilde{\epsilon}^{(l)} \cdot p) = (\tilde{\epsilon}^{(l)*} \cdot p) = 0$
and
$(\tilde{\epsilon}^{(0)} \cdot p_\alpha) = (\tilde{\epsilon}^{(0)*} \cdot p_\alpha) =\alpha m$
in the second and third lines 
so that the last expression vanishes as the whole.
Thus, we have shown that the expression (\ref{eqn:MassVStueckSol})
together with Eqs. (\ref{eqn:StuckelExpressMomenta}, \ref{eqn:StueckelPolVecs})
surely satisfies Stueckelberg's field equation (\ref{eqn:StueckelbergEOM}).
}\\ %------------------------------------ small //
\verb/-----------.-----------.-----------.-----------.-----------/\\

We quantize the field (\ref{eqn:MassVStueckSol}) by requiring canonical
commutation relations which reads
\begin{eqnarray}
\begin{array}{l}
[a_\lambda (\bld{p}), a_{\lambda'}^\dagger (\bld{p}') ]
=
[b_{\lambda} (\bld{p}), b_{\lambda'}^\dagger (\bld{p}') ]
=
-g_{\lambda \lambda'}  \, 2 p^0_{(\lambda)} \,\delta^3(\bld{p} - \bld{p})
\vspace{2mm}
\\
\mbox{others} = 0
\end{array}
\end{eqnarray}

We are now at a position to compute the propagator.
\begin{eqnarray}
&&
\left.
\bra 0 \braend \varphi_\mu (x) \varphi^\dagger_\nu (y) \ketend 0 \ket
\right|_{x^0 > y^0}
\nonumber\\
&&
=
\int
\frac{d^3 \bld{p} d^3 \bld{p}' }{(2\pi)^3}
\sum_{\lambda, \lambda' = 0}^3
\frac{1}{4 p^0_{(\lambda)} p^0_{(\lambda')}}
\bra 0 \braend 
a_\lambda (\bld{p}) \tilde{\epsilon}_\mu^{(\lambda)} (\bld{p}) e^{-i p_{(\lambda)} x}
\nonumber\\
&&
\hspace{35mm}
a_{\lambda'}^\dagger (\bld{p}') \tilde{\epsilon}_\nu^{(\lambda')} (\bld{p}') e^{i p_{(\lambda')}' y}
\ketend 0 \ket
\nonumber\\
&&
=
\int
\frac{d^3 \bld{p} }{(2\pi)^3}
\sum_{\lambda, \lambda' = 0}^3
\frac{-g_{\lambda \lambda'}}{2  p^0_{(\lambda)}}
 \tilde{\epsilon}_\mu^{(\lambda)} (\bld{p})
\tilde{\epsilon}_\nu^{(\lambda')} (\bld{p})  e^{-i p_{(\lambda)}( x- y)}
\nonumber\\
&&
=
\int
\frac{d^3 \bld{p} }{(2\pi)^3}
\left(
\frac{-1}{2 p^0}
\left[
g_{\mu \nu} - g_{00} \frac{p_\mu p_\nu}{m^2}
\right] e^{- ip (x - y)}
-
g_{00}
\frac{p_{\alpha \mu} p_{\alpha \nu} / m^2}{2 p^0_\alpha}
e^{- p_\alpha (x-y)}
\right)
\nonumber\\
\end{eqnarray}
and
\begin{eqnarray}
D_F^{\mu \nu} (x-y)
&\leftdef&
\bra 0 \braend T\left[ \varphi_\mu (x) \varphi^\dagger_\nu (y)\right] \ketend 0 \ket
\nonumber\\
&=&
-i
\int
\frac{d^4 p }{(2\pi)^4}
\left(
\frac{g_{\mu \nu} -  p_\mu p_\nu / m^2}{p^2 - m^2 + i\epsilon}
+
\frac{p_\mu p_\nu / m^2}{p^2 - \alpha m^2 + i\epsilon}
\right)
e^{- ip (x - y)}
\nonumber\\
\end{eqnarray}


\bigskip
<<<<<<<<<<<<<<<<<<<<<<



\begin{comment}
\\
&&
(= -\partial_mu \varphi_\nu^* \cdot \partial^\mu \varphi^\nu + m^2 \varphi_\mu^* \varphi^\mu
+ \partial_mu \varphi_\nu^* \cdot \partial^\nu \varphi^\mu)
\nonumber
\end{comment}