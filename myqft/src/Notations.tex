%\setcounter{subsection}{-1}
%\subsection{Notations}
$\blacksquare$
metric of 4-vector space:
\begin{equation*}
   g = 
 \left( \begin{array}{cccc}
       1 & 0&0&0\\
       0 & -1&0&0\\
       0 & 0&-1&0\\
       0 & 0&0&-1\\
              \end{array} \right)\hspace{10mm}({\mbox{\small "Bjorken-Drell metric"}})
\label{eqn:metric}
\end{equation*}
The square of a 4-vector $\underline{p} = (p^0, \bld{p}) = (p^0, p^1, p^2, p^3)$ is
\[ \underline{p}^2 = \underline{p}^T g\, \underline{p} = (p^0)^2 - (p^1)^2 - (p^2)^2 - (p^3)^2 = (p^0)^2 - \bld{p}^2\,, \]
where a superscript $T$ denotes the transpose. An underline on a quantity (e.g. $\underline{p}$) indicates a 
%contravariant 
4-vector but it is frequently suppressed.
% and $\underline{p}$ is written just as $p$, when there is no possibility of confusions.
When components of a 4-vector are given, they are contravariant components by default.
The contravariant components of a 4-vector $\underline{p}$ are denoted as $p^{\mu}$ with
upper suffixes. To make sure that components are contravariant, we often denote a 4-vector
itself as $p^{\mu}$. For given contravariant components, covariant components are obtained
as
\[ p_{\mu} = g_{\mu \nu} p^\nu \,,\]
where $g_{\mu \nu}$ is the ($\mu \nu$) element of $g$.
Thus, for the $\underline{p}$ given in the above, we write $p_{\mu} = (p^0, -\bld{p})$.
The square of a 4-vector $\underline{p}$ is now also written as
\[ \underline{p}^2 = p_{\mu} p^{\mu} \,,\]
where same suffixes $\mu$ in lower and upper positions must be summed over $\mu \in [0,3]$.

\bigskip

\bigskip


\noindent
$\blacksquare$ Energy-momentum of a particle of the mass $m$ is written as
\begin{equation*}
	p^\mu \;\leftdef
         \left( \begin{array}{c} E/ c \\ 
          \bld{p}  \end{array} \right)  
          =
	mc
         \left( \begin{array}{c} \gamma \\ 
          \gamma\bld{\beta}  \end{array} \right)  
          = m u^\mu \,,
\label{eqn:enmomvec}
\end{equation*}
where $\gamma \equiv 1 /\sqrt{1 - \bld{\beta}^2}$ denotes the Lorentz factor of motion of the particle,
$\bld{\beta}$ stands for the velocity nomalized by one of the light, $c$,
and $u^\mu$ is the four-velocity.

\bigskip

\bigskip


\noindent
$\blacksquare$ Natural unit ($c = \hbar = 1$) is employed throughout our discussions.
E.g. for the energy-momentum given in the above, we write
\[ E = \sqrt{\bld{p}^2 + m^2} 
\hspace{3mm}
\mbox{and}
\hspace{3mm}
\underline{p}^2 = m^2 \,.
\]

\noindent
$\blacksquare$
Nomalization of states\\
\begin{equation*}
\begin{array}{l}
\mbox{Energy-momentum eigenstate:}\hspace{3mm}
 \bra p \braketend p' \ket = 2 E \delta^3 ( \bld{p} - \bld{p}' )$,\hspace{2mm}$E = \sqrt{m^2 + \bld{p}^2}\\
\mbox{Completeness:}\hspace{3mm}
{\displaystyle   \int \frac{d^3 \bld{p}}{2E}} \ketend p \ket \bra p \braend = 1\,.
\end{array}
\label{eqn:statenormalization}
\end{equation*}
Many authors including ones of Refs. \cite{ref:Donnachie} and \cite{ref:Collins} 
put a factor $(2\pi)^3$ in the front of $\delta$ function and in the denominator 
of the integration volume element. It is just a matter of convention.
Note that
\begin{equation*}
Dim[ \ketend p \ket ] = [E] ^{-1}
\end{equation*}

\noindent
$\blacksquare$
Levi-Civita symbols\\
These are totally antisymmetric tensor densities.\\

\noindent
3 dimensional
\begin{eqnarray*}
\epsilon_{ijk}
=
\left\{
\begin{array}{l}
\raisebox{-.3\height}{${\stackrel{\displaystyle+}{-}}$} \;1
\hspace{5mm}
(i, j, k) =
\raisebox{-.3\height}{$\stackrel{\mbox{even}}{\mbox{odd}}$} \,\;\mbox{permutation of } (1,2,3)
\vspace{2mm}
\\
0
\hspace{5mm}
\mbox{otherwise}
\end{array}
\right.
\label{eqn:LeviCivita3dim}
\end{eqnarray*}
For the purpose of applying the Einstein's contraction rule to 3 vector suffices,
we may use symbols like $\epsilon^{ijk}$, $\epsilon_{ij}^{\;\;\;k}$ and so on.
All these symbols are equivalent with $\epsilon_{ijk}$.
%Eq. (\ref{eqn:LeviCivita3dim}).
The vertical positions of indexes have no meaning.\\

\noindent
4 dimensional
\begin{eqnarray*}
\epsilon^{\mu\nu\rho\sigma}
=
\left\{
\begin{array}{l}
+ 1
\hspace{5mm}
(\mu, \nu, \rho, \sigma, ) = \mbox{even parmutation of } (0, 1, 2, 3)
\\
- 1
\hspace{5mm}
(\mu, \nu, \rho, \sigma, ) = \mbox{odd parmutation of } (0, 1, 2, 3)
\\
0
\hspace{5mm}
\mbox{otherwise}
\end{array}
\right.
\end{eqnarray*}
Symbols with covariant suffices are defined with the metric tensor 
in such a way like $\epsilon_\mu^{\;\;\nu\rho\sigma} = g_{\mu \tau} \epsilon^{\tau \nu\rho\sigma}$.
Therefore, we have $\epsilon_{\mu\nu\rho\sigma} = - \epsilon^{\mu\nu\rho\sigma}$.
This notation follows \cite{ref:Peskin-Schroeder}.
