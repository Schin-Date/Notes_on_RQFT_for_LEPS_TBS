\indent

Observables are hermitian operators acting on a space of state vectors.
Complete set of observables is a set of observables that commute with one another
and for which there exists only one simultaneous eigenvector when the normalization is fixed.
Set of all simultaneous eigenvectors of a complete set spans a Hilbert space
\footnote{%--------------------------------------------------------------------- footnote >
When an observable of the set has continuous eigenvalues,
the space is not a Hilbert space in a mathematically rigorous sense.
However, we may follow a physics convention to use the term
including such cases.
}%--------------------------------------------------------------------- footnote //
 of
state vectors associated with a quantum mechanical system under consideration. 
Therefore, the set constitutes a complete bases of the Hilbert space.
Different choices of a particular set of complete bases, namely, that of a complete set of observables
define different representations of the Hilbert space.
Suppose $\{ \hat{A}_1, \hat{A}_2, \dots \}$ is a complete set
and $\ketend  a_1 a_2 \cdots \ket$ is a simultaneous eigenvector
associated with a set of eigen values $\{ a_1, a_2, \cdots \}$.
Then an arbitrary state vector $\ketend \Psi \ket$ can be expressed as
\begin{eqnarray}
\ketend \Psi \ket
=
\int \left[da_1\right] \left[da_2 \right] \cdots
\bra a_1 a_2 \cdots \braketend \Psi \ket
\ketend  a_1 a_2 \cdots \ket
\label{eqn:completerepresentation}
\end{eqnarray}
where $\int \left[ da_i \right]$ denotes sum or integration over a variable $a_i$ with a measure
chosen so that
$\int \left[ da_i \right] \ketend a_i \ket \bra a_i \braend$ is the projection of
a space spanned by $\ketend a_i \ket$.
Factor $\bra a_1 a_2 \cdots \braketend \Psi \ket$ in Eq. (\ref{eqn:completerepresentation}) 
is the wave function in the representation defined by the complete set.
When we define another representation by a complete set $\{ B_1, B_2, \dots \}$ and
its simultaneous eigenvectors $\ketend  b_1 b_2 \cdots \ket$,
it follows from Eq. (\ref{eqn:completerepresentation}) that
the wave function in the new representation is given as
\begin{eqnarray}
\bra b_1 b_2 \cdots \braketend \Psi \ket
= 
\int \left[da_1\right] \left[da_2 \right] \cdots
\bra a_1 a_2 \cdots \braketend \Psi \ket
\bra b_1 b_2 \cdots \braketend  a_1 a_2 \cdots \ket
\label{eqn:changeofrepresentation}
\end{eqnarray}

In quantum mechanics, canonically conjugate variables $\hat{A}$ and $\hat{B}$
satisfy a commutation relation instead of  corresponding  Poisson bracket in
classical mechanics. 


For a system composed of one particle, cartesian components $(\hat{x}_1, \hat{x}_2, \hat{x}_3) \equiv \hat{\bld{x}}$
of particle position are observables. 

Eigenvector of a physical quantity $\hat{A}$ ($=\hat{A}^\dagger$ ) associated with 
an eigenvalue a ($= a^*$) is called eigenstate $\ketend a \ket$;
$\hat{A} \ketend a \ket = a \ketend a \ket$.
Physical quantities 

To a system composed of one particle, 
there associate coordinates of the position $\hat{\bld{x}}$ that are
physical quantities.


%To a particle, there associates a physical quantity called position $\hat{\bld{x}}$.

\bigskip

%Operators acting in a Hilbert space. Physical quantities are hermitian operators.
%Born's probability interpretation is employed.\\
\begin{eqnarray}
\mbox{Schr\"odinger eq.} \hspace{24mm} 
&\hspace{-5mm}&
\hspace{-20mm}
i \partial_t \Psi(t, \bld{x}) = {H} \Psi(t, \bld{x}) \,,
%i \partilal_t \Psi (t, \bld{x}) = H \Psi (t, \bld{x})
\hspace{3mm}
H = - \frac{\bld{\partial}^2}{2m} + V(\bld{x}) = H^\dagger
\label{eqn:SchroedingerEq}
\\
\mbox{Eigenstates of } H \hspace{22mm} 
&\hspace{-5mm}&
\hspace{-20mm}
H \varphi_l (\bld{x} ) = \epsilon_l \varphi_l (\bld{x} )\,,
\hspace{3mm}
\{ \varphi_l (\bld{x} ) \} \mbox{: orthogonal, complete}
\label{eqn:energyEigenSt}
\\
\mbox{Solution of the Schr\"odinger eq.}
&\hspace{-5mm}&
\Psi(t, \bld{x})
=
\sum_l \Psi_l e^{-i \epsilon_l t} \varphi_l (\bld{x})
\label{eqn:Psiexpansion}
\end{eqnarray}
%-------------------------------------------------------------------
%
Bras and kets
\begin{equation}
\Psi(t, \bld{x}) = \bra \bld{x} \braend \Psi(t) \ket\,,
\hspace{5mm}
\varphi_l(\bld{x})  = \bra \bld{x} \braend \epsilon_l \ket\,, 
\label{eqn:coordrepbraket}
\end{equation}
\begin{equation*}
\hspace{21mm}
i \partial_t  \ketend \Psi(t) \ket = \hat{H} \ketend \Psi(t) \ket\,,
\hspace{5mm}
\hat{H} = \frac{\hat{\bld{p}}^2}{2 m} + V(\bld{x}) = \hat{H}^\dagger
\hspace{20mm}
\bra \ref{eqn:SchroedingerEq} \ket
\end{equation*}
\begin{equation*}
\hspace{13mm}
\hat{H} 
\ketend \epsilon_l \ket
= 
\epsilon_l \ketend \epsilon_l \ket\,,
\hspace{3mm}
\bra \epsilon_l \ketend \epsilon_{l'} \ket = 0 \mbox{ if } l\neq l'\,,
\hspace{3mm}
\sum_l \ketend \epsilon_{l} \ket \bra \epsilon_{l} \braend = 1
\hspace{10mm}
\bra \ref{eqn:energyEigenSt} \ket
\end{equation*}
\begin{equation*}
\hspace{32mm}
\ketend \Psi(t) \ket =
\sum_l \bra \epsilon_l \braend   \Psi(0) \ket
e^{-i \epsilon_l t} \ketend \epsilon_l \ket
\hspace{30mm}
\bra \ref{eqn:Psiexpansion} \ket
\end{equation*}
%-------------------------------------------------------------------
%
1st quantization
\begin{equation}
[\hat{x}_i, \hat{p}_j] = i \delta_{ij}
\end{equation}
Eigen states
\begin{equation}
\begin{array}{l}
\hat{\bld{x}} \ketend \bld{x} \ket = \bld{x} \ketend \bld{x} \ket
\\
\hat{\bld{p}} \ketend \bld{p} \ket = \bld{p} \ketend \bld{p} \ket
\end{array}
\end{equation}
Conventional ("half relativistic") normalization
\begin{equation}
\bra \bld{x} \braend \bld{x}' \ket = \delta^3(\bld{x}-\bld{x}')\,,
\hspace{5mm}
\int \ketend \bld{x} \ket d^3 \bld{x} \bra \bld{x} \braend = 1
\end{equation}
\begin{equation}
\bra \bld{p} \braend \bld{p}' \ket = 2E \delta^3(\bld{p}-\bld{p}')\,,
\hspace{5mm}
\int \ketend \bld{p} \ket \frac{d^3 \bld{p}}{2E} \bra \bld{p} \braend = 1
\label{eqn:norm_p_state_rel}
\end{equation}
Coordinate representation of the momentum operator
\begin{equation}
\bra \bld{x} \braend \hat{\bld{p}}
=
\frac{1}{i} \bld{\partial} \bra \bld{x} \braend
\end{equation}
and of the momentum eigenstate
\begin{equation}
\bra \bld{x} \braend \bld{p} \ket
=
\sqrt{\frac{2E}{(2\pi)^3}}\, e^{i \bld{p} \cdot \bld{x}}
\end{equation}
Requirement for the normalization factor is understood by employing
the first equation of (\ref{eqn:norm_p_state_rel}) in the $l.h.s$ of
$\nolinebreak{\bra \bld{p} \braend \bld{p}' \ket}= $
$\nolinebreak{\int \bra \bld{p} \braend \bld{x} \ket d^3\bld{x} \bra \bld{x} \ketend \bld{p}' \ket}$
and remembering a formula
\[
\int {d^3 \bld{x}}\, e^{\pm i \bld{p} \cdot \bld{x}}= (2\pi)^3 \delta^3 (\bld{p})
\]
