\subsection{Dirac matrices}
Traceless matrices $\gamma^\mu$ ($\mu =$ 0, 1, 2, 3) are called Dirac matrices
when they satisfy the Clifford algebra,
\begin{eqnarray}
&\{ \gamma^\mu, \gamma^\nu \}_+ = 2 g^{\mu \nu}\,.& 
\label{eqn:DiracGammaClifford}
\end{eqnarray}
Their hermiticity are chosen to satisfy
\begin{eqnarray}
\gamma^{\mu \dagger} =
\gamma^0 \gamma^\mu \gamma^0
\label{eqn:DiracHermitisity}
\end{eqnarray}
so that $\bld{\alpha}$ and $\beta$ in Eq. (\ref{eqn:gammabyalphabeta}) become hermitians.
When $d$ is the rank of an irreducible matrix representation of an algebra composed of hypercomplex numbers,
the number of linearly independent elements of the algebra is $n = d^2$. Therefore, the minimal algebra composed
of $\gamma$'s has $4^2 = 16$ linearly independent elements. They are counted as
\begin{eqnarray}
\left\{
\begin{array}{l}
\Gamma^S
=
1
\\
\Gamma^V_\mu
=
\gamma_\mu
\\
\Gamma^T_{\mu \nu}
=
\sigma_{\mu \nu} 
= - \sigma_{\nu \mu} 
\\
\Gamma^A_{\mu}
=
\gamma_5 \gamma_\mu
\\
\Gamma^P
=
\gamma_5
\end{array}
\right.
\end{eqnarray}
The $\gamma_5$ matrix is defined by
\begin{eqnarray}
\gamma_5 \equiv \gamma^5 \leftdef
i \gamma^0 \gamma^1 \gamma^2 \gamma^3
=
- \frac{i}{4!}
\epsilon_{\mu \nu \rho \sigma}
\gamma^\mu \gamma^\nu \gamma^\rho \gamma^\sigma \,.
\label{eqn:defgamma5}
\end{eqnarray}
We refer to sec.\ref{sec:Notations} for the notation of the Levi-Civita tensor.
It may be useful to note
\begin{eqnarray}
\epsilon^{\rho \sigma \mu \nu }
\epsilon_{\rho \sigma \alpha \beta }
=
-2 \left(
g^\mu_\alpha g^\nu_\beta 
-
g^\mu_\beta  g^\nu_\alpha
\right)
\end{eqnarray}
The $\gamma_5$ satisfies
\begin{eqnarray}
\{
\gamma_5, \gamma_\mu 
\}_+ = 0\,,
\hspace{3mm}
(\gamma_5)^2 = 1\,,
\hspace{3mm}
\gamma_5^\dagger = \gamma_5 \,.
\end{eqnarray}
%-----------------------------------------------------------
Also,
\begin{eqnarray}
&&\gamma_5 
=
i \gamma^3 \gamma^2 \gamma^1 \gamma^0
\hspace{3mm}
[\mbox{and any even permutation of }(0, 1, 2, 3)],
%\vspace*{8mm}
\\
\nonumber\\
&&\tr( \gamma_5 ) 
=
\tr( \gamma^0 \gamma_5 \gamma^0) 
=
-\tr( \gamma_5 ) 
= 0
\,.
\end{eqnarray}
%-----------------------------------------------------------
The $\sigma^{\mu \nu}$ is defined as
\begin{eqnarray}
\sigma_{\mu \nu} &\leftdef&
\frac{i}{2}
[ \gamma_\mu, \gamma_\nu]
\label{eqn:DefDiracsigmamunu}
\end{eqnarray}
which has a property $\sigma_{\mu \nu}^\dagger = \gamma^0 \sigma_{\mu \nu} \gamma^0$
according to Eq. (\ref{eqn:DiracHermitisity}).
A direct computation yields
\begin{eqnarray}
\gamma^i \gamma^j
=
g^{ij} + 
i \epsilon^{ijk}\gamma^0 \gamma^5 \gamma^k\,,
\end{eqnarray}
and the spatial part of $\sigma_{\mu \nu}$ in Eq. (\ref{eqn:DefDiracsigmamunu}) may be expressed as
\begin{eqnarray}
\sigma^{i j} 
=
\epsilon^{i j k} \gamma^5 \gamma^0 \gamma^k
\,.
\end{eqnarray}
We define
\begin{eqnarray}
\bld{\sigma} &\leftdef& \frac{1}{2} \epsilon^{\uparrow i j} \sigma^{i j} 
\label{eqn:defDiracSigma}
\\
&=&
\gamma^5 \gamma^0 \bld{\gamma}
\,,
\end{eqnarray}
so that $\sigma^{ij} = \epsilon^{ijk} \sigma^k$ holds.
%==============================================
\bigskip

\noindent
\underline{a convenient relationship}
\begin{eqnarray}
\gamma^\mu \gamma^\nu
&=&
\frac{1}{2}
\left(
\{\gamma^\mu, \gamma^\nu \}_+ +
\left[ \gamma^\mu,  \gamma^\nu \right]
\right)
=
g^{\mu \nu} - i \sigma^{\mu \nu}
\end{eqnarray}

\bigskip

\noindent
\underline{contract formulae}
\begin{equation}
\begin{split}
\gamma_\mu \gamma^\mu 
&=
g_{\mu \nu} \gamma^\mu \gamma^\nu
=
\frac{1}{2} g_{\mu \nu} \{\gamma^\mu, \gamma^\nu\}
= 4
\\
\gamma_\mu \gamma^\alpha \gamma^\mu 
&=
\gamma_\mu
\left(
\{ \gamma^\alpha, \gamma^\mu \}_+ 
- \gamma^\mu \gamma^\alpha
\right)
= -2 \gamma^\alpha
\\
\gamma_\mu \gamma^\alpha \gamma^\beta \gamma^\mu 
&=
\gamma_\mu \gamma^\alpha
\left(
2 g^{\mu \beta} - \gamma^\mu \gamma^\beta
\right)
\\
&=
2\gamma^\beta \gamma^\alpha
-
\gamma_\mu
\left(
2g^{\alpha \mu} - \gamma_\mu \gamma^\alpha
\right)
\gamma^\beta
=
4 g^{\alpha \beta}
%----------------------------------------------------------------------
\\
\gamma_\mu \gamma^\alpha \gamma^\beta \gamma^\gamma \gamma^\mu 
&=
2 \gamma^\gamma \gamma^\alpha \gamma^\beta
-
\gamma_\mu \gamma^\alpha \gamma^\beta \gamma^\mu  \gamma^\gamma
%\\
%&
=
2 \gamma^\gamma \gamma^\alpha \gamma^\beta
-
4g^{\alpha \beta} \gamma^\gamma
\\
&=
- 2 \gamma^\gamma \gamma^\beta \gamma^\alpha
%----------------------------------------------------------------------
\\
\gamma_\mu \gamma^\alpha \gamma^\beta \gamma^\gamma \gamma^\delta \gamma^\mu 
&=
2 \gamma^\delta  \gamma^\alpha \gamma^\beta \gamma^\gamma 
-
\gamma_\mu \gamma^\alpha \gamma^\beta \gamma^\gamma \gamma^\mu  \gamma^\delta
\\
&=
2\left(
 \gamma^\delta  \gamma^\alpha \gamma^\beta \gamma^\gamma 
+
 \gamma^\gamma \gamma^\beta \gamma^\alpha \gamma^\mu  \gamma^\delta
\right)
\end{split}
\end{equation}

\bigskip

\noindent
\underline{trace formulae}
\begin{equation}
\begin{split}
\tr( \gamma^\mu \gamma^\nu ) 
&=
\frac{1}{2} \tr( \{ \gamma^\mu \gamma^\nu \}_+ ) 
=
g^{\mu \nu} \tr 1
= 4 g^{\mu \nu} 
%----------------------------------------------------------------------
\\
\tr( \gamma^\mu \cdots \mbox{\tiny (odd numbers)}) 
&=
\tr((\gamma_5)^2 \gamma^\mu \cdots \mbox{\tiny (odd numbers)}) 
\\
&=
\tr(\gamma_5 \gamma^\mu \cdots \mbox{\tiny (odd numbers)} \gamma_5) 
\\
&=
(-1)^{\mbox{\tiny (odd numbers)}}\tr(\gamma^\mu \cdots \mbox{\tiny (odd numbers)}) 
= 0
%----------------------------------------------------------------------
\\
\tr( \gamma^\mu \gamma^\nu \gamma^\rho \gamma^\sigma ) 
&=
2g^{\mu \nu} 4g^{\rho \sigma} - \tr( \gamma^\nu \gamma^\mu \gamma^\rho \gamma^\sigma ) 
\\
&=
8g^{\mu \nu} g^{\rho \sigma} - 8g^{\mu \rho}g^{\sigma \nu}
+ \tr( \gamma^\nu \gamma^\rho \gamma^\mu \gamma^\sigma ) 
\\
&=
8g^{\mu \nu} g^{\rho \sigma} - 8g^{\mu \rho}g^{\sigma \nu}
+ 8g^{\mu \sigma}g^{\nu \rho}
-\tr( \gamma^\nu \gamma^\rho \gamma^\sigma \gamma^\mu ) 
\\
&=
4 (
g^{\mu \nu} g^{\rho \sigma} - g^{\mu \rho}g^{\sigma \nu}
+ g^{\mu \sigma}g^{\nu \rho}
)
\end{split}
\label{eqn:DiracTraceFormulae}
\end{equation}
If we denote $\tr{\gamma^{\mu_1} \cdots \gamma^{\mu_{2n}} } = \left[ 1 \cdots 2n \right]$ and
$g^{\mu_i \mu_j } = (ij)$,
the first and last equation in Eq. (\ref{eqn:DiracTraceFormulae}) can be simply written as
\begin{equation}
\begin{split}
[12] &= 4(12)
\\
[1234]
&=
(12)[34] - (13)[24] + (14)[23]
\end{split}
\label{eqn:DiracTraceSimpleNotation}
\end{equation}
The four gamma trace $[1234]$ is invariant under exchanges 
$P$ and $Q$ of four symbols defined as
\begin{equation}
P[1234] = [2143]\,,
\hspace{3mm}
Q[1234] = [3412]\,.
\end{equation}
Since $PQ[1234] = [4321]$, we have
\begin{equation}
[4321] = [1234]
\label{eqn:DiracTrace4gammaDecOrder}
\end{equation}
For larger numbers of gamma matrices,
we have a reduction formula
\begin{eqnarray}
\left[1 \cdots 2n \right] 
&=&
\sum_{\substack{j \in [1, 2n] \\ j \neq i}}
(-1)^{j - i +1} (ij) \left[ 1 \cdots \xcancel{i} \cdots \xcancel{j} \cdots 2n \right]\,,
\label{eqn:DiracGammaReductionFormula}
\end{eqnarray}
where $n \geq 2$ and $i \in [1, 2n]$ is arbitrarily fixed. This formula can be derived by
moving $\gamma^{\mu_i}$ one by one to the right or left by means of Eq. (\ref{eqn:DiracGammaClifford})
until it comes back to the original position. Putting $2n = 4$, we recover the second formula in
Eq. (\ref{eqn:DiracTraceSimpleNotation}).
If  a relationship $[2(n-1) \cdots 1 ] = [1 \cdots 2(n-1) ]$ holds, Eq. (\ref{eqn:DiracGammaReductionFormula})
leads 
\begin{equation}
[2n \cdots 1] = [1 \cdots 2n]\,.
\label{eqn:DiracGammaTraceDecOrder}
\end{equation}
However, the assumed relationship holds for $n = 4$ as in Eq. (\ref{eqn:DiracTrace4gammaDecOrder})
and Eq. (\ref{eqn:DiracGammaTraceDecOrder}) is proved in terms of induction.

\bigskip

\noindent
\underline{trace formulae involving a $\gamma_5$}
\begin{equation}
\begin{split}
\tr(\gamma_5 \gamma^{\mu})
&=
-\tr(\gamma^{\mu} \gamma_5 )
= 0
\\
\tr(\gamma_5 \gamma^{\mu} \gamma^{\nu})
&=
i \tr( \gamma^0 \gamma^1 \gamma^2 \gamma^3  \gamma^{\mu} \gamma^{\nu} )
=
i \tr(\gamma^{\nu} \gamma^{\mu}  \gamma^3 \gamma^2 \gamma^1 \gamma^0  )
\\
&=
\tr( \gamma^{\nu} \gamma^{\mu}  \gamma_5)
=
\tr( \gamma^{\mu}  \gamma_5 \gamma^{\nu} )
=
-\tr( \gamma_5 \gamma^{\mu}  \gamma^{\nu})
= 0
\\
\tr(\gamma_5 \gamma^{\mu} \gamma^{\nu} \gamma^{\rho})
&=
\tr(\mbox{\small odd number of } \gamma's)
= 0
\\
\tr(\gamma_5 \gamma^{\mu} \gamma^{\nu} \gamma^{\rho} \gamma^{\sigma})
&=
\tr(\gamma_5  \gamma^0 \gamma^1 \gamma^2 \gamma^3)
\epsilon^{\mu \nu \rho \sigma}
= -i \tr({\gamma_5  \gamma_5}) \epsilon^{\mu \nu \rho \sigma}
= -4 i \epsilon^{\mu \nu \rho \sigma}
\end{split}
\end{equation}


\begin{comment}
%\\
%\mbox{ Denoting }\gamma^\mu &\mbox{ as just }\mu\mbox{ and so on, }
%----------------------------------------------------------------------
\\
\tr ( \gamma^{\mu_1} \gamma^{\mu_2} \cdots \gamma^{\mu_{2n}} )
&=
2g^{\mu_1 \mu_2} \tr ( \gamma^{\mu_3} \cdots \gamma^{\mu_{2n}} )
-
\tr ( \gamma^{\mu_2} \gamma^{\mu_1}  \gamma^{\mu_3} \cdots \gamma^{\mu_{2n}} )
\\
&=
2g^{12} \tr(3 \cdots 2n) 
-
2g^{13} \tr(2\cdot 4 \cdots 2n)
+
\tr ( 2 \cdot 3 \cdot 1 \cdot 4 \cdots 2n )
\\
&\cdots
\\
&=
\sum_{i = 2}^{2n} (-1)^i 2g^{1i} \tr( 2\cdots \xcancel{i} \cdots 2n)
+(-1)^{2n -1} \tr ( 2 \cdots 2n \cdot 1)
\\
&=
\sum_{i = 2}^{2n} (-1)^i g^{\mu_1 \mu_i} 
\tr( \gamma^{\mu_2} \cdots \xcancel{\gamma^{\mu_i}} \cdots \gamma^{\mu_{2n}})

%\mbox{writing}
\end{comment}


\bigskip
%------------------------------------------------------------------------------------------------------------------------------------------------
\noindent
\subsection{Lorentz transformation property}
\label{sec:AppDirac_LorentzTransf}

For a Lorentz transformation $L^\mu_{\;\;\nu}: p^{(0)\mu} = (m, \bld{0}) \mapsto p^\mu$, we write
$\psi^{(\pm)}(\bld{p}) = S(L) \psi^{(\pm)}(\bld{0})$.
Eq. (\ref{eqn:LorentzTransfgamma}) reads
\begin{eqnarray*}
S^{-1}(L) \slashed{p} S(L)
=
p_\mu L^\mu_{\;\;\nu} \gamma^\nu
=
L_\mu^{\;\;\rho} L^\mu_{\;\;\nu} p^{(0)}_\rho \gamma^\nu
=
g^\rho_\nu p^{(0)}_\rho \gamma^\nu
=
m \gamma^0
\end{eqnarray*}
and we have

\bigskip
%---------------------------------------------------
\begin{eqnarray}
\det(A)
&=&
- \frac{1}{4!}
\epsilon_{\mu \nu \rho \sigma}\epsilon^{\mu' \nu' \rho' \sigma'}
A^\mu_{\;\;\mu'}A^\nu_{\;\;\nu'}A^\rho_{\;\;\rho'}A^\sigma_{\;\;\sigma'}
\nonumber\\
&=&
- \frac{1}{4!}
\epsilon_{\mu \nu \rho \sigma}\epsilon^{\mu' \nu' \rho' \sigma'}
A^\mu_{\;\;0}A^\nu_{\;\;1}A^\rho_{\;\;2}A^\sigma_{\;\;3}
\end{eqnarray}

%---------------------------------------------------
\begin{eqnarray}
S^{-1}(L) \gamma_5 S(L)
&=&
\det(L) \gamma_5
\end{eqnarray}
%---------------------------------------------------
\bigskip

\bigskip

Under a proper homogeneous Lorentz transformation
\footnote{%--------------------------------------------------------
\begin{eqnarray*}
L_\mu^{\;\;\rho} L^\mu_{\;\;\sigma} = g^\rho_\sigma\,,
\hspace{5mm}
\end{eqnarray*}
Denoting a matrix with its elements given by $L^\mu_{\;\;\nu}$ as $L$,
we have
\begin{eqnarray*}
L_\mu^{\;\;\rho} = (L^{-1})^\rho_{\;\;\mu}
\end{eqnarray*}
}%-------------------------------------------------------- end of footnote
\begin{equation}
x^\mu \mapsto x^{\mu'} = L(\bld{\beta}, \bld{\theta})^\mu_{\;\nu}\; x^\nu\,,
\end{equation}
the Dirac field transforms as
\begin{eqnarray}
\psi(x) \mapsto \psi'(x') = S(L) \psi(x)\,,
\hspace{5mm}
S(L) = \exp[ -\frac{i}{4}\sigma_{\mu \nu} \omega^{\mu \nu}]\,,
\end{eqnarray}
where $\omega^{\nu \mu} = -\omega^{\mu \nu}$ and
\begin{eqnarray}
\omega^{0i} &=& \xi_i\,,
\hspace{5mm}
\bld{\xi} = \xi \bld{\beta}/\beta\,,
\hspace{3mm}
\xi = \frac{1}{2} \ln \frac{1+\beta}{1-\beta}
\\
\omega^{ij}
&=&
-\epsilon_{ijk} \theta_k
\end{eqnarray}
We may examine the properties under boosts and rotations separately by considering
\begin{eqnarray}
S_{boost}(L(\bld{\beta},\bld{0})) &=& \exp[
-\frac{i}{2}\sigma_{0i}\xi_i ]
\\
&=&
\cosh \frac{\xi}{2} - \frac{\bld{\beta}\cdot \bld{\alpha}}{\beta} \sinh \frac{\xi}{2}
\end{eqnarray}
and
\begin{eqnarray}
S_{rotation}(L(\bld{0},\bld{\theta}))&=& \exp[
\frac{i}{2} \bld{\sigma}\cdot \bld{\theta} ]
\\
&=&
\cos \frac{\theta}{2}
+ i \frac{\bld{\theta}\cdot\bld{\sigma}}{\theta} \sin \frac{\theta}{2}
\end{eqnarray}
where three components of
\begin{eqnarray}
\bld{\sigma}  \leftdef 
\frac{1}{2} \epsilon^{\uparrow ij}\sigma_{ij}
=
\frac{i}{2} \epsilon^{\uparrow ij} \gamma^i \gamma^j
=
\gamma_5 \gamma^0 \bld{\gamma}
\label{eqn:DiGenRotation}
\end{eqnarray}
are the spinor (fundamental) reperesentation of generators of the spatial rotation.
They certainly satisfy $\{ \sigma^i, \sigma^j \}_+ = 2 \delta^{ij}$. Also,
$S^\dagger_{boost}(L) = S_{boost}(L)$,
$S^\dagger_{rotation}(L) = S^{-1}_{rotation}(L)$ and
$\gamma^0 S^\dagger_{}(L) \gamma^0 = S^{-1}_{}(L)$.
The covariance of the Dirac equation (\ref{eqn:DiracEqstandard}) is guaranteed by
a relationship
\begin{eqnarray}
S^{-1}(L) \gamma^\mu S(L) = L^\mu_{\;\;\nu} \gamma^\nu
\label{eqn:LorentzTransfgammaApp}
\end{eqnarray}
$S(L)$ has a property
\begin{eqnarray}
\gamma^0 S^\dagger(L) \gamma^0 = S^{-1}(L)\,,
\end{eqnarray}
so that
\begin{eqnarray}
\overline{\psi}(x) \mapsto {\overline{\psi}}'(x') = \psi^{*t} (x) S^\dagger(L) \gamma^0
= \overline{\psi}(x) S^{-1}(L)
\end{eqnarray}
and $\overline{\psi} \psi$ is a Lorentz scalar.

For later use, we write down a particular form of $S_{boost}$.
For a Lorentz transformation
\begin{eqnarray}
L(-\bld{\beta}) = \left(
\begin{array}{cc}
\gamma &
 \gamma \bld{\beta} \cdot
\\
\gamma  \bld{\beta} &
1 + \hat{\bld{\beta}} (\gamma - 1) \hat{\bld{\beta}} \cdot
\end{array}
\right)\,,
\end{eqnarray}
which transforms $p^{(0)}= (m, \bld{0}) \mapsto p = L p^{(0)} = (E, \bld{p})$,
we have
\begin{eqnarray}
\begin{array}{l}
S_{boost}(L)
=
\frac{1}{\sqrt{2m(E + m)}}
[ \slashed{p} \gamma^0 + m ]\,,
\\
S_{boost}^{-1}(L)
=
\frac{1}{\sqrt{2m(E + m)}}
[ \gamma^0 \slashed{p}  + m ]\,,
\end{array}
\end{eqnarray}
Since $L^{-1}$ corresponds to the change of the sign of the spacial momentum,
we have $S_{boost}^{-1}(L) = S_{boost}(L^{-1})$.


\bigskip
%------------------------------------------------------------------------------------------------------------------------------------------------
\noindent
{\bf Projections}\\
%and we have
%\begin{eqnarray*}
%(\gamma^0 - 1) \psi^{(+)}(\bld{0}) = 0\,,
%\hspace{5mm}
%(\gamma^0 + 1) \psi^{(-)}(\bld{0}) = 0
%\end{eqnarray*}
%\bigskip
%We choose their normalization
%\footnote{%-------------------------------------------------
%Izykson: $\{ b(\bld{p}),b^\dagger(\bld{p}') \}_+ = \frac{2E}{2m}\delta^3(\bld{p}-\bld{p}')$,
%$\bar{u}^\alpha u^\beta = \delta^{\alpha \beta}$,\\
%\hspace{20mm}$\psi(x) \sim \int d^3\bld{p}\frac{2m}{2E}[ b u e^{-ipx} \dots]$\\
%Hioki, Tong: $\{ b(\bld{p}),b^\dagger(\bld{p}') \}_+ = (2\pi)^3 2E\delta^3(\bld{p}-\bld{p}')$,
%$\bar{u}^\alpha u^\beta = 2m \delta^{\alpha \beta}$,\\
%\hspace{20mm}$\psi(x) = \int \frac{d^3\bld{p}}{(2\pi)^32E}[ b u e^{-ipx} \dots]$\\
%dim $\psi = E^{3/2}$
%}%-------------------------------------------end of footnote
%------------------------------------------------------------------------------
\underline{Energy state projection}
\begin{eqnarray}
\hat{\Omega}_+(\bld{p})
=
\frac{\slashed{p} + m}{2m}
\,, \hspace{5mm}
\hat{\Omega}_-(\bld{p})
=
\frac{- \slashed{p} + m}{2m}
\end{eqnarray}
For a Lorentz transformation $L: p^{(0)} = (m, \bld{0}) \mapsto p$, 
%we write $\psi^{(\pm)}(\bld{p}) = S(L) \psi^{(\pm)}(\bld{0})$.
Eq. (\ref{eqn:LorentzTransfgamma}) reads
\begin{eqnarray*}
S^{-1}(L) \slashed{p} S(L)
=
p_\mu L^\mu_{\;\;\nu} \gamma^\nu
=
L_\mu^{\;\;\rho} L^\mu_{\;\;\nu} p^{(0)}_\rho \gamma^\nu
=
g^\rho_\nu p^{(0)}_\rho \gamma^\nu
=
m \gamma^0
\end{eqnarray*}

%------------------------------------------------------------------------------
\noindent
\underline{Spin state projection}\\
The spin states are defined in the rest frame of particle.
Consider the generator of spatial rotation given in Eq. (\ref{eqn:DiGenRotation}).
For $s^{(0)}= (0, \bld{s})$ with a spatial 3 vector $\bld{s}$, 
with helps of Eq. (\ref{eqn:LorentzTransfgamma}) and a relation
\begin{eqnarray}
S^{-1}(L) \gamma_5 S^{}(L) = \det (L) \gamma_5\,,
\end{eqnarray}
we have 
\begin{eqnarray}
S^{-1}(L) \bld{\sigma} \cdot \bld{s} S^{}(L) 
&=&
S^{-1}(L) \gamma_5 \gamma^0 \bld{\gamma} \cdot \bld{s} S^{}(L) 
\nonumber\\
&=&
\det (L) \gamma_5 L^0_{\;\;\mu} \gamma^\mu L^{i}_{\;\;\nu} \gamma^\nu s^i
\nonumber\\
&=&
\det (L) \gamma_5 (p^{(0)}_\rho / m) L^\rho_{\;\;\mu} \gamma^\mu (- s^{(0)}_\sigma) L^{\sigma}_{\;\;\nu} \gamma^\nu 
\nonumber\\
&=&
\det (L) \gamma_5  (L^{-1})^{\;\;\rho}_{\mu} (p^{(0)}_\rho / m) \gamma^\mu  
(L^{-1})^{\;\;\sigma}_{\nu} \gamma^\nu (- s^{(0)}_\sigma)\,,
\nonumber
\end{eqnarray}
where $p^{(0)}= (m, \bld{0})$.
Then, for $L= L(\bld{\beta}): p = (E, \bld{p}) \mapsto p^{(0)}$,
we have
\begin{eqnarray}
S^{-1}(L) \bld{\sigma} \cdot \bld{s} S^{}(L) 
&=&
- (1/m) \det (L) \gamma_5  p^{}_\mu  \gamma^\mu   \gamma^\nu  s^{}_\nu\,,
\nonumber\\
&=&
\det (L) \frac{\gamma_5 \slashed{s} \slashed{p}}{m}\,,
\end{eqnarray}
where we have used a fact $\slashed{p} \slashed{s} = - \slashed{s} \slashed{p} $
for $p\cdot s = 0$.
Thus, we have for a proper Lorentz transformation that
\begin{eqnarray}
\frac{\gamma_5 \slashed{s} \slashed{p}}{m}
u(\bld{p})
=
S^{-1}(L) \bld{\sigma} \cdot \bld{s} S^{}(L) S^{-1}(L) u(\bld{0})
=
S^{-1}(L) \bld{\sigma} \cdot \bld{s} u(\bld{0})
\end{eqnarray}
and similar relationship for $v$. We set two linearly independent spinors 
in the rest frame as eigenstates of $\bld{\sigma}\cdot \bld{s}$ 
for $\bld{s}^2 = 1$
and write
\begin{eqnarray}
\begin{array}{l}
\bld{\sigma}\cdot \bld{s}\; u(\bld{0}, \bld{s} ) = u(\bld{0}, \bld{s} )
\\
\bld{\sigma}\cdot \bld{s}\; u(\bld{0}, -\bld{s} ) = -u(\bld{0}, -\bld{s} )
\\
- \bld{\sigma}\cdot \bld{s}\; v(\bld{0}, \bld{s} ) =  v(\bld{0}, \bld{s} )
\\
- \bld{\sigma}\cdot \bld{s}\; v(\bld{0}, -\bld{s} ) =  - v(\bld{0}, -\bld{s} )
\end{array}
\end{eqnarray}
Since $S^{-1}(L) u(\bld{0},\bld{s}) = u(\bld{p},\bld{s})$ and so on, we have
a set of similar relations by replacing $\bld{0}$ by $\bld{p}$ and
$\bld{\sigma}\cdot \bld{s}$ by $\gamma_5 \slashed{s} \slashed{p} / m$.
Considering relationships $(\slashed{p}/m)u(\bld{p}) = u(\bld{p})$
and $(-\slashed{p}/m)v(\bld{p}) = v(\bld{p})$, an operator matirx
\begin{eqnarray}
\hat{\Sigma}(s)
 = 
\frac{1 + \gamma_5 \slashed{s}}{2}
%\,,\hspace{5mm}
%s = L s^{(0)}\,,
%\hspace{2mm}
%s^{(0)} = (0, \bld{s})\,,
 \end{eqnarray}
projects out 
spinstates along $\bld{s}$ as
\begin{eqnarray}
\begin{array}{l}
\hat{\Sigma}(s) u(\bld{p}, \bld{s}) = u(\bld{p}, \bld{s})
\\
\hat{\Sigma}(s) u(\bld{p}, -\bld{s}) = 0
\\
\hat{\Sigma}(s) v(\bld{p}, \bld{s}) = v(\bld{p}, \bld{s})
\\
\hat{\Sigma}(s) v(\bld{p}, -\bld{s}) = 0
\end{array}
\end{eqnarray}
In the Dirac representation,
\begin{eqnarray}
\bld{\sigma}
&=&
1 \otimes \bld{\sigma}_{Pauli}
=
\left[
\begin{array}{cc}
\bld{\sigma}_{Pauli} & 0 \\
0 & \bld{\sigma}_{Pauli}
\end{array}
\right]
\end{eqnarray}


\bigskip

For $p = Lp^{(0)}$, $p^{(0)} = (m, \bld{0})$,
\begin{eqnarray}
\hat{\Sigma}(s) u(\bld{p})
=
\hat{\Sigma}(s)
S_{boost}(L)
u(\bld{0})
=
\frac{1 + \gamma_5 \slashed{s}}{2}
\frac{\slashed{p}\gamma^0 + m}{\sqrt{2m(E+m)}}
\end{eqnarray}


\noindent
\underline{Propagator}\\
\begin{eqnarray}
i S_{\xi \eta} &\equiv&
\{
\psi_\xi (x) , \overline{\psi}_\eta (y) \}
\end{eqnarray}
\begin{eqnarray}
iS(x - y)
&=& 
\dots
=
(i \slashed{\partial}_x + m)
\left[
D(x-y) - D(y - x)
\right]
\end{eqnarray}

\begin{eqnarray}
S_F(x - y) &\leftdef&
\bra 0 \braend T[ \psi(x) \overline{\psi}(y) \ketend 0 \ket
\\
&=&
i \int \frac{d^4 p}{(2\pi)^4}
\frac{e^{-ip(x-y)}}{\slashed{p} - m + i\epsilon}
\end{eqnarray}
%------------------------------------------------------------- comment out starts
\begin{comment}
\begin{eqnarray}
S_F(q) &=& i \int d^4x e^{iqx}
\bra 0 \braend T[\psi(x) \bar{\psi}(0)]
\ketend 0 \ket
\nonumber\\
&=&
\frac{-1}{\slashed{q} - m + i \epsilon}
=
- \frac{\slashed{q} + m}{ q^2 - m^2 + i \epsilon}
\end{eqnarray}
\end{comment}
%------------------------------------------------------------- comment out ends

%=======================================================================
\subsection{Choices of Spinor normalizations }
When we write the normalization of orthonormal basis of Dirac spinors as
\begin{eqnarray}
\begin{array}{c}
\overline{u}^{(r)}(\bld{p})
u^{(s)}(\bld{p})
= N \delta^{rs}\,,
\hspace{5mm}
\overline{v}^{(r)}(\bld{p})
v^{(s)}(\bld{p})
= -N \delta^{rs}\,,
\vspace{2mm}
\end{array}
\label{eqn:ArbitraryDiracspinorNormalization}
\end{eqnarray}
where a constant $N$ may depends on $p_0$,
relationships of the completeness of basis are wtitten as
\begin{eqnarray}
\begin{array}{l}
\displaystyle
\sum_{r=1}^2 u^{(r)}(\bld{p}) 
\overline{u}^{(r)}(\bld{p})
=
\frac{\slashed{p} + m}{2m}N\,,
\vspace{2mm}
%\\
%\hspace{-40mm}\mbox{and}
\\
\displaystyle
\sum_{r=1}^2 v^{(r)}(\bld{p}) 
\overline{v}^{(r)}(\bld{p})
=
\frac{\slashed{p} - m}{2m}N\,,
\end{array}
\label{eqn:DiracspinorPolSumforN}
\end{eqnarray}
as a consequence of the Dirac equations satisfied by these spinor basis.
If we denote the measure of momentum space as
\begin{eqnarray}
[ d^3 \bld{p} ] = F d^3 \bld{p}\,,
\label{eqn:MeasureMomSp}
\end{eqnarray}
where $F$ may depends on $p_0$, so that the Fourier expansion of a Dirac field is written as
\begin{equation}
\psi(x) 
=
\int [ d^3 \bld{p} ]
\sum_{r = 1,2}
\left[
c_r(\bld{p}) u^{(r)}(\bld{p}) e^{-ip \cdot x}
+
d_r^\dagger(\bld{p}) v^{(r)}(\bld{p}) e^{ip \cdot x}
\right]\,,
\label{eqn:DiracFourierArbMeasure}
\end{equation}
canonical quantization condition (\ref{eqn:canonicalQ_psi}) gives
normalizations of operators $c$ and $d$ as
\begin{equation}
\{ c_{r}(\bld{p}), c_{r'}^\dagger(\bld{p}') \}
=
\{ d_{r}(\bld{p}), d_{r'}^\dagger(\bld{p}') \}
=
C \delta_{rr'} 
\delta^3(\bld{p} - \bld{p}')\,,
\label{eqn:Dirc_creann_with_C}
\end{equation}
where $C$ is given through
\begin{eqnarray}
F^2 N C =
\frac{2m}{(2\pi)^2 2p_0}\,.
\end{eqnarray}
%----------------------------------------------------------------------------------------------------
\begin{table}[h]
\begin{center}
\begin{tabular}{|c||c|c|c|}
\hline
Reference & $F$ & $N$ & $C$
\\
\hline
\hline
Nish\cite{ref:NIsh.} & & &  1
\\
\hline
BogShi\cite{ref:Bogoliubov}  & $\frac{1}{\sqrt{(2\pi)^3}}$ &  $\frac{2m}{2p_0}$ &1
\\
\hline
ItzZub\cite{ref:Itzykson-Zuber} & $\frac{2m}{ (2\pi)^3 2p_0}$ &  1 & $ (2\pi)^3\frac{2m}{2p_0}$ 
\\
\hline
Weinb\cite{ref:Weinberg} & $\frac{1}{(2\pi)^3}$ & &  
\\
\hline
Mandl\cite{ref:Mandl-Shaw} & $\frac{\sqrt{V}}{(2\pi)^3} \sqrt{\frac{2m}{2p_0}}$ & 1 & $\frac{(2\pi)^3}{V}$
\\
\hline
Hioki\cite{ref:Hioki} & $\frac{1}{(2\pi)^3 2p_0}$ & $2m$ &  $(2\pi)^3 2p_0$
\\
\hline
\begin{tabular}{c}
Pes-Sch\cite{ref:Peskin-Schroeder} \\ Tong\cite{ref:Tong} 
\end{tabular}
  & $\frac{1}{(2\pi)^3 \sqrt{2p_0}}$ & $2m$ &  $(2\pi)^3$
\\
\hline
This & $\frac{1}{ \sqrt{(2\pi)^3} 2p_0}$ & $2m$ &  $2 p_0$
\\
\hline
\end{tabular}
\end{center}
\caption{Choices of normalizatin constants}
\label{tbl:SpinorNormChoicesTBL}
\end{table}
We list some typical choices of these constants in Table \ref{tbl:SpinorNormChoicesTBL}.
Among these cases, Mandl\cite{ref:Mandl-Shaw} introduces a finite spatial volume $V$
and uses a discrete momentum space with a unit volume $\Delta^3 \bld{p} = (2\pi)^3/V$. 
To reproduce that case, one needs to perform replacements
\begin{eqnarray}
\int d^3 \bld{p}
\longrightarrow
\sum_{\bld{p}} \Delta^3 \bld{p}
\end{eqnarray}
in Eq. (\ref{eqn:DiracFourierArbMeasure})
and 
\begin{eqnarray}
\delta^3(\bld{p} - \bld{p}')
\longrightarrow
\delta_{\bld{p}\bld{p}'} / \Delta^3 \bld{p}
\end{eqnarray}
in Eq. (\ref{eqn:Dirc_creann_with_C}).
%<<<<<<<<<<<<<<<<<<<<<<<<<<<<<<<<<<<<<<<<<<<<<<<<<<<<<<<<<<<<<


%{\bf The Dirac spinors}
%------------------------------------------------------------- comment out starts
\begin{comment}
We choose their normalization
\footnote{%-------------------------------------------------
Izykson: $\{ b(\bld{p}),b^\dagger(\bld{p}') \}_+ = \frac{2E}{2m}\delta^3(\bld{p}-\bld{p}')$,
$\bar{u}^r u^s = \delta^{rs}$,\\
\hspace{20mm}$\psi(x) \sim \int d^3\bld{p}\frac{2m}{2E}[ b u e^{-ipx} \dots]$\\
Hioki, Tong: $\{ b(\bld{p}),b^\dagger(\bld{p}') \}_+ = (2\pi)^3 2E\delta^3(\bld{p}-\bld{p}')$,
$\bar{u}^r u^s = 2m \delta^{rs}$,\\
\hspace{20mm}$\psi(x) = \int \frac{d^3\bld{p}}{(2\pi)^32E}[ b u e^{-ipx} \dots]$\\
dim $\psi = E^{3/2}$
}%-------------------------------------------end of footnote

\end{comment}
%------------------------------------------------------------- comment out end


%=======================================================================

\subsection{Representations of Dirac matrices}
All representations given below are based upon
a notation of the Pauli matrices written as
\footnote{%------------------------------- footnote >>
It was already given in Eq. (\ref{eqn:SU2GenFundamental}).
}%------------------------------- footnote //
\begin{eqnarray}
\sigma_1 =
\left(
\begin{array}{cc}
0 & 1 \\ 1 & 0
\end{array}
\right)\,,
\hspace{3mm}
\sigma_2 =
\left(
\begin{array}{cc}
0 & -i \\ i & 0
\end{array}
\right)\,,
\hspace{3mm}
\sigma_3 =
\left(
\begin{array}{cc}
1 &  0\\ 0 & -1
\end{array}
\right)\,,
\end{eqnarray}
which satisfy the following relationships
\begin{eqnarray}
\{
\sigma_i, \sigma_j \}_+ = 2 \delta_{ij}\,,
\hspace{3mm}
\sigma_i \sigma_j = i\epsilon_{ijk} \sigma_k + \delta_{ij}
\end{eqnarray}
%----------------------------------------------------------------------------

The following three representations are unitary equivalent.\\
(1) Dirac representation\\
\begin{eqnarray}
\gamma^0
=
\left[
\begin{array}{cc}
1 & 0 \\ 0 & -1
\end{array}
\right]
= \sigma_3 \otimes 1
\,,
\hspace{5mm}
\gamma^i
=
\left[
\begin{array}{cc}
0 & \sigma_i \\ -\sigma_i & 0
\end{array}
\right]
=
i\sigma_2 \otimes \sigma_i
\end{eqnarray}
Elements of $\left[\;\ddots\;\right]$ are $2 \times 2$ matrices.
Expressions with the direct product makes calculations more transparent. 
For instance,
\begin{eqnarray}
\begin{array}{l}
\alpha_i = \beta \gamma^i = \gamma^0 \gamma^i
= (\sigma_3 \otimes 1)(i\sigma_2 \otimes \sigma_i)
= -i \sigma_2 \sigma_3 \otimes \sigma_i
= \sigma_1 \otimes \sigma_i
\\
\hspace{35mm}
=
\left[
\begin{array}{cc}
0 & \sigma_i
\\
\sigma_i & 0
\end{array}
\right]
\end{array}
\end{eqnarray}
Other matrices are similarly obtained as
\begin{eqnarray}
\gamma_5
=
\left[
\begin{array}{cc}
0 & 1 \\ 1 & 0
\end{array}
\right]
= \sigma_1 \otimes 1\,,
\hspace{3mm}
\sigma_{0 i} = -i  \sigma_1 \otimes \sigma_i\,,
\hspace{3mm}
\sigma_{i j} = \epsilon_{ijk} \otimes \sigma_k
\end{eqnarray}
When we write the Dirac field as composed of two component 
large and small components,
\begin{eqnarray}
\psi =
\left[
\begin{array}{c}
\varphi \\ \chi
\end{array}
\right]\,
\end{eqnarray}
the Dirac equation for these components reads
\begin{eqnarray}
\left\{
\begin{array}{l}
i \partial_t \varphi(x)
=
m \varphi(x) + \frac{1}{i} \bld{\sigma} \cdot \bld{\partial} \chi(x)
\vspace{2mm}
\\
i \partial_t \chi(x)
=
-m \chi(x) + \frac{1}{i} \bld{\sigma} \cdot \bld{\partial} \varphi(x)
\end{array}
\right.
\end{eqnarray}
\\

%----------------------------------------------------------
\noindent
(2) Majorna representation\\
\begin{eqnarray}
\begin{array}{l}
\gamma^0
=
\left[
\begin{array}{cc}
0 & \sigma_2 \\ \sigma_2 & 0
\end{array}
\right]
= \sigma_1 \otimes \sigma_2
\,,
\hspace{5mm}
\gamma^1
=
\left[
\begin{array}{cc}
i\sigma_3 & 0 \\  0 & i\sigma_3\,,
\end{array}
\right]
=
1 \otimes i\sigma_3
\vspace{2mm}
\\ %------------------------
\gamma^2
=
\left[
\begin{array}{cc}
0 & -\sigma_2 \\ \sigma_2 & 0
\end{array}
\right]
= -i\sigma_2 \otimes \sigma_2
\,,
\hspace{5mm}
\gamma^3
=
\left[
\begin{array}{cc}
-i\sigma_1 & 0 \\  0 & -i\sigma_1
\end{array}
\right]
=
1 \otimes -i\sigma_1\,,
\end{array}
\end{eqnarray}
and $\gamma_5 = \sigma_3 \otimes \sigma_2$.
In this representation, we have $(\gamma^\mu)^* = - \gamma^\mu$ and
the Dirac equation becomes real. The Dirac fields are given as linear combinations
of real sollutions. This representation is related with the Dirac one by
\begin{eqnarray}
\gamma^\mu_{\mbox{\tiny Majorana}}
= U \gamma^\mu_{\mbox{\tiny Dirac}} U^{\dagger}\,,
\hspace{5mm}
U = U^\dagger =
\frac{1}{\sqrt{2}}
\left[
\begin{array}{cc}
1 & \sigma_2 \\
\sigma_2 & -1
\end{array}
\right]
\end{eqnarray}\\

%----------------------------------------------------------
\noindent
(3) Chiral representation\\
\begin{eqnarray}
\begin{array}{l}
\gamma^0
=
\left[
\begin{array}{cc}
0 & -1 \\ -1 & 0
\end{array}
\right]
= \sigma_1 \otimes -1
\,,
\hspace{5mm}
\gamma^i
=
\left[
\begin{array}{cc}
0  & \sigma_i \\  -\sigma_i & 0\,,
\end{array}
\right]
=
i\sigma_2 \otimes \sigma_i
\vspace{2mm}
\\ %------------------------
\gamma_5
=
\left[
\begin{array}{cc}
1 & 0 \\ 0 & -1
\end{array}
\right]
= \sigma_3 \otimes 1
\,,
\hspace{5mm}
\sigma_{0i}
=
\left[
\begin{array}{cc}
-i\sigma_i & 0 \\  0 & i\sigma_i
\end{array}
\right]
=
-i\sigma_3 \otimes \sigma_i\,,
\vspace{2mm}
\\ %------------------------
\sigma_{ij}
=
\left[
\begin{array}{cc}
\epsilon_{ijk}\sigma_k & 0 \\  0 & \epsilon_{ijk}\sigma_k
\end{array}
\right]
=
\epsilon_{ijk} \otimes \sigma_k\,,
\end{array}
\end{eqnarray}
In this representation, spatial rotators and Lorentz boosters, namely, proper Lorentz transformations
take forms as diagonal in the $\left[\;\ddots\;\right]$ space so that $\varphi$ and $\chi$ fields
are transformed independently. This representation is related with the Dirac one by
\begin{eqnarray}
\gamma^\mu_{\mbox{\tiny Chiral}}
= U \gamma^\mu_{\mbox{\tiny Dirac}} U^{\dagger}\,,
\hspace{5mm}
U = 
\frac{1}{\sqrt{2}}
(1 - \gamma_5 \gamma^0)
=
\frac{1}{\sqrt{2}}
\left[
\begin{array}{cc}
1 & 1 \\
-1 & 1
\end{array}
\right]
\end{eqnarray}\\




\subsection{Majorana and Chiral representations}

In the following, two more representations 
%other than Dirac representation
for Dirac matrices are given.\\
%----------------------------------------------------------
\noindent
\underline{Majorna representation}\\
\begin{eqnarray}
\begin{array}{l}
\gamma^0
=
\left[
\begin{array}{cc}
0 & \sigma_2 \\ \sigma_2 & 0
\end{array}
\right]
= \sigma_1 \otimes \sigma_2
\,,
\hspace{5mm}
\gamma^1
=
\left[
\begin{array}{cc}
i\sigma_3 & 0 \\  0 & i\sigma_3\,,
\end{array}
\right]
=
1 \otimes i\sigma_3
\vspace{2mm}
\\ %------------------------
\gamma^2
=
\left[
\begin{array}{cc}
0 & -\sigma_2 \\ \sigma_2 & 0
\end{array}
\right]
= -i\sigma_2 \otimes \sigma_2
\,,
\hspace{5mm}
\gamma^3
=
\left[
\begin{array}{cc}
-i\sigma_1 & 0 \\  0 & -i\sigma_1
\end{array}
\right]
=
1 \otimes -i\sigma_1\,,
\end{array}
\end{eqnarray}
and $\gamma_5 = \sigma_3 \otimes \sigma_2$.
In this representation, we have $(\gamma^\mu)^* = - \gamma^\mu$ and
the Dirac equation becomes real. The Dirac fields are given as linear combinations
of real sollutions. This representation is related with the Dirac one by
\begin{eqnarray}
\gamma^\mu_{\mbox{\tiny Majorana}}
= U \gamma^\mu_{\mbox{\tiny Dirac}} U^{\dagger}\,,
\hspace{5mm}
U = U^\dagger =
\frac{1}{\sqrt{2}}
\left[
\begin{array}{cc}
1 & \sigma_2 \\
\sigma_2 & -1
\end{array}
\right]
\end{eqnarray}\\

%----------------------------------------------------------
\noindent
\underline{Chiral representation}\\
\begin{eqnarray}
\begin{array}{l}
\gamma^0
=
\left[
\begin{array}{cc}
0 & -1 \\ -1 & 0
\end{array}
\right]
= \sigma_1 \otimes -1
\,,
\hspace{5mm}
\gamma^i
=
\left[
\begin{array}{cc}
0  & \sigma_i \\  -\sigma_i & 0\,,
\end{array}
\right]
=
i\sigma_2 \otimes \sigma_i
\vspace{2mm}
\\ %------------------------
\gamma_5
=
\left[
\begin{array}{cc}
1 & 0 \\ 0 & -1
\end{array}
\right]
= \sigma_3 \otimes 1
\,,
\hspace{5mm}
\sigma_{0i}
=
\left[
\begin{array}{cc}
-i\sigma_i & 0 \\  0 & i\sigma_i
\end{array}
\right]
=
-i\sigma_3 \otimes \sigma_i\,,
\vspace{2mm}
\\ %------------------------
\sigma_{ij}
=
\left[
\begin{array}{cc}
\epsilon_{ijk}\sigma_k & 0 \\  0 & \epsilon_{ijk}\sigma_k
\end{array}
\right]
=
\epsilon_{ijk} \otimes \sigma_k\,,
\end{array}
\end{eqnarray}
In this representation, spatial rotators and Lorentz boosters, namely, proper Lorentz transformations
take forms as diagonal in the $\left[\;\ddots\;\right]$ space so that $\varphi$ and $\chi$ fields
are transformed independently. This representation is related with the Dirac one by
\begin{eqnarray}
\gamma^\mu_{\mbox{\tiny Chiral}}
= U \gamma^\mu_{\mbox{\tiny Dirac}} U^{\dagger}\,,
\hspace{5mm}
U = 
\frac{1}{\sqrt{2}}
(1 - \gamma_5 \gamma^0)
=
\frac{1}{\sqrt{2}}
\left[
\begin{array}{cc}
1 & 1 \\
-1 & 1
\end{array}
\right]
\end{eqnarray}\\



%==============================================================

%\subsubsection{Quantized Free Field}


%------------------------------------------------------------- comment out
\begin{comment}
\subsubsection{Majorana Field}
Massless, spin1/2, selfconjugate.
\begin{equation}
i \slashed{\partial} \psi(x) = 0\,,
\hspace{3mm}
\bar{\psi}(x) i \stackrel{\leftarrow}{\slashed{\partial}}   = 0
\end{equation}
\end{comment}
%------------------------------------------------------------- comment out end

