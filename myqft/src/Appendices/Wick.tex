We write our field as
\begin{eqnarray}
\varphi(x) = a(x) + b^\dagger(x)\,,
\end{eqnarray}
where $a(x)$ includes an annihilation operator and $b^\dagger(x)$ includes a creation operator.
Particularly, we write
\begin{eqnarray}
a(x) = \int \frac{d^3 \bld{k}}{\sqrt{(2\pi)^3}2k^0} a(\bld{k}) e^{- ikx}\,,
\hspace{5mm}
b^\dagger(x) = \int \frac{d^3 \bld{k}}{\sqrt{(2\pi)^3}2k^0} b^\dagger(\bld{k}) e^{ikx}
\end{eqnarray}
Usual notation of hermite conjugates is applied to these objects and, for instance, 
meanings to write $\varphi^\dagger(x)$ and $b(x)$ are obvious.
The creation and annihilation operators satisfy,
\begin{eqnarray}
\begin{array}{c}
[a(\bld{k}), a^\dagger(\bld{k}')] = [b(\bld{k}), b^\dagger(\bld{k}')] = 2k^0 \delta^3(\bld{k}-\bld{k}')
\\
\mbox{all other }[\cdots] \mbox{ among }a, a^\dagger, b, b^\dagger \mbox{ are } 0.
\end{array}
\end{eqnarray}
$[\cdots]$ denotes the (anti-) commutator when we are considering
a boson (fermion) field. Fields of different kinds are assumed to be commuting to each other.
We introduce a signature factor $s_f$ which reads $+1$ for bosons and $-1$ for fermions.
In the following, we abbreviate different arguments by suffices. For instance, 
$\varphi(x_1)$ is denoted by $\varphi_1$. $a_1$ can mean both $a(x_1)$ and $a(\bld{k}_1)$.
Also, we will write $\Phi$ to represent either of $\varphi$ and $\varphi^\dagger$.

%======================================================================
\subsection{Normal ordered product}
Denoted by $\normalprod{\cdots}$.
All creation operators stack to the left and all annihilation operators stack to the right.
When the field is fermionic, an exchange of neighboring operators produce a $s_f$.
For instance, 
\begin{eqnarray}
\normalprod{a_1 a^\dagger_1 b^\dagger_2 b_2} = s_f^{2} a^\dagger_1 b^\dagger_2 a_1 b_2,
\end{eqnarray}
where suffices stand for different arguments of $x$ or $\bld{k}$.
Another example reads,
\begin{eqnarray}
\normalprod{\varphi_1 \varphi_2 \varphi_3 \varphi_4}
&=&
a_1^\dagger a_2^\dagger a_3^\dagger a_4^\dagger 
+ s_f^3 a_2^\dagger a_3^\dagger a_4^\dagger  a_1 + \dots
\nonumber\\
&&
+ s_f^4 a_3^\dagger a_4^\dagger  a_1  a_2 
+ s_f^3 a_2^\dagger a_4^\dagger  a_1  a_3 + 
\dots
\nonumber\\
&& {\small ( 2^4\mbox{ terms})}
\label{eqn:normalprod4fields}
\end{eqnarray}
%Here $\normalprod{\cdots}$ pretends to be a linear operation but it is not.
It is important to note that $\bra 0 \braend \normalprod{\cdots} \ketend 0 \ket = 0$.
A relationship
\begin{eqnarray}
\normalprod{a_1 a_2^\dagger}
= s_f a_2^\dagger a_1 
= a_1 a_2^\dagger - [a_1, a_2^\dagger]
\end{eqnarray}
remind us a caution 
for wrong manipulations such as follows.
Assuming the linearity of $\normalprod{\cdots}$ and a relationship 
$\normalprod{\mbox{\it c-number}\,} = $ {\it c-number},
we might write
\begin{eqnarray}
\mbox{\bf Wrong!}
\hspace{5mm}
\normalprod{\normalprod{a_1 a_2^\dagger}} = \normalprod{a_1 a_2^\dagger - [a_1, a_2^\dagger]}
= \normalprod{a_1 a_2^\dagger} + \mbox{\it c-number}
\end{eqnarray}
which contradicts with the idempotency $\normalprod{\normalprod{abc\dots}} = \normalprod{abc\dots}$.
We made it wrong when we replace a symbol $[a_1, a_2^\dagger]$ by a {\it c-number}.
The $\normalprod{\cdots}$ is an operation on symbols and one can not replace them
by using mathematical equations. In fact, we have
\begin{eqnarray}
\normalprod{[a_1, a_2^\dagger]} = \normalprod{ a_1 a_2^\dagger - s_f a_2^\dagger a_1 } = 0
\end{eqnarray}
and thus recover the idempotency by keeping expressions in symbols.
The same caution is applied to relationships
\begin{eqnarray}
\normalprod{\normalprod{\Phi_1 \Phi_2 \cdots}\normalprod{\Phi_a \Phi_b \cdots}}
= \normalprod{\Phi_1 \Phi_2 \cdots \Phi_a \Phi_b \cdots }\,,
\hspace{3mm}
\mbox{etc.}
\end{eqnarray}
where each $\Phi$ stands for a $\varphi$ or $\varphi^\dagger$.
The following is a frequently used relationship:
\begin{eqnarray}
\normalprod{\Phi_1 \Phi_2} = \Phi_1 \Phi_2 -
\bra 0 \braend \Phi_1 \Phi_2 \ketend 0 \ket\,.
\end{eqnarray}
%======================================================================
\subsection{Time ordered product}
Denoted by $T[\cdots]$.
This is also an idempotent manipulation on a series of symbols 
and a caution as one in the normal ordered product is necessary.
In this product, a product of fields are ordered in such a way that
a field on the left has the time variable larger than the right one.
\begin{eqnarray}
T[\Phi_a \Phi_b \cdots] = s_f(1,2,\cdots;a,b,\cdots) \Phi_1 \Phi_2 \cdots
\end{eqnarray}
where $\Phi_l$ denotes one of $\varphi(x_l)$ or $\varphi^\dagger(x_l)$,
$t_1 > t_2 > \cdots$ and 
$s_f(1,2,\cdots;a,b,\cdots)$ = +1 (-1) when $\Phi$ is fermionic and
a permutation $(a,b,\cdots) \to (1, 2, \cdots)$ is even (odd).
When $\Phi$ is bosonic, one can omit $s_f$.
In more general form, we write the definition of T-product as
\begin{eqnarray}
T[\Phi_1 \Phi_2 \cdots] =
\sum_{P} \theta(t_{P_1} > t_{P_2} > \cdots) 
s_f(P_1, P_2, \cdots; 1, 2, \cdots)
\Phi_{P_1} \Phi_{P_2} \cdots 
\end{eqnarray}
where the sum is taken over all permutations of $(1, 2, \dots) \to (P_1,P_2, \dots)$.
Particularly, 
\begin{eqnarray}
T[\varphi_1 \varphi_2] &=&
\theta(t_1 > t_2) \varphi_1 \varphi_2 + \theta(t_2 > t_1) s_f \varphi_2 \varphi_1
\nonumber\\
&=&
\theta(t_1 > t_2) \left\{ 
(a_1 + b_1^\dagger)(a_2 + b_2^\dagger) \right\}
+
\theta(t_2 > t_1) s_f  \left\{ 
(a_2 + b_2^\dagger) (a_1 + b_1^\dagger) \right\}
\nonumber\\
&=&
\theta(t_1 > t_2) \left\{ 
a_1 a_2 + [a_1, \underline{b_2^\dagger] + s_f   b_2^\dagger  a_1} + b_1^\dagger (a_2 + b_2^\dagger)
\right\}
+ 
\theta(t_2 > t_1) s_f  \left\{  1 \leftrightarrow 2 \right\}
\nonumber\\
&=&
\theta(t_1 > t_2) \left\{ 
\normalprod{\varphi_1 \varphi_2} + [a_1, b_2^\dagger]
\right\}
+ 
\theta(t_2 > t_1) s_f  \left\{  1 \leftrightarrow 2 \right\}
\nonumber\\
&=&
\normalprod{\varphi_1 \varphi_2}
+
\theta(t_1 > t_2) [a_1, b_2^\dagger]
+ 
\theta(t_2 > t_1) s_f  [a_2, b_1^\dagger]
\nonumber\\
&=&
\normalprod{\varphi_1 \varphi_2}
+
\bra 0 \braend
T[\varphi_1 \varphi_2]
\ketend 0 \ket
\nonumber
\end{eqnarray}
In deriving the last line, we have employed the expression in the first line.
Though this result itself is useless since $\bra 0 \braend
T[\varphi_1 \varphi_2]
\ketend 0 \ket = 0$
for complex scalar fields, remember that the T-product is a manipulation on symbols.
This means the above equation can be generalized to arbitrary field to write
\begin{eqnarray}
T[\Phi_1 \Phi_2] &=&
\normalprod{\Phi_1 \Phi_2}
+
\bra 0 \braend
T[\Phi_1 \Phi_2]
\ketend 0 \ket
\label{eqn:Tprodphiphi}
\end{eqnarray}
and we can substitute either $\varphi$ or $\varphi^\dagger$ to each of $\Phi$
to write, for instance,
\begin{eqnarray*}
T[\varphi_1^\dagger \varphi_2] &=&
\normalprod{\varphi_1^\dagger \varphi_2}
+
\bra 0 \braend
T[\varphi_1^\dagger \varphi_2]
\ketend 0 \ket
\end{eqnarray*}
which relates a T-product to the propagator of complex scalar fields.

%======================================================================
\subsection{Wick's theorem}
We write Eq. (\ref{eqn:Tprodphiphi}) as
\begin{eqnarray}
T[\Phi_1 \Phi_2] &=&
\normalprod{\Phi_1 \Phi_2}
+
\acontraction[1ex]{}{\Phi}{{}_1}{\Phi}
\Phi_1 \Phi_2
\end{eqnarray}
and call the last term a contract of fields.
Wick's theorem gives a way to express a T-product of more than two
fields in terms of contracts and normal ordered products:
\begin{eqnarray}
T[\Phi_1 \Phi_2 \cdots ]
&=&
\normalprod{\Phi_1 \Phi_2 \cdots }
\nonumber\\
&+&
\sum_{i < j}  s_f(i,j,1,2,\cdots; 1, 2, \cdots)
%\normalprod{\cdots  \Phi_i \cdots \Phi_j \cdots}
\normalprod{
\acontraction[1ex]{\cdots}{\Phi}{{}_i\cdots}{\Phi}
\cdots \Phi_i \cdots \Phi_j \cdots
}%end normalprod
\nonumber\\
&+&
\sum_{i < j, k<l}  s_f(i,j, k, l, 1,2,\cdots; 1, 2, \cdots)
\normalprod{
\acontraction[1ex]{\cdots}{\Phi}{{}_i\cdots \Phi_k \cdots}{\Phi}
\acontraction[2ex]{\cdots \Phi_i \cdots }{\Phi}{{}_k\cdots \Phi_j \cdots}{\Phi}
\cdots \Phi_i \cdots \Phi_k \cdots  \Phi_j \cdots \Phi_l \cdots
}%end normalprod
\nonumber\\
&+&
\cdots  \mbox{ (all possible contracts)}
\nonumber\\
&=&
\normalprod{\Phi_1 \Phi_2 \cdots }
\nonumber\\
&+&
\sum_{i < j}  s_f(i,j,1,2,\cdots; 1, 2, \cdots)
\bra 0 \braend T[\Phi_i \Phi_j] \ketend 0 \ket
\normalprod{
\cdots \xout{\Phi_i} \cdots \xout{\Phi_j} \cdots
}%end normalprod
\nonumber\\
&+&
\sum_{i < j, k<l}  s_f(i,j, k, l, 1,2,\cdots; 1, 2, \cdots)
\nonumber\\
&&
\hspace{10mm}
\bra 0 \braend T[\Phi_i \Phi_j] \ketend 0 \ket
\bra 0 \braend T[\Phi_k \Phi_l] \ketend 0 \ket
%\hspace{50mm}
\normalprod{
\cdots \xout{\Phi_i} \cdots \xout{\Phi_k} \cdots  \xout{\Phi_j} \cdots \xout{\Phi_l} \cdots
}%end normalprod
\nonumber\\
&+&
\cdots
\nonumber\\
\end{eqnarray}
Each of $\Phi$ can be either $\varphi$ or $\varphi^\dagger$.
Since contracts appear to compensate the difference between T-product and
normal ordered product, no contracts should be taken among fields inside a normal orderd
product.
For instance, 
\begin{eqnarray}
T[\normalprod{\Phi_1 \Phi_2} \normalprod{\Phi_3 \Phi_4} ]
&=&
\normalprod{\Phi_1 \Phi_2 \Phi_3 \Phi_4}
+
s_f \normalprod{
\acontraction[1ex]{}{\Phi}{{}_1 \Phi_2}{\Phi}
\Phi_1 \Phi_2 \Phi_3 \Phi_4}
+
s_f^2 
\normalprod{
\acontraction[1ex]{}{\Phi}{{}_1 \Phi_2 \Phi_3}{\Phi}
\Phi_1 \Phi_2 \Phi_3 \Phi_4}
\nonumber\\
&&+
\normalprod{\acontraction[1ex]{\Phi_1 }{\Phi}{{}_2}{\Phi}\Phi_1 \Phi_2 \Phi_3 \Phi_4}
+
s_f
\normalprod{\acontraction[1ex]{\Phi_1 }{\Phi}{{}_2 \Phi_3}{\Phi}\Phi_1 \Phi_2 \Phi_3 \Phi_4}
\nonumber\\
&&
+
s_f
\normalprod{
\acontraction[1ex]{}{\Phi}{{}_1 \Phi_2}{\Phi}
\acontraction[2ex]{\Phi_1}{\Phi}{{}_2 \Phi_3}{\Phi}
\Phi_1 \Phi_2 \Phi_3 \Phi_4}
+
\normalprod{\acontraction[1ex]{}{}{}{}
\acontraction[2ex]{}{\Phi}{{}_1 \Phi_2 \Phi_3}{\Phi}
\acontraction[1ex]{\Phi_1}{\Phi}{{}_2}{\Phi}
\Phi_1 \Phi_2 \Phi_3 \Phi_4}
\label{eqn:TofNormals}
\end{eqnarray}


%====================================================================
\newpage

\subsubsection{Propagator}
When $x-y$ is spacelike, there exists a Lorentz frame where $x^0 - y^0$.
Then the $r.h.s.$ of Eq. (\ref{eqn:scfieldcommrel}) vanishes. 
The whole expression is Lorentz invariant and it must vanish for all $(x-y)^2 < 0$.
Nevertheless, $D(x-y)$ itself does not vanish even $x-y$ is spacelike.

The Feynman propagator is defined as
\begin{eqnarray}
\bra 0 \braend T[\varphi(x) \varphi(y)] \ketend 0 \ket
&=&
\theta(x^0 - y^0) D(x-y)
+
\theta(y^0 - x^0) D(y-x)
\label{eqn:scFeynPropbyD-A}
\\
&=&
i \int \frac{d^4 p}{(2\pi)^4}
\frac{e^{-ip (x-y)}}{p^2 - m^2 + i\epsilon}
\label{eqn:scFeynmanProp-A}
\\
&\rightdef&
\Delta_F(x-y)
\nonumber
\end{eqnarray}
{\it Proof of Eq.(\ref{eqn:scFeynmanProp-A}) $\Leftrightarrow$ Eq. (\ref{eqn:scFeynPropbyD-A})}\\
%\begin{proof}
\begin{eqnarray}
\frac{1}{p^2 - m^2 + i\epsilon}
=
\frac{1}{2 E_{\bld{p}}} \left(
\frac{1}{p^0 - E_{\bld{p}} + i\epsilon}
-
\frac{1}{p^0 + E_{\bld{p}} - i\epsilon}
\right)
\,,
\end{eqnarray}
\begin{eqnarray}
\Delta_F(x-y)
&=&
i \int \frac{d^3 \bld{p}}{(2\pi)^3 2 E_{\bld{p}}} e^{i\bld{p}\cdot (\bld{x} - \bld{y})}
\nonumber\\
&&
\times \int \frac{d p^0}{2\pi} e^{- ip^0 (x^0 - y^0)}
\left(
\frac{1}{p^0 - E_{\bld{p}} + i\epsilon}
-
\frac{1}{p^0 + E_{\bld{p}} - i\epsilon}
\right)
\nonumber\\
&=&
\int \frac{d^3 \bld{p}}{(2\pi)^3 2 E_{\bld{p}}} e^{i\bld{p}\cdot (\bld{x} - \bld{y})}
\frac{-1}{2\pi i} \left(
\theta(x^0 - y^0) (-2\pi i e^{-i E_{\bld{p}}(x^0 - y^0)})
\right.
\nonumber\\
&&
\left.
-
\theta(y^0 - x^0) (+2\pi i e^{i E_{\bld{p}}(x^0 - y^0)}
\right)
\nonumber\\
&=&
\int \frac{d^3 \bld{p}}{(2\pi)^3 2 E_{\bld{p}}} 
\left(
\theta(x^0 - y^0) e^{-ip(x-y)}
+
\theta(y^0 - x^0) e^{i p(x-y)}
\right)
\nonumber\\
&=&
\theta(x^0 - y^0) D(x-y)
+
\theta(y^0 - x^0) D(y-x)
\nonumber\\
&& 
\hspace{65mm}
\blacksquare
\end{eqnarray}
%\end{proof}


\subsubsection{T product}
We decompose a real scalar field as 
\begin{eqnarray}
\varphi(x) = \varphi^{(+)}(x) + \varphi^{(-)}(x)
\end{eqnarray}
where $\varphi^{(+)}$ ($\varphi^{(-)}$) is the term which contain
annihilation (creation) operator in Eq. (\ref{eqn:scYfields}).
If $x^0 > y^0$,
\begin{eqnarray}
T[ \varphi(x) \varphi(y)]
&=& (\varphi^{(+)}(x) + \varphi^{(-)}(x))(\varphi^{(+)}(y) + \varphi^{(-)}(y))
\nonumber\\
&=&
\varphi^{(+)}(x) \varphi^{(+)}(y)
+
( [\varphi^{(+)}(x), \varphi^{(-)}(y)] 
+ \varphi^{(-)}(y) \varphi^{(+)}(x) )
\nonumber\\
&&
+
\varphi^{(-)}(x) \varphi^{(+)}(y)
+
\varphi^{(-)}(x) \varphi^{(-)}(y)
\nonumber\\
&=&
\normalprod{\varphi(x) \varphi(y)} + D(x-y)
\nonumber
\end{eqnarray}
and if $y^0 > x^0$,
\begin{eqnarray}
T[ \varphi(x) \varphi(y)]
=
\normalprod{\varphi(x) \varphi(y)} + D(y-x)
\nonumber
\end{eqnarray}
Then, for arbitrary $x^0$ and $y^0$,
\begin{eqnarray}
T[ \varphi(x) \varphi(y)]
=
\normalprod{\varphi(x) \varphi(y)} + \Delta_F(x-y)
\end{eqnarray}
A similar evaluation shows for a complex scalar field that
\begin{eqnarray}
T[ \varphi(x) \varphi^\dagger(y)]
=
\normalprod{\varphi(x) \varphi^\dagger(y)} + \Delta_F(x-y)
\end{eqnarray}

\noindent
$\bullet$Wick's theorem\\
\begin{eqnarray}
&&T[\varphi_1(x_1) \varphi_2(x_2) \varphi_3(x_3) \varphi_4(x_4) \dots]
\nonumber\\
&=& \normalprod{\varphi_1 \varphi_2 \varphi_3 \varphi_4 \dots}
\nonumber\\
&+&
\sum_{k<l} \Delta_F(x_k - x_l)
\normalprod{
%%\contraction[1ex]{\dots}{\varphi_k}{\dots}{\varphi_l}
%%%\contraction[1ex]{\dots}{\varphi}{{}_k \dots}{\varphi}
%%%
{\varphi_1\dots}{\xout{\varphi_k}}{\dots}{\xout{\varphi_l}} \dots
%%{\varphi_1\dots}{\varphi_k}{\dots}{\varphi_l} \dots
%%
}%end normalprod
%\contraction[1ex]{\dots}{\varphi}{{}_k \dots}{\varphi}
%\nomathglue{%
%%\normalprod{
%%%\acontraction[1ex]{\dots}{\varphi_k\hphantom{\varphi_k}}{}{\dots\varphi_l}
%%\acontraction[1ex]{\dots}{\varphi_k\hphantom{\varphi_k}}{}{\dots\varphi_l}
%%{\varphi_1\dots}{\varphi_k}{\dots}{\varphi_l} \dots
%%%{\varphi_1\dots}{\xout{\varphi_k}}{\dots}{\xout{\varphi_l}} \dots
%%}%end normalprod
%%%}%end of nomathglue
\nonumber\\
&+&
\sum_{k<l} \sum_{m<n} \Delta_F(x_k - x_l) \Delta_F(x_m - x_n)
\normalprod{\dots \xout{\varphi_k} \dots \xout{\varphi_m} \dots \xout{\varphi_l} \dots
\xout{\varphi_n} \dots}
\nonumber\\
&+& \cdots
\label{eqn:WickTheorem}
\end{eqnarray}

