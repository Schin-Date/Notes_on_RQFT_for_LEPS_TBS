\noindent
These fields describe massive, charged, spin 1/2 particles.\\
((Put the following line somewhere: Spinors belong to the fundamental representation of the Poincar\'e Group SL(2,C))).
\subsection{Classical Free Field}
%\noindent
%{\bf Dirac equation}

\subsubsection{Dirac equation}

The Dirac equation is written as
\begin{eqnarray}
i \partial_t \psi (x) = 
\left(
\frac{1}{i} \bld{\alpha} \cdot \bld{\partial} + \beta m \right) \psi(x) \,.
\label{eqn:TheDiracEq}
\end{eqnarray}
Solution $\psi(x)$ of this equation satisfies the Klein-Gordon equation
\footnote{%------------------------------- footnote >>
The Dirac equation is not the unique equation that
is the  first order in time derivative and whose solution satisfies the K-G eq.
The Duffin-Kemmer equation  also satisfies these requirements
\cite{ref:Itzykson-Zuber}.
 } %------------------------------- footnote //
if and only if coefficients $\bld{\alpha}$ and $\beta$ satisfy
\begin{eqnarray}
\{ \alpha_i, \alpha_j \}_+ = 2\delta_{ij}\,,
\hspace{3mm}
\{ \alpha_i, \beta \}_+ = 0\,,
\hspace{3mm}
\beta^2 = 1\,,
\label{eqn:Dialphbetacond}
\end{eqnarray}
as one may confirm it by taking time derivative of the both hand sides of Eq. (\ref{eqn:TheDiracEq})
and use the equation again in the $r.h.s.$
It follows from Eq. (\ref{eqn:Dialphbetacond}) that they must be traceless:
\begin{eqnarray}
\tr \alpha_i &=& \tr \beta^2 \alpha_i = \tr \beta \alpha_i \beta = - \tr \alpha_i = 0 \,,
\nonumber\\
\tr \beta &=& \tr \alpha_i^2 \beta  = \tr \alpha_i \beta \alpha_i  = - \tr \beta = 0 \,.
\nonumber
\end{eqnarray}
The minimum rank of $\alpha$'s and $\beta$ as matrices to satisfy the conditions 
(\ref{eqn:Dialphbetacond}) is found to be 4. 
Introduce $\gamma$ matrices by
\begin{eqnarray}
\gamma^0 = \beta\,,
\hspace{3mm}
\gamma^i = \beta \alpha_i\ \,.
\end{eqnarray}
In terms of these matrices, Eq. (\ref{eqn:Dialphbetacond}) reads
\begin{equation}
\{ \gamma^\mu, \gamma^\nu \}_+ = 2 g^{\mu \nu}\,.
\end{equation}
This is called the Clifford algebra. This relationship means $(\gamma^0)^2 = 1$ and $(\gamma^i)^2 = -1$ for $i = 1,2,3$.
$\gamma^0$ is obviously traceless and $\gamma^i$'s are also for
\begin{eqnarray}
\tr \gamma^i = \tr \beta \alpha_i = \tr \alpha_i \beta = -\tr \beta \alpha_i = 0\,.
\end{eqnarray}
The Dirac equation is then written as
\begin{eqnarray}
\left( i\slashed{\partial} - m \right) \psi(x) = 0\,,
\label{eqn:DiracEqstandard}
\end{eqnarray}
where $\slashed{a} \leftdef \gamma_\mu a^\mu = \gamma \cdot a$ and
$\slashed{\partial} = \gamma^0 \partial_0 + \gamma^i \partial_i = \gamma^0 \partial_0 + \bld{\gamma}\cdot \bld{\partial}$.

We may choose $\alpha_i$'s and $\beta$ as hermitian matrices.
In that case we have
\begin{eqnarray}
\gamma^{0\dagger} = \gamma^0\,,
\hspace{3mm}
\gamma^{i\dagger} = - \gamma^i
\hspace{3mm}
\Longleftrightarrow
\hspace{3mm}
{\gamma^\mu}^\dagger = \gamma^0 \gamma^\mu \gamma^0 \,.
\label{eqn:hermitealphabeta}
\end{eqnarray}
Define the conjugate Dirac field by
\begin{equation}
\overline{\psi} = \psi^{*t} \gamma^0\,
\label{eqn:DefConjDiracField}
\end{equation}
where the superscript $t$ denotes transpose
\footnote{%------------------------------- footnote >>
We keep the dagger field notation
for the hermite conjugate of quntized fields.
}%------------------------------- footnote >>
.
Taking complex conjugate and transpose of Eq. (\ref{eqn:DiracEqstandard}), we obtain
\begin{eqnarray}
0 &=&  
\psi^{*t}(x) 
\left( -i \stackrel{\leftarrow}{\partial} \cdot \gamma^\dagger - m \right) 
\nonumber\\
&=&
\psi^{*t}(x) (\gamma^0)^2
\left( -i \stackrel{\leftarrow}{\partial} \cdot \gamma^\dagger - m \right) 
\nonumber\\
&=&
\overline{\psi}(x)
\left( -i \stackrel{\leftarrow}{\partial} \cdot \gamma^0\gamma^\dagger \gamma^0 - m \right) \gamma^0
\nonumber\\
&=&
- \overline{\psi}(x)
\left( i \stackrel{\leftarrow}{\partial} \cdot \gamma  + m \right) \gamma^0 \,.
\nonumber
\end{eqnarray}
Thus, the conjugate Dirac field satisfies
\begin{eqnarray}
\overline{\psi}(x)
\left( i \stackrel{\leftarrow}{\slashed{\partial}}  + m \right)
= 0 \,.
\label{eqn:DiracEqBarstandard}
\end{eqnarray}
This equation is equivalent with Eq. (\ref{eqn:DiracEqstandard}) in the classical level
\footnote{%------------------------------- footnote >>
In the quantum level, this equation should be satisfied by the Heisenberg operator 
$\overline{\psi}(x)$ which is independent of the field ${\psi}(x)$.
}%------------------------------- footnote \\
provided we choose $\gamma$'s in such a way that Eq. (\ref{eqn:hermitealphabeta}) holds.

\bigskip
%------------------------------------------------------------------------------------------------------------------------------------------------
%\noindent
%{\bf Lagrangian density}
\subsubsection{Lagrangian formalism}

Lagrangian density that leads Eqs. (\ref{eqn:DiracEqstandard}) and (\ref{eqn:DiracEqBarstandard})
is written as
\begin{eqnarray}
{\cal L}_{Dirac}
=
\overline{\psi}(x)
\left(i \slashed{\partial} - m \right)
\psi(x) \,.
\label{eqn:DiracLagrangean}
\end{eqnarray}
Canonical momentum conjugate to the field $\psi$ is given as
\begin{eqnarray}
\pi_{\psi}(x) = \frac{\partial {\cal L}_{Dirac}}{\partial \dot{\psi}}
=
i \psi^{*t} (x) 
\end{eqnarray}
and Hamiltonian density yields
\begin{eqnarray}
{\cal H}_{Dirac}
&=&
i \psi^{*t}  \dot {\psi} - {\cal L}_{Dirac}
= 
i \overline{\psi}(x) \gamma^0 \dot {\psi}
- 
\overline{\psi}(x)
\left(i \slashed{\partial} - m \right)
\psi(x)
\nonumber\\
&=&
\overline{\psi}(x)
\left(
 \bld{ \frac{1}{i}\gamma}\cdot\bld{\partial} + m 
 \right)
 \psi(x)\,,
 \\
 &=&
\psi^{*t}(x)
\left(
 \bld{ \frac{1}{i}\alpha}\cdot\bld{\partial} + \beta m 
 \right)
 \psi(x)\,.
\end{eqnarray}
One may feel uneasy for Eq. (\ref{eqn:DiracLagrangean})
does not involve kinetic term of $\overline{\psi}$ field and
conjugate momentum $\pi_{\overline{\psi}}$ of this field disappears. It turns out
that this does not cause any serious problem. In practice,
Eq. (\ref{eqn:DiracLagrangean}) may be rewritten by means
of a surface integration in the action as \cite{ref:Itzykson-Zuber}
\begin{eqnarray}
{\cal L}_{Dirac}
=
\overline{\psi}(x)
\left(
\frac{i}{2} \slashed{\partial}
- \frac{i}{2}  \stackrel{\leftarrow}{\slashed{\partial}}
 - m \right)
\psi(x) \,,
\label{eqn:DiracLagrangeanIZ}
\end{eqnarray}
which yields $\pi_{\overline{\psi}}$ ($= -i \gamma^0 \psi / 2$)
that is certainly not identically zero.

\bigskip
%------------------------------------------------------------------------------------------------------------------------------------------------
%\noindent
%{\bf Lorentz transformation property}
\subsubsection{Lorentz transformation property}

Requiring ${\cal L}_{Dirac}$ to be invariant under proper Lorentz transformations,
one may find the transformation property of the field $\psi$.
It turns out that $\psi$ transforms as a Lorentz spinor, which 
belongs to the fundamental representation of the Lorentz group.
Leaving details to Appendix \ref{sec:AppDirac_LorentzTransf},
we merely mention here that, under a (proper homogeneous) Lorentz transformation 
\begin{equation}
x^\mu \mapsto x^{\mu'} = L^\mu_{\;\;\nu}\; x^\nu\,,
\end{equation}
the four component filed $\psi(x)$ is transformed linearly 
with a matrix $S(L)$ as
\begin{eqnarray}
\psi(x)
\mapsto
\psi'(x') 
= 
S(L) \psi(x)\,.
\end{eqnarray}
It turns out that the conjugate field $\overline{\psi}$
is transformed as
\begin{eqnarray}
\overline{\psi}(x)
\mapsto
\overline{\psi}'(x')
= 
\overline{\psi}(x) S(L)^{-1}\,,
\end{eqnarray}
so that a bilinear quantity $\overline{\psi} \psi$ is a Lorentz scalar.
It is also shown in Appendix \ref{sec:AppDirac_LorentzTransf} that
\begin{eqnarray}
S^{-1}(L) \gamma^\mu S(L) = L^\mu_{\;\;\nu} \gamma^\nu \,,
\label{eqn:LorentzTransfgamma}
\end{eqnarray}
so that a quantity $\overline{\psi} \gamma^\mu \psi$ is a Lorentz vector.
It is straightforward with these properties to confirm the Lorentz invariance of ${\cal L}_{Dirac}$.


%\bigskip
%------------------------------------------------------------------------------------------------------------------------------------------------
%\noindent
\subsubsection{Dirac matrices}
%{\bf Dirac matrices}

Since one needs calculus of $\gamma$ matrices to deal with Dirac spinors,
we summarize a few basic points leaving further details to Appendix \ref{sec:App_Dirac}.
First, we define
\begin{eqnarray}
\gamma_5 \equiv \gamma^5 \leftdef
i \gamma^0 \gamma^1 \gamma^2 \gamma^3
=
- \frac{i}{4!}
\epsilon_{\mu \nu \rho \sigma}
\gamma^\mu \gamma^\nu \gamma^\rho \gamma^\sigma
\end{eqnarray}
The $\gamma_5$ satisfies
\begin{eqnarray}
\{
\gamma_5, \gamma_\mu 
\}_+ = 0\,,
\hspace{3mm}
(\gamma_5)^2 = 1\,,
\hspace{3mm}
\gamma_5^\dagger = \gamma_5
\end{eqnarray}
We also define
\begin{eqnarray}
\sigma_{\mu \nu} \equiv
\frac{i}{2}
[ \gamma_\mu, \gamma_\nu]\,,
\end{eqnarray}
which has a property $\sigma_{\mu \nu}^\dagger = \gamma^0 \sigma_{\mu \nu} \gamma^0$.
The reason why we are defining these quantities lays in the follwing fact:
When $d$ is the rank of an irreducible matrix representation of an algebra composed of hypercomplex numbers,
the number of linearly independent elements of the algebra is $n = d^2$. Therefore, the minimal algebra composed
of $\gamma$'s has $4^2 = 16$ linearly independent elements. They are given as
\begin{eqnarray}
\left\{
\begin{array}{l}
\Gamma^S
=
1
\\
\Gamma^V_\mu
=
\gamma_\mu
\\
\Gamma^T_{\mu \nu}
=
\sigma_{\mu \nu} \leftdef
\frac{i}{2}
[ \gamma_\mu, \gamma_\nu]
\\
\Gamma^A_{\mu}
=
\gamma_5 \gamma_\mu
\\
\Gamma^P
=
\gamma_5
\end{array}
\right.
\end{eqnarray}

For particular representations of  $\alpha_i$'s, $\beta$ and $\gamma$'s as
explicit numeric matrices, we refer again to Appendix \ref{sec:App_Dirac}
and confine ourselves to mention that all such representations are based upon
a notation of the Pauli matrices written as
\footnote{%------------------------------- footnote >>
It was already given in Eq. (\ref{eqn:SU2GenFundamental}).
}%------------------------------- footnote //
\begin{eqnarray}
\sigma_1 =
\left(
\begin{array}{cc}
0 & 1 \\ 1 & 0
\end{array}
\right)\,,
\hspace{3mm}
\sigma_2 =
\left(
\begin{array}{cc}
0 & -i \\ i & 0
\end{array}
\right)\,,
\hspace{3mm}
\sigma_3 =
\left(
\begin{array}{cc}
1 &  0\\ 0 & -1
\end{array}
\right)\,,
\end{eqnarray}
which satisfy the following relationships
\begin{eqnarray}
\{
\sigma_i, \sigma_j \}_+ = 2 \delta_{ij}\,,
\hspace{3mm}
\sigma_i \sigma_j = i\epsilon_{ijk} \sigma_k + \delta_{ij}
\end{eqnarray}
%----------------------------------------------------------------------------

%------------------------------------------------------------------------------------------------------------------------------------------------
\subsubsection{Dirac spinors}
%\bigskip
%\noindent
%{\bf Dirac spinors}

We write solutions of the Dirac equation (\ref{eqn:DiracEqstandard}) as
\begin{eqnarray}
\psi(x) = \int \frac{d^3 \bld{p}}{\sqrt{(2\pi)^3} 2 p^0}
 \left[
\psi^{(+)}(\bld{p}) e^{-ipx} + \psi^{(-)}(\bld{p}) e^{ipx}
\right]\,,
\label{eqn:Diracfieldbypsipm}
\end{eqnarray}
where $p^0 = \sqrt{\bld{p}^2 + m^2}$.
The Dirac equation demands
\footnote{%------------------------------- footnote >>
%------------------------------------------------
As  a solution of the K-G eq., we can write
\begin{eqnarray*}
\psi(x) &=& \int \frac{d^4 p}{\sqrt{(2\pi)^3}}
\delta(p^2 - m^2) 
 \left[
 \theta(p^0) +  \theta(-p^0) \right]
 \psi(p) e^{-ipx}
\\
&=&
\int \frac{d^4 p}{\sqrt{(2\pi)^3}}
\delta(p^2 - m^2)  
{\large \mbox{[}}
\underbrace{\psi(p)\theta(p^0)}_{\psi^{+}}  e^{-ipx} + 
\underbrace{\psi(-p)\theta(p^0)}_{\psi^{-}}  e^{ipx}
{\large \mbox{]}}
\\
&=&
\int \frac{d^3 \bld{p}}{\sqrt{(2\pi)^3}2p^0}
\left[
\psi^{+}(\bld{p})  e^{-ipx} + \psi^{-}(\bld{p})  e^{ipx}
\right]
\end{eqnarray*}
The Dirac eq. (\ref{eqn:DiracEqstandard}) requires
\begin{eqnarray*}
\left. (\slashed{p} - m) \psi(p) \right|_{p^2 = m^2} = 0\,,
\end{eqnarray*}
from which Eq. (\ref{eqn:DiracEqMomentumamp}) follows.
}%---------------------------------- end of footnote //
\begin{eqnarray}
(\slashed{p} - m) \psi^{(+)}(\bld{p}) = 0\,,
\hspace{5mm}
(\slashed{p} + m) \psi^{(-)}(\bld{p}) = 0 \,.
\label{eqn:DiracEqMomentumamp}
\end{eqnarray}
Setting $p = (m, \bld{0})$, we have 
\begin{eqnarray}
(\gamma^0 - 1) \psi^{(+)}(\bld{0}) = 0\,,
\hspace{5mm}
(\gamma^0 + 1) \psi^{(-)}(\bld{0}) = 0 \,.
\label{eqn:DiracEqMomentumampatzeromomentum}
\end{eqnarray}
Solutions of Eqs. (\ref{eqn:DiracEqMomentumamp}) and
(\ref{eqn:DiracEqMomentumampatzeromomentum})
are related by a Lorentz transformation $L$
connecting Lorentz frames where
spacial momenta of a particle with the mass $m$ are
$\bld{p}$ and $\bld{0}$, respectively,
as
\begin{eqnarray}
\psi^{(\pm)}(\bld{p}) = S(L) \psi^{(\pm)}(\bld{0}) \,.
\end{eqnarray}
Considering that $\gamma_0^2 =1$, $\tr \gamma_0 = 0$ and rank$(\gamma_0) = 4$,
we recognize that each of equations in Eq. (\ref{eqn:DiracEqMomentumampatzeromomentum})
has  two  linearly independent solutions.
By acting $S(L)$ on 
these solutions, we obtain linearly independent solutions of Eq. (\ref{eqn:DiracEqMomentumamp})
which we write $u^{(1, 2)}(\bld{p})$ for $\psi^{(+)}(\bld{p})$ and $v^{(1,2)}(\bld{p})$
 for $\psi^{(-)}(\bld{p})$, respectively. 
We may choose them as to be orthonormal basis in a sense that
\footnote{%------------------------------------------------- footnote >>
This footnote should be rewritten later.\\
Choices of normalizations by different authors:\\
Izykson: $\{ b(\bld{p}),b^\dagger(\bld{p}') \}_+ = \frac{2E}{2m}\delta^3(\bld{p}-\bld{p}')$,
$\bar{u}^\alpha u^\beta = \delta^{\alpha \beta}$,\\
\hspace{20mm}$\psi(x) \sim \int d^3\bld{p}\frac{2m}{2E}[ b u e^{-ipx} \dots]$\\
Hioki, Tong: $\{ b(\bld{p}),b^\dagger(\bld{p}') \}_+ = (2\pi)^3 2E\delta^3(\bld{p}-\bld{p}')$,
$\bar{u}^\alpha u^\beta = 2m \delta^{\alpha \beta}$,\\
\hspace{20mm}$\psi(x) = \int \frac{d^3\bld{p}}{(2\pi)^32E}[ b u e^{-ipx} \dots]$\\
dim $\psi = E^{3/2}$
}%-------------------------------------------end of footnote //
\begin{eqnarray}
\begin{array}{c}
\overline{u}^{(r)}(\bld{p})
u^{(s)}(\bld{p})
= 2m \delta^{rs}\,,
\hspace{5mm}
\overline{v}^{(r)}(\bld{p})
v^{(s)}(\bld{p})
= -2m \delta^{rs}\,,
\vspace{2mm}
\\
\overline{u}^{(r)}(\bld{p})
v^{(s)}(\bld{p})
= 0\,,
\hspace{5mm}
\overline{v}^{(r)}(\bld{p})
u^{(s)}(\bld{p})
= 0 \,,
\end{array}
\label{eqn:DiracspinorNormalization}
\end{eqnarray}
where $r$ and $s$ take values $1, 2$ and
$\overline{u}^{(r)}$  and $\overline{v}^{(r)}$ 
are conjugate fields defined in a same manner as Eq. (\ref{eqn:DefConjDiracField}).
To be indubitable, we write Dirac equations satisfied by these basis below.
\begin{eqnarray}
\begin{array}{l}
(\slashed{p} - m) u^{(r)}(\bld{p}) = 0\,,
\hspace{5mm}
(\slashed{p} + m) v^{(r)}(\bld{p}) = 0\,,
\vspace{3mm}
\\
\overline{u}^{(r)}(\bld{p}) (\slashed{p} - m) = 0\,,
\hspace{5mm}
\overline{v}^{(r)}(\bld{p}) (\slashed{p} + m) = 0\,.
\end{array}
\label{eqn:DiracEqDiracspinors}
\end{eqnarray}
They also satisfy identities
\begin{eqnarray}
\begin{array}{l}
\displaystyle
\sum_{r=1}^2 u^{(r)}(\bld{p}) 
\overline{u}^{(r)}(\bld{p})
=
\slashed{p} + m\,,
\vspace{2mm}
%\\
%\hspace{-40mm}\mbox{and}
\\
\displaystyle
\sum_{r=1}^2 v^{(r)}(\bld{p}) 
\overline{v}^{(r)}(\bld{p})
=
\slashed{p} - m\,.
\end{array}
\label{eqn:DiracspinorPolSum}
\end{eqnarray}
The physical contents of each of these basis 
in particular representations of Dirac matrices
are examined by
constructing energy and spin state projections.
We refer to Appendix \ref{sec:App_Dirac} for detailes.

Finally, we may write Eq. (\ref{eqn:Diracfieldbypsipm})  in a form with these basis as
\begin{eqnarray}
\psi(x) = \int \frac{d^3 \bld{p}}{\sqrt{(2\pi)^3} 2 p^0}
\sum_{r = 1, 2} \left[
c_r(\bld{p}) u^{(r)}(\bld{p}) e^{-ip\cdot x} + d_r^*(\bld{p}) v^{(r)}(\bld{p}) e^{ip\cdot x}
\right] \,.
\end{eqnarray}

%\newpage 
%======================================================================
\subsection{Quantized Free Field}
\begin{equation}
\psi(x) 
=
\int \frac{d^3 \bld{p}}{\sqrt{(2\pi)^3} 2 p^0}
\sum_{r = 1,2}
\left[
c_r(\bld{p}) u^{(r)}(\bld{p}) e^{-ip \cdot x}
+
d_r^\dagger(\bld{p}) v^{(r)}(\bld{p}) e^{ip \cdot x}
\right]
\end{equation}

%------------------------------------------------------------------------------------------------------------------------------------------------
\subsubsection{Canonical quantization}

\begin{equation}
\{ c_{r}(\bld{p}), c_{r'}^\dagger(\bld{p}') \}
=
\{ d_{r}(\bld{p}), d_{r'}^\dagger(\bld{p}') \}
=
\delta_{rr'} 2p^0 
\delta^3(\bld{p} - \bld{p}')\,,
\label{eqn:Dirc_creann}
\end{equation}
and other anticommutators are equal to zero.

%------------------------------------------------------------------------------------------------------------------------------------------------
\subsubsection{Particle states and statistics}

%------------------------------------------------------------------------------------------------------------------------------------------------
\subsubsection{Discrete symmetries}

%------------------------------------------------------------------------------------------------------------------------------------------------
\subsubsection{Propagetor}
We write 
%Eq. (\ref{eqn:Dirc_creann})
\begin{eqnarray}
\psi_{\alpha}(x) = c_{\alpha}(x) + d_{\alpha}^\dagger(x)\,.
\hspace{5mm}
\overline{\psi}_{\alpha}(x) = \overline{c}_{\alpha}(x) + \overline{d_{\alpha}^\dagger}(x)\,,
\end{eqnarray}
%-----------------------------
VEV of quadratic expressions.
\begin{eqnarray}
\begin{array}{l}
\bra 0 \braend \psi(x) \psi(y) \ketend 0 \ket
=
\bra 0 \braend c(x) d^\dagger(y) \ketend 0 \ket
= 0 \,,
%---------------------------------------------------------
\vspace{2mm}
\\
\bra 0 \braend \overline{\psi}(x) \overline{\psi}(y) \ketend 0 \ket
=
\bra 0 \braend \overline{d^\dagger}(x) \overline{c}(y) \ketend 0 \ket
= 0 \,.
\end{array}
\end{eqnarray}
%---------------------------------------------------------

\begin{eqnarray}
%\begin{array}{l}
&&
\vspace{2mm}
%\\
 \bra 0 \braend \psi(x) \overline{\psi}(y) \ketend 0 \ket 
=
\bra 0 \braend c(x) \overline{c}(y) \ketend 0 \ket =
\nonumber\\
&&
%\displaystyle
\hspace{5mm} 
%= 
 \int \frac{d^3 \bld{p}}{(2\pi)^3 2 p^0} \frac{d^3 \bld{p}'}{2 {p^0}'}
\sum_{r, r'} 
\{ c_r(\bld{p}), c^\dagger_{r'}(\bld{p}) \}
u^{(r)}(\bld{p}) \overline{u}^{(r')}(\bld{p}') e^{-ip \cdot x + ip' \cdot y} =
\nonumber\\
&&
%\displaystyle
\hspace{9mm} 
 \int \frac{d^3 \bld{p}}{(2\pi)^3 2 p^0} 
\sum_{r} 
u^{(r)}(\bld{p}) \overline{u}^{(r)}(\bld{p}) e^{-ip \cdot ( x - y)} =
\nonumber\\
&&
%\displaystyle
\hspace{13mm} 
 \int \frac{d^3 \bld{p}}{(2\pi)^3 2 p^0} 
(\slashed{p} + m)  e^{-ip \cdot ( x - y)} =
(i \slashed{\partial} + m) D(x - y)
%\end{array}
\end{eqnarray}
%---------------------------------------------------------
Spinor suffices should be specified to avoid confusions in the following one
\footnote{%------------------------------- footnote >>
Suppressing these suffices, the result may be written as
\begin{eqnarray}
\bra 0 \braend \overline{\psi}(x)  \psi(y) \ketend 0 \ket
=
(i \slashed{\partial}^T - m) D(x - y)\,,
\end{eqnarray}
of which the $l.h.s.$ may lead a confusion.
We may use this expression though when there is no possibility of confusions.
}%------------------------------- footnote //
;
\begin{eqnarray}
%\begin{array}{l}
%\vspace{2mm}
%\\
&&\bra 0 \braend \overline{\psi}_\alpha(x)  \psi_\beta(y) \ketend 0 \ket
=
\bra 0 \braend \overline{d^\dagger}_\alpha(x)  d_\beta^\dagger(y) \ketend 0 \ket =
\nonumber\\
&&
%\displaystyle
\hspace{5mm} 
%= 
\int \frac{d^3 \bld{p}}{(2\pi)^3 2 p^0} \frac{d^3 \bld{p}'}{2 {p^0}'}
\sum_{r, r'} 
\{ d_r(\bld{p}), d^\dagger_{r'}(\bld{p}) \}
\overline{v}_\alpha^{(r)}(\bld{p}) v_\beta^{(r')}(\bld{p}')
 e^{-ip \cdot x + ip' \cdot y} =
\nonumber\\
&&
%\displaystyle
\hspace{9mm} 
\int \frac{d^3 \bld{p}}{(2\pi)^3 2 p^0} 
\sum_{r} 
 v_\beta^{(r)}(\bld{p}) \overline{v}_\alpha^{(r)}(\bld{p})
e^{-ip \cdot ( x - y)} =
\nonumber\\
&&
%\displaystyle
\hspace{13mm} 
\int \frac{d^3 \bld{p}}{(2\pi)^3 2 p^0} 
(\slashed{p} - m)_{\beta \alpha}  e^{-ip \cdot ( x - y)} =
(i \slashed{\partial} - m)_{\beta \alpha} D(x - y)
%\end{array}
\end{eqnarray}
T-product of fermionic operators involves a minus sign 
when operators are exchanged:
\begin{eqnarray}
T[\psi_\alpha(x) \overline{\psi}_\beta(y)] =
\theta(x^0 > y^0) \psi_\alpha(x) \overline{\psi}_\beta(y)
-
\theta(y^0 > x^0)  \overline{\psi}_\beta(y) \psi_\alpha(x)
\label{eqn:TprodDiracFields}
\end{eqnarray}
The Feynman propagator of the Dirac field is given as
\footnote{%----------------------------------------------footnote >>
Our definition of $S_F$ is the same as one in Ref. \cite{ref:Tong}.
$S_F[$Ref.\cite{ref:Itzykson-Zuber}$] = -i S_F[ours]$,
$S_F[$Ref.\cite{ref:Mandl-Shaw}$] = -iS_F[ours]$.
}%----------------------------------------------footnote //
\begin{eqnarray}
S_F(x-y) &\leftdef&
\bra 0 \braend T[\psi(x) \bar{\psi}(y)] \ketend 0 \ket
\nonumber\\
&=&
\left[ \theta(x^0 - y^0) (i \slashed{\partial} + m) D(x-y)
-
\theta(y^0 - x^0) (-i \slashed{\partial} - m) D(y-x)
\right]
\nonumber\\
&=&
\int \frac{d^3 \bld{p}}{(2\pi)^3 2 p^0} 
\left[
\theta(x^0 - y^0) (\slashed{p} + m)  e^{-ip \cdot ( x - y)}  -
\theta(y^0 - x^0) (\slashed{p} - m) e^{ip \cdot ( x - y)} 
\right]  
\nonumber\\
&=&
i \int \frac{d^4 p}{(2\pi)^4} \frac{\slashed{p} + m}{2 p^0}
\left[
\frac{1}{p^0 - E_{\bld{p}} + i\epsilon}  +
\frac{1}{p^0 + E_{\bld{p}} - i\epsilon} 
\right]  e^{-ip \cdot ( x - y)} 
\nonumber\\
&=&
i \int \frac{d^4 p}{(2\pi)^4} \frac{\slashed{p} + m}{p^2 - m^2 + i\epsilon}
e^{-ip \cdot ( x - y)} 
%\nonumber\\
\;\rightdef\;
i \int \frac{d^4 p}{(2\pi)^4} \frac{ e^{-ip \cdot ( x - y)}  }{\slashed{p} - m + i\epsilon}\,,
\label{eqn:DefFeynmanPropDirac}
\end{eqnarray}
where the last expression is a convention regarding 
a relationship\\ 
$(\slashed{p}~+~m)(\slashed{p}~-~m)~=~p^2~-~m^2$.
Fourier transform of $S_F(x)$,
\begin{eqnarray}
\tilde{S}_F(q) 
=
\int d^4x e^{iqx} S_F(x)
=
\frac{i}{\slashed{q} - m + i\epsilon}\,
\label{eqn:DiracPropMomRep}
\end{eqnarray}
is frequently referred as $S_F(q)$ without a symbol to distinguish
from the original function
when there is no possibility of confusion. 

To find the field equation of which  $S_F(x)$ is a solution, let us
operate the Dirac operator to the T-product in Eq. (\ref{eqn:TprodDiracFields}).
Since the field $\psi(x)$ itself satisfies the Dirac equation (\ref{eqn:TheDiracEq}),
non vanishing contributions arise from differentiation of theta functions:
\begin{eqnarray}
(i \slashed{\partial} - m)_{\alpha \xi} T[\psi_\xi(x) \overline{\psi}_\beta(y)] 
&=&
i (\gamma^0)_{\alpha \xi} \delta(x^0 - y^0)
\left[
\psi_\xi(x) \overline{\psi}_\beta(y)
+
\overline{\psi}_\beta(y) \psi_\xi(x) 
\right]
\nonumber\\
&=&
\end{eqnarray}



%<<<<<<<<<<<<<<<<<<<<<<<<<<<<<<<<<<<<<<<<<< 



