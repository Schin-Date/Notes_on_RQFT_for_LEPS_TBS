\begin{comment}%================================================== comment <<
Provided with ${\cal H}_{int}$ in terms of quantized fields in the interaction picture, 
we write 
%we write Eq. (\ref{eqn:SmatrixPertSer}) as
\begin{eqnarray}
\bra f \braend S -1 \ketend i \ket
&=&
 -i \int d^4 x' \, \bra f \braend {\cal H}_{int}(x') \ketend i \ket
\nonumber\\
&&+
\frac{(-i)^2}{2}
\int d^4 x_1' d^4 x_2'
\;\bra f \braend T[ {\cal H}_{int}(x_1') {\cal H}_{int}(x_2')] \ketend i \ket
+ \cdots
\nonumber\\
\label{eqn:SmatrixPertSerCov}
\end{eqnarray}
To proceed beyond the second term requires some further preparations
that we describe in the following.
\end{comment}
%================================================== comment //
\subsection{Prescription for time ordered products}
Since the second and higher order terms in Eq. (\ref{eqn:SmatrixPertSer})
are written in terms of time ordered products of $H_I$'s and
each $H_I$ is given in terms of a normal product of fields like
one in the first line of Eq. (\ref{eqn:scYHIterms}), 
we need a way to deal with time ordered products 
of fields. A technique to do this is provided by Wick's theorem, 
which we have described in Appendix \ref{sec:App_Wick}.
With the Feynman propagator defined in Eq. (\ref{eqn:FeynmanPropDefSc}),
Wick's theorem (\ref{eqn:WickTheorem}) for the real scalar field reads
\begin{eqnarray}
T[\phi_1 \phi_2 \cdots ]
&=&
\normalprod{\phi_1 \phi_2 \cdots }
\nonumber\\
&+&
\sum_{i < j}  
\normalprod{
\acontraction[1ex]{\cdots}{\phi}{{}_i\cdots}{\phi}
\cdots \phi_i \cdots \phi_j \cdots
}%end normalprod
\nonumber\\
&+&
\sum_{i < j, k<l}  
\normalprod{
\acontraction[1ex]{\cdots}{\phi}{{}_i\cdots \phi_k \cdots}{\Phi}
\acontraction[2ex]{\cdots \phi_i \cdots }{\phi}{{}_k\cdots \Phi_j \cdots}{\phi}
\cdots \phi_i \cdots \phi_k \cdots  \phi_j \cdots \phi_l \cdots
}%end normalprod
\nonumber\\
&+&
\cdots  \mbox{ (all possible contracts)}
\nonumber\\
&=&
\normalprod{\phi_1 \phi_2 \cdots }
\nonumber\\
&+&
\sum_{i < j}  
\Delta_F(x_i - x_j)
\normalprod{
\cdots \xout{\phi_i} \cdots \xout{\phi_j} \cdots
}%end normalprod
\nonumber\\
&+&
\sum_{i < j, k<l}  
\Delta_F(x_i - x_j)
\Delta_F(x_k - x_l)
%\hspace{50mm}
\normalprod{
\cdots \xout{\phi_i} \cdots \xout{\phi_k} \cdots  \xout{\phi_j} \cdots \xout{\phi_l} \cdots
}%end normalprod
\nonumber\\
&+&
\cdots
\nonumber\\
\label{eqn:WickTheoremRealSc}
\end{eqnarray}
where $\phi_i$ stands for $\phi(x_i)$.
For the complex scalar field, we have a similar formula as the above but
contracts are taken only among $\varphi$ and $\varphi^\dagger$ since
ones among the same kind disappears as it can be read from Eq.  (\ref{eqn:CplxScProp}).
Let us examine an example in which normal products of complex scalar fields 
are involved inside a time ordered product:
\begin{eqnarray}
T[\normalprod{\varphi_1^\dagger \varphi_2} \normalprod{\varphi_3^\dagger \varphi_4}]
&=&
\normalprod{\varphi_1^\dagger \varphi_2 \varphi_3^\dagger \varphi_4}
+
\acontraction[1ex]{:\!}{\varphi}{_1^\dagger \varphi_2 \!:\, : \!\varphi_3^\dagger}{\varphi}
:\!\varphi_1^\dagger \varphi_2 \!:\, : \!\varphi_3^\dagger \varphi_4 \!:
+
\acontraction[1ex]{:\!\varphi_1^\dagger}{\varphi}{_2 \!:\, : \!}{\varphi}
:\!\varphi_1^\dagger \varphi_2 \!:\, : \!\varphi_3^\dagger \varphi_4 \!:
\nonumber\\
&& +
\acontraction[1ex]{:\!\varphi_1^\dagger}{\varphi}{_2 \!:\, : \!}{\varphi}
\acontraction[2ex]{:\!}{\varphi}{_1^\dagger \varphi_2 \!:\, : \!\varphi_3^\dagger}{\varphi}
:\!\varphi_1^\dagger \varphi_2 \!:\, : \!\varphi_3^\dagger \varphi_4 \!:
\nonumber\\
&=&
\normalprod{\varphi_1^\dagger \varphi_2 \varphi_3^\dagger \varphi_4}
+
\Delta_F(x_1 - x_4) \normalprod{\varphi_2 \varphi_3^\dagger}
+
\Delta_F(x_2 - x_3) \normalprod{\varphi_1^\dagger \varphi_4}
\nonumber\\
&& +
\Delta_F(x_1 - x_4) \Delta_F(x_2 - x_3)
\end{eqnarray}


