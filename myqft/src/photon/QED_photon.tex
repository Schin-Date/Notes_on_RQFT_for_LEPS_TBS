\subsection{Interacting Photon Field}
Let us consider QED, namely the photon field interacting with the massive Dirac field.
We write the Lagrangian density as
\begin{eqnarray}
{\cal L}_{\mbox{\scriptsize QED}}
 = 
{\cal L}_{{\scriptsize F^2}}
+ 
{\cal L}_{\mbox{\scriptsize Dirac}}
+ 
{\cal L}_{\mbox{\scriptsize int}}\,,
\label{eqn:QEDLagr.symbolic}
\end{eqnarray}
where
\begin{eqnarray}
{\cal L}_{{\scriptsize F^2}}
=
- \frac{1}{4}F^2\,,
\hspace{4mm}
{\cal L}_{\mbox{\scriptsize Dirac}}
= 
 \overline{\psi}(i \slashed{\partial} - m) \psi \,.
\end{eqnarray}
To determine ${\cal L}_{\mbox{\scriptsize int}}$,
we require that ${\cal L}_{\mbox{\scriptsize QED}}$
is invariant under the following gauge transformations:
%\begin{eqnarray}
\begin{gather}
\psi (x) \mapsto \psi' (x) = e^{-ie \lambda(x)} \psi(x)
\,,
\hspace{3mm}
\psi^\dagger  (x) \mapsto  \psi^{\dagger}{}' (x)
=
\psi^\dagger  (x) e^{ie \lambda(x)}\,,
\\
%\vspace{6mm}
A^\mu \mapsto A'{}^{\mu} = A^\mu + \partial^\mu \lambda(x)\,,
\end{gather}
%\end{eqnarray}
Due to the locality of the transformation, ${\cal L}_{\mbox{\scriptsize Dirac}}$ is
not invariant since,
\begin{eqnarray}
 \overline{\psi} \partial_\mu  \psi 
 \mapsto
 \overline{\psi}' \partial_\mu  \psi'
 =
 \overline{\psi} \left(\partial_\mu  - ie (\partial_\mu \lambda) \right)\psi
 \neq
  \overline{\psi} \left(\partial_\mu ) \right)\psi
\end{eqnarray}
We may make this differential bilinear form invariant by replacing 
the derivative by a covariant derivative, which is defined as
\begin{eqnarray}
D_\mu \equiv \partial_\mu + ie A_\mu\,.
\end{eqnarray}
We surely find
\begin{eqnarray*}
 \overline{\psi} D_\mu  \psi 
 \mapsto
 \overline{\psi}' D_\mu'  \psi'
 =
 \overline{\psi} D_\mu  \psi \,,
\end{eqnarray*}
for
\begin{eqnarray}
D_\mu' = D_\mu + ie \partial_\mu \lambda(x) \,.
\end{eqnarray}
This substitution of a partial derivative by the covariant derivative
is called minimum substitution.
Applying the substitution in ${\cal L}_{\mbox{\scriptsize Dirac}}$,
we obtain
\begin{eqnarray}
 \overline{\psi}(i \slashed{D} - m) \psi 
= 
 \overline{\psi}(i \slashed{\partial} - m) \psi 
- e  \overline{\psi} \slashed{A} \psi \,.
\end{eqnarray}
Thus, in Eq. (\ref{eqn:QEDLagr.symbolic}),
\begin{eqnarray}
{\cal L}_{\mbox{\scriptsize int}}
= 
- e  \overline{\psi} \slashed{A} \psi
\label{eqn:LintQED}
\end{eqnarray}
gives a gauge invariant ${\cal L}_{\mbox{\scriptsize QED}}$
which we were seeking for.
This form of ${\cal L}_{\mbox{\scriptsize int}}$ is called
the minimal coupling.
Now we have an explicit form of the full Lagrangian density.
However, we already know it is not solvable because
of the lack of the photon propagator.
Therefore we introduce a gauge fixing term which
violates the gauge invariance but it will be recovered 
for all physically meaningful quantities.
We write our Lagrangian density as
\begin{eqnarray}
{\cal L}_{\mbox{\scriptsize QED, GF}}
 &=& 
{\cal L}_{{\scriptsize F^2}}
+ 
{\cal L}_{{\scriptsize GF}}
+ 
{\cal L}_{\mbox{\scriptsize Dirac}}
+ 
{\cal L}_{\mbox{\scriptsize int}}\,,
\nonumber\\
 &=& 
-\frac{1}{4} F^2
-\frac{1}{2\alpha} (\partial_\mu A^\mu)^2
+ \overline{\psi}(i \slashed{\partial} - m) \psi
- e \overline{\psi} \slashed{A} \psi \,.
\label{eqn:QEDLagr.w.GF}
\end{eqnarray}
Identifying ${\cal L}_{int}$ as in Eq. (\ref{eqn:LintQED}),
the photon and Dirac fields in the interaction picture obey 
free field equations for which Feynman propagators are
given by
Eq. (\ref{eqn:DiracPropagator}) and Eq. (\ref{eqn:photonPropalpha}),
respectively.


