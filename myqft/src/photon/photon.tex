\noindent
\subsection{Classical Theory}
Maxwell's equations,
a set of classical field equations of the electromagnetism,
in a vacuum, 
are written 
\footnote{%------------------------------- footnote >>
We use the Heaviside-Lorentz Gauss unit system of the electrodynamics and
write the light speed $c$ explicitly for a while.
}%------------------------------------------ footnote //
in a Lorentz covariant form as
\begin{eqnarray}
\partial_\nu F^{\nu \mu}(x) = \frac{1}{c} j^{\mu}(x)\,,
\label{eqn:inhomMxwF}
\\
\frac{1}{2}\epsilon^{\mu \rho \sigma \tau}
\partial_\rho F_{\sigma \tau}(x) = 0
\,,
\label{eqn:homMxwF}
\end{eqnarray}
where
$j^\mu(x) = (c \rho, \bld{j})$ is the four charge current density
and
$\epsilon^{\mu \rho \sigma \tau}$ is the Levi-Civita tensor in the 4 dimension.
%with a choice of signature $\epsilon^{0123}$ = 1.
The field strength tensor $F^{\mu \nu}$ is defined in terms of
the four electromagnetic potential
$A^\mu(x) = (\phi, \bld{A})$ as
\begin{eqnarray}
F^{\mu \nu} = \partial^{\mu} A^{\nu} - \partial^{\nu} A^{\mu}\,,
\label{eqn:FdefinedbyA}
\end{eqnarray}
which is apparently an antisymmetric tensor.
The field strength tensor is related with
the electric and magnetic fields
as
\begin{eqnarray}
\left\{
\begin{array}{l}
F^{\uparrow 0} = \bld{E}\,,
\vspace{2mm}
\\
F^{ij} =  \epsilon^{ijk} B_k
\hspace{2mm}
( \Leftrightarrow 
\hspace{1mm}\bld{B} = -\frac{1}{2}\epsilon^{\uparrow ij}F^{ij}  )
\,,
\end{array}
\right.
\label{eqn:FbyEB}
\end{eqnarray}
or explicitly in a matrix form as
\begin{equation}
F^{\mu\nu}
=
\left(
\begin{array}{cccc}
0 & -E^1& -E^2& -E^3 \\
E^1 &0&-B^3& B^2\\
E^2  &B^3&0&-B^1 \\
E^3 &-B^2&B^1& 0
\end{array}
\right)\,.
\label{eqn:FcontentbyEandB} 
\end{equation}
In the first equation in Eq. (\ref{eqn:FbyEB}), an upper arrow as a
superscript represents cartesian components of a spatial vector
in the right hand side.
\footnote{%------------------------------------------ footnote 
This notation is convenient in dealing with
calculus among spatial vectors.
}
With these correspondences, we may easily reproduce
the familiar form of Maxwell's equations in a vacuum:
\[
\left\{
\begin{array}{l}
\bld{\partial}\cdot \bld{E} =  \rho 
\hspace{30mm}\mbox{Coulomb}
\\
\bld{\partial} \times \bld{B}  - \partial_0 \bld{E} = \frac{1}{c}\bld{j}
\hspace{13.5mm}\mbox{Amp\`ere-Maxwell}
\\
\bld{\partial} \times \bld{E}  +  \partial_0 \bld{B} = \bld{0}
\hspace{15.5mm}\mbox{Faraday}
\\
\bld{\partial}\cdot \bld{B} = 0
\hspace{30mm}\mbox{Coulomb (magnetic)}
\\
\end{array}
\right.
\]

It follows from the definition (\ref{eqn:FdefinedbyA}) that
the set of
homogeneous Maxwell's equations (\ref{eqn:homMxwF})
is satisfied automatically and becomes an identity.
This identity  is called the Bianchi identity.
Now, problems of electromagnetism reduce themselves to solving inhomogeneous
equations (\ref{eqn:inhomMxwF}). Substituting Eq. (\ref{eqn:FdefinedbyA}) into
Eq. (\ref{eqn:inhomMxwF}), we have
\begin{eqnarray}
(\partial^2 g^{\mu \nu} - \partial^\mu \partial^\nu) A_\nu = j^\mu
%(\partial^2 g^\mu_\nu - \partial_\nu \partial^\mu) A^\nu = j^\mu
\label{eqn:MxwA}
\end{eqnarray}
Here and hereafter, we set $c = 1$ as usual.
A famous problem (and the sake) of the electromagnetism is that
the operator acting on $A_\nu$ in the left hand side
is not invertible.
\footnote{%------------------------------------------ footnote >>
It is equivalent to say Green's function satisfying 
\[(\partial^2 g^\mu_\nu
 - \partial_\nu \partial^\mu) G^\nu_\lambda (x-x')=
 g^\mu_\lambda \delta(x -x')
\]
does not exist. If it does, one get
$\partial_\lambda \delta(x - x') = \partial_\nu 
(\partial^2 g^\mu_\nu - \partial_\nu \partial^\mu) G^\nu_\lambda
= 0$,
which is mathematically contradicting.}%------------------------------------------ footnote //
This is due to a symmetry of Eq. (\ref{eqn:MxwA}) under the gauge transformation
of the second kind:
\begin{eqnarray}
A^\mu \mapsto A'{}^{\mu} = A^\mu + \partial^\mu \Lambda\,,
\end{eqnarray}
where $\Lambda$ is an arbitrary second order differentiable function.
\begin{comment}
\end{comment}
One can confirm that the physically observable fields $F^{\mu \nu}$ in Eq. (\ref{eqn:FdefinedbyA}) 
are certainly invariant under this transformation. 
This is why we are insisted (and allowed) to pick a representative of the field $A^\mu$
among those connected by gauge transformations
and, as a result, obtain two degrees of freedom corresponding to photon polarizations
from the  four of the Lorentz vector $A^\mu$.

The Lagrangian density which
 reproduces Maxwell's equations (\ref{eqn:MxwA}) may be written as
%for $j^\mu = 0$ is written as
\begin{eqnarray}
%{\cal L} = -\frac{1}{4}F_{\mu \nu} F^{\mu \nu}
{\cal L} = -\frac{1}{4}F_{\mu \nu} F^{\mu \nu} - j^\mu A_\mu
\,.
\label{eqn:LagdenMx}
\end{eqnarray}
The first term is manifestly gauge invariant.
The second term is not manifest but
the gauge invariance is guarantied in the level of action
through the current conservation law
%of the action is guarantied by the current conservation law 
$\partial \cdot j = 0$, which is a consistency condition of the Maxwell equation
(\ref{eqn:MxwA}).
To fix a gauge, we impose condition(s) on the field $A_\mu$ and
modify the first term of the Lagrangian density (\ref{eqn:LagdenMx})
accordingly. This modification is not accounted for as a change by
a total derivative, which does not change the field equation.
We need to modify the field equation itself to make it solvable.
In the classical theory, this can be achieved as described below.
In the quantum theory, however, 
an special care is required to set condition(s) on the field.
% should be treated carefully.

Before discussing particular choices of a gauge, 
we make notes on the Lagrangian density (\ref{eqn:LagdenMx}).
The first term has a few different expressions as follows:
\begin{eqnarray}
{\cal L}_{F^2} 
&\equiv&
 - \frac{1}{4} F^2
\nonumber\\
&=&
- \frac{1}{2} \partial_\mu A_\nu F^{\mu \nu}
=
- \frac{1}{2}
\left[
(\partial_\mu A_\nu)^2 
- \partial_\mu A_\nu \partial^\nu A^\mu
\right] \,,
\label{eqn:LF2expressbyA}
\\
&=&
\frac{1}{2}
( \bld{E}^2 - \bld{B}^2 ) \,,
\label{eqn:LF2expressbyEB}
\\
&=&
\frac{1}{2}
\left[
%A_\nu \partial_\mu F^{\mu \nu}
A_\mu 
(\partial^2 g^{\mu \nu} - \partial^\mu \partial^\nu) A_\nu
-
\partial_\mu (A_\nu F^{\mu \nu})
\right] \,.
\label{eqn:LF2expressMx}
\end{eqnarray}
The equivalence among these expressions
can be checked in a straightforward manner.
 \footnote{%------------------------------- footnote >>
It may help for doing so to write
\begin{eqnarray}
\bld{B}^2 = \frac{1}{2} (F^{ij})^2 = \partial_i A_j F^{ij} \,.
\end{eqnarray}
}%------------------------------- footnote //
In these expressions, $F^{\mu \nu}$ abbreviates its expression in terms of the field $A$.
Eq. (\ref{eqn:LF2expressbyEB})
shows through Eq. (\ref{eqn:FbyEB})
that ${\cal L}_{F^2}$
has no kinetic term proportional with $\dot{A}_0$ 
so that canonical conjugate
momontum $\Pi_0$ associated with $A_0$ is identically zero.
This may cause a problem when one apply canonical quantization method.
The canonical conjugate momentum $\Pi^{\uparrow}$ associated with $A_{\uparrow} = -\bld{A}$ is 
immediately obtained
from Eq. (\ref{eqn:LF2expressbyEB}) as
\begin{eqnarray}
\bld{\Pi} = 
- \frac{\partial {\cal L}}{\partial \dot{\bld{A}}}
=
\bld{E} \,.
\end{eqnarray}
Finally, Eq. (\ref{eqn:LF2expressMx}) shows explicitly that Maxwell's equations are
deduced from ${\cal L}_{F^2}$ by requiring action to be unchanged by
varying $A_\nu$.
\footnote{%------------------------------- footnote >>
In such a treatment, we adopt the variational principle considering ${\cal L}$ is 
a functional of the field $A$ solely and its derivatives are not independent variables.
We must take variation of both of $A$ in places before and after differential operators. 
For the form of Eq. (\ref{eqn:LF2expressMx}), we make use of partial integrations twice.
}%------------------------------- footnote //

Hamiltonian density is written in a variety of expressions as
\begin{eqnarray}
{\cal H}_{F^2} 
&=&
 \Pi^i \dot{A}_i - {\cal L}_{F^2}
\nonumber\\
&=&
(F^{i0})^2 - \partial^i A^0 F^{i0} - \frac{1}{2}(\bld{E}^2 - \bld{B}^2) \,,
\nonumber\\
&=&
\frac{1}{2} (\bld{E}^2 + \bld{B}^2) 
- A^0 \bld{\partial}\cdot \bld{E}
+ \partial_i
\left\{
A^0 E^i
\right\} \,,
\label{eqn:F2HamiltDensEB}
\\
&=&
-\frac{1}{2}
\left[
(\partial^i A^0)^2 - (\partial^0 A^i)^2
\right]
+
\frac{1}{2}
\partial_i A_j F^{ij} \,.
\label{eqn:F2HamiltDensA}
\end{eqnarray}

\bigskip
%------------------------------------------------------------------------------------- Lorentz
\noindent
\underline{Lorentz gauge}\\
The gauge condition is written as
\begin{eqnarray}
\partial_\mu A^\mu = 0 \,.
\label{eqn:LorentzCond}
\end{eqnarray}
Since
\begin{eqnarray}
\partial_\mu F^{\mu \nu} = \partial^2 A^\nu - \partial_\nu (\partial_\mu A^\mu)
\,,
\end{eqnarray}
the field equation (\ref{eqn:MxwA}) reduces to the massless Klein-Gordon equation
$\partial^2 A^\nu~=~j^\nu$ under this condition.
Making use of an expression Eq. (\ref{eqn:LF2expressbyA}), Lagrangian density under 
the Lorentz condition reads
\begin{eqnarray}
{\cal L}_{F^2}
&=&
- \frac{1}{2}
\left[
(\partial_\mu A_\nu)^2 + 
%\slashed
\cancel
{
A_\nu \partial^\nu 
\partial_\mu A^\mu
}
- \partial_\mu (A_\nu \partial^\nu A^\mu)
\right]
\nonumber\\
&=&
{\cal L}_{(\partial A)^2} + \mbox{total derivative}
\,,
\label{eqn:LagF2Lorentz}
\end{eqnarray}
where we have defined a new Lagrangian density
\begin{eqnarray}
{\cal L}_{(\partial A)^2}
=
- \frac{1}{2}
(\partial_\mu A_\nu)^2 \,,
\label{eqn:LagdenDA2}
\end{eqnarray}
that leads to the massless Klein-Gordon equation.
Appart from the explicit way to modify ${\cal L}$ as in the above, 
the Lorentz gauge condition can be imposed in terms of Lagrangian
multiplier $\eta$ as follows:
\begin{eqnarray}
{\cal L}_{Lor}
&=&
{\cal L}_{F^2}
-
\frac{\eta}{2} (\partial \cdot A)^2
\nonumber\\
&=&
\frac{1}{2}
\left[
A_\nu \partial_\mu F^{\mu \nu}
+
\eta A_\nu \partial^\nu \partial^\mu A_\mu
\right]
+ \mbox{total derivative} \,,
\nonumber\\
&=&
\frac{1}{2}
\left[
A_\nu \partial^2 A^\nu - (1 - \eta) A_\nu \partial^\mu \partial^\nu A_\mu
\right]
+ \mbox{total derivative} \,,
\label{eqn:LorentzLagmultip}
\end{eqnarray}
that induces
\begin{eqnarray}
0
&=&
\partial_\mu F^{\mu \nu}
+ \eta \partial^\nu \partial^\mu A_\mu
- j^\nu
\nonumber\\
&=&
\partial^2 A^\nu
- (1 - \eta)
\partial^\nu \partial^\mu A_\mu
- j^\nu
\label{eqn:EOMwitheta}
\end{eqnarray}
for the full Lagrangian density ${\cal L} = {\cal L}_{Lor} - j^\mu A_\mu$.
Euler-Lagrange equation for $\eta$ results in the Lorentz condition and
field equations of the system is composed of that and the massless Klein-Gordon equation.

With the Lorentz condition, the gauge is not completely fixed. We still have freedom of
making gauge transformation for an arbitrary function $\Lambda$ satisfying 
$\partial^2 \Lambda = 0$.

\bigskip
%------------------------------------------------------------------------------------- Coulomb
\noindent
\underline{Coulomb gauge}\\
\begin{eqnarray}
\bld{\partial} \cdot \bld{A} = 0
\label{eqn:CoulmbGaugeCond}
\end{eqnarray}
Inhomogeneous Maxwell's equations read
\begin{eqnarray}
%\left\{
%\begin{array}{l}
\partial_\mu F^{\mu 0} 
&=&
- \bld{\partial}^2 A^0 - \partial_0 
\cancel{
\bld{\partial}\cdot \bld{A}
}
= \rho
\label{eqn:CoulmbEOMrho}
\\
\partial_\mu F^{\mu \uparrow} 
&=&
\partial^2 \bld{A} + \bld{\partial}\partial_0 A^0 +
\bld{\partial}
\cancel{
 (\bld{\partial}\cdot \bld{A})
}
= \bld{j}
%\end{array}
%\right.
\label{eqn:CoulmbEOMj}
\end{eqnarray}
\begin{comment}
\begin{eqnarray}
\left\{
\begin{array}{l}
\partial_\mu F^{\mu 0} =
- \bld{\partial}^2 A^0 - \partial_0 
\cancel{
\bld{\partial}\cdot \bld{A}
}
= \rho
\\
\partial_\mu F^{\mu \uparrow} =
\partial^2 \bld{A} + \bld{\partial}\partial_0 A^0 +
\bld{\partial}
\cancel{
 (\bld{\partial}\cdot \bld{A})
}
= \bld{j}
\end{array}
\right.
\label{eqn:CoulmbEOMj}
\end{eqnarray}
\end{comment}
The solution of the first (nondynamical) equation for $A^0$ is given as
\begin{eqnarray}
A^0 (x) = \frac{1}{4\pi} \int d^3 \bld{x}' 
\frac{\rho(\bld{x}', t)}{| \bld{x} - \bld{x}' |}
\label{eqn:CoulmbA0}
\end{eqnarray}
The solution is unique under the boundary condition that $A^0$ and $\rho$ vanish at
spatial infinity.
Cancelling out terms with $\partial_0 A_0$ in the Lagrangian density ${\cal L}_{F^2}$,  
we read from Eq. (\ref{eqn:LF2expressbyA}) that
\begin{eqnarray}
{\cal L}_{F^2}
&=&
-\frac{1}{2}
\left[
(\partial_\mu A_0)^2
- (\partial_\mu  A_i)^2 
- \partial_0 A_\nu \partial^\nu A_0
- \partial_i A_\nu \partial^\nu A^i
\right]
\nonumber
\\
&=&
-\frac{1}{2}
\left[
-(\partial_i A_0)^2
- (\partial_\mu  A_i)^2 
- \partial_0 A^i \partial_i A_0
- \partial_i A_\nu \partial^\nu A^i
\right]
\nonumber
\\
&=&
-\frac{1}{2}
\left[
\left\{A_0 \bld{\partial}^2 A_0  - \partial_i (A_0 \partial_i A_0)\right\}
-\left\{ A_i \partial^2 A^i + \partial_\mu(A_i \partial^\mu A_i)\right\}
\right.
\nonumber\\
&&
\hspace{5mm}
\left.
+2\left\{
A^i \partial_i \partial_0 A_0 - \partial_0(A^i \partial_i A_0)
\right\}
+ \left\{
A^i \partial_i \partial_j A^j - \partial_i (A^j \partial_j A^i)
\right\}
\right]\,.
\nonumber
\label{eqn:CoulmbLinA}
\end{eqnarray}
where we have used abbreviations  $(a_i)^2 = a_i a_i = \bld{a}^2$ while
$(a_\mu)^2 = a_\mu a^\mu = a^2$.
The last expression in the above equation has a form easy to take 
variations in $A$. 
In the Hamiltonian density of a form equivalent to Eq. (\ref{eqn:F2HamiltDensA}),
\begin{eqnarray*}
{\cal H}_{F^2} 
&=&
\frac{1}{2}
(\partial^0 A^i)^2
+
\frac{1}{2}
\partial_i A_j F^{ij}
+
\frac{1}{2}
A^0 \bld{\partial}^2 A^0
-
\frac{1}{2}
\partial_i \left( A^0 \partial_i A^0 \right)\,,
\end{eqnarray*}
we may substitute the solution (\ref{eqn:CoulmbA0}) of 
non-dynamical variable $A^0$ in the third term to write
\begin{eqnarray}
{\cal H}_{F^2,\mbox{\scriptsize Coul}} 
&=&
\frac{1}{2}
(\partial^0 A^i)^2
+
\frac{1}{2}
\partial_i A_j F^{ij}
-
\frac{1}{2}
A^0 \rho
-
\frac{1}{2}
\partial_i \left( A^0 \partial_i A^0 \right) \,.
\nonumber
\end{eqnarray}
Adding the source term $j \cdot A = \rho A^0 - \bld{j} \cdot \bld{A}$, we obtain the
full Hamiltonian density in the Coulomb gauge as
\begin{eqnarray}
{\cal H} 
&=&
\frac{1}{2}
(\partial^0 A^i)^2
+
\frac{1}{2}
\partial_i A_j F^{ij}
+
{\cal H}_{\mbox{\scriptsize Coul}}
- \bld{j} \cdot \bld{A}
-
\frac{1}{2}
\partial_i \left( A^0 \partial_i A^0 \right)\,,
\label{eqn:HdensCoul}
\end{eqnarray}
where 
\begin{eqnarray}
{\cal H}_{\mbox{\scriptsize Coul}}
=
\frac{1}{2} A^0 \rho
=
\frac{1}{8\pi} \int
d^3 \bld{x}' 
\frac{\rho(\bld{x}, t) \rho(\bld{x}', t)}{|\bld{x} - \bld{x}'|}\,.
\end{eqnarray}
The last term in Eq. (\ref{eqn:HdensCoul}) is a total derivative
and will not contribute to the Hamiltonian.


The gauge condition (\ref{eqn:CoulmbGaugeCond}) is
equivalent with writing
\begin{eqnarray}
\bld{A} =
\left(
1
-
\frac{\bld{\partial} \bld{\partial} \cdot}
{\bld{\partial}^2}
\right)
\bld{A}\,.
\label{eqn:CoulmbProjection}
\end{eqnarray}
From the current conservation law 
%$\partial_0 j^0 + \bld{\partial}\cdot \bld{j}$ 
and Eq. (\ref{eqn:CoulmbEOMj}), we have
\begin{eqnarray}
\bld{\partial}\cdot \bld{j}
= -\partial_0 \rho
= \bld{\partial}^2 \partial_0 A^0
\end{eqnarray}
and
\begin{eqnarray}
\bld{j}_T
&\leftdef&
\left(
1 -
\frac{\bld{\partial} \bld{\partial} \cdot}
{\bld{\partial}^2}
\right)
\bld{j}
=
\bld{j} -
\bld{\partial} \partial_0 A^0
\end{eqnarray}
so that Eq. (\ref{eqn:CoulmbEOMj}) 
%(\ref{eqn:CoulmbEOMrho}) 
can be written
\footnote{%------------------------------- footnote >>
An expression of $\bld{j}_T$ and $\bld{j}_L = \bld{j} -\bld{j}_T$ in terms of integration is given in
\cite{ref:Jackson}.
We note that $A^0 = 0$ when $\rho = 0$ for our boundary condition at spatial infinity and
thus $\bld{j}_T = \bld{0}$ for the case of no source $j^\mu = 0$.
}%------------------------------- footnote \\
 as
\begin{eqnarray}
\partial^2 \bld{A}
=
\bld{j}_T \,.
\end{eqnarray}

When there is no source current, we have $A^0 = 0$ from Eq. (\ref{eqn:CoulmbA0}) and
as the dynamical field equation $\partial^2 \bld{A}  = \bld{0}$.
In this case the Coulomb gauge condition (\ref{eqn:CoulmbGaugeCond}) can be
realized as a further restriction $\bld{\partial}^2 \Lambda = \bld{\partial} \cdot \bld{A}$ for a
given $A$ in the Lorentz condition case.
Now $A^0$ and longitudinal component of $\bld{A}$ are dropped from dynamical degree of
freedom and we have 2 degree of freedom remaining. They will be identified as the two
polarization states of the photon.

%=====================================================================
\subsection{Quantization of Free Field}
%------------------------------------------------------------------------------------- Coulomb
\noindent
\underline{Coulomb gauge}\\

The dynamical field equation for free field is
\begin{eqnarray}
\partial^2 \bld{A} = \bld{0}\,,
\end{eqnarray}
for which we may write the solution as
\begin{eqnarray}
\bld{A}(x) = \int \frac{d^3 \bld{k}}{\sqrt{(2\pi)^3}2\omega} 
\left[
%\bld{A}^{(+)}(\bld{k}) e^{-i k x} + \bld{A}^{(-)}(\bld{k}) e^{i k x} 
\bld{a}^{}(\bld{k}) e^{-i k x} + \bld{a}^{\dagger}(\bld{k}) e^{i k x} 
\right] \,,
\label{eqn:CoulPhoAA}
\end{eqnarray}
where $\omega = k^0 = |\bld{k}|$ and $\bld{a}^{\dagger}$ is
complex (hermite) conjugate of $\bld{a}$ in the classical (quantum) level.
\footnote{%------------------------------- footnote >>>>>>>>>>>>>>>
Expressions below follow:
\begin{eqnarray}
\begin{array}{l}
\dot{\bld{A}} =
\int
\frac{d^3 \bld{k}}{\sqrt{(2\pi)^3} 2\omega}
(-i\omega)
\left[
\bld{a} e^{-ikx} - \bld{a}^\dagger e^{ikx}
\right]
\vspace{2mm}
\\
F^{ij} = 
\int
\frac{d^3 \bld{k}}{\sqrt{(2\pi)^3} 2\omega}
\left[
(-ik^i a^j + ik^j a^i )  e^{-ikx}
+
(ik^i a^{j \dagger} - ik^j a^{i \dagger} )  e^{ikx}
\right]
\vspace{2mm}
\\
\bld{B} = 
\int
\frac{d^3 \bld{k}}{\sqrt{(2\pi)^3} 2\omega}
i \bld{k} \times
\left[
\bld{a} e^{-ikx} - \bld{a}^\dagger e^{ikx}
\right]
\end{array}
\end{eqnarray}
}%------------------------------- footnote //
The gauge condition (\ref{eqn:CoulmbGaugeCond}) reads
\begin{eqnarray}
\bld{k} \cdot \bld{a}^{} (\bld{k}) = 0 \,.
\label{eqn:CoulmbGCondMom}
\end{eqnarray}
Considering there are two independent dynamical degrees of freedom, we introduce
polarization 3 vectors $\bld{e}^{(\lambda)}$, $\lambda = 1, 2$ which satisfy
the following three conditions:
\begin{eqnarray}
\begin{array}{l}
\bld{k} \cdot \bld{e}^{(\lambda)} = 0\,,
\vspace{1mm}
\\
{\bld{e}^{(\lambda)}}^* \cdot \bld{e}^{(\lambda')} = \delta^{\lambda \lambda'} \,,
\vspace{1mm}
\\
\sum_\lambda \bld{e}^{(\lambda)} {\bld{e}^{(\lambda)}}^* \cdot = 1 - \hat{\bld{k}} \hat{\bld{k}} \cdot \,.
\end{array}
\end{eqnarray}
where we wrote $\bld{k} / |\bld{k}| = \hat{\bld{k}}$.
Decomposing $\bld{a}$ with these basis, we write
\begin{eqnarray}
\bld{A}(x) = \int \frac{d^3 \bld{k}}{\sqrt{(2\pi)^3}2\omega} 
\sum_{\lambda = 1, 2}
\left[
a_\lambda (\bld{k}) \bld{e}^{(\lambda)} e^{-i k x} 
+ 
a_\lambda^{\dagger}(\bld{k}) {\bld{e}^{(\lambda)}}^* e^{i k x} 
\right] \,.
\label{eqn:CoulPhoAApol}
\end{eqnarray}
In a coordinate system where the wave vector $\bld{k}$ lays in the direction of the
3rd axix,
one may choose $\bld{e}^{(1)}$ and $\bld{e}^{(2)}$ as unit vectors in directions of
the 1st and 2nd axes, respectively. One may also choose two polarization vectors
$\bld{e}^{(\pm)} =  (1, \pm i , 0)/\sqrt{2}$  for right- and left- hanaded circular
polarizations.
\footnote{%------------------------------- footnote >>
\cite{ref:Lifshitz} takes
$\bld{e}^{(\pm)} = \mp i (1, \pm i , 0)/\sqrt{2}$.
}%------------------------------- footnote \\

On applying the canonical quantization method,
one needs to take constraints of the gauge condition on canonical
variables $\bld{A}$ and $\bld{\Pi} = \bld{E}$ into account.
Since $\bld{\partial}\cdot \bld{A} = \bld{\partial}\cdot \bld{E} = 0$ shold
hold, these variables are restricted in a space projected by
an operator in Eq. (\ref{eqn:CoulmbProjection}). Therefore we write
the equal-time commutation relation among them as
\begin{eqnarray}
[ A^i (\bld{x}, t), \Pi^j (\bld{y}, t) ]
&=&
\left(
\delta^{ij}
-
\frac{\partial^i \partial^j}
{\bld{\partial}^2}
\right)
i \delta^3 (\bld{x} - \bld{y})
\nonumber\\
&=&
i \int \frac{d^3 \bld{k}}{(2\pi)^3}
\left(
\delta^{ij}
-
\frac{k^i k^j}{\bld{k}^2}
\right)
e^{i \bld{k} \cdot (\bld{x} - \bld{y})}
\label{eqn:CoulmbCanonicalAPi}
\end{eqnarray}
One may check the consistency with the constraints by taking 
divergences of the both hand sides.
Other commutation relations are not altered since the right hand sides are
proportional to 0:
\begin{eqnarray}
[ A^i (\bld{x}, t), A^j (\bld{y}, t) ]
=
[ \Pi^i (\bld{x}, t), \Pi^j (\bld{y}, t) ]
=
0
\label{eqn:CoulmbCanonicalAA}
\end{eqnarray}
Substituting Eq. (\ref{eqn:CoulPhoAApol}) in an expression
$\bld{\Pi} = - \partial_0 \bld{A}$ and Eqs. (\ref{eqn:CoulmbCanonicalAPi}, \ref{eqn:CoulmbCanonicalAA}),
we find
\begin{eqnarray}
\begin{array}{l}
[a_\lambda(\bld{k}), a_{\lambda '}(\bld{k}')]
=
[a^\dagger_\lambda(\bld{k}), a^\dagger_{\lambda '}(\bld{k}')]
=0 \,,
\vspace{3mm}
\\
{[}a_\lambda (\bld{k}), a_{\lambda '} (\bld{k}'){]}
=
2 \omega \delta^3(\bld{k} - \bld{k}') \delta_{\lambda \lambda'} \,,
\end{array}
\end{eqnarray}
for $\lambda = 1, 2$.
Now we can interpret $a^\dagger_\lambda (\bld{k})$ and $a_\lambda (\bld{k})$
as creation and annihilation operators for dynamical photons of definite polarization represented by $\lambda$.

The Hamiltonian corresponding to Eq. (\ref{eqn:HdensCoul}) 
before taking normal product reads
\footnote{%------------------------------- footnote >>
Denoting $d^3 \tilde{\bld{k}} =  d^3\bld{k}/\{ \sqrt{(2\pi)^3} 2\omega\}$, 
$\bld{a}' = \bld{a}(\bld{k}')$ and $\omega' = {k'}^0 = |\bld{k}'|$,
we write
\begin{eqnarray*}
(\partial_0 A_i)^2
&=&
\int
d^3 \tilde{\bld{k}} d^3 \tilde{\bld{k}}'
(-i \omega)(-i \omega')
\left[
\bld{a} e^{-ikx} - \bld{a}^\dagger e^{ikx}
\right]
\cdot
\left[
\bld{a}' e^{-ik'x} - {\bld{a}'}^\dagger e^{ik'x}
\right]
\end{eqnarray*}
%--------------------------------------------------- inside a footnote 
\begin{eqnarray*}
\int d^3 \bld{x}
\bld{B}^2
&=&
\int
d^3 \tilde{\bld{k}} d^3 \tilde{\bld{k}}'
\int d^3 \bld{x}\,
i \bld{k} \times
\left[
\bld{a} e^{-ikx} - \bld{a}^\dagger e^{ikx}
\right]
\cdot
i \bld{k}' \times
\left[
\bld{a}' e^{-ik'x} - {\bld{a}'}^\dagger e^{ik'x}
\right]
\\
&=&
-\int
d^3 \tilde{\bld{k}} d^3 \tilde{\bld{k}}'
\int d^3 \bld{x}\,
\left[
\{
(\bld{k}\cdot \bld{k}')
(\bld{a}\cdot \bld{a}')
-
(\bld{k}'\cdot \bld{a})
(\bld{k}\cdot \bld{a}')
\}
e^{-i(k+ k')x}
\right.
\\
&&
\hspace{25mm}
+
\{
(\bld{k}\cdot \bld{k}')
(\bld{a}^\dagger\cdot {\bld{a}'}^\dagger)
-
(\bld{k}'\cdot \bld{a}^\dagger)
(\bld{k}\cdot {\bld{a}'}^\dagger)
\}
e^{i(k+ k')x}
\\&&
\hspace{25mm}
-
\{
(\bld{k}\cdot \bld{k}')
(\bld{a}\cdot {\bld{a}'}^\dagger)
-
(\bld{k}'\cdot \bld{a})
(\bld{k}\cdot {\bld{a}'}^\dagger)
\}
e^{-i(k- k')x}
\\
&&
\left.
\hspace{25mm}
-
\{
(\bld{k}\cdot \bld{k}')
(\bld{a}^\dagger\cdot {\bld{a}'})
-
(\bld{k}'\cdot \bld{a}^\dagger)
(\bld{k}\cdot {\bld{a}'})
\}
e^{i(k- k')x}
\right]
\end{eqnarray*}
Adding the above two reads
\begin{eqnarray*}
&&\int d^3 \bld{x} \left\{(\partial_0 A_i)^2 + \bld{B}^2\right\}
\\
&&
=
-\int
d^3 \tilde{\bld{k}} d^3 \tilde{\bld{k}}'
\int d^3 \bld{x}\,
\left[
\{
(\omega \omega' +
\bld{k}\cdot \bld{k}')
(\bld{a}\cdot\bld{a}')
-
(\bld{k}'\cdot \bld{a})
(\bld{k}\cdot \bld{a}')
\}
e^{-i(k+ k')x}
+ \cdots
\right.
\\
&&
\hspace{25mm}
\left.
-
\{
(\omega \omega' +
\bld{k}\cdot \bld{k}')
(\bld{a}\cdot {\bld{a}'}^\dagger)
-
(\bld{k}'\cdot \bld{a})
(\bld{k}\cdot {\bld{a}'}^\dagger)
\}
e^{-i(k- k')x}
+ \cdots
\right]
\\
&&
=
- \int
\frac{d^3 \bld{k}} 
{4 \omega^2}
\left[
\{
(\omega^2 
-
\bld{k}^2)
(\bld{a}(\bld{k})\cdot\bld{a}(-\bld{k}))
-
(\bld{k}\cdot \bld{a}(\bld{k}))
(-\bld{k}\cdot \bld{a}(-\bld{k}))
\}
e^{-2i\omega t}
+ \cdots
\right.
\\
&&
\hspace{25mm}
\left.
-
\{
(\omega^2 +
\bld{k}^2)
(\bld{a}\cdot {\bld{a}}^\dagger)
-
(\bld{k}\cdot \bld{a})
(\bld{k}\cdot {\bld{a}}^\dagger)
\}
+ \cdots
\right]
\\
&&
=
\int
\frac{d^3 \bld{k}}{2 \omega}
\omega
\left(
\bld{a}^\dagger \cdot \bld{a}
+
\bld{a} \cdot \bld{a}^\dagger
\right)
\end{eqnarray*}
This leads to Eq. (\ref{eqn:HamiltCoul}).
}%------------------------------- footnote //
\begin{eqnarray}
H 
&=&
\int
\frac{d^3 \bld{k}}{2 \omega} 
\frac{\omega}{2}
(\bld{a}^\dagger\cdot \bld{a} + \bld{a}\cdot \bld{a}^\dagger)
+
H_{\mbox{\scriptsize Coul}}
- 
\int
d^3 \bld{x}
\bld{j} \cdot \bld{A}
\label{eqn:HamiltCoul}
\end{eqnarray}

To obtain the Feynman propagator in this gauge, we first derive the following
expression:
\begin{eqnarray}
D^{ij}(x-y)
&\leftdef&
\bra 0 \braend A^i(x) A^j(y) \ketend 0 \ket
\nonumber\\
&=&
\int
d^3 \tilde{\bld{k}} d^3 \tilde{\bld{k}}'
\bra 0 \braend \left[ a^i (\bld{k}), {a^j}^\dagger (\bld{k}') \right] \ketend 0 \ket
e^{-ikx + ik'y}
\nonumber\\
&=&
\int
d^3 \tilde{\bld{k}} d^3 \tilde{\bld{k}}'
\bra 0 \braend \left[ a_{\lambda} (\bld{k}), {a_{\lambda'}}^\dagger (\bld{k}') \right] \ketend 0 \ket
e^{(\lambda)i} {e^{(\lambda') j}}^*
e^{-ikx + ik'y}
\nonumber\\
&=&
\int
\frac{d^3 \bld{k}}{(2\pi)^3 4 \omega^2} 2\omega
\sum_\lambda 
e^{(\lambda)i} {e^{(\lambda) j}}^*
e^{-ik(x -y)}
\\
&=&
\int
\frac{d^3 \bld{k}}{(2\pi)^3 2 \omega} 
(\delta^{ij} - \frac{k^i k^j}{\bld{k}^2})
e^{-ik(x -y)} \,.
\end{eqnarray}
The Feynman propagator is then obtained as
\footnote{%------------------------------- footnote >>
The derivation goes similar way as that for scalar fields.
For checking the result, it is straightforward to obtain
\begin{eqnarray*}
D_T^{ij}(x-y) = \theta(x^0 > y^0) D^{ij}(x-y) +\theta(y^0 > x^0) D^{ij}(y - x)
\end{eqnarray*}
by
performing the contour integration of the variable $k_0$ in Eq. (\ref{eqn:FeynPropCoulgauge}),
}%------------------------------- footnote //
\begin{eqnarray}
D^{ij}_T(x - y)
&\leftdef&
\bra 0 \braend 
T\left[
A^i(x) A^j(y)
\right]
\ketend 0 \ket
%\nonumber\\
%&=&
%\theta(x^0 - y^0) D^{ij}(x-y)
%+
%\theta(y^0 - x^0) D^{ji}(y-x)
\nonumber\\
&=&
\int
\frac{d^4 k}{(2\pi)^4}
\frac{i}{k^2 + i\epsilon}
\left(
\delta^{ij}
-
\frac{k^i k^j}{\bld{k}^2}
\right)
e^{-ik \cdot (x- y)} \,,
\nonumber\\
\label{eqn:FeynPropCoulgauge}
\end{eqnarray}
where we have distinguished this propagator with a subscript $T$
to indicate this one is for transverse photons.

\bigskip

\bigskip

%------------------------------------------------------------------------------------- Lorentz quantization
\noindent
\underline{Lorentz gauge}\\

We consider a  prescription where we start with the Lagrangian density in Eq. (\ref{eqn:LorentzLagmultip})
with $\eta = 1/ \alpha = 1$ (Feynman gauge). Thus, our Lagrangian density is written as
\begin{eqnarray}
{\cal L}
&=&
{\cal L}_{F^2} - \frac{1}{2}(\partial_\mu A^\mu)^2
\nonumber\\
&=&
{\cal L}_{(\partial A)^2} 
+ \frac{1}{2}
\partial_\nu (
A_\mu \partial^\mu A^\nu
-
A_\nu \partial_\mu A^\mu
) \,.
\label{eqn:LorQFermiLag}
\end{eqnarray}
This Lagrangian density is equivalent with ${\cal L}_{F^2}$ provided
the Lorentz gauge condition is imposed. That means we modify the 
Lagrangian density explicitly so that the EOM is solvable.
We quantize the field first without constraints and later impose
the gauge condition.
Since the field equation is the massless Klein-Goldon equation
\begin{eqnarray}
\partial^2 A^\mu = j^\mu
\,,
\end{eqnarray}
we write the solution for the free field as
\begin{eqnarray}
A_\mu (x)
=
\int
\frac{d^3 \bld{k}}
{\sqrt{(2\pi)^3} 2\omega}
\sum_{\lambda = 0}^3
\left[
a_\lambda (\bld{k})
e_\mu^{(\lambda)} e^{-ikx}
+
a^\dagger_\lambda (\bld{k})
{e_\mu^{(\lambda)}}^* e^{ikx}
\right] \,.
\label{eqn:photonFieldExpansionLorentz}
\end{eqnarray}
As the field $A_\mu$ is not constrained, the polarization four vectors $e_\mu^{(\lambda)}(\bld{k})$ 
can be chosen as 4 linearly independent basis:
\begin{comment}
%--------------------------------------------------------------------comment >>
\begin{eqnarray}
\begin{array}{l}
\sum_{\mu = 0}^3 e_\mu^{(\lambda)} e_\mu^{(\lambda')} 
=
\delta^{\lambda \lambda'}
\vspace{2mm}
\\
\sum_{\lambda=0}^3
e_\mu^{(\lambda)} e_\nu^{(\lambda)} 
=
\delta_{\mu \nu}
\hspace{5mm}
\mbox{completeness}
\vspace{2mm}
\\
e_\mu^{(\lambda)} e^{(\lambda')\mu}
= g^{\lambda \lambda'} 
\hspace{5mm}
\mbox{orthonormality}
\end{array}
\end{eqnarray}
\begin{eqnarray}
\sum_{\lambda=0}^3
\frac{ e_\mu^{(\lambda)} {e_\nu^{(\lambda)}}^* }
{e^{(\lambda)} \cdot {e^{(\lambda)}}^* }
= g_{\mu \nu}
\,,
\hspace{5mm}
e^{(\lambda)} \cdot {e^{(\lambda')}}^* 
=
g^{\lambda \lambda'}
\end{eqnarray}
\end{comment}
%--------------------------------------------------------------------comment //
\begin{eqnarray}
\left\{
\begin{array}{l}
g^{\mu \nu }{e_\mu^{(\lambda)}}^*   e_\nu^{(\lambda')}
=
g^{\lambda \lambda'}
\hspace{5mm}
\mbox{(orthonormal)}
\vspace{2mm}
\\
g_{\lambda \lambda'} e_\mu^{(\lambda)} {e_\nu^{(\lambda')}}^* 
= g_{\mu \nu}
\hspace{5mm}
\mbox{(complete)}
\end{array}
\right.
\end{eqnarray}
In a standard choice of the polarization vectors,
$e^{(3)}$ is taken along the spacial momentum vector $\bld{k}$ 
and other two are in the plane orthogonal to $k$:
\begin{eqnarray}
e^{(0)} =
\left(
\begin{array}{c}
1\\ \bld{0}
\end{array}
\right)
\hspace{2mm}
e^{(1)} =
\left(
\begin{array}{c}
0\\ \bld{e}_1
\end{array}
\right)
\hspace{2mm}
e^{(2)} =
\left(
\begin{array}{c}
0\\ \bld{e}_2
\end{array}
\right)
\hspace{2mm}
e^{(3)} =
\left(
\begin{array}{c}
0\\ \hat{\bld{k}}
\end{array}
\right)
\label{eqn:phopolvecstandardchoice}
\end{eqnarray}
where $k = (k^0, \bld{k}) = \omega (1, \hat{\bld{k}})$ and
$\bld{e}_2 = \hat{\bld{k}} \times \bld{e}_1$ with
$\bld{e}_1$ being an arbitrary chosen spacial unit vector perpendicular 
to $\hat{\bld{k}}$.
In these expressions, we have to notice that
spatial components of $e_\mu^{(\lambda)}$ have opposite
signatures from ones indicated in the above which are ones for
 contravariant vectors $e^{(\lambda)\mu}$ in our notation.
Another standard choice of the polarization vector basis is to take
the transverse ones as
\begin{eqnarray}
e^{(\pm)} = \frac{1}{\sqrt{2}} 
\left(
e^{(1)} \pm i e^{(2)}
\right)
\end{eqnarray}
to represent the two helicities of the photon.

For our Lagrangian density (\ref{eqn:LorQFermiLag}), we have
components of the momentum canonically conjugate to $A_\mu$ as
\begin{eqnarray}
\begin{array}{l}
\displaystyle
\pi^0 
= 
\frac{\partial L_{}}{\partial \dot{A_0}}
= 
- \partial_\mu A^\mu \,,
\vspace{2mm}
\\
\displaystyle
\pi^i =
\frac{\partial L_{}}
{\partial \dot{A_i}}
=
\partial^i A^0 - \partial^0 A^i
= 
E^i \,.
\end{array}
\end{eqnarray}
Since the free field $A$ is not constrained, we write canonical quantization conditions 
as usual
\footnote{%------------------------------- footnote >>
Since $A_\mu$ and $\pi^\mu$ are canonically conjugate to each other,
we write
\begin{eqnarray*}
\left[
A_\mu(\bld{x}, t),  \pi^\nu(\bld{y}, t)
\right]
=
i \delta_\mu^{\nu} \delta^3( \bld{x} - \bld{y}) \,.
\end{eqnarray*}
Uprising the subscript of $A_\mu$, one obtain
Eq. (\ref{eqn:LortzCanonQcond}).
}%------------------------------- footnote \\
 as
\begin{eqnarray}
\begin{array}{l}
\left[
A^\mu(\bld{x}, t),  A^\nu(\bld{y}, t)
\right]
=
\left[
\pi^\mu(\bld{x}, t),  \pi^\nu(\bld{y}, t)
\right]
= 0 \,,
\vspace{2mm}
\\
\left[
A^\mu(\bld{x}, t),  \pi^\nu(\bld{y}, t)
\right]
=
i g^{\mu \nu} \delta^3( \bld{x} - \bld{y}) \,.
\label{eqn:LortzCanonQcond}
\end{array}
\end{eqnarray}
In terms of creation and annihilation operators in Eq. (\ref{eqn:photonFieldExpansionLorentz}), 
these relationships are equivalent with
\begin{eqnarray}
\begin{array}{l}
\left[
a_\lambda(\bld{k}), a_{\lambda'}(\bld{k}')
\right]
=
\left[
a^\dagger_\lambda(\bld{k}), a^\dagger_{\lambda'}(\bld{k}')
\right]
= 0 \,,
\vspace{2mm}
\\
\left[
a_\lambda(\bld{k}), a^\dagger_{\lambda'}(\bld{k}')
\right]
=
- g_{\lambda \lambda'} 2\omega \delta^3(\bld{k} - \bld{k}') \,.
\end{array}
\label{eqn:creanncommLorGauge}
\end{eqnarray}
The signature of the $r.h.s.$ in the second equation for $\lambda = \lambda' = 0$
is quite twisted for it makes the norm of a timelike polarized photon state negative.
Here is the place where we need to impose constraints on the field 
by virtue of the gauge freedom to surpress such an oscillation.
However the Lorentz gauge condition in Eq. (\ref{eqn:LorentzCond}) 
as an operator equation is not consistent with the second equation in
Eq. (\ref{eqn:LortzCanonQcond}) as a derivative of the delta function
in the $r.h.s.$ is not vanishing.

Gupta and Bleuler solved this problem by imposing the Lorentz condition 
(\ref{eqn:LorentzCond}) not as an operator equation but
as a restriction on the physical Hilbert space;
\begin{eqnarray}
\bra \Psi_{Phys}' \braend
\partial_\mu A^\mu
\ketend \Psi_{Phys} \ket
=
0 \,.
\label{eqn:GuptaBleulerCond}
\end{eqnarray}
We may decompose any state $\ketend \Psi \ket$ of the Hilbert space {\cal H}
into a direct product of states $\ketend \psi_T \ket$ containing transverse photons and 
states $\ketend \phi \ket$ containing the timelike and longitudinal photons.
Since
\begin{eqnarray*}
\partial \cdot A
\sim
(a_3 - a_0)  e^{-ikx} + h.c.\,,
\end{eqnarray*}
for our choice (\ref{eqn:phopolvecstandardchoice}) of basis,
the condition (\ref{eqn:GuptaBleulerCond}) reduces to
\begin{eqnarray}
(a_3 - a_0) \ketend \phi \ket = 0
\label{eqn:GuptaBlCondonannopbase}
\end{eqnarray}
Regarding the signature in Eq. (\ref{eqn:creanncommLorGauge}), the number operator
in the Fock space of $\ketend \phi \ket$ may be defined as
\begin{eqnarray}
N' = \int
\frac{d^3 \bld{k}}{2\omega}
\left[
a^\dagger_3(\bld{k}) a_3(\bld{k}) 
- 
a^\dagger_0(\bld{k}) a_0(\bld{k})
\right]
\end{eqnarray}
and the condition (\ref{eqn:GuptaBlCondonannopbase})
requires
\begin{eqnarray}
\bra \phi' \braend
N' \ketend \phi \ket
&=&
\bra \phi' \braend
\left\{
\int
\frac{d^3 \bld{k}}{2\omega}
\left[
a^\dagger_3(\bld{k}) 
- 
a^\dagger_0(\bld{k}) 
\right]
a_3(\bld{k}) 
\ketend \phi \ket
\right\}
\nonumber\\
&=&
\int
\frac{d^3 \bld{k}}{2\omega}
\left\{
\bra \phi' \braend
\left[
a^\dagger_3(\bld{k}) 
- 
a^\dagger_0(\bld{k}) 
\right]
\right\}
a_3(\bld{k}) 
\ketend \phi \ket
\nonumber\\
&=&
0
\end{eqnarray}
Thus, $\bra \phi' \braend \phi \ket = 0$
unless both of $\ketend \phi \ket$ and $\ketend \phi' \ket$ are
eigenstates of  $N'$ belonging to the eigenvalue 0, namely,
the vacuum state $\ketend 0 \ket_L$ of timelike and longitudinal photons.
This also mean $||\ketend \phi \ket \!\!|| = 0$ when the state
$\ketend \phi \ket$ involves more than one timelike or longitudinal photon.
Such states will never contribute to expectation values of 
gauge invariant physical observables
\footnote{%------------------------------- footnote >>
Such a quantity ${\cal O}$ will behave as
\begin{eqnarray*}
{\cal O} \sim \sum_{i = 1}^3 a^\dagger_i(\bld{k}) a_i(\bld{k})
-
a^\dagger_0(\bld{k}) a_0(\bld{k})
\end{eqnarray*}
and the longitudinal and timelike photons cancel among themselves
for 
%\[\bra \Psi_{Phys} \braend a_3^\dagger a_3 \ketend \Psi_{Phys} \ket
\[\bra \phi \braend a_3^\dagger a_3 \ketend \phi \ket
=\bra \phi \braend a_0^\dagger a_0 \ketend \phi \ket
\]
is ensured by the condition (\ref{eqn:GuptaBlCondonannopbase}).
}%------------------------------- footnote \\
 and we may just disregard
them by imposing $\ketend \phi \ket \equiv \ketend 0 \ket_L$.

The propagator in Lorentz, Feynman gauge is evaluated from
Eqs. (\ref{eqn:photonFieldExpansionLorentz}) and (\ref{eqn:creanncommLorGauge})
with a help of Eq. (\ref{eqn:creanncommLorGauge}) as
\begin{eqnarray}
\bra 0 \braend
T[
A_\mu (x) A_\nu (y)
\ketend 0 \ket
=
\int
\frac{d^4 k}{(2\pi)^4}
\frac{-i g_{\mu \nu}}
{k^2 + \i\epsilon}
e^{-ik \cdot (x - y)}
\label{eqn:photonProp}
\end{eqnarray}
Without fixing the gauge parameter $\alpha$, we would obtain
\footnote{%------------------------------- footnote >>
It is straightforward to observe
\begin{eqnarray}
\left[
\partial^2 g^{\mu \nu} - \left(
1 - 1/\alpha \right)
\partial^\mu \partial^\nu
\right]
D_{F\nu \rho}(x)
= i g^\nu_{\!\rho} \delta^4(x) 
\end{eqnarray}
showing $D_F^{\mu \nu}$ is certainly the Green's function of 
Eq. (\ref{eqn:EOMwitheta}) .
}%------------------------------- footnote \\
\begin{eqnarray}
D_F^{\mu \nu}(x - y)
=
\int
\frac{d^4 k}{(2\pi)^4}
\frac{-i}
{k^2 + \i\epsilon}
\left(
g^{\mu \nu} + (\alpha - 1)
\frac{k^\mu k^\nu}{k^2+ i\epsilon}
\right)
e^{-ik \cdot (x - y)}
\label{eqn:photonPropalpha}
\end{eqnarray}
Physical results should not depend on the choice of the value of $\alpha$
and one may choose it so that computations get simple.

%=====================================================================
\subsection{Interacting Photon Field}
Let us consider QED, namely photon field interacting with massive Dirac charge.
Lagrangian density is written as
\begin{eqnarray}
{\cal L}_{\mbox{\scriptsize QED}}
 = 
-\frac{1}{4} F^2
+ \overline{\psi}(i \slashed{\partial} - m) \psi
- e \overline{\psi} \slashed{A} \psi
-\frac{1}{2\alpha} (\partial_\mu A^\mu)^2
\label{eqn:LagdensQED}
\end{eqnarray}
This Lagrangian density is derived as follows. First, we write terms
for the free photon field with a gauge fixing term. Second, we add
a term for the free Dirac field. We now have the one except for 
interaction term $\mathcal{L}_{int} = - e \overline{\psi} \slashed{A} \psi$.
This form of interaction is delivered by requiring that
$\mathcal{L}_{\mbox{\scriptsize QED}}$
without the gauge fixing term
to be locally gauge invariant. A local gauge transformation of the Dirac field is
written as
\begin{eqnarray}
\psi (x) \mapsto \psi' (x) = e^{-ie \lambda(x)} \psi(x)
\,,
\hspace{3mm}
\psi^\dagger  (x) \mapsto  \psi^{\dagger}{}' (x)
=
\psi^\dagger  (x) e^{ie \lambda(x)}
\end{eqnarray}
The free Dirac Lagrangian density is not invariant under this transformation
due to the locality of the phase.
The way out of this is to introduce the covariant derivative:
\begin{eqnarray}
D_\mu = \partial_\mu + ie A_\mu
\label{eqn:QEDcovDeriv}
\end{eqnarray}
Since we have
\begin{eqnarray}
 \overline{\psi}(i \slashed{D} - m) \psi
&\mapsto&
 \overline{\psi}
 \left(
 i \gamma^\mu [\partial_\mu + ie A_\mu - i e (\partial_\mu \lambda)]
 - m
 \right)
 \psi
 \nonumber\\
 &=&
 \overline{\psi}
(i \slashed{D}' - m) \psi
\,,
\end{eqnarray}
where
\begin{eqnarray}
D_\mu' = \partial_\mu + ieA_\mu'
\,,
\hspace{3mm}
A_\mu' = A_\mu + \partial_\mu \lambda
\,
 \end{eqnarray}
 the change $D_\mu \mapsto D'_\mu$ is absorbed as a gauge transformation
 of $A_\mu$
 to which the $\mathcal{L}_{F^2}$ term in Eq. (\ref{eqn:LagdensQED}) is
 invariant. Thus, we obtain a locally gauge invariant Lagrangian density
 by adding $\mathcal{L}_{F^2}$ and the free Dirac Lagrangian with
 partial derivative replaced by the covariant derivative.
 The $\mathcal{L}_{int}$ term in Eq. (\ref{eqn:LagdensQED}) is
 delivered from the "connection" term in the covariant derivative in
Eq. (\ref{eqn:QEDcovDeriv}). Such an interaction is called
"minimum coupling".
In the following, we will use the Feynman gauge ($\alpha = 1$) so that
the photon propagator $D_F^{\mu \nu}$ is given by Eq. (\ref{eqn:photonProp}).
 