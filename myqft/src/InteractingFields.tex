\subsection{The Interaction Picture}
●Schr\"odinger Picture\\
For a given Hamiltonian $H$,
\begin{eqnarray}
\left\{
\begin{array}{l}
i \partial_t 
\ketend \Psi \ket_S
=
H \ketend \Psi \ket_S
\vspace{2mm}
\\
i \partial_t 
{\cal O}_S = 0
\end{array}
\right.
\end{eqnarray}
It is assumed that ${\cal O}_S$ does not depend on $t$ explicitly.
Formal solution to the Schr\"odinger equation is written as
\begin{eqnarray}
\ketend \Psi \ket_S = e^{-iHt} \ketend \Psi_0 \ket_S
\label{eqn:SchrPicFormalSol}
\end{eqnarray}
Expectation value of ${\cal O}_S$ in a state $\ketend \Psi \ket_S$ is given as
\begin{equation}
{}_S\!\bra \Psi \braend {\cal O}_S \ketend \Psi \ket_S
\end{equation}

\bigskip

\noindent
●Heisenberg Picture\\
Related to the Schr\"odinger picture by
\begin{eqnarray}
\left\{
\begin{array}{l}
\ketend \Psi \ket_H
=
e^{iHt}
\ketend \Psi \ket_S
\vspace{2mm}
\\
{\cal O}_H = 
e^{iHt} {\cal O}_S e^{-iHt}
\end{array}
\right.
\end{eqnarray}
so that
\begin{equation}
{}_H\!\bra \Psi \braend {\cal O}_H \ketend \Psi \ket_H
=
{}_S\!\bra \Psi \braend {\cal O}_S \ketend \Psi \ket_S
\end{equation}
They evolve in time as
\begin{eqnarray}
\left\{
\begin{array}{l}
i \partial_t 
\ketend \Psi \ket_H
=
0
\vspace{2mm}
\\
i \partial_t 
{\cal O}_H = 
[ {\cal O}_H, H]
\end{array}
\right.
\end{eqnarray}

\bigskip

\noindent
●The Interaction Picture
\begin{eqnarray}
H = H_0 + H_{int}
\label{eqn:HamiltonianDecomposed}
\end{eqnarray}
Related to the Schr\"odinger picture by
\begin{eqnarray}
\left\{
\begin{array}{l}
\ketend \Psi \ket_I
=
e^{iH_0 t}
\ketend \Psi \ket_S
\vspace{2mm}
\\
{\cal O}_I = 
e^{iH_0 t} {\cal O}_S e^{-iH_0 t}
\end{array}
\right.
\end{eqnarray}
so that
\begin{equation}
{}_I\!\bra \Psi \braend {\cal O}_I \ketend \Psi \ket_I
=
{}_S\!\bra \Psi \braend {\cal O}_S \ketend \Psi \ket_S
\end{equation}
The interaction Hamiltonian in this picture is time dependent;
\begin{eqnarray}
H_I \leftdef (H_{int})_I = 
e^{iH_0 t} H_{int} e^{-iH_0 t}
\label{eqn:defHintI}
\end{eqnarray}
$\ketend \Psi \ket_I$ and ${\cal O}_I$ 
evolve in time as
\begin{eqnarray}
\left\{
\begin{array}{l}
i \partial_t 
\ketend \Psi \ket_I
=
e^{i H_0 t} (-H_0 + H) \ketend \Psi \ket_S
\vspace{2mm}
\\
\hspace{15mm}
=
e^{i H_0 t} H_{int} e^{-i H_0 t}e^{i H_0 t} \ketend \Psi \ket_S
\vspace{2mm}
\\
\hspace{15mm}
=
H_I \ketend \Psi \ket_I
\vspace{2mm}
\\
i \partial_t 
{\cal O}_I = 
[ {\cal O}_I, H_0]
\end{array}
\right.
\label{eqn:InteractionPicTime}
\end{eqnarray}

%=====================================================
\subsection{Dyson's Formula}
Formal solution of Eq. (\ref{eqn:InteractionPicTime}) 
can not be written in the form like one in Eq. (\ref{eqn:SchrPicFormalSol})
since $H_I$ is time dependent.
Writing the solution as
\begin{eqnarray}
\begin{array}{l}
\ketend \Psi(t) \ket_I
=
U(t,t_0)
\ketend \Psi(t_0) \ket_I
\vspace{2mm}
\\
U(t,t) = 1
\hspace{3mm} \mbox{and}
\hspace{3mm}
 U(t_3,t_2) U(t_2,t_1) = U(t_3,t_1) \,,
\end{array}
\label{eqn:timeEvolOpDef}
\end{eqnarray}
the time evolution unitary operator $U(t,t_0)$ is given as
\begin{eqnarray}
U(t, t_0) =
T exp \left(
-i \int_{t_0}^t H_I (t') dt'
\right)\,,
\label{eqn:DysonFormula}
\end{eqnarray}
where $T$ stands for time ordered product 
\cite{ref:Peskin-Schroeder, ref:Itzykson-Zuber, ref:NIsh.1-2, ref:Hioki, ref:Tong}
\begin{eqnarray}
T[ {\cal O}_2(t')  {\cal O}_1 (t) ]=
\theta(t' - t) {\cal O}_2(t')  {\cal O}_1 (t) +
\theta(t - t'){\cal O}_1(t)  {\cal O}_2 (t') \,.
\label{eqn:TProdDef}
\end{eqnarray}
In fact, for $t > t_0$
\begin{eqnarray}
i\partial_t U(t, t_0)
&=&
T \left[
H_I(t) 
 exp \left(
-i \int_{t_0}^t H_I (t') dt'
\right)
\right]
\nonumber\\
&=&
H_I(t) 
T exp \left(
-i \int_{t_0}^t H_I (t') dt'
\right)
\nonumber\\
&=&
H_I(t) 
U(t, t_0)
\end{eqnarray}
and the conditions for $U(t, t_0)$ in Eq. (\ref{eqn:timeEvolOpDef})
is obviously satisfied.
We may write Eq. (\ref{eqn:DysonFormula}) in a form of series as
\begin{eqnarray}
U(t, t_0) &=&
1 -i \int_{t_0}^t H_I (t') dt'
\nonumber\\
&&+
\frac{(-i)^2}{2}
\int_{t_0}^t dt'
\int_{t_0}^t dt''
\;T[ H_I (t') H_I (t'')]
+ \cdots
\label{eqn:TimeDevPerturbSerTprod}
\end{eqnarray}
However,
\begin{eqnarray*}
\int_{t_0}^t dt'
\int_{t_0}^t dt''
\;T[ {\cal O} (t') {\cal O} (t'')]
&=&
\int_{t_0}^t dt'
\int_{t_0}^{{t'}} dt''
\; {\cal O} (t') {\cal O} (t'')
+
\int_{t_0}^t dt'
\int_{t'}^t dt''
\; {\cal O} (t'') {\cal O} (t')
\nonumber\\
&=&
2
\int_{t_0}^t dt'
\int_{t_0}^{{t'}} dt''
\; {\cal O} (t') {\cal O} (t'')
\end{eqnarray*}
and
\begin{eqnarray}
U(t, t_0) &=&
1 -i \int_{t_0}^t H_I (t') dt'
\nonumber\\
&&+
(-i)^2
\int_{t_0}^t dt'
\int_{t_0}^{t'} dt''
 H_I (t') H_I (t'')
+ \cdots
\label{eqn:TimeDevOpPerturbSer}
\end{eqnarray}

%=====================================================
\subsection{The S-matrix}
Dealing with the scattering, we {\bf \textit{assume}} that initial and final states are
eigenstates of $H_0$. To be more concrete, let ${\cal O}_0$ be a complete set of
observables that includes $H_0$ and no one of them depends on $t$ in an explicit manner.
In the interaction picture, observables in ${\cal O}_0$ are constant as operators. [See Eq.
(\ref{eqn:InteractionPicTime}).] What we have assumed is that the initial and final states
are eigenstates of ${\cal O}_0$.

The scattering process takes place as follows. At $t_1 \to -\infty$, the system is in the initial
state $\ketend i \ket$, the system evolves in time by $U(t_2, t_1)$ under the effect of
$H_I$, then, at $t_2 \to \infty$, the system turns in to the scattered state.
The amplitude to find a final state $\ketend f \ket$ in the scattered state is 
\begin{eqnarray}
\lim_{\stackrel{\scriptstyle t_1 \to -\infty}{t_2 \to \infty}}
\bra f \braend U(t_2, t_1) \ketend i \ket
\rightdef
\bra f \braend S \ketend i \ket
\label{eqn:SmatrixDef}
\end{eqnarray}
This is the definition of the S-matrix. Obviously, $S$ is unitary.
Substituting Eq. (\ref{eqn:TimeDevPerturbSerTprod}) or (\ref{eqn:TimeDevOpPerturbSer}) 
in Eq. (\ref{eqn:SmatrixDef}), we 
may consider the operator $S$ as given in the form of a perturbation series.
%When the expression of Eq.  (\ref{eqn:TimeDevPerturbSerTprod}) is used, it reads
It reads
\begin{eqnarray}
 S 
&=&
1 
 -i \int_{-\infty}^\infty H_I (t') dt'
\nonumber\\
&&+
\frac{(-i)^2}{2}
\int_{-\infty}^\infty dt'
 dt''
\;T[ H_I (t') H_I (t'')]
+ \cdots
\label{eqn:SmatrixPertSer}
\end{eqnarray}
In the following discussions, it will be licit in most cases
to write $H_I(t) = \int d^3 \bld{x} {\cal H}_{int}(x)$.
If we adopt this expression, the $n$th order $S$-matrix reads
\footnote{
${\cal H}_{int}(x)$ denotes interaction Hamiltonian density
and it should be transformed to the interaction picture according to 
Eq. (\ref{eqn:defHintI}) before adopted in the formula.
However, in most cases in the following discussions,
this transformation is achieved just by replacing fields in
${\cal H}_{int}(x)$ by ones in the interaction picture
and we may use the same notation for the interaction Hamiltonian
density in the interaction picture as one for the Heisenberg picture.
%as advertised in the above, 
An exception is a case when ${\cal H}_{int}(x)$ involves
derivative couplings.
}
\begin{eqnarray}
 S^{(n)} 
&=&
\frac{(-i)^n}{n!}
\int d^4x_1' \cdots d^4x_n'
\;T[ 
{\cal H}_{int}(x_1') \cdots {\cal H}_{int}(x_n')
]
\label{eqn:nthorderSmatrix}
\end{eqnarray}

