Irreducible representations of the Loerentz group solely 
determines forms of free field equations.
Irreducible representations are classified by
spins of particles.
\subsection{Real Scalar Free Field}
Spinless and neutral.
\subsubsection{Lagrangian Formalism}
\begin{eqnarray}
{\cal L} &=& \frac{1}{2} 
\left( \underline{\partial} \varphi(\underline{x}) \right)^2
-\frac{1}{2} m^2 \varphi^2(\underline{x})
\nonumber\\
&=& \frac{1}{2} 
\partial_\mu \varphi(\underline{x})\cdot \partial^\mu \varphi(\underline{x})
-\frac{1}{2} m^2 \varphi^2(\underline{x})
\label{eqn:RSCLagrangiandensity}
\end{eqnarray}
where a dot in the last equation indicates the former derivative acting only on the first $\varphi$.
Under a Lorentz transformation $\underline{x} \mapsto \underline{x}' = L\underline{x}$,
the field $\varphi(\underline{x})$ transforms as
\begin{equation}
\varphi(x) \mapsto \varphi'(x') = \varphi(x)\,.
\end{equation}
Here and hereafter, we omitt underlines on Lorentz vectors.
\begin{equation}
\begin{array}{c}
\displaystyle
\frac{\partial {\cal L}}{\partial(\partial_\mu \varphi)} = \partial^\mu \varphi
\\
\displaystyle
\pi(x) = \frac{ \partial {\cal L}}{ \partial \dot{\varphi}(x) } = \dot{\varphi}
%\pi_{}(x) = \partial_0 \varphi_{cl}(x)
%=
%\int \frac{d^3 \bld{k}}{\sqrt{(2\pi)^3}} (- i k^0 ) \left[
%a(\bld{k}) e^{-i k \cdot x} - b^\dagger(\bld{k})e^{i k \cdot x} \right]
\end{array}
\label{eqn:conjmomentumRscalar}
\end{equation}
Euler-Lagrange equation
\begin{equation}
\left( \Box + m^2 \right) \varphi_{}(x) = 0
\hspace{10mm}
\mbox{Klein-Gordon}
\label{eqn:Klein-Gordon}
\end{equation}
where $\Box \leftdef \partial^2 = \partial_\mu \partial^\mu = \partial_0^2 - \bld{\partial}^2$.\\
Classical solution of the field equation (\ref{eqn:Klein-Gordon}) is written as
\footnote{%------------------------------- footnote >>
The following consideration lays behind:
\begin{eqnarray*}
\varphi(x)
&=&
\frac{1}{\sqrt{(2\pi)^3}}
\int 
d^4 k
\delta(k^2 - m^2)
\left[
\theta(k^0) + \theta(-k^0)
\right]
\tilde{\varphi} (k)
e^{-i k \cdot x}
\nonumber\\
&=&
\frac{1}{\sqrt{(2\pi)^3}}
\int 
d^4 k
\delta(k^2 - m^2)
\left[
\theta(k^0)
\tilde{\varphi} (k)
e^{-i k \cdot x}
 + \theta(k^0)
\tilde{\varphi} (-k)
e^{i k \cdot x}
\right]
\nonumber\\
&=&
\int 
\frac{d^3 \bld{k}}{\sqrt{(2\pi)^3}2 k^0}
\left[
\tilde{\varphi} (k)
e^{-i k \cdot x}
 + 
\tilde{\varphi} (-k)
e^{i k \cdot x}
\right]
\nonumber\\
&\rightdef&
\int 
\frac{d^3 \bld{k}}{\sqrt{(2\pi)^3}2 k^0}
\left[
a (\bld{k})
e^{-i k \cdot x}
 +
 a^*(\bld{k}) 
e^{i k \cdot x}
\right]
\end{eqnarray*}
}%------------------------------- footnote //
\begin{equation}
\varphi_{cl}(x) = \int \frac{d^3 \bld{k}}{\sqrt{(2\pi)^3}2k^0} \left[
a(\bld{k}) e^{-i k \cdot x} + a^*(\bld{k})e^{i k \cdot x} \right]
\label{eqn:KGclassicalsol}
\end{equation}
where $k^0 = + \sqrt{\bld{k}^2 + m^2}$. 
Canonical conjugate field reads,
\begin{equation}
\pi_{cl}(x) = \dot{\varphi}_{cl}(x)
=
\int \frac{d^3 \bld{k}}{\sqrt{(2\pi)^3}2k^0}  \left[
- i k^0 a(\bld{k}) e^{-i k \cdot x} + i k^0 a^*(\bld{k})e^{i k \cdot x} \right]
\label{eqn:KGclassicalpi}
\end{equation}
Hamiltonian density in the classical level is evaluated from Eq. (\ref{eqn:Hamiltoniandensity}) as
\begin{eqnarray}
{\cal H} &=& \pi(x) \dot{\varphi}(x) - \frac{1}{2} \left\{
\left( \dot{\varphi}(x) \right)^2 - 
\left( \bld{\partial} \varphi(x) \right)^2 \right\}
+ \frac{1}{2} m^2 \varphi^2(x)
\nonumber\\
&=&
\frac{1}{2} \left\{
\pi^2 (x) + \left( \bld{\partial} \varphi(x) \right)^2 + m^2 \varphi^2(x)
\right\}
\label{eqn:KGHamiltoniandens}
\end{eqnarray}
%===============================================================
\subsubsection{Canonical quantization}
\begin{equation}
\begin{array}{c}
[ \varphi(t,\bld{x}), \pi(t, \bld{y})] = i \delta^3 (\bld{x} - \bld{y} )\,,
\vspace{2mm}
\\
\left[ \varphi(t,\bld{x}), \varphi(t,\bld{y}) \right] = 0\,,
\;\;\;\;\left[ \pi(t,\bld{x}), \pi(t,\bld{y}) \right] = 0\,.
\end{array}
\label{eqn:cancomm_eqtime_RS}
\end{equation}
The fields $\varphi(x)$ and $\pi(x)$ are settled as operators at a time $t$.
As Heisenberg operators,
they obey the Heisenberg equation (\ref{eqn:HeisenbergEOM}) with quantum Hamiltonian
%$\hat{H} = \int d^3 \bld{x} \hat{{\cal H}}(x)$ with $\hat{{\cal H}}(x)$ given as
%the normal product of the $r.h.s$ of Eq. (\ref{eqn:KGHamiltoniandens})
% fields replaced by 
given by
\begin{equation}
\hat{H} = \int d^3 \bld{x} \hat{{\cal H}}(x)\,,
\hspace{3mm}
\hat{{\cal H}}(x) = 
\normord{ \frac{1}{2} \left\{
\pi^2 (x) + \left( \bld{\partial} \varphi(x) \right)^2 + m^2 \varphi^2(x)
\right\}
}\,,
\label{eqn:KG_Hamiltonian}
\end{equation}
where $\normord{\dots}$ denotes the normal product.
Since we have the Hamiltonian,
the Heisenberg equation (\ref{eqn:HeisenbergEOM}) is equivalent to the
quantum level Euler equation (\ref{eqn:Klein-Gordon}) and its solution 
and the canonical conjugate field are written as
\begin{equation}
\varphi_{}(x) = \int \frac{d^3 \bld{k}}{\sqrt{(2\pi)^3}2k^0} \left[
a(\bld{k}) e^{-i k \cdot x} + a^\dagger(\bld{k})e^{i k \cdot x} \right]\,,
\label{eqn:RS_fieldexpansion}
\end{equation}
\begin{equation}
\pi_{}(x) 
=
\frac{-i}{2}
\int \frac{d^3 \bld{k}}{\sqrt{(2\pi)^3}}  \left[
a(\bld{k}) e^{-i k \cdot x} - a^\dagger(\bld{k})e^{i k \cdot x} \right]\,,
\label{eqn:RS_pifieldexpansion}
\end{equation}
They have the same form as Eqs. (\ref{eqn:KGclassicalsol}) and (\ref{eqn:KGclassicalpi})
but now the coefficients $a(\bld{k})$ and $a^\dagger(\bld{k})$ are
quantum operators defined by the equal-time commutation relations (\ref{eqn:cancomm_eqtime_RS})
at time $x^0 = t$.
The requirements of Eq. (\ref{eqn:cancomm_eqtime_RS}) is equivalent with
requiring
\begin{equation}
\begin{array}{c}
[ a(\bld{k}), a^\dagger(\bld{k}')] = 2k^0 \delta^3 (\bld{k} - \bld{k'} )\;,
\vspace{2mm}
\\
\left[ a(\bld{k}), a(\bld{k}') \right] = 0\;,\;\;\;\;\left[ a^\dagger(\bld{k}), a^\dagger(\bld{k}') \right] = 0
\end{array}
\label{eqn:cancomm_RS}
\end{equation}

\verb/-----------.-----------.-----------.-----------.-----------/\\
\vspace{-3mm}
{\small
\begin{center}
Addendum: Proof of (\ref{eqn:cancomm_eqtime_RS}) $\Leftrightarrow$ (\ref{eqn:cancomm_RS})
\end{center}
Proof of the necessity of Eq. (\ref{eqn:cancomm_RS}) is straightforward,
We show the sufficiency
of Eq. (\ref{eqn:cancomm_eqtime_RS}) in the following.
We may write Eqs. (\ref{eqn:RS_fieldexpansion}) and (\ref{eqn:RS_pifieldexpansion})
as
\begin{equation}
\begin{array}{l}
\displaystyle
\varphi(x) = \int \frac{d^3\bld{k}}{\sqrt{(2\pi)^3}} Q_{\bld{k}} (t) e^{i \bld{k} \cdot \bld{x}}\,,
\\
\displaystyle
\pi_{}(x)  = \int \frac{d^3\bld{k}}{\sqrt{(2\pi)^3}} P_{\bld{k}} (t) e^{- i \bld{k} \cdot \bld{x}}\,,
\end{array}
\end{equation}
with
\begin{equation}
Q_{\bld{k}}(t)
=
\frac{1}{2k^0} \left[
a(\bld{k}) e^{-i k^0 t} + a^\dagger(-\bld{k}) e^{i k^0 t} \right]
\label{eqn:RSC_Qk_def}
\end{equation}
\begin{equation}
P_{\bld{k}}(t)
=
\frac{i}{2} \left[
a^\dagger(\bld{k}) e^{i k^0 t} - a(-\bld{k}) e^{- i k^0 t} \right]
\label{eqn:RSC_Pk_def}
\end{equation}
Relationships $Q^\dagger_{\bld{k}} = Q_{-\bld{k}}$ and
$P^\dagger_{\bld{k}} = P_{-\bld{k}}$ ensure 
that $\varphi$ and $\pi$ are real.
From the linear independence of Fourier components, we have
\begin{equation}
\begin{array}{l}
0 = [\varphi(t, \bld{x}), \varphi(t, \bld{y})]
\;\Longleftrightarrow\;
[Q_{\bld{k}}(t), Q_{\bld{k'}}(t)] = 0\,,
\vspace{2mm}
\\
0 = [\pi(t, \bld{x}), \pi(t, \bld{y})]
\;\Longleftrightarrow\;
[P_{\bld{k}}(t), P_{\bld{k'}}(t)] = 0\,,
\end{array}
\label{eqn:RSC_QQcomm}
\end{equation}
and
\begin{eqnarray}
i\delta^3(\bld{x} - \bld{y})
&=&
\int \frac{d^3 \bld{k} d^3 \bld{k}'}{(2\pi)^3}
i \delta^3 (\bld{k} - \bld{k}')
e^{i \bld{k} \cdot \bld{x} - i \bld{k}' \bld{y}}
\nonumber\\
&=&
[\varphi(t, \bld{x}), \pi(t, \bld{y}) ]
\nonumber\\
&=&
\int \frac{d^3 \bld{k} d^3 \bld{k}'}{(2\pi)^3} [Q_{\bld{k}}, P_{\bld{k}'}]
e^{i \bld{k} \cdot \bld{x} - i \bld{k}' \cdot \bld{y}}
\nonumber\\
&\Longleftrightarrow&    
\nonumber\\
\left[ Q_{\bld{k}}, P_{\bld{k}'} \right] &=& i \delta^3 (\bld{k} - \bld{k}')
\label{eqn:RSC_QPcomm}
\end{eqnarray}
Eqs. (\ref{eqn:RSC_Qk_def}) and (\ref{eqn:RSC_Pk_def})
reads,
\begin{equation}
a(\bld{k}) = \left( k^0 Q_{\bld{k}} (t) + i P_{\bld{k}}^\dagger (t) \right)
e^{i k^0 t}\,,
\label{eqn:RSC_a_by_QP} 
\end{equation}
\begin{equation}
a^\dagger(\bld{k}) = \left( k^0 Q_{\bld{k}}^\dagger (t) - i P_{\bld{k}} (t) \right)
e^{- i k^0 t}\,,
\label{eqn:RSC_a_by_QP} 
\end{equation}
and Eq. (\ref{eqn:cancomm_RS}) follows
from Eqs. (\ref{eqn:RSC_QQcomm}) and (\ref{eqn:RSC_QPcomm}).
}\\
\verb/-----------.-----------.-----------.-----------.-----------/\\


%=================================

Notice that $dim [\varphi] = E^1$ as can be seen from Eq. (\ref{eqn:RSCLagrangiandensity})
and $dim [\pi] = E^2$. Thus they are physical quantities  quite different from
those in the case of the Schr\"odinger field theory.\\

%===============================================================
\subsubsection{Particle states and wave functions}
We are now making use of a fact that operators in Eq. (\ref{eqn:cancomm_RS}) satisfiy
the condition of the bosonic creation-annihilation operators, (\ref{eqn:Nbodycreanncomm}), 
for the case of continuous eigenvalues.
The total number operator can be defined as
\begin{equation}
\hat{N} = \int \frac{d^3 \bld{k}}{2k^0} a^\dagger(\bld{k}) a(\bld{k})
\label{eqn:KG_totalNumberOp}
\end{equation}
It holds that
\begin{equation}
\begin{array}{l}
\displaystyle
\hat{N} a(\bld{k}) = 
\int \frac{d^3 \bld{k}'}{2k^{0'}} \left\{
a(\bld{k}) a^\dagger(\bld{k}') - 2 k^0 \delta^3(\bld{k} - \bld{k}') \right\} a(\bld{k}')\\
\hspace{12mm} 
= a(\bld{k}) (\hat{N} - 1)\,,
\\
\hat{N} a^\dagger(\bld{k}) 
= a^\dagger(\bld{k}) (\hat{N} + 1)
\end{array}
\end{equation}
The Hamiltonian (\ref{eqn:KG_Hamiltonian}) reads from 
Eqs. (\ref{eqn:RS_fieldexpansion}) and (\ref{eqn:RS_pifieldexpansion}) as,
\begin{equation}
\hat{H} = \int \frac{d^3 \bld{k}}{2k^0} k^0 a^\dagger(\bld{k}) a(\bld{k})
\label{eqn:KG_Hamiltonian_aadagg}
\end{equation}
□Total momentum comes here
%-------------------------------------------
\footnote{
Applying  Noether's theorem to the invariance under space-time translations,
we have an expression for the conserved energy-momentum vector as
\begin{equation*}
P^\mu = \int T^{0 \mu} (x) d^3 \bld{x}
\end{equation*}
where the conserved Noether current is given as
\begin{equation*}
T^{\mu}_{\;\nu} (x) =
\frac{\partial {\cal L}}{\partial (\partial_\mu \varphi(x))}
\partial_\nu \varphi(x) 
- g^\mu_\nu \,{\cal L}
\end{equation*}
\begin{equation*}
\hat{P}^\mu =  \int \frac{d^3 \bld{k}}{2k^0} k^\mu a^\dagger(\bld{k}) a(\bld{k})
\end{equation*}
} % footnote end
\\
We have
\begin{equation}
[ \hat{P}^\mu, \hat{N}] = 0
\label{eqn:commutingHandN}
\end{equation}
and this relationship establishes particle interpretation.
Namely a state $a^\dagger({\bld{k}})\ketend 0 \ket$ is interpreted as
one particle eigenstate of the momentum associated with an eigenvalue $\bld{k}$.
\begin{equation}
\hat{\bld{P}} \hat{a}^\dagger(\bld{k}) \ketend 0 \ket
=
\bld{k} \hat{a}^\dagger(\bld{k}) \ketend 0 \ket
\end{equation}
so that we may write
\begin{equation}
\hat{a}^\dagger(\bld{k}) \ketend 0 \ket = \ketend \bld{k} \ket
\end{equation}
State normalization
\[
\bra \bld{p} \braend \bld{p}' \ket
= \bra 0 \braend [a(\bld{p}), a^\dagger(\bld{p}')] \ketend 0 \ket
= 2 k^0 \delta^3 (\bld{p} - \bld{p}')
\]
The Hamiltonian (\ref{eqn:KG_Hamiltonian_aadagg}) is diagonalized
by creation-annihilation operators:
\begin{equation}
[ \hat{H}, \hat{a}^\dagger(\bld{k}) ] = k^0 \hat{a}^\dagger(\bld{k})
\,, \hspace{3mm}
[ \hat{H}, \hat{a}(\bld{k}) ] = - k^0 \hat{a}(\bld{k})
\label{eqn:KGcommHanda}
\end{equation}
so that
\begin{equation}
\hat{H} \hat{a}^\dagger(\bld{k}) \ketend 0 \ket
=
k^0 \hat{a}^\dagger(\bld{k}) \ketend 0 \ket
\end{equation}
Wave function:\\
For
\begin{equation*}
\ketend \Psi^{(1)} \ket
\leftdef
\int \frac{d^3 \bld{k}}{2k^0}  \Psi^{(1)} (\bld{k}) a^\dagger(\bld{k}) \ketend 0 \ket\,,
\end{equation*}
\begin{equation*}
\hat{N} \ketend \Psi^{(1)} \ket = \ketend \Psi^{(1)} \ket\,,
\end{equation*}
\begin{equation*}
\bra \bld{k} \braend \Psi^{(1)} \ket
=  \Psi^{(1)} (\bld{k})
\end{equation*}
The state is normalized through a relationship
\begin{eqnarray*}
\bra \Psi^{(1)} \braend \Psi^{(1)} \ket
=
\int \frac{d^3 \bld{k}}{2k^0}  |\Psi^{(1)}(\bld{k})|^2
\end{eqnarray*}

A state of $n$ particles at momenta $\bld{k}_1$, $\bld{k}_2, \dots \bld{k}_n$,  
is written as
\begin{eqnarray*}
\ketend \bld{k}_1 \cdots \bld{k}_n \ket
=
\frac{1}{\sqrt{n!}}
a^{\dagger}(\bld{k}_1) \cdots a^{\dagger}(\bld{k}_n)
\ketend 0 \ket
\end{eqnarray*}
Since $a^{\dagger}(\bld{k}_1), \dots a^{\dagger}(\bld{k}_n)$
commute among themselves, the state is symmetric under
change of orders of momenta.
We have
\footnote{%----------------------------------------------footnote >>
Useful relationships
\begin{eqnarray*}
\left[
a, a_1^\dagger \cdots a_n^\dagger
\right]
=
\sum_{i=1}^n
a_1^\dagger \cdots [a, a_i^\dagger] \cdots a_n^\dagger
\end{eqnarray*}
\begin{eqnarray*}
\left[
a_1 \cdots a_n, a^\dagger
\right]
=
\sum_{i=1}^n
a_1 \cdots [a_i, a^\dagger] \cdots a_n
\end{eqnarray*}

}%----------------------------------------------footnote //
\begin{eqnarray*}
\hat{N} \ketend \bld{k}_1 \cdots \bld{k}_n \ket
&=&
\frac{1}{\sqrt{n!}}
 \int \frac{d^3 \bld{k}}{2k^0} a^\dagger(\bld{k}) 
 \left[ a(\bld{k}), 
a^{\dagger}(\bld{k}_1) \cdots a^{\dagger}(\bld{k}_n)
\right]
\ketend 0 \ket
 \\
&=&
\frac{1}{\sqrt{n!}}
 \int \frac{d^3 \bld{k}}{2k^0} a^\dagger(\bld{k}) 
 \left(
 \left[ a(\bld{k}), 
a^{\dagger}(\bld{k}_1)
\right]
a^{\dagger}(\bld{k}_2) \cdots a^{\dagger}(\bld{k}_n)
\right.
\\
&&+
\left.
a^{\dagger}(\bld{k}_1)
 \left[ a(\bld{k}), 
a^{\dagger}(\bld{k}_2) \cdots a^{\dagger}(\bld{k}_n)
\right]
\right)
\ketend 0 \ket
 \\
&=&
\cdots
 \\
&=&
n \ketend \bld{k}_1 \cdots \bld{k}_n \ket
\end{eqnarray*}
It is normalized as
\begin{eqnarray}
&&\bra \bld{k}_1 \cdots \bld{k}_n 
\ketend \bld{k}'_1 \cdots \bld{k}'_n \ket
\nonumber\\
&=&
\frac{1}{n!}
\sum_{i'_1 = 1}^n
\bra 0 \braend
a_2 \cdots a_{n} a_{1'} \cdots [a_1, a_{i'_1}^\dagger ] \cdots a_{n'}^\dagger
\ketend 0 \ket
\nonumber\\
&=&
\frac{1}{n!}
\sum_{i'_1 = 1}^n
\sum_{i'_2 \neq i'_i}^n
\bra 0 \braend
a_3 \cdots a_{n} a_{1'} \cdots \cancel{a_{i'_1}} \cdots \cancel{a_{i'_2}} \cdots a_{n'}^\dagger
\ketend 0 \ket 
\left[ a_1, a_{i'_1}^\dagger \right]
\left[ a_2, a_{i'_2}^\dagger \right]
\nonumber\\
&=&
\cdots
\nonumber\\
&=&
\frac{1}{n!}
\sum_{i'_1 \dots i'_n = perm(1\dots n)}^{n! \mbox{ terms}}
\prod_{l}^n
2 k_l^0 \delta^3 (\bld{k}_l - \bld{k}_{i'_l})
\label{eqn:nscalarnorm}
\end{eqnarray}
We may construct a state of $n$ scalar particles described by a wave function
$\Psi^{(N)}(\bld{k}_1, \dots, \bld{k}_n)$ as
\begin{eqnarray}
\ketend \Psi^{(N)} \ket
=
\int \prod_i^N 
\frac{d^3 \bld{k}_i}{2 k_i^0}
 \Psi^{(N)} (\bld{k}_1,\dots,\bld{k}_N) 
 \ketend
 \bld{k}_1,\cdots,\bld{k}_N
\ket
\label{eqn:NscalarState}
\end{eqnarray}
Since $n$ particle momentum satate is symmetric under exchange of momenta, 
we may presuppose the function $\Psi^{(N)}(\bld{k}_1, \dots, \bld{k}_n)$ is also
symmetric. Then we have
\begin{eqnarray*}
\bra  \bld{k}_1 \cdots \bld{k}_n \braend
\Psi^{(N)} \ket
= 
 \Psi^{(N)} (\bld{k}_1,\dots,\bld{k}_N) 
 \end{eqnarray*}
When the number of particles is fixed, normalization of the state  is written as
\begin{eqnarray*}
\| \ketend \Psi^{(N)} \ket \|^2
=
\int \prod_i^N 
\frac{d^3 \bld{k}_i}{2 k_i^0}
\| \Psi^{(N)} (\bld{k}_1,\dots,\bld{k}_N) \|^2
=
1
\end{eqnarray*}
The most general state 
in the Fock space
%============================= comment >>
%\begin{comment}
may be written as
\begin{equation*}
\ketend \Psi^{} \ket
=
\sum_N
\ketend \Psi^{(N)} \ket
\end{equation*}
In this case, $\| \ketend \Psi^{(N)} \ket \|^2$
gives the probability to find the system 
with $n$ particles.
%\end{comment}
%============================= comment //

Particle states composed like this are these in the Heisenberg picture.
If we choose $e^{-i k^0 t_0} \hat{a}(\bld{k})$ and $e^{i k^0 t_0} \hat{a}^\dagger(\bld{k})$
as initial values of Heisenberg operators $\hat{a}_H(\bld{k}, t)$ and $\hat{a}_H^\dagger(\bld{k}, t)$
respectively at $t = t_0$, we have from Eq. (\ref{eqn:Heisenbergformalsol}) that
\footnote{%----------------------------------------------footnote >>
From Eq. (\ref{eqn:KGcommHanda}), we have
\begin{equation*}
H a^\dagger (\bld{k}) = a^\dagger (H + k^0)
\,,\hspace{3mm}
H^2 a^\dagger (\bld{k}) = a^\dagger (H + k^0)^2\,,\dots
\end{equation*}
so that
\begin{eqnarray*}
e^{i H (t-t_0)} a^\dagger (\bld{k})
&=&
\sum_n^\infty \frac{i^n}{n!} (t-t_0)^n H^n a^\dagger (\bld{k})
\\
&=&
a^\dagger (\bld{k}) \sum_n^\infty \frac{i^n}{n!} (t-t_0)^n (H + k^0)^n 
\\
&=&
a^\dagger (\bld{k})  e^{i (H + k^0) (t-t_0)}
\end{eqnarray*}
} %----------------------------------------------footnote //
\begin{equation}
a_H(\bld{k}, t) = e^{-i k^0 t} a (\bld{k})\,,
\hspace{3mm}
a_H^\dagger(\bld{k}, t) = e^{i k^0 t} a^\dagger (\bld{k})
\end{equation}
We introduce state vectors in the Schr\"odinger picture by
\begin{equation}
{}_S\bra \Psi(t) \braend {\cal O} \ketend \Psi(t) \ket_S
=
\bra \Psi \braend {\cal O}_H(t) \ketend \Psi \ket
\end{equation}
State vector in the Schr\"odinger picture obeys 
\begin{equation}
i \partial_t \ketend \Psi(t) \ket_S = \hat{H} \ketend \Psi(t) \ket_S
\end{equation}

\newpage
%================================================== 
\subsubsection{Quadratic forms and the propagator}
Let us consider a vacuum expectation value (VEV) of 
a quadratic of the field in Eq. (\ref{eqn:RS_fieldexpansion}).
Eq. (\ref{eqn:cancomm_RS}) gives
\begin{eqnarray}
\bra 0 \braend \varphi(x) \varphi(y) \ketend 0 \ket
=
\int \frac{d^3 \bld{k} }{(2\pi)^3 2 k^0} 
e^{-ik(x - y)}
\,\rightdef\,
D(x-y)\,,
\label{eqn:InvDfuncRscalar}
\end{eqnarray}
where we have defined the invariant $D$ function by the last equation.
Though it is straightforward to derive the  first equation above, 
it will be instructive to illustrate a convenient abbreviation for later,
more complicated calculations. Corresponding to the structure in Eq. (\ref{eqn:RS_fieldexpansion})
that the field can be divided into a positive frequency part which contain an annihilation operator
$a$ and a negative frequency part containing a creation operator $a^\dagger$,
we introduce abbreviations by writing
% Eq. (\ref{eqn:RS_fieldexpansion}) as 
\begin{eqnarray}
\varphi_{}(x) = 
\varphi^{(+)}(x) + \varphi^{(-)}(x)
=
a(x) + a^\dagger (x) \,.
\label{eqn:RscalarFiledAbbrev}
\end{eqnarray}
The first expression by positive and negative frequency parts is after Pauli and Villars
and employed by many authors. The second expression is not commonly used
and sometimes dangerous but is quite convenient in many cases. We shall use
the both expressions depending on the situations. 
Using these expressions, a derivation of the first equation in Eq. (\ref{eqn:InvDfuncRscalar}) goes as follows:
\begin{eqnarray}
\bra 0 \braend \varphi(x) \varphi(y) \ketend 0 \ket
&=&
\bra 0 \braend (a(x) + a^\dagger(x) )( a(y) + a^\dagger(y) ) \ketend 0 \ket
\nonumber\\
&=&
\bra 0 \braend a(x) a^\dagger(y) \ketend 0 \ket
=
\bra 0 \braend \left[ a(x), a^\dagger(y) \right] \ketend 0 \ket
\nonumber\\
&=&
\int \frac{d^3 \bld{k} d^3 \bld{k}'}{(2\pi)^3 4 k^0 {k^0}'} 
\bra 0 \braend \left[ a(\bld{k}), a^\dagger(\bld{k}') \right] \ketend 0 \ket
e^{-i(kx - k'y)}
\nonumber\\
&=&
r.h.s\mbox{ of the 1st equation in Eq. (\ref{eqn:InvDfuncRscalar})}.
\end{eqnarray}
As another example of evaluating quadratic forms of the field, let us
compute a commutator of the field:
\begin{eqnarray}
[\varphi(x), \varphi(y)]
&=&
[a(x) + a^\dagger(x),  a(y) + a^\dagger(y) ]
\nonumber\\
&=&
[a(x),  a^\dagger(y) ] + [a^\dagger(x),  a(y) ]
\nonumber\\
&=&
\int \frac{d^3 \bld{k} d^3 \bld{k}'}{(2\pi)^3 4 k^0 {k^0}'} 
\left(
[a(\bld{k}), a^\dagger(\bld{k}')]e^{-i(kx - k'y)}
\right.
\nonumber\\
&&
\left.
\hspace{35mm}
+
[a^\dagger(\bld{k}), a(\bld{k}')]e^{i(kx - k'y)}
\right)
\nonumber\\
&=&
\int \frac{d^3 \bld{k} }{(2\pi)^3 2 k^0} 
\left(
e^{-ik(x - y)} - e^{ik(x - y)}
\right)
\nonumber\\
&=&
D(x-y) - D(y-x)
\label{eqn:scfieldcommrel}
\end{eqnarray}
Now comes a comment on this result and the invariant D function.
When $x-y$ is spacelike, there exists a Lorentz frame where $x^0 = y^0$.
Then an expression in the second from the last line shows 
this commutator vanishes in that frame.
The whole expression is Lorentz invariant and it must vanish for all $(x-y)^2 < 0$.
Nevertheless, $D(x-y)$ itself does not vanish even $x-y$ is spacelike.


The Feynman propagator of the scalar field is defined as 
\footnote{%----------------------------------------------footnote >>
$\Delta_F[$Ref.\cite{ref:Mandl-Shaw}$] = -i\Delta_F[ours]$,
$G_F[$Ref.\cite{ref:Itzykson-Zuber}$] = i \Delta_F[ours]$.
}%----------------------------------------------footnote //
%the VEV of time ordering product of field operators:
\begin{eqnarray}
\Delta_F(x-y)
&\leftdef&
\bra 0 \braend T[\varphi(x) \varphi(y)] \ketend 0 \ket
\nonumber\\
&=&
\theta(x^0 - y^0) D(x-y)
+
\theta(y^0 - x^0) D(y-x)\,,
\label{eqn:DefFeynmanPropRscalar}
\end{eqnarray}
where $T[\dots]$ stands for time ordering product and it is written for
bosonic field operators as
\begin{equation}
T[\varphi(x) \varphi(y)] =
\theta(x^0 - y^0)\varphi(x) \varphi(y)
+
\theta(y^0 - x^0)\varphi(y) \varphi(x)\,.
\label{eqn:definitionTprodBoson}
\end{equation}
Since we have the following relationship
deduced by contour integrations
\footnote{%----------------------------------------------footnote >>
Contour of the integration should go through the real axis from $- \infty$ to $\infty$ and
turn lower (upper) half plane (anti) clockwise when $x^0  > (<)\, 0$ in the 
middle expression in Eq. (\ref{eqn:FeynmanPropContourInteg}).
Opposite signatures of a factor $2k^0$ at poles in the two terms are absorbed
in opposite signatures of their residues.
}%----------------------------------------------footnote //
,
\begin{eqnarray}
\int d k^0 \frac{e^{-ik^0x^0}}{k^2 - m^2 + i\epsilon}
&=&
\int \frac{dk^0}{2k^0}
\left(
\frac{1}{k^0 - E_{\bld{k}} + i\epsilon
}
+
\frac{1}{k^0 + E_{\bld{k}} - i\epsilon}
\right)
e^{-ik^0x^0}
\nonumber\\
&=&
\frac{-2\pi i}{2E_{\bld{k}}}
\left(
\theta(x^0 > 0) e^{-iE_{\bld{k}}x^0}
+
\theta(x^0 < 0) e^{iE_{\bld{k}}x^0}
\right)\,,
\nonumber\\
\label{eqn:FeynmanPropContourInteg}
\end{eqnarray}
for an infinitesimal positive $\epsilon$ and $E_{\bld{k}} = \sqrt{\bld{k}^2 + m^2}$,
a substitution of Eq. (\ref{eqn:InvDfuncRscalar}) in Eq. (\ref{eqn:DefFeynmanPropRscalar})
yields
\begin{eqnarray}
\Delta_F(x)
&=&
\int \frac{d^3\bld{k}}{(2\pi)^3 2k^0}
\left(
\theta(x^0 > 0) e^{-i k\cdot x}
+
\theta(x^0 < 0) e^{i k\cdot x}
\right)
\nonumber\\
&=&
i \int \frac{d^4 k}{(2\pi)^4 2 k^0}
\left(
\frac{1}{k^0 - E_{\bld{k}} + i\epsilon
}
+
\frac{1}{k^0 + E_{\bld{k}} - i\epsilon}
\right)
e^{-ik \cdot x}
\nonumber\\
&=&
i \int \frac{d^4 k}{(2\pi)^4}
\frac{e^{-ik \cdot x}}{k^2 - m^2 + i\epsilon}\,.
\end{eqnarray}
We observe $\Delta_F$ is an even function
as it should be from a fact $T[\varphi(y)\varphi(x)] = T[\varphi(x)\varphi(y)]$
for bosonic fields.
%\begin{proof}
%\begin{eqnarray}
%\hspace{65mm}
%\blacksquare
%\end{eqnarray}
%\end{proof}
Fourier transform of $\Delta_F(x)$,
\begin{eqnarray}
\tilde{\Delta}_F(q) 
=
\int d^4x e^{iqx} \Delta_F(x)
=
\frac{i}{q^2 - m^2 + i\epsilon}\,
\end{eqnarray}
is frequently referred as $\Delta_F(q)$ without a symbol to distinguish
from the original function
when there is no possibility of confusion. 

Being a propagator should mean that is a Green's function of 
field equation. To convince ourselves that $\Delta_F(x)$ is certainly a propagator,
let us operate the Klein-Gordon operator,
which involves a second order differentiation with respect to time,
 to the T-product in Eq. (\ref{eqn:definitionTprodBoson}).
When one differentiates the T-product with respect to $x^0$, there appear terms 
not only from differentiations of fields but also from that of theta functions, which read
$\delta(x^0 - y^0)(\varphi(x)\varphi(y) - \varphi(y)\varphi(x)$).
This contribution vanishes however because the field operators commutes to each other
according to Eq. (\ref{eqn:cancomm_eqtime_RS}).
We may therefore write
\begin{eqnarray}
\partial_{x^0} T[\varphi(x) \varphi(y)]
&=&
\theta(x^0 - y^0)\partial_{x^0} \varphi(x) \varphi(y)
+
\theta(y^0 - x^0)\varphi(y) \partial_{x^0} \varphi(x)
\nonumber\\
&=&
\theta(x^0 - y^0) \pi(x) \varphi(y)
+
\theta(y^0 - x^0)\varphi(y) \pi(x)\,,
\nonumber
\end{eqnarray}
where we have adopted Eq. (\ref{eqn:conjmomentumRscalar}) in the  last equation.
Further differentiation with respect to $x^0$ generates a contribution from
differentiations of theta functions, 
$\delta(x^0 - y^0)(\pi(x)\varphi(y) - \varphi(y)\pi(x)) = - i \delta^4(x -y)$,
where we have adopted Eq.(\ref{eqn:cancomm_eqtime_RS}).
Thus, the contribution does not vanish this time but gives
a factor of delta function.
Considering that 
spatial differential operators 
%other than one with respect to time
commute with the theta functions and 
that $\varphi(x)$ obays the field equation (\ref{eqn:Klein-Gordon}), 
we finally obtain a relationship
\begin{eqnarray}
\left( \Box_x + m^2 \right) T[\varphi(x) \varphi(y)]
=
-i \delta^4(x - y)\,,
\end{eqnarray}
which means $\Delta_F(x-y)$ is a inhomogeneous Green's function
of the Klein-Gordon equation and it is surely a propagator.
%<<<<<<<<<<<<<<<<<<<<<<<<<<<<<<<<<<<<<




