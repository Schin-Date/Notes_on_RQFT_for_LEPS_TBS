To proceed beyond the second term in Eq. (\ref{eqn:SmatrixPertSer}) requires some
further preparations.

\subsubsection{Propagator}
Let's take a real scalar field.
\begin{eqnarray}
\bra 0 \braend \varphi(x) \varphi(y) \ketend 0 \ket
&=&
\int \frac{d^3 \bld{k} d^3 \bld{k}'}{(2\pi)^3 4 k^0 {k^0}'} 
\bra 0 \braend a(\bld{k}) a^\dagger(\bld{k}') \ketend 0 \ket
e^{-i(kx - k'y)}
\nonumber\\
&=&
\int \frac{d^3 \bld{k} }{(2\pi)^3 2 k^0} 
e^{-ik(x - y)}
\rightdef
D(x-y)
\end{eqnarray}
\begin{eqnarray}
[\varphi(x), \varphi(y)]
&=&
\int \frac{d^3 \bld{k} d^3 \bld{k}'}{(2\pi)^3 4 k^0 {k^0}'} 
\left(
[a(\bld{k}), a^\dagger(\bld{k}')]e^{-i(kx - k'y)}
\right.
\nonumber\\
&&
\left.
\hspace{35mm}
+
[a\dagger(\bld{k}), a(\bld{k}')]e^{i(kx - k'y)}
\right)
\nonumber\\
&=&
\int \frac{d^3 \bld{k} }{(2\pi)^3 2 k^0} 
\left(
e^{-ik(x - y)} - e^{ik(x - y)}
\right)
\nonumber\\
&=&
D(x-y) - D(y-x)
\label{eqn:scfieldcommrel}
\end{eqnarray}
When $x-y$ is spacelike, there exists a Lorentz frame where $x^0 - y^0$.
Then the $r.h.s.$ of Eq. (\ref{eqn:scfieldcommrel}) vanishes. 
The whole expression is Lorentz invariant and it must vanish for all $(x-y)^2 < 0$.
Nevertheless, $D(x-y)$ itself does not vanish even $x-y$ is spacelike.

The Feynman propagator is defined as
\begin{eqnarray}
\bra 0 \braend T[\varphi(x) \varphi(y)] \ketend 0 \ket
&=&
\theta(x^0 - y^0) D(x-y)
+
\theta(y^0 - x^0) D(y-x)
\label{eqn:scFeynPropbyD}
\\
&=&
i \int \frac{d^4 p}{(2\pi)^4}
\frac{e^{-ip (x-y)}}{p^2 - m^2 + i\epsilon}
\label{eqn:scFeynmanProp}
\\
&\rightdef&
\Delta_F(x-y)
\nonumber
\end{eqnarray}
{\it Proof of Eq.(\ref{eqn:scFeynmanProp}) $\Leftrightarrow$ Eq. (\ref{eqn:scFeynPropbyD})}\\
%\begin{proof}
\begin{eqnarray}
\frac{1}{p^2 - m^2 + i\epsilon}
=
\frac{1}{2 E_{\bld{p}}} \left(
\frac{1}{p^0 - E_{\bld{p}} + i\epsilon}
-
\frac{1}{p^0 + E_{\bld{p}} - i\epsilon}
\right)
\,,
\end{eqnarray}
\begin{eqnarray}
\Delta_F(x-y)
&=&
i \int \frac{d^3 \bld{p}}{(2\pi)^3 2 E_{\bld{p}}} e^{i\bld{p}\cdot (\bld{x} - \bld{y})}
\nonumber\\
&&
\times \int \frac{d p^0}{2\pi} e^{- ip^0 (x^0 - y^0)}
\left(
\frac{1}{p^0 - E_{\bld{p}} + i\epsilon}
-
\frac{1}{p^0 + E_{\bld{p}} - i\epsilon}
\right)
\nonumber\\
&=&
\int \frac{d^3 \bld{p}}{(2\pi)^3 2 E_{\bld{p}}} e^{i\bld{p}\cdot (\bld{x} - \bld{y})}
\frac{-1}{2\pi i} \left(
\theta(x^0 - y^0) (-2\pi i e^{-i E_{\bld{p}}(x^0 - y^0)})
\right.
\nonumber\\
&&
\left.
-
\theta(y^0 - x^0) (+2\pi i e^{i E_{\bld{p}}(x^0 - y^0)}
\right)
\nonumber\\
&=&
\int \frac{d^3 \bld{p}}{(2\pi)^3 2 E_{\bld{p}}} 
\left(
\theta(x^0 - y^0) e^{-ip(x-y)}
+
\theta(y^0 - x^0) e^{i p(x-y)}
\right)
\nonumber\\
&=&
\theta(x^0 - y^0) D(x-y)
+
\theta(y^0 - x^0) D(y-x)
\nonumber\\
&& 
\hspace{65mm}
\blacksquare
\end{eqnarray}
%\end{proof}


\subsubsection{T product}
We decompose a real scalar field as 
\begin{eqnarray}
\varphi(x) = \varphi^{(+)}(x) + \varphi^{(-)}(x)
\end{eqnarray}
where $\varphi^{(+)}$ ($\varphi^{(-)}$) is the term which contain
annihilation (creation) operator in Eq. (\ref{eqn:scYfields}).
If $x^0 > y^0$,
\begin{eqnarray}
T[ \varphi(x) \varphi(y)]
&=& (\varphi^{(+)}(x) + \varphi^{(-)}(x))(\varphi^{(+)}(y) + \varphi^{(-)}(y))
\nonumber\\
&=&
\varphi^{(+)}(x) \varphi^{(+)}(y)
+
( [\varphi^{(+)}(x), \varphi^{(-)}(y)] 
+ \varphi^{(-)}(y) \varphi^{(+)}(x) )
\nonumber\\
&&
+
\varphi^{(-)}(x) \varphi^{(+)}(y)
+
\varphi^{(-)}(x) \varphi^{(-)}(y)
\nonumber\\
&=&
\normalprod{\varphi(x) \varphi(y)} + D(x-y)
\nonumber
\end{eqnarray}
and if $y^0 > x^0$,
\begin{eqnarray}
T[ \varphi(x) \varphi(y)]
=
\normalprod{\varphi(x) \varphi(y)} + D(y-x)
\nonumber
\end{eqnarray}
Then, for arbitrary $x^0$ and $y^0$,
\begin{eqnarray}
T[ \varphi(x) \varphi(y)]
=
\normalprod{\varphi(x) \varphi(y)} + \Delta_F(x-y)
\end{eqnarray}
A similar evaluation shows for a complex scalar field that
\begin{eqnarray}
T[ \varphi(x) \varphi^\dagger(y)]
=
\normalprod{\varphi(x) \varphi^\dagger(y)} + \Delta_F(x-y)
\end{eqnarray}

\noindent
$\bullet$Wick's theorem\\
\begin{eqnarray}
&&T[\varphi_1(x_1) \varphi_2(x_2) \varphi_3(x_3) \varphi_4(x_4) \dots]
\nonumber\\
&=& \normalprod{\varphi_1 \varphi_2 \varphi_3 \varphi_4 \dots}
\nonumber\\
&+&
\sum_{k<l} \Delta_F(x_k - x_l)
%%\normalprod{
%%\contraction[1ex]{\dots}{\varphi_k}{\dots}{\varphi_l}
%%%\contraction[1ex]{\dots}{\varphi}{{}_k \dots}{\varphi}
%%%{\varphi_1\dots}{\xout{\varphi_k}}{\dots}{\xout{\varphi_l}} \dots
%%{\varphi_1\dots}{\varphi_k}{\dots}{\varphi_l} \dots
%%}%end normalprod
%\contraction[1ex]{\dots}{\varphi}{{}_k \dots}{\varphi}
%\nomathglue{%
\normalprod{
%\acontraction[1ex]{\dots}{\varphi_k\hphantom{\varphi_k}}{}{\dots\varphi_l}
\acontraction[1ex]{\dots}{\varphi_k\hphantom{\varphi_k}}{}{\dots\varphi_l}
{\varphi_1\dots}{\varphi_k}{\dots}{\varphi_l} \dots
%{\varphi_1\dots}{\xout{\varphi_k}}{\dots}{\xout{\varphi_l}} \dots
}%end normalprod
%}%end of nomathglue
\nonumber\\
&+&
\sum_{k<l} \sum_{m<n} \Delta_F(x_k - x_l) \Delta_F(x_m - x_n)
\normalprod{\dots \xout{\varphi_k} \dots \xout{\varphi_m} \dots \xout{\varphi_l} \dots
\xout{\varphi_n} \dots}
\nonumber\\
&+& \cdots
\end{eqnarray}


