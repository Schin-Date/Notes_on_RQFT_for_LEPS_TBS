\noindent
massive, charged, spin 1/2
%Spinors belong to the fundamental representation of the Poincar\'e Group SL(2,C).
\subsubsection{Classical Free Field}
\noindent
{\bf Dirac equation}

The Dirac equation is written as
\begin{eqnarray}
i \partial_t \psi (x) = 
\left(
\frac{1}{i} \bld{\alpha} \cdot \bld{\partial} + \beta m \right) \psi(x)
\end{eqnarray}
Solution $\psi(x)$ of this equation satisfies the Klein-Gordon equation
\footnote{
The Dirac equation is not the unique form which is
the first order in time derivative and the solution satisfies the K-G eq.
The Duffin-Kemmer equation satisfies these requirements
\cite{ref:Itzykson-Zuber}.
 } % end of footnote
if and only if
\begin{eqnarray}
\{ \alpha_i, \alpha_j \}_+ = 2\delta_{ij}\,,
\hspace{3mm}
\{ \alpha_i, \beta \}_+ = 0\,,
\hspace{3mm}
\beta^2 = 1
\label{eqn:Dialphbetacond}
\end{eqnarray}
They must be traceless:
\begin{eqnarray}
\tr \alpha_i &=& \tr \beta^2 \alpha_i = \tr \beta \alpha_i \beta = - \tr \alpha_i = 0
\nonumber\\
\tr \beta &=& \tr \alpha_i^2 \beta  = \tr \alpha_i \beta \alpha_i  = - \tr \beta = 0
\nonumber
\end{eqnarray}
The minimum dimension of $\alpha$'s and $\beta$ as matrices to satisfy the conditions 
(\ref{eqn:Dialphbetacond}) is 4. 
Introduce $\gamma$ matrices by
\begin{eqnarray}
\gamma^0 = \beta\,,
\hspace{3mm}
\gamma^i = \beta \alpha_i\
\end{eqnarray}
In terms of $\gamma$ matrices, Eq. (\ref{eqn:Dialphbetacond}) reads
\begin{equation}
\{ \gamma^\mu, \gamma^\nu \}_+ = 2 g^{\mu \nu}
\end{equation}
This is the Clifford algebra. This means $(\gamma^0)^2 = 1$ and $(\gamma^i)^2 = -1$ for $i = 1,2,3$.
$\gamma^0$ is obviously traceless and $\gamma^i$'s are also for
\begin{eqnarray}
\tr \gamma^i = \tr \beta \alpha_i = \tr \alpha_i \beta = -\tr \beta \alpha_i = 0\,.
\end{eqnarray}
The Dirac equation is then written as
\begin{eqnarray}
\left( i\slashed{\partial} - m \right) \psi(x) = 0
\label{eqn:DiracEqstandard}
\end{eqnarray}
where $\slashed{a} \leftdef \gamma_\mu a^\mu = \gamma \cdot a$ and
$\slashed{\partial} = \gamma^0 \partial_0 + \gamma^i \partial_i = \gamma^0 \partial_0 + \bld{\gamma}\cdot \bld{\partial}$.

We may choose $\alpha_i$'s and $\beta$ as hermitian matrices.
In that case we have
\begin{eqnarray}
\gamma^{0\dagger} = \gamma^0
\hspace{2mm}
\mbox{and}
\hspace{2mm}
\gamma^{i\dagger} = - \gamma^i
\hspace{3mm}
\mbox{i.e.}
%\hspace{3mm}
%\Longleftrightarrow
\hspace{3mm}
{\gamma^\mu}^\dagger = \gamma^0 \gamma^\mu \gamma^0
\label{eqn:hermitealphabeta}
\end{eqnarray}
Define the conjugate Dirac field by
\begin{equation}
\overline{\psi} = \psi^{*t} \gamma^0
\label{eqn:DefConjDiracField}
\end{equation}
Taking complex conjugate and transpose of Eq. (\ref{eqn:DiracEqstandard}), we obtain
\begin{eqnarray}
0 &=&  
\psi^{*t}(x) 
\left( -i \stackrel{\leftarrow}{\partial} \cdot \gamma^\dagger - m \right) 
\nonumber\\
&=&
\psi^{*t}(x) (\gamma^0)^2
\left( -i \stackrel{\leftarrow}{\partial} \cdot \gamma^\dagger - m \right) 
\nonumber\\
&=&
\overline{\psi}(x)
\left( -i \stackrel{\leftarrow}{\partial} \cdot \gamma^0\gamma^\dagger \gamma^0 - m \right) \gamma^0
\nonumber\\
&=&
- \overline{\psi}(x)
\left( i \stackrel{\leftarrow}{\partial} \cdot \gamma  + m \right) \gamma^0
\nonumber
\end{eqnarray}
Thus, the conjugate Dirac field satisfies
\begin{eqnarray}
\overline{\psi}(x)
\left( i \stackrel{\leftarrow}{\slashed{\partial}}  + m \right)
= 0
\label{eqn:DiracEqConjstandard}
\end{eqnarray}
This equation is equivalent with Eq. (\ref{eqn:DiracEqstandard}) in the classical level
provided we choose $\gamma$'s in such a way that Eq. (\ref{eqn:hermitealphabeta}) holds.

\bigskip
%------------------------------------------------------------------------------------------------------------------------------------------------
\noindent
{\bf Dirac matrices}

The number of linearly independent $4 \times 4$ complex matrices
in the world of Dirac matrices $\gamma^\mu$ is 16.
We count them as
\begin{eqnarray}
\left\{
\begin{array}{l}
\Gamma^S
=
1
\\
\Gamma^V_\mu
=
\gamma_\mu
\\
\Gamma^T_{\mu \nu}
=
\sigma_{\mu \nu} \leftdef
\frac{i}{2}
[ \gamma_\mu, \gamma_\nu]
\\
\Gamma^A_{\mu}
=
\gamma_5 \gamma_\mu
\\
\Gamma^P
=
\gamma_5
\end{array}
\right.
\end{eqnarray}
where $\gamma_5$ is defined as
\begin{eqnarray}
\gamma_5 \equiv \gamma^5 \leftdef
i \gamma^0 \gamma^1 \gamma^2 \gamma^3
=
- \frac{i}{4!}
\epsilon_{\mu \nu \rho \sigma}
\gamma^\mu \gamma^\nu \gamma^\rho \gamma^\sigma
\end{eqnarray}
This matrix satisfies
\begin{eqnarray}
\{
\gamma_5, \gamma_\mu 
\}_+ = 0\,,
\hspace{3mm}
(\gamma_5)^2 = 1\,,
\hspace{3mm}
\gamma_5^\dagger = \gamma_5
\end{eqnarray}
$\sigma_{\mu \nu}$ has a property $\sigma_{\mu \nu}^\dagger = \gamma^0 \sigma_{\mu \nu} \gamma^0$.

\bigskip

Pauli matrices are written as
\footnote{
They were already given in Eq. (\ref{eqn:SU2GenFundamental}).
}
\begin{eqnarray}
\sigma_1 =
\left(
\begin{array}{cc}
0 & 1 \\ 1 & 0
\end{array}
\right)\,,
\hspace{3mm}
\sigma_2 =
\left(
\begin{array}{cc}
0 & -i \\ i & 0
\end{array}
\right)\,,
\hspace{3mm}
\sigma_3 =
\left(
\begin{array}{cc}
1 &  0\\ 0 & -1
\end{array}
\right)
\end{eqnarray}
They satisfy relationships
\begin{eqnarray}
\{
\sigma_i, \sigma_j \}_+ = 2 \delta_{ij}\,,
\hspace{3mm}
\sigma_i \sigma_j = i\epsilon_{ijk} \sigma_k + \delta_{ij}
\end{eqnarray}
A set of Dirac matrices shown below is the one called Dirac representation.
For other frequently used expressions, refer to Appendix \ref{sec:App_Dirac}.
%----------------------------------------------------------------------------
\begin{eqnarray}
\gamma^0
=
\left[
\begin{array}{cc}
1 & 0 \\ 0 & -1
\end{array}
\right]
= \sigma_3 \otimes 1
\,,
\hspace{5mm}
\gamma^i
=
\left[
\begin{array}{cc}
0 & \sigma_i \\ -\sigma_i & 0
\end{array}
\right]
=
i\sigma_2 \otimes \sigma_i
\end{eqnarray}
Elements of $\left[\;\ddots\;\right]$ are $2 \times 2$ matrices.
Expressions with the direct product makes calculations more transparent. 
For instance,
\begin{eqnarray}
\begin{array}{l}
\alpha_i = \beta \gamma^i = \gamma^0 \gamma^i
= (\sigma_3 \otimes 1)(i\sigma_2 \otimes \sigma_i)
= -i \sigma_2 \sigma_3 \otimes \sigma_i
= \sigma_1 \otimes \sigma_i
\\
\hspace{35mm}
=
\left[
\begin{array}{cc}
0 & \sigma_i
\\
\sigma_i & 0
\end{array}
\right]
\end{array}
\end{eqnarray}
Other matrices are similarly obtained as
\begin{eqnarray}
\gamma_5
=
\left[
\begin{array}{cc}
0 & 1 \\ 1 & 0
\end{array}
\right]
= \sigma_1 \otimes 1\,,
\hspace{3mm}
\sigma_{0 i} = -i  \sigma_1 \otimes \sigma_i\,,
\hspace{3mm}
\sigma_{i j} = \epsilon_{ijk} \otimes \sigma_k
\end{eqnarray}
When we write the Dirac field as composed of two component 
large and small components,
\begin{eqnarray}
\psi =
\left[
\begin{array}{c}
\varphi \\ \chi
\end{array}
\right]\,
\end{eqnarray}
the Dirac equation for these components reads
\begin{eqnarray}
\left\{
\begin{array}{l}
i \partial_t \varphi(x)
=
m \varphi(x) + \frac{1}{i} \bld{\sigma} \cdot \bld{\partial} \chi(x)
\vspace{2mm}
\\
i \partial_t \chi(x)
=
-m \chi(x) + \frac{1}{i} \bld{\sigma} \cdot \bld{\partial} \varphi(x)
\end{array}
\right.
\end{eqnarray}
\\



\bigskip
%------------------------------------------------------------------------------------------------------------------------------------------------
\noindent
{\bf The Dirac spinors}

We write solutions of the Dirac equation (\ref{eqn:DiracEqstandard}) as
\begin{eqnarray}
\psi(x) = \int \frac{d^3 \bld{p}}{\sqrt{(2\pi)^3} 2 p^0}
 \left[
\psi^{(+)}(\bld{p}) e^{-ipx} + \psi^{(-)}(\bld{p}) e^{ipx}
\right]\,,
\end{eqnarray}
where $p^0 = \sqrt{\bld{p}^2 + m^2}$.
The Dirac equation demands
\footnote{
%------------------------------------------------
As  a solution of the K-G eq., we can write
\begin{eqnarray*}
\psi(x) &=& \int \frac{d^4 p}{\sqrt{(2\pi)^3}}
\delta(p^2 - m^2) 
 \left[
 \theta(p^0) +  \theta(-p^0) \right]
 \psi(p) e^{-ipx}
\\
&=&
\int \frac{d^4 p}{\sqrt{(2\pi)^3}}
\delta(p^2 - m^2)  
{\large \mbox{[}}
\underbrace{\psi(p)\theta(p^0)}_{\psi^{+}}  e^{-ipx} + 
\underbrace{\psi(-p)\theta(p^0)}_{\psi^{-}}  e^{ipx}
{\large \mbox{]}}
\\
&=&
\int \frac{d^3 \bld{p}}{\sqrt{(2\pi)^3}2p^0}
\left[
\psi^{+}(\bld{p})  e^{-ipx} + \psi^{-}(\bld{p})  e^{ipx}
\right]
\end{eqnarray*}
The Dirac eq. (\ref{eqn:DiracEqstandard}) requires
\begin{eqnarray*}
\left. (\slashed{p} - m) \psi(p) \right|_{p^2 = m^2} = 0\,,
\end{eqnarray*}
from which Eq. (\ref{eqn:DiracEqMomentumamp}) follows.
}%---------------------------------- end of footnote
\begin{eqnarray}
(\slashed{p} - m) \psi^{(+)}(\bld{p}) = 0\,,
\hspace{5mm}
(\slashed{p} + m) \psi^{(-)}(\bld{p}) = 0
\label{eqn:DiracEqMomentumamp}
\end{eqnarray}
For $\bld{p} = \bld{0}$, we have
\begin{eqnarray*}
(\gamma^0 - 1) \psi^{(+)}(\bld{0}) = 0\,,
\hspace{5mm}
(\gamma^0 + 1) \psi^{(-)}(\bld{0}) = 0
\end{eqnarray*}
\bigskip
Each of these equations have two linearly independent solutions
which we write $u^{(r)}(\bld{p})$ for $\psi^{(+)}(\bld{p})$ and
$v^{(r)}(\bld{p})$ for $\psi^{(-)}(\bld{p})$ respectively for 
$r = 1, 2$.
We write
\begin{eqnarray}
\psi(x) = \int \frac{d^3 \bld{p}}{\sqrt{(2\pi)^3} 2 p^0}
\sum_{r = 1, 2} \left[
b_r(\bld{p}) u^{(r)}(\bld{p}) e^{-ipx} + d_r^*(\bld{p}) v^{(r)}(\bld{p}) e^{ipx}
\right]
\label{eqn:DiracFieldClExpand}
\end{eqnarray}
Eq. (\ref{eqn:DiracEqMomentumamp}) reads
\begin{eqnarray}
(\slashed{p} - m) u^{(r)}(\bld{p}) = 0\,,
\hspace{5mm}
(\slashed{p} + m) v^{(r)}(\bld{p}) = 0
\label{eqn:DiracEqDiracspinors}
\end{eqnarray}
For conjugate (adjoint) spinors defined in Eq. (\ref{eqn:DefConjDiracField}) we have
\begin{eqnarray}
\overline{u}^{(r)}(\bld{p}) (\slashed{p} - m) = 0\,,
\hspace{5mm}
\overline{v}^{(r)}(\bld{p}) (\slashed{p} + m) = 0
\label{eqn:DiracEqConjspinors}
\end{eqnarray}
We choose their normalization
 as
\begin{eqnarray}
\begin{array}{c}
\overline{u}^{(r)}(\bld{p})
u^{(s)}(\bld{p})
= 2m \delta^{rs}\,,
\hspace{5mm}
\overline{v}^{(r)}(\bld{p})
v^{(s)}(\bld{p})
= -2m \delta^{rs}\,,
\\
\overline{u}^{(r)}(\bld{p})
v^{(s)}(\bld{p})
= 0\,,
\hspace{5mm}
\overline{v}^{(r)}(\bld{p})
u^{(s)}(\bld{p})
= 0
\end{array}
\end{eqnarray}
%------------------------------------------------------------------------------

\noindent
{\bf Lagrangian density}
\begin{equation}
{\cal L} =  \bar{\psi}(x) \left(
i \slashed{\partial}  - m \right)
 \psi(x)
\end{equation}
where $\slashed{\partial} = \gamma^\mu \partial_\mu$ and
$\bar{\psi}(x) = \psi^\dagger (x) \gamma^0$.
\begin{eqnarray}
\pi(x) =
\frac{\partial {\cal L}}{\partial \dot{\psi}}
=
i \overline{\psi} \gamma^0
= i \psi^\dagger
\end{eqnarray}
The Euler-Lagrange equation reads
%\begin{equation}
%%\slashed{\partial} = 0
%\left(i \slashed{\partial} + m \right) \psi(x) = 0\,,
%\hspace{3mm}
%\bar{\psi}(x) \left(i \stackrel{\leftarrow}{\slashed{\partial}} + m \right)  = 0
%\end{equation}
Eqs. (\ref{eqn:DiracEqstandard}, \ref{eqn:DiracEqConjstandard}).\\
(*A comment on $\pi_{\overline{\psi}}$ and independent $\psi$ and $\overline{\psi}$ treatment.)
(*Conserved Dirac charge)

\subsubsection{Quantized Free Field}
Canonical quantization requires
\begin{eqnarray}
\{
\psi_\alpha (t, \bld{x}), \pi_\beta(t, \bld{y}) \}
=
i \delta_{\alpha \beta }
\delta^3(\bld{x} - \bld{y})
\end{eqnarray}
Writing a quantized free Dirac field
corresponding to Eq. (\ref{eqn:DiracFieldClExpand}) as
\begin{eqnarray}
\psi(x) 
&=&
\int \frac{d^3 \bld{p}}{\sqrt{(2\pi)^3} 2 p^0}
\sum_{r = 1,2}
\left[
c_r(\bld{p}) u^{(r)}(\bld{p}) e^{-ipx}
+
d_r^\dagger(\bld{p}) v^{(r)}(\bld{p}) e^{ipx}
\right]
\,,
\nonumber\\
\label{eqn:DiracFieldExpand}
\end{eqnarray}
creation and annihilation operators satisfy
\begin{eqnarray}
\begin{array}{l}
\{ c_r(\bld{p}), c^\dagger_s(\bld{p}') \}_+
=
2 p^0 \delta_{rs}
\delta^3(\bld{p}-\bld{p}')\,,
\hspace{3mm}
\{ d_r(\bld{p}), d^\dagger_s(\bld{p}') \}_+
=
2 p^0 \delta_{rs}
\delta^3(\bld{p}-\bld{p}')
\vspace{2mm}
\\
\{ c_r(\bld{p}), c_s(\bld{p}') \}_+
=
\{ d_r(\bld{p}), d_s(\bld{p}') \}_+
=
\{ c_r(\bld{p}), d^\dagger_s(\bld{p}') \}_+
=
\dots
= 0
\end{array}
\end{eqnarray}
%-----------------------------------------
%------------------------------------------------------------- comment out
\begin{comment}
\begin{equation}
\{ c(\bld{p}, s), c^\dagger(\bld{p}', s') \}
=
\{ d(\bld{p}, s), d^\dagger(\bld{p}', s') \}
=
2p^0 \delta_{ss'}
\delta^3(\bld{p} - \bld{p}')
\end{equation}
\begin{equation}
\begin{array}{l}
(\slashed{p} - m) u(\bld{p}, s) = 0\,,
\hspace{3mm}
\bar{u}(\bld{p}, s)(\slashed{p} - m)  = 0\,,
\\
(\slashed{p} + m) v(\bld{p}, s) = 0\,,
\hspace{3mm}
\bar{v}(\bld{p}, s)(\slashed{p} + m)  = 0\,,
\end{array}
\end{equation}
\end{comment}
%------------------------------------------------------------- comment out end

\bigskip

\noindent
\underline{Propagator}\\
\begin{eqnarray}
i S_{\xi \eta} &\equiv&
\{
\psi_\xi (x) , \overline{\psi}_\eta (y) \}
\end{eqnarray}
\begin{eqnarray}
iS(x - y)
&=& 
%\dots
%=
(i \slashed{\partial}_x + m)
\left[
D(x-y) - D(y - x)
\right]
\end{eqnarray}

\begin{eqnarray}
S_F(x - y) &\leftdef&
\bra 0 \braend T[ \psi(x) \overline{\psi}(y) \ketend 0 \ket
=
i \int \frac{d^4 p}{(2\pi)^4}
\frac{e^{-ip(x-y)}}{\slashed{p} - m + i\epsilon}
\end{eqnarray}
%------------------------------------------------------------- comment out
\begin{comment}
\begin{eqnarray}
S_F(q) &=& i \int d^4x e^{iqx}
\bra 0 \braend T[\psi(x) \bar{\psi}(0)]
\ketend 0 \ket
\nonumber\\
&=&
\frac{-1}{\slashed{q} - m + i \epsilon}
=
- \frac{\slashed{q} + m}{ q^2 - m^2 + i \epsilon}
\end{eqnarray}
\end{comment}
%------------------------------------------------------------- comment out end

%------------------------------------------------------------- comment out
\begin{comment}
\subsubsection{Majorana Field}
Massless, spin1/2, selfconjugate.
\begin{equation}
i \slashed{\partial} \psi(x) = 0\,,
\hspace{3mm}
\bar{\psi}(x) i \stackrel{\leftarrow}{\slashed{\partial}}   = 0
\end{equation}
\end{comment}
%------------------------------------------------------------- comment out end

