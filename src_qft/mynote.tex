\documentclass{article}
%\documentclass[12pt]{article}
%\pagestyle{empty}          s       % <======= surpress page numbers
\newcommand{\bld}[1]{\mbox{\boldmath $#1$}}
\newcommand{\utilde}[1]{\mbox{$#1$}\hspace{-.8em}\raisebox{-1ex}{$\sim$}}
\newcommand{\leftdef}[0]{\,\mbox{$\stackrel{\leftarrow}{=}$}\,}
\newcommand{\rightdef}[0]{\,\mbox{$\stackrel{\rightarrow}{=}$}\,}
\newcommand{\bra}[0]{<\!}
\newcommand{\braketend}[0]{\,|\,}
\newcommand{\braend}[0]{|\,}
\newcommand{\ketend}[0]{\,|}
\newcommand{\ket}[0]{\!>}
\newcommand{\ctext}[1]{\raise0.2ex\hbox{\textcircled{\scriptsize{#1}}}}
\newcommand{\maru}[1]{\ooalign{
\hfil\resizebox{.8\width}{\height}{\scriptsize{#1}}\hfil
\crcr
\raise0.2ex\hbox{\large$\bigcirc$}}}
\newcommand{\normord}[1]{%
  {:\mathrel{\mspace{1mu}#1\mspace{1mu}}:}%
}
\newcommand{\normalprod}[1]{%
  {:\mathrel{\mspace{1mu}#1\mspace{1mu}}:}%
}


%--------------------------   Source Directories ---------------
\newcommand{\srcdirectory}[0]{../src}
\newcommand{\mfdirectory}[0]{../src/mfonts}
\newcommand{\qftsrcdirectory}[0]{./}


%\oddsidemargin 10pt
%\evensidemargin 10pt
\setlength{\textheight}{42\baselineskip}
%\textheight 600pt
\topmargin -5mm
\footskip 25mm
 \renewcommand\footnoterule{
 %\vspace*{-3pt}%
 \vspace*{9pt}%
     \hrule width 2in height 0.4pt
     \vspace*{2.6pt}}
%\marginparsep 5cm
%\marginparwidth5cm

\newcounter{exercise}

\newcounter{problem}
\newcounter{refprob}
%080219 Problems are numbered by a new counter "problem".
%       One additional counter "refprob" is defined to refer
%       to a previous problem. Find an example in
%       set07/sec02F_3dim_x.tex
%\oddsidemargin10pt
\newcounter{exple}
\newcounter{refexple}
%100104 Examples are numbered by a new counter "exple".
%       One additional counter "refexple" is defined to refer
%       to a previous example. 

\usepackage{graphicx}
\usepackage{latexsym}
\usepackage{amssymb}
\usepackage{ulem}
\usepackage{cancel}
\usepackage{comment}
%\usepackage{multicol}
%\setlength{\columnseprule}{1pt}
%\usepackage{vwcol}
\usepackage{amsmath}
\usepackage{arydshln}  %dashed line in a matrix

\usepackage{feynmp}%%%{feynmp}

\usepackage{color}
\newcommand{\bl}[0]{\color{blue}}
\newcommand{\re}[0]{\color{red}}
\newcommand{\gr}[0]{\color{green}}
\newcommand{\cy}[0]{\color{cyan}}
\newcommand{\ma}[0]{\color{magenta}}
\newcommand{\ye}[0]{\color{yellow}}

\include{hugenholtz}  % macros for hugenholtz diagrams

\usepackage{slashed}

\usepackage[vcentermath]{youngtab}


\begin{document}

\begin{titlepage} 
\vspace*{3cm}
\vspace*{2cm}
\noindent
\begin{center}
{\Huge \bf Notes on Relativistic Quantum Field Theories}   \\
-- For supplements to LEPS Theoretical Base Set --
\vspace{1cm}
{\large Year 2017}
\vspace{1cm}
\\
{\Large 
Schin Dat\'e}
\end{center}
\end{titlepage} 

\setcounter{section}{-1}
\setcounter{page}{0}
%\setcounter{page}{-2}

\tableofcontents   %<<<<<<<<<<<<<<<<<<<<<<<
%\includegraphics[width=5cm,clip]{page-0.pdf}
%\includegraphics[scale=0.5,bb=0 0 200 200,clip]{page-0.pdf}

%\setcounter{page}{-1}
\setcounter{footnote}{0}
\setcounter{equation}{0}
\setcounter{figure}{0}
\setcounter{problem}{0}

\newpage % to keep a space for the contents


\vspace*{-18mm}
%========================================  section 0
\setcounter{section}{-1}
\setcounter{page}{1}
\section{Notations}
%\setcounter{subsection}{-1}
%\subsection{Notations}
$\blacksquare$
metric of 4-vector space:
\begin{equation*}
   g = 
 \left( \begin{array}{cccc}
       1 & 0&0&0\\
       0 & -1&0&0\\
       0 & 0&-1&0\\
       0 & 0&0&-1\\
              \end{array} \right)\hspace{10mm}({\mbox{\small "Bjorken-Drell metric"}})
\label{eqn:metric}
\end{equation*}
The square of a 4-vector $\underline{p} = (p^0, \bld{p}) = (p^0, p^1, p^2, p^3)$ is
\[ \underline{p}^2 = \underline{p}^T g\, \underline{p} = (p^0)^2 - (p^1)^2 - (p^2)^2 - (p^3)^2 = (p^0)^2 - \bld{p}^2\,, \]
where a superscript $T$ denotes the transpose. An underline on a quantity (e.g. $\underline{p}$) indicates a 
%contravariant 
4-vector but it is frequently suppressed.
% and $\underline{p}$ is written just as $p$, when there is no possibility of confusions.
When components of a 4-vector are given, they are contravariant components by default.
The contravariant components of a 4-vector $\underline{p}$ are denoted as $p^{\mu}$ with
upper suffixes. To make sure that components are contravariant, we often denote a 4-vector
itself as $p^{\mu}$. For given contravariant components, covariant components are obtained
as
\[ p_{\mu} = g_{\mu \nu} p^\nu \,,\]
where $g_{\mu \nu}$ is the ($\mu \nu$) element of $g$.
Thus, for the $\underline{p}$ given in the above, we write $p_{\mu} = (p^0, -\bld{p})$.
The square of a 4-vector $\underline{p}$ is now also written as
\[ \underline{p}^2 = p_{\mu} p^{\mu} \,,\]
where same suffixes $\mu$ in lower and upper positions must be summed over $\mu \in [0,3]$.

\bigskip

\bigskip


\noindent
$\blacksquare$ Energy-momentum of a particle of the mass $m$ is written as
\begin{equation*}
	p^\mu \;\leftdef
         \left( \begin{array}{c} E/ c \\ 
          \bld{p}  \end{array} \right)  
          =
	mc
         \left( \begin{array}{c} \gamma \\ 
          \gamma\bld{\beta}  \end{array} \right)  
          = m u^\mu \,,
\label{eqn:enmomvec}
\end{equation*}
where $\gamma \equiv 1 /\sqrt{1 - \bld{\beta}^2}$ denotes the Lorentz factor of motion of the particle,
$\bld{\beta}$ stands for the velocity nomalized by one of the light, $c$,
and $u^\mu$ is the four-velocity.

\bigskip

\bigskip


\noindent
$\blacksquare$ Natural unit ($c = \hbar = 1$) is employed throughout our discussions.
E.g. for the energy-momentum given in the above, we write
\[ E = \sqrt{\bld{p}^2 + m^2} 
\hspace{3mm}
\mbox{and}
\hspace{3mm}
\underline{p}^2 = m^2 \,.
\]

\noindent
$\blacksquare$
Nomalization of states\\
\begin{equation*}
\begin{array}{l}
\mbox{Energy-momentum eigenstate:}\hspace{3mm}
 \bra p \braketend p' \ket = 2 E \delta^3 ( \bld{p} - \bld{p}' )$,\hspace{2mm}$E = \sqrt{m^2 + \bld{p}^2}\\
\mbox{Completeness:}\hspace{3mm}
{\displaystyle   \int \frac{d^3 \bld{p}}{2E}} \ketend p \ket \bra p \braend = 1\,.
\end{array}
\label{eqn:statenormalization}
\end{equation*}
Many authors including ones of Refs. \cite{ref:Donnachie} and \cite{ref:Collins} 
put a factor $(2\pi)^3$ in the front of $\delta$ function and in the denominator 
of the integration volume element. It is just a matter of convention.
Note that
\begin{equation*}
Dim[ \ketend p \ket ] = [E] ^{-1}
\end{equation*}

\noindent
$\blacksquare$
Levi-Civita symbols\\
These are totally antisymmetric tensor densities.\\

\noindent
3 dimensional
\begin{eqnarray*}
\epsilon_{ijk}
=
\left\{
\begin{array}{l}
\raisebox{-.3\height}{${\stackrel{\displaystyle+}{-}}$} \;1
\hspace{5mm}
(i, j, k) =
\raisebox{-.3\height}{$\stackrel{\mbox{even}}{\mbox{odd}}$} \,\;\mbox{permutation of } (1,2,3)
\vspace{2mm}
\\
0
\hspace{5mm}
\mbox{otherwise}
\end{array}
\right.
\label{eqn:LeviCivita3dim}
\end{eqnarray*}
For the purpose of applying the Einstein's contraction rule to 3 vector suffices,
we may use symbols like $\epsilon^{ijk}$, $\epsilon_{ij}^{\;\;\;k}$ and so on.
All these symbols are equivalent with $\epsilon_{ijk}$.
%Eq. (\ref{eqn:LeviCivita3dim}).
The vertical positions of indexes have no meaning.\\

\noindent
4 dimensional
\begin{eqnarray*}
\epsilon^{\mu\nu\rho\sigma}
=
\left\{
\begin{array}{l}
+ 1
\hspace{5mm}
(\mu, \nu, \rho, \sigma, ) = \mbox{even parmutation of } (0, 1, 2, 3)
\\
- 1
\hspace{5mm}
(\mu, \nu, \rho, \sigma, ) = \mbox{odd parmutation of } (0, 1, 2, 3)
\\
0
\hspace{5mm}
\mbox{otherwise}
\end{array}
\right.
\end{eqnarray*}
Symbols with covariant suffices are defined with the metric tensor 
in such a way like $\epsilon_\mu^{\;\;\nu\rho\sigma} = g_{\mu \tau} \epsilon^{\tau \nu\rho\sigma}$.
Therefore, we have $\epsilon_{\mu\nu\rho\sigma} = - \epsilon^{\mu\nu\rho\sigma}$.
This notation follows \cite{ref:Peskin-Schroeder}.


%-------------------------------------------------------- comment block start
\begin{comment}
%========================================  section 1
\section{Kinematics}
\label{sec:kinematics}
\input{../src/2bodysystem}
\input{../src/2to2kinema}

\subsection{3 body final states}
\subsubsection{$\gamma N \rightarrow K^+ K^- N$}
\subsubsection{Dalitz Plot}
\vspace{1cm}

\subsection{Invariant Phase Volume}
\input{../src/InvariantPhaseVolume}
\subsection{Cross Section}
\input{../src/cross_section}

%========================================  section 2
\section{Scattering amplitudes}
\label{sec:ScattAmp}
\input{../src/unitarity}

\subsection{Potential scattering and scattering length}
\input{../src/analyticity}
\input{../src/crossing}
\subsection{Dispersion relations}
%-------------------------------------------------------- comment block end
\end{comment}
%==============================================
\section{Introduction to Relativistic Quantum Field Theory}
\subsection{Starting from Quantum Mechanics}
\subsubsection{Quantum Mechanics}
\indent

Observables are hermitian operators acting on a space of state vectors.
Complete set of observables is a set of observables that commute with one another
and for which there exists only one simultaneous eigenvector when the normalization is fixed.
Set of all simultaneous eigenvectors of a complete set spans a Hilbert space
\footnote{%--------------------------------------------------------------------- footnote >
When an observable of the set has continuous eigenvalues,
the space is not a Hilbert space in a mathematically rigorous sense.
However, we may follow a physics convention to use the term
including such cases.
}%--------------------------------------------------------------------- footnote //
 of
state vectors associated with a quantum mechanical system under consideration. 
Therefore, the set constitutes a complete bases of the Hilbert space.
Different choices of a particular set of complete bases, namely, that of a complete set of observables
define different representations of the Hilbert space.
Suppose $\{ \hat{A}_1, \hat{A}_2, \dots \}$ is a complete set
and $\ketend  a_1 a_2 \cdots \ket$ is a simultaneous eigenvector
associated with a set of eigen values $\{ a_1, a_2, \cdots \}$.
Then an arbitrary state vector $\ketend \Psi \ket$ can be expressed as
\begin{eqnarray}
\ketend \Psi \ket
=
\int \left[da_1\right] \left[da_2 \right] \cdots
\bra a_1 a_2 \cdots \braketend \Psi \ket
\ketend  a_1 a_2 \cdots \ket
\label{eqn:completerepresentation}
\end{eqnarray}
where $\int \left[ da_i \right]$ denotes sum or integration over a variable $a_i$ with a measure
chosen so that
$\int \left[ da_i \right] \ketend a_i \ket \bra a_i \braend$ is the projection of
a space spanned by $\ketend a_i \ket$.
Factor $\bra a_1 a_2 \cdots \braketend \Psi \ket$ in Eq. (\ref{eqn:completerepresentation}) 
is the wave function in the representation defined by the complete set.
When we define another representation by a complete set $\{ B_1, B_2, \dots \}$ and
its simultaneous eigenvectors $\ketend  b_1 b_2 \cdots \ket$,
it follows from Eq. (\ref{eqn:completerepresentation}) that
the wave function in the new representation is given as
\begin{eqnarray}
\bra b_1 b_2 \cdots \braketend \Psi \ket
= 
\int \left[da_1\right] \left[da_2 \right] \cdots
\bra a_1 a_2 \cdots \braketend \Psi \ket
\bra b_1 b_2 \cdots \braketend  a_1 a_2 \cdots \ket
\label{eqn:changeofrepresentation}
\end{eqnarray}

In quantum mechanics, canonically conjugate variables $\hat{A}$ and $\hat{B}$
satisfy a commutation relation instead of  corresponding  Poisson bracket in
classical mechanics. 


For a system composed of one particle, cartesian components $(\hat{x}_1, \hat{x}_2, \hat{x}_3) \equiv \hat{\bld{x}}$
of particle position are observables. 

Eigenvector of a physical quantity $\hat{A}$ ($=\hat{A}^\dagger$ ) associated with 
an eigenvalue a ($= a^*$) is called eigenstate $\ketend a \ket$;
$\hat{A} \ketend a \ket = a \ketend a \ket$.
Physical quantities 

To a system composed of one particle, 
there associate coordinates of the position $\hat{\bld{x}}$ that are
physical quantities.


%To a particle, there associates a physical quantity called position $\hat{\bld{x}}$.

\bigskip

%Operators acting in a Hilbert space. Physical quantities are hermitian operators.
%Born's probability interpretation is employed.\\
\begin{eqnarray}
\mbox{Schr\"odinger eq.} \hspace{24mm} 
&\hspace{-5mm}&
\hspace{-20mm}
i \partial_t \Psi(t, \bld{x}) = {H} \Psi(t, \bld{x}) \,,
%i \partilal_t \Psi (t, \bld{x}) = H \Psi (t, \bld{x})
\hspace{3mm}
H = - \frac{\bld{\partial}^2}{2m} + V(\bld{x}) = H^\dagger
\label{eqn:SchroedingerEq}
\\
\mbox{Eigenstates of } H \hspace{22mm} 
&\hspace{-5mm}&
\hspace{-20mm}
H \varphi_l (\bld{x} ) = \epsilon_l \varphi_l (\bld{x} )\,,
\hspace{3mm}
\{ \varphi_l (\bld{x} ) \} \mbox{: orthogonal, complete}
\label{eqn:energyEigenSt}
\\
\mbox{Solution of the Schr\"odinger eq.}
&\hspace{-5mm}&
\Psi(t, \bld{x})
=
\sum_l \Psi_l e^{-i \epsilon_l t} \varphi_l (\bld{x})
\label{eqn:Psiexpansion}
\end{eqnarray}
%-------------------------------------------------------------------
%
Bras and kets
\begin{equation}
\Psi(t, \bld{x}) = \bra \bld{x} \braend \Psi(t) \ket\,,
\hspace{5mm}
\varphi_l(\bld{x})  = \bra \bld{x} \braend \epsilon_l \ket\,, 
\label{eqn:coordrepbraket}
\end{equation}
\begin{equation*}
\hspace{21mm}
i \partial_t  \ketend \Psi(t) \ket = \hat{H} \ketend \Psi(t) \ket\,,
\hspace{5mm}
\hat{H} = \frac{\hat{\bld{p}}^2}{2 m} + V(\bld{x}) = \hat{H}^\dagger
\hspace{20mm}
\bra \ref{eqn:SchroedingerEq} \ket
\end{equation*}
\begin{equation*}
\hspace{13mm}
\hat{H} 
\ketend \epsilon_l \ket
= 
\epsilon_l \ketend \epsilon_l \ket\,,
\hspace{3mm}
\bra \epsilon_l \ketend \epsilon_{l'} \ket = 0 \mbox{ if } l\neq l'\,,
\hspace{3mm}
\sum_l \ketend \epsilon_{l} \ket \bra \epsilon_{l} \braend = 1
\hspace{10mm}
\bra \ref{eqn:energyEigenSt} \ket
\end{equation*}
\begin{equation*}
\hspace{32mm}
\ketend \Psi(t) \ket =
\sum_l \bra \epsilon_l \braend   \Psi(0) \ket
e^{-i \epsilon_l t} \ketend \epsilon_l \ket
\hspace{30mm}
\bra \ref{eqn:Psiexpansion} \ket
\end{equation*}
%-------------------------------------------------------------------
%
1st quantization
\begin{equation}
[\hat{x}_i, \hat{p}_j] = i \delta_{ij}
\end{equation}
Eigen states
\begin{equation}
\begin{array}{l}
\hat{\bld{x}} \ketend \bld{x} \ket = \bld{x} \ketend \bld{x} \ket
\\
\hat{\bld{p}} \ketend \bld{p} \ket = \bld{p} \ketend \bld{p} \ket
\end{array}
\end{equation}
Conventional ("half relativistic") normalization
\begin{equation}
\bra \bld{x} \braend \bld{x}' \ket = \delta^3(\bld{x}-\bld{x}')\,,
\hspace{5mm}
\int \ketend \bld{x} \ket d^3 \bld{x} \bra \bld{x} \braend = 1
\end{equation}
\begin{equation}
\bra \bld{p} \braend \bld{p}' \ket = 2E \delta^3(\bld{p}-\bld{p}')\,,
\hspace{5mm}
\int \ketend \bld{p} \ket \frac{d^3 \bld{p}}{2E} \bra \bld{p} \braend = 1
\label{eqn:norm_p_state_rel}
\end{equation}
Coordinate representation of the momentum operator
\begin{equation}
\bra \bld{x} \braend \hat{\bld{p}}
=
\frac{1}{i} \bld{\partial} \bra \bld{x} \braend
\end{equation}
and of the momentum eigenstate
\begin{equation}
\bra \bld{x} \braend \bld{p} \ket
=
\sqrt{\frac{2E}{(2\pi)^3}}\, e^{i \bld{p} \cdot \bld{x}}
\end{equation}
Requirement for the normalization factor is understood by employing
the first equation of (\ref{eqn:norm_p_state_rel}) in the $l.h.s$ of
$\nolinebreak{\bra \bld{p} \braend \bld{p}' \ket}= $
$\nolinebreak{\int \bra \bld{p} \braend \bld{x} \ket d^3\bld{x} \bra \bld{x} \ketend \bld{p}' \ket}$
and remembering a formula
\[
\int {d^3 \bld{x}}\, e^{\pm i \bld{p} \cdot \bld{x}}= (2\pi)^3 \delta^3 (\bld{p})
\]

\subsubsection{$N$-body quantum mechanics}
Schr\"odinger eq.
\begin{equation}
\begin{array}{l}
i \partial_t
\Psi^{(N)}(t; \bld{x}_1, \dots \bld{x}_N) = H^{(N)} \Psi^{(N)}(t; \bld{x}_1, \dots \bld{x}_N)\,,
\vspace{2mm}
\\
\displaystyle
H^{(N)} = \sum_i^N H_i\,,
\hspace{3mm}
H_i = - \frac{\bld{\partial}^2}{2m} + V(\bld{x}_i)
\hspace{5mm}
\mbox{(for simplicity)}
\end{array}
\end{equation}
\begin{equation}
H_i \varphi_l^{(i)} (\bld{x} ) = \epsilon_l \varphi_l^{(i)} (\bld{x} )\,,
\hspace{5mm}
i = 1, \dots, N
\end{equation}
\begin{equation}
\begin{array}{l}
\Psi^{(N)}(t; \bld{x}_1, \dots \bld{x}_N)
=
\hspace{50mm}
\\
\hspace{10mm}
%{\stackrel{\displaystyle \sum}{\scriptstyle \epsilon_l^{(1)} + \dots + \epsilon_l^{(n)} = E^{(n)} } }
{\mathop{\displaystyle \sum}_{\scriptstyle  l_1, \dots, l_N} }
\Psi^{(N)}(l_1,\dots, l_N)
e^{-i E^{(N)} t}
\left\{
\varphi_{l_1}^{(1)} (\bld{x}_1 ) \cdots \varphi_{l_N}^{(N)} (\bld{x}_N )
\right\}_P\,,
\end{array}
\end{equation}
where $E^{(N)} = \epsilon_{l_1} + \dots + \epsilon_{l_N}$ and
$\{\cdots \}_P$ denotes symmetrization  for systems of identical bosons
and antisymmetrization for identical fermions.
Thus $\Psi^{(N)}$ is defined in the direct product of $N$ Hilbert space.

\bigskip
\noindent
{\bf ■Theorem of bosonic creation and annihilation operators}\cite{ref:Takahashi_bussei1}\\
If an operator $\hat{a}$ and its hermite conjugate $\hat{a}^\dagger$ satisfy
\begin{equation}
[\hat{a}, \hat{a}^\dagger] = 1\,,
\label{eqn:boson_commutation_relation}
\end{equation}
then
\begin{enumerate}
\item Eigenvalues of an operator $\hat{N} \equiv \hat{a}^\dagger \hat{a}$ is
nonnegative integers $\{0,1,\dots,\infty\}$ and we can call it number operator.
\item Vaccum state $\ketend 0 \ket$ with respect to the dynamical freedom described by $\hat{a}$
and $\hat{a}^\dagger$ can be defined as the eigenstate of
$\hat{N}$ belonging to its eigenvalue 0.
\item
If we normalize the vacuum state by $\bra 0 \braend 0 \ket = 1$, then the eigenstate
of $\hat{N}$ belonging to an eigenvalue $n$ is given below and 
it constitutes an orthonormal set.
\begin{equation}
\ketend n \ket = \frac{1}{\sqrt{n!}} (\hat{a}^\dagger )^n \ketend  0 \ket \,,
\hspace{3mm}
\bra n \braend m \ket = \delta_{nm}
\label{eqn:nbosonstate}
\end{equation}
{\it Proof}

We will make use of useful relationships
\begin{eqnarray*}
\left[ab, c\right] &=& a\left[b,c\right] + \left[a, c\right] b
\nonumber\\
\mbox{and}\hspace{3mm}
\left[a, bc\right] &=& \left[ a, b \right] c + b\left[ a,c \right] 
\end{eqnarray*}
which hold regardless the type of operators $a$, $b$ and $c$.
For a bosonic operator $\hat{a}$ satisfying  Eq. (\ref{eqn:boson_commutation_relation}),
we have
\begin{eqnarray*}
\left[ \hat{a}, \hat{a}^{\dagger n} \right]
= n \hat{a}^{\dagger n - 1}
\end{eqnarray*}
so that
\begin{eqnarray}
\left[ \hat{a}^{\dagger} \hat{a}, \hat{a}^{\dagger n} \right]
= n \hat{a}^{\dagger n }
\label{eqn:nbosonnumber}
\end{eqnarray}
and
\begin{eqnarray*}
( \hat{a}^{\dagger} \hat{a} )(\hat{a}^{\dagger})^n \ketend 0 \ket
 =
 \left[ \hat{a}^{\dagger} \hat{a}, \hat{a}^{\dagger n} \right]\ketend 0 \ket
= n \hat{a}^{\dagger n } \ketend 0 \ket\,.
\end{eqnarray*}
This proves that $\ketend n \ket$ in Eq. (\ref{eqn:nbosonstate}) is an eigenstate of $\hat{N}$
belonging to an eigenvalue $n$:
\begin{eqnarray*}
\hat{N} \ketend n \ket = n \ketend n \ket
\end{eqnarray*}
Next, we compute
\begin{eqnarray}
\hat{a}^m \hat{a}^{\dagger n} \ketend 0 \ket
&=&
\hat{a}^{m-1} \left[ \hat{a}, \hat{a}^{\dagger n} \right] \ketend 0 \ket
\nonumber\\
&=&
n \hat{a}^{m-1} \hat{a}^{\dagger n-1} \ketend 0 \ket
\nonumber\\
&=&
n \hat{a}^{m-2} \left[ \hat{a}, \hat{a}^{\dagger n-1} \right] \ketend 0 \ket
\nonumber\\
&=&
n (n-1 )\hat{a}^{m-2} \hat{a}^{\dagger n-2} \ketend 0 \ket
\nonumber\\
&=&
\dots
\nonumber\\
&=&
\left\{
\begin{array}{l}
n (n-1) \cdots (n-m+1) \hat{a}^{\dagger n - m} \ketend 0 \ket \hspace{2mm}(m < n)
\\
n!\ketend 0 \ket \hspace{2mm}(m = n)
\\
n! \hat{a}^{m- n} \ketend 0 \ket \hspace{2mm}(m > n)
\end{array}
\right.
\nonumber
\end{eqnarray}
and we obtain
\begin{eqnarray*}
\bra 0 \braend \hat{a}^m \hat{a}^{\dagger n} \ketend 0 \ket
= 
n!  \delta_{n,m}
\end{eqnarray*}
that shows $\ketend n \ket$ are orthonormal. 
\end{enumerate}




\bigskip
\noindent
{\bf ■Theorem of fermionic creation and annihilation operators}\cite{ref:Takahashi_bussei1}\\\
If an operator $\hat{c}$ and its hermite conjugate $\hat{c}^\dagger$ satisfy
\begin{equation}
\{\hat{c}, \hat{c}^\dagger\} = 1
\hspace{3mm}
\mbox{and}
\hspace{3mm}
\{\hat{c}, \hat{c} \} = 0\,,
\end{equation}
then
\begin{enumerate}
\item Eigenvalues of an operator $\hat{N} \equiv \hat{c}^\dagger \hat{c}$ is
0 or 1 and we can call it number operator.
\item Vaccum state $\ketend 0 \ket$ with respect to the dynamical freedom described by $\hat{c}$
and $\hat{c}^\dagger$ can be defined as the eigenstate of
$\hat{N}$ belonging to its eigenvalue 0.
\item
If we normalize the vacuum state by $\bra 0 \braend 0 \ket = 1$, then the eigenstates
of $\hat{N}$ are $\ketend 0 \ket$ and $\ketend 1 \ket = c^\dagger \ketend 0 \ket$.
\end{enumerate}

\bigskip
Suppose now we have a set of $\hat{a}_l$  and $\hat{a}_l^\dagger$ for $l = 1, 2, \dots, \infty$
corresponding to energy eigenvalues $\epsilon_1, \epsilon_2,\dots$.
We assume that we can have a set of operators such that each pair of $\hat{a}_l$  and $\hat{a}_l^\dagger$
satisfies the condition of creation-annihilation operators mentioned above and they are independent
for different suffices:
\begin{equation}
\begin{array}{l}
[\hat{a}_l, \hat{a}^\dagger_m] = \delta_{lm}
\\
\left[ \hat{a}_l, \hat{a}_m \right] 
= [\hat{a}^\dagger_l, \hat{a}^\dagger_m] = 0
\end{array}
\label{eqn:Nbodycreanncomm}
\end{equation}
Having with these operators, we define
\begin{equation}
\hat{\varphi}(\bld{x}) = \sum_l \hat{a}_l \varphi_l (\bld{x})\,,
\label{eqn:SchrFieldprimitive}
\end{equation}
where $\varphi_l (\bld{x})$ is eigenvector of $H = -\bld{\partial}^2/2m + V(\bld{x})$ belonging
to the $l$th eigenvalue $\epsilon_l$.
Assign $\hat{a}_l^\dagger$ as operator to create a particle in the $l$th energy eigenstate:
\begin{equation}
\ketend \epsilon_l \ket = \hat{a}_l^\dagger \ketend 0 \ket
\end{equation}
Then we find
\begin{eqnarray}
\bra 0 \braend \hat{\varphi}(\bld{x}) \ketend \epsilon_l \ket
&=&
\sum_{l'} \varphi_{l'} (\bld{x}) 
\bra 0 \braend \hat{a}_{l'} \hat{a}_l^\dagger \ketend 0 \ket
\nonumber\\
&=&
\sum_{l'} \varphi_{l'} (\bld{x}) 
\bra 0 \braend [\hat{a}_{l'}, \hat{a}_l^\dagger] \ketend 0 \ket
\nonumber\\
&=&
\varphi_{l} (\bld{x}) \,,
\end{eqnarray}
where we have used relationships $\hat{a} \ketend 0 \ket = \bra 0 \braend \hat{a}^\dagger= 0$.
Comparing this result with the second equation in Eq. (\ref{eqn:coordrepbraket}), we may write
\begin{equation}
\bra \bld{x} \braend = \bra 0 \braend \hat{\varphi}(\bld{x})
\end{equation}
If we denote by $\hat{a}^\dagger_{\bld{x}}$ the creation operator 
that creates a a particle at a position $\bld{x}$, we can write
\begin{equation}
\hat{a}^\dagger_{\bld{x}} = \hat{\varphi}^\dagger(\bld{x})
\end{equation}
The operator $\hat{\varphi}(\bld{x})$ defined in Eq. (\ref{eqn:SchrFieldprimitive})
is a primitive form of field operators discussed later.
 
 N particle states can be constructed as
 \begin{equation}
 \ketend \epsilon_{l_1}, \dots, \epsilon_{l_N} \ket
 =
 \hat{a}_{l_1}^\dagger \dots  \hat{a}_{l_N}^\dagger \ketend 0 \ket
 \label{eqn:NpartEnergystate}
 \end{equation}
 and
 \begin{equation}
\bra \bld{x}_1, \dots  \bld{x}_N \braend
= \bra 0 \braend
\hat{\varphi}(\bld{x}_1) \dots \hat{\varphi}(\bld{x}_N)
\label{eqn:NpartCoordstate}
\end{equation}
We read
\begin{equation}
\Psi^{(N)}(t; \bld{x}_1, \dots, \bld{x}_N)
= \bra \bld{x}_1, \dots  \bld{x}_N \braend \Psi^{(N)}(t) \ket
\end{equation}
and 
\begin{equation}
\ketend \Psi^{(N)}(t) \ket
=
{\mathop{\displaystyle \sum}_{\scriptstyle {l_1},  \dots, {l_N} } }
\Psi^{(N)}(l_1,\dots, l_N)
e^{-i E^{(N)} t}
\ketend
\epsilon_{l_1}, \dots, \epsilon_{l_N} >
\end{equation}
When $N$ particles are identical bosons,
these expressions in Eq. (\ref{eqn:NpartEnergystate}) and (\ref{eqn:NpartCoordstate})
are redundant because all operators in them are commuting with each other
and the order of variables in these bra and ket have no meaning.
Suppose we have $n_1$ particle in state of energy $\epsilon_1$, $n_2$ in $\epsilon_2$ 
and so on, we may write the $l.h.s.$ of Eq. (\ref{eqn:NpartEnergystate}) as
$\ketend n_1, n_2, \dots \ket$. This notation is commonly used in condensed matter
and nuclear physics.

Basis $\ketend \bld{x}_1, \dots  \bld{x}_N \ket$ or 
$\ketend \epsilon_{l_1}, \dots, \epsilon_{l_N} \ket$ 
span N-particle Hilbert space. Set of all N-particle
Hilbert space spanned by $\ketend \bld{x}_1, \bld{x}_2, \dots   \ket$
is called the Fock space.
A state vector $\ketend \Psi \ket$ of the Fock space can be
expanded as
\begin{equation}
\ketend \Psi \ket =
\sum_N \int \prod_i^N d^3 \bld{x}_i \ketend \bld{x}_1, \dots,  \bld{x}_N \ket
\bra \bld{x}_1, \dots,  \bld{x}_N \braend \Psi \ket
\end{equation}


%<<<<<<<<<<<<<<<<<<<<<<<<<<<<<<<<<
\begin{comment}
\bigskip

\bigskip

\bigskip

\begin{equation}
\ketend p_1, p_2, \dots, p_n \ket
= \{a^\dagger(p_1) a^\dagger(p_2) \dots a^\dagger(p_n)\}
\ketend 0 \ket
\end{equation}
\end{comment}
\subsubsection{Schr\"odinger field theory}
In place of Eq. (\ref{eqn:SchrFieldprimitive}), if we define
\begin{equation}
\hat{\Psi}(t, \bld{x}) = \sum_l \hat{a}_l e^{-i \epsilon_l t}\varphi_l (\bld{x})\,,
\label{eqn:SchWaveOp}
\end{equation}
this operator satisfies the single particle Schr\"odinger equation:
\begin{equation*}
\hspace{40mm}
i \partial_t \hat{\Psi}(t, \bld{x}) = {H} \hat{\Psi}(t, \bld{x}) \,
\hspace{40mm}
(\widehat{\ref{eqn:SchroedingerEq}})
\end{equation*}
We already know that any number of Schr\"odinger particles can be
generated by $\hat{\varphi}(\bld{x}) = \hat{\Psi}(0, \bld{x})$.
Comparing Eq. (\ref{eqn:SchWaveOp}) with
Eq. (\ref{eqn:Psiexpansion}), we observe 
$\hat{\Psi}(t, \bld{x})$ is obtained by replacing 
c-number amplitude $\Psi_l$ in ${\Psi}(t, \bld{x})$
by annihilation operator $\hat{a}_l$.
The field operator  $\hat{\Psi}$ satisfies the following
equaltime commutation relations:
\begin{equation}
\begin{array}{l}
[\hat{\Psi}(t, \bld{x}), \hat{\Psi}^\dagger(t, \bld{x'})]
=
\delta^3 (\bld{x} - \bld{x}')
\vspace{1mm}\\ \relax
[\hat{\Psi}(t, \bld{x}), \hat{\Psi}(t, \bld{x'}) ]
=
[\hat{\Psi}^\dagger(t, \bld{x}), \hat{\Psi}^\dagger(t, \bld{x'})]
= 0
\end{array}
\label{eqn:SchFieldcommlutationRel}
\end{equation}
If we require Eq. (\ref{eqn:SchFieldcommlutationRel}), then Eq. (\ref{eqn:Nbodycreanncomm})
follows.
Schr\"odinger equation ($\widehat{\ref{eqn:SchroedingerEq}}$) follows from a
{\cal L}agrangian density
\begin{equation}
{\cal L} = i \Psi^* \partial_t \Psi + \frac{1}{2m} \Psi^* \bld{\partial}^2 \Psi - V \Psi^* \Psi
\label{eqn:SchroedingerFieldLagrangian}
\end{equation}
Here we note $dim [{\cal L}] = E^4$ and $dim [\Psi] = E^{3/2}$.
Canonical momentum field is given as
\begin{equation}
\Pi \leftdef \frac{\partial \cal L}{\partial \dot{\Psi}} = i \Psi^*\,,
\end{equation}
which has the same dimension as $\Psi$.
The set (\ref{eqn:SchFieldcommlutationRel}) is equivalent with
\begin{equation}
\begin{array}{l}
[\hat{\Psi}(t, \bld{x}), \hat{\Pi}(t, \bld{x'})]
=
\delta^3 (\bld{x} - \bld{x}')
\vspace{1mm}\\ \relax
[\hat{\Psi}(t, \bld{x}), \hat{\Psi}(t, \bld{x'}) ]
=
[\hat{\Pi}(t, \bld{x}), \hat{\Pi}(t, \bld{x'})]
= 0
\end{array}
\label{eqn:SchFieldCanonicalcommlutationRel}
\end{equation}
This is the canonical commutation relation.
We may reverse our argument starting from Eq. (\ref{eqn:SchroedingerFieldLagrangian}),
writing down the "field equation" ($\widehat{\ref{eqn:SchroedingerEq}}$) and
requiring the equaltime canonical commutation relation (\ref{eqn:SchFieldCanonicalcommlutationRel}).
This procedure is called the 2nd quantizaton.
%-------------------------------------------------------------------- comment block below
\begin{comment}
%Operators acting in a Hilbert space. Physical quantities are hermitian operators.
%Born's probability interpretation is employed.\\
\begin{eqnarray}
\mbox{Schr\"odinger eq.} \hspace{24mm} 
&\hspace{-5mm}&
\hspace{-20mm}
i \partial_t \Psi(t, \bld{x}) = {H} \Psi(t, \bld{x}) \,,
%i \partilal_t \Psi (t, \bld{x}) = H \Psi (t, \bld{x})
\hspace{3mm}
H = - \frac{\bld{\partial}^2}{2m} + V(\bld{x}) = H^\dagger
\label{eqn:SchroedingerEq}
\\
\mbox{Eigenstates of } H \hspace{22mm} 
&\hspace{-5mm}&
\hspace{-20mm}
H \varphi_l (\bld{x} ) = \epsilon_l \varphi_l (\bld{x} )\,,
\hspace{3mm}
\{ \varphi_l (\bld{x} ) \} \mbox{: orthogonal, complete}
\label{eqn:energyEigenSt}
\\
\mbox{Solution of the Schr\"odinger eq.}
&\hspace{-5mm}&
\Psi(t, \bld{x})
=
\sum_l \Psi_l e^{-\epsilon_l t} \varphi_l (\bld{x})
\label{eqn:Psiexpansion}
\end{eqnarray}
%-------------------------------------------------------------------
%
Bras and kets
\begin{equation}
\Psi(t, \bld{x}) = \bra \bld{x} \braend \Psi(t) \ket\,,
\hspace{5mm}
\varphi_l(\bld{x})  = \bra \bld{x} \braend \epsilon_l \ket\,, 
\end{equation}
\begin{equation*}
\hspace{21mm}
i \partial_t  \ketend \Psi(t) \ket = \hat{H} \ketend \Psi(t) \ket\,,
\hspace{5mm}
\hat{H} = \frac{\hat{\bld{p}}^2}{2 m} + V(\bld{x}) = \hat{H}^\dagger
\hspace{20mm}
\bra \ref{eqn:SchroedingerEq} \ket
\end{equation*}
\begin{equation*}
\hspace{13mm}
\hat{H} 
\ketend \epsilon_l \ket
= 
\epsilon_l \ketend \epsilon_l \ket\,,
\hspace{3mm}
\bra \epsilon_l \ketend \epsilon_{l'} \ket = 0 \mbox{ if } l\neq l'\,,
\hspace{3mm}
\sum_l \ketend \epsilon_{l} \ket \bra \epsilon_{l} \braend = 1
\hspace{10mm}
\bra \ref{eqn:energyEigenSt} \ket
\end{equation*}
\begin{equation*}
\hspace{32mm}
\ketend \Psi(t) \ket =
\sum_l \bra \epsilon_l \braend   \Psi(0) \ket
e^{-\epsilon_l t} \ketend \epsilon_l \ket
\hspace{30mm}
\bra \ref{eqn:Psiexpansion} \ket
\end{equation*}
%-------------------------------------------------------------------
%
1st quantization
\begin{equation}
[\hat{x}_i, \hat{p}_j] = i \delta_{ij}
\end{equation}
Eigen states
\begin{equation}
\begin{array}{l}
\hat{\bld{x}} \ketend \bld{x} \ket = \bld{x} \ketend \bld{x} \ket
\\
\hat{\bld{p}} \ketend \bld{p} \ket = \bld{p} \ketend \bld{p} \ket
\end{array}
\end{equation}
Conventional ("half relativistic") normalization
\begin{equation}
\bra \bld{x} \braend \bld{x}' \ket = \delta^3(\bld{x}-\bld{x}')\,,
\hspace{5mm}
\int \ketend \bld{x} \ket d^3 \bld{x} \bra \bld{x} \braend = 1
\end{equation}
\begin{equation}
\bra \bld{p} \braend \bld{p}' \ket = 2E \delta^3(\bld{p}-\bld{p}')\,,
\hspace{5mm}
\int \ketend \bld{p} \ket \frac{d^3 \bld{p}}{2E} \bra \bld{p} \braend = 1
\label{eqn:norm_p_state_rel}
\end{equation}
Coordinate representation of the momentum operator
\begin{equation}
\bra \bld{x} \braend \hat{\bld{p}}
=
\frac{1}{i} \bld{\partial} \bra \bld{x} \braend
\end{equation}
and of the momentum eigenstate
\begin{equation}
\bra \bld{x} \braend \bld{p} \ket
=
\sqrt{\frac{2E}{(2\pi)^3}}\, e^{i \bld{p} \cdot \bld{x}}
\end{equation}
Requirement for the normalization factor is understood by employing
the first equation of (\ref{eqn:norm_p_state_rel}) in the $l.h.s$ of
$\nolinebreak{\bra \bld{p} \braend \bld{p}' \ket}= $
$\nolinebreak{\int \bra \bld{p} \braend \bld{x} \ket d^3\bld{x} \bra \bld{x} \ketend \bld{p}' \ket}$
and remembering a formula
\[
\int {d^3 \bld{x}}\, e^{\pm i \bld{p} \cdot \bld{x}}= (2\pi)^3 \delta^3 (\bld{p})
\]
\end{comment}
\subsection{Canonical quantizaton}
${\cal L}$agrangian density ($dim = E^4$) is given as a functional of field $\varphi(x)$
and its space time derivatives
\begin{equation}
	{\cal L} = {\cal L}[\varphi(x), \partial^\mu\varphi(x)]
\end{equation}
Euler-Lagrange Eq.
\begin{equation}
 \frac{ \partial {\cal L}}{ \partial {\varphi}(x) } - 
\partial_\mu \frac{ \partial {\cal L}}{ \partial (\partial_\mu {\varphi}(x) )} = 0
 \;\;
 \label{eqn:EulerLagrange}
\end{equation}
Canonical momentum field
\begin{equation}
\pi(x) = \frac{ \partial {\cal L}}{ \partial \dot{\varphi}(x) }
\label{eqn:conjmomentum}
\end{equation}
where $\dot{\varphi}(x) = \partial_0 \varphi(x) = \partial^0 \varphi(x)$.\\
${\cal H}$amiltonian density
\begin{equation}
{\cal H} = \pi(x) \dot{\varphi}(x) -  {\cal L}\;\;.
\label{eqn:Hamiltoniandensity}
\end{equation}
By solving Eq. (\ref{eqn:conjmomentum}) for $\dot\varphi(x)$,
${\cal H}$ is a function solely of $\pi(x), \varphi(x)$ and $ \partial_i \varphi(x)$.
In the classical field theory, temporal developments of $\varphi(x)$ and $\pi(x)$
are given by Hamiltonian $H = \int \mbox{d}^3\bld{x} \; {\cal H}$
through the canonical equation of motion
\begin{equation}
\dot{\varphi}(x) = -i [\varphi(x), H ]\;, \; \; \; \; \; \dot{\pi}(x) = -i [\pi(x), H ]
\label{eqn:canonicalEOM}
\end{equation}
where $-i[\cdots]$ is the Poisson braket.
The Euler equation (\ref{eqn:EulerLagrange}) and 
the canonical equation (\ref{eqn:canonicalEOM})
are equivalent.

\bigskip

Following the canonical quantization method, 
fields $\varphi(x)$ and $\pi(x')$ at a time 
$x^0 = x'^0 = t_0$ (which is called the time of quantization)
are postulated to satisfy an equal time commutation relation
\begin{eqnarray}
&&\;\;\;\;\;\;[ \varphi(x), \pi(x')] = i \delta^3 (\bld{x} - \bld{x'} )\;,
\label{eqn:cancomm}
\\
&&\left[ \varphi(x), \varphi(x') \right] = 0\;,\;\;\;\;\left[ \pi(x), \pi(x') \right] = 0
\nonumber
\end{eqnarray}
Assuming the existence of the Hamiltonian and using this equal time
commutation relation, it is shown that
the Euler equation (\ref{eqn:EulerLagrange}) and
the Heisenberg equation (\ref{eqn:canonicalEOM})
are equivalent in the level of quantum theory.
\cite{ref:NIsh.1-2}
Here, we shold read $[\cdots]$ in Eq. (\ref{eqn:canonicalEOM}) as
commutation relation.
In that case, using EOM of fields, it is shown that
the commutation relation (\ref{eqn:cancomm}) holds at
arbitrary time once it is set.\cite{ref:NIsh.1-2}
In this sense, the quantization time is arbitral when
the canonical quantization method works in usual manner.
Formal solution of $\varphi(x)$ for the Heisenberg Eq. (\ref{eqn:canonicalEOM})
is written as 
\begin{equation}
\varphi(x) = e^{iH(t-t_0)} \varphi(\bld{x}, t_0) e^{-iH(t-t_0)} 
\label{eqn:timeevophi}
\end{equation}

\subsection{Noethers Theorem}
For each degree of freedom of continuous transformation of fields against which
the action remains invariant, there exist a conserved current $J^\mu (x)$.
Particularly, let a infinitesimal transformation with $s$ parameters $\delta \alpha_\lambda$ be
written as
\begin{eqnarray}
\begin{array}{l}
x^\mu \mapsto x^{\mu '} = x^\mu + \delta x^\mu\,,
\hspace{3mm}
\delta x^\mu = \sum_{\lambda=1}^s X_\lambda^\mu \delta \alpha_\lambda
\vspace{2mm}
\\
\varphi (x) \mapsto \varphi' (x')
=
\varphi(x) + \delta \varphi(x)\,,
\hspace{3mm}
\delta \varphi(x) =
\sum_{\lambda=1}^s \Phi_{ \lambda} (x) \delta \alpha_\lambda
\end{array}
\label{eqn:Noeth_infsimtransf}
\end{eqnarray}
This transformation may be associated with a space-time transformation
specified by $X_\lambda^\mu$ as shown in the first line in Eq. (\ref{eqn:Noeth_infsimtransf}).
When the transformation is related only with  internal degree of freedom of the field , 
one may set $X_\lambda^\mu = 0$.
When the action ${\cal A}$ is invariant under this transformation,
$s$ currents
\begin{eqnarray}
J_\lambda^\mu (x)
&\leftdef&
- \frac{\partial {\cal L}}{\partial (\partial_\mu \varphi)}
\left(
\Phi_{\lambda} (x) - \partial_\nu \varphi(x) X_\lambda^\nu 
\right)
-
{\cal L}(x) 
X_\lambda^\mu 
\label{eqn:NoetherCurrentGeneral}
\end{eqnarray}
are all conserved ones. Namely,
\begin{eqnarray}
\partial_\mu J_\lambda^\mu (x) = 0\,,
\hspace{3mm}
\lambda = 1, \dots, s
\end{eqnarray}
hold and corresponding charges
\begin{eqnarray}
Q_\lambda = 
\int d^3 \bld{x}
J_\lambda^0 (x)
\label{eqn:NoetherChargeGeneral}
\end{eqnarray}
are conserved.
Eq. (\ref{eqn:NoetherCurrentGeneral}) defines the Noether currents
and quantities $Q_\lambda$ in Eq. (\ref{eqn:NoetherChargeGeneral})
are called Noether charges.

%=====================================
\subsubsection{Energy-Momentum tensor}
Our action must be invariant under space-time translations.
The infinitesimal transformation
\begin{equation}
x^\mu \mapsto x^{\mu '} = x^\mu + \delta a^\mu
\end{equation}
has 4 continuous parameters $\delta a^\mu$ and in our notation
$X_\nu^\mu = \delta_\nu^\mu$. 
The field is invariant
\begin{equation}
\varphi (x) 
\mapsto
\varphi' (x') 
=
\varphi (x) 
\end{equation}
and $\Phi_{\nu} = 0$.
The Noether current (\ref{eqn:NoetherCurrentGeneral}) reads
\begin{eqnarray}
T^\mu_{\;\;\nu}
=
 \frac{\partial {\cal L}}{\partial (\partial_\mu \varphi)}
\partial_\nu \varphi(x) 
-
{\cal L}(x) 
g_\nu^\mu 
\label{eqn:NoetrherEnergyMomentum}
\end{eqnarray}

\subsubsection{Angular Momentum tensor}
Consider an infinitesimal spatial rotation
\begin{equation}
x^\mu \mapsto x^{\mu '} = x^\mu + \delta \omega^\mu_{\;\nu} x^\nu
\end{equation}
This transformation has 6 continuous parameters $\delta \omega^{\mu \nu} = -\delta \omega^{\nu \mu}$.
$\delta x^\mu$ in Eq. (\ref{eqn:Noeth_infsimtransf}) is given as
\begin{eqnarray}
\delta x^\mu &=& \delta \omega^\mu_{\;\nu} x^\nu
\nonumber\\
&=&
g^\mu_\xi g_{\nu \eta} \delta \omega^{\xi \eta} x^\nu
\nonumber\\
&=&
\frac{1}{2}( g^\mu_\xi g_{\nu \eta} - g^\mu_\eta g_{\nu \xi} ) x^\nu \delta \omega^{\xi \eta}
\nonumber\\
&=&
\frac{1}{2} (a_{\xi \eta})^\mu_{\;\nu} x^\nu \delta \omega^{\xi \eta}
\end{eqnarray}
where
\begin{eqnarray}
(a_{\xi \eta})^\mu_{\; \nu}
=
g^\mu_\xi g_{\nu \eta} - g^\mu_\eta g_{\nu \xi}
\end{eqnarray}
We have 
\begin{eqnarray}
X^\mu_{\;\;\xi \eta}
=
\frac{1}{2} (a_{\xi \eta})^\mu_{\;\nu} x^\nu
\end{eqnarray}
The field is transformed as
\begin{eqnarray}
\varphi_\alpha (x) 
\mapsto
\varphi_\alpha' (x') 
=
\delta \varphi_\alpha (x) 
\,,
\hspace{3mm}
\delta \varphi_\alpha (x) 
= 
\frac{1}{2} (S_{\xi \eta})_\alpha^{\;\beta}
\varphi_\beta (x) 
\delta \omega^{\xi \eta}
\end{eqnarray}
where
\begin{eqnarray}
(S_{\mu \nu})_\alpha^{\;\beta}
=
\left\{
\begin{array}{l}
0
\hspace{3mm}
\cdots
\mbox{scalar}
\\
(a_{\mu \nu})_\alpha^{\;\;\beta}
\hspace{3mm}
\cdots
\mbox{vector}
\\
\frac{1}{4} [ \gamma_\mu, \gamma_\nu ]_{\alpha}^{\;\;\beta}
\hspace{3mm}
\cdots
\mbox{Dirac spinor}
\end{array}
\right.
\end{eqnarray}
We have
\begin{eqnarray}
\Phi_{\alpha \xi \eta} = 
\frac{1}{2} (S_{\xi \eta})_\alpha^{\;\beta}
\varphi_\beta (x) 
\end{eqnarray}
The Noether current
\begin{eqnarray}
M^\mu_{\xi \eta}
&\leftdef&
2 J^\mu_{\;\xi \eta}
\nonumber\\
&=&
-2 \frac{\partial {\cal L}}{\partial \varphi_{\alpha : \mu}}
\left(
\Phi_{\alpha \xi \eta} (x) - \partial_\nu \varphi_\alpha(x) X_{\;\xi \eta}^\nu 
\right)
-2
{\cal L}(x) 
X_{\;\xi \eta}^\mu 
\nonumber\\
&=&
\left(
 \frac{\partial {\cal L}}{\partial \varphi_{\alpha : \mu}}
\partial_\nu \varphi_\alpha(x) 
- 
{\cal L} g^\mu_\nu
\right)
\cdot 2X^\nu_{\;\;\xi \eta}
- 2  \frac{\partial {\cal L}}{\partial \varphi_{\alpha : \mu}}
\Phi_{\alpha \xi \eta} (x)
\nonumber\\
&=&
(a_{\xi \eta})^\nu_{\;\rho} x^\rho \cdot
T^\mu_{\;\nu}
-  \frac{\partial {\cal L}}{\partial \varphi_{\alpha : \mu}}
(S_{\xi \eta})_\alpha^{\;\beta}
\varphi_\beta (x) 
\\
&\rightdef&
L^\mu_{(\xi \eta)} + S^\mu_{(\xi \eta)}
\end{eqnarray}
The last equation defines orbital and spin parts of the current.
The total angular momentum vector is defined as
\begin{eqnarray}
J^{k} = \frac{1}{2} \epsilon^{kij} M^{ij}\,,
\hspace{3mm}
M_{\xi \eta} = \int d^3 \bld{x} M^0_{\xi \eta}
\label{eqn:NoetherAngularMomentum}
\end{eqnarray}

%===================================================================
\subsubsection{Electric Charge}
For a complex field, gauge transformation of the first kind is defined as
\begin{eqnarray}
\varphi(x) \mapsto \varphi'_\alpha(x) = e^{ie \theta} \varphi(x)\,,
\hspace{3mm}
\varphi^*(x) \mapsto {\varphi^{*}}'(x)= e^{-ie \theta} \varphi^*(x)
\end{eqnarray}
This transformation consists a U(1). Considering infinitesimal $\theta$, 
we read in Eq. (\ref{eqn:Noeth_infsimtransf}) that
\begin{eqnarray}
X^\mu = 0\,,
\hspace{3mm}
\Phi = i e \varphi\,,
\hspace{3mm}
\Phi^* = -i e \varphi^*\,,
\end{eqnarray}
Invariance of the ${\cal L}$agragian density leads
\begin{equation}
\partial_\mu J^\mu = 0
\end{equation}
for
\begin{eqnarray}
J^\mu = 
ie \left(
\varphi^*_{} \frac{\partial {\cal L}}{\partial (\partial_\mu \varphi^*_{})} 
-\frac{\partial {\cal L}}{\partial (\partial_\mu \varphi_{})} \varphi_{}
\right)
\label{eqn:NoetherElectricCharge}
\end{eqnarray}
The corresponding charge is $Q = \int d^3 \bld{x} J^0$.

%===================================================================
\subsubsection{Internal Global Symmetries}
We consider a internal global SU(n) symmetry. 
The field $\varphi_a(x)$ is subject to a transformation
\begin{eqnarray}
\varphi_a(x)
\mapsto
\varphi'_a(x)
=
(e^{i \alpha_i G_i})_a^{\;b}
\varphi_b(x)
\label{eqn:LieGrTransfFields}
\end{eqnarray}
where $\alpha_i, (i = 1, \dots, n^2-1)$ are continuous real parameters, $G_i$ are matrix representations of
generators. 
Considering infinitesimal $\alpha_i$, we have in Eq. (\ref{eqn:Noeth_infsimtransf}) that
\begin{equation}
\Phi_{a i} = i (G_i)_a^{\;b} \varphi_b\,,
\end{equation}
The Noether current is given by
\begin{eqnarray}
J^\mu_i
&=&
-
i
\frac{\partial {\cal L}}{\partial (\partial_\mu \varphi_{a })}
(G_i)_a^{\;b} \varphi_b
\label{eqn:NoetherCurrGeneral}
\end{eqnarray}
and the corresponding $n^2 - 1$ charges are
\begin{eqnarray}
C_i
=
-i
\int d^3 \bld{x}
\frac{\partial {\cal L}}{\partial \dot{\varphi_{a }}}
(G_i)_a^{\;b} \varphi_b
\label{eqn:NoetherChrgGeneral}
\end{eqnarray}
Among $n^2-1$ generators $G_i$, only $n-1$ commutes to each other.
Accordingly, $n-1$ charges among $n^2-1$ can be diagonalized at the same time.

\bigskip

\noindent
■SU(2)\\
The conserved (internal) vector is the isospin $\bld{I} = (C_1, C_2, C_3)$.
\begin{eqnarray}
\varphi_a \in
\yng(1)&:&
\hspace{3mm}
G_i = \frac{1}{2} \sigma_i\,
\hspace{3mm}
{\scriptstyle \tiny
\sigma_1 = \left[
\begin{array}{cc}
0&1 \\ 1&0
\end{array}
\right]\,,
\sigma_2 = \left[
\begin{array}{cc}
\scriptstyle
0&-i \\ i&0
\end{array}
\right]\,,
\sigma_3 = \left[
\begin{array}{cc}
\scriptstyle
1&0 \\ 0&-1
\end{array}
\right]
} %scriptstyle, tiny
\label{eqn:SU2GenFundamental}
\\
\varphi_a \in
\yng(2)&:&
\hspace{3mm}
G_i = t_i\,,
\hspace{3mm}
(t_i)_{jk} = -i \epsilon_{ijk}
\end{eqnarray}


\noindent
■SU(3)\\
\hspace{15mm}$\varphi_a \in \yng(1):$
\hspace{7mm}$G_i = \frac{1}{2} \lambda_i$
\begin{eqnarray}
\begin{array}{ccc}
\lambda_1 =
\left(
\begin{array}{ccc}
0&1&0\\
1&0&0\\
0&0&0\\
\end{array}
\right)
&
\lambda_2 =
\left(
\begin{array}{ccc}
0&-i&0\\
i&0&0\\
0&0&0\\
\end{array}
\right)
&
\lambda_3 =
\left(
\begin{array}{ccc}
1&0&0\\
0&-1&0\\
0&0&0\\
\end{array}
\right)
\vspace{2mm}
\\
\lambda_4 =
\left(
\begin{array}{ccc}
0&0&1\\
0&0&0\\
1&0&0\\
\end{array}
\right)
&
\lambda_5 =
\left(
\begin{array}{ccc}
0&0&-i\\
0&0&0\\
i&0&0\\
\end{array}
\right)
&
\lambda_6 =
\left(
\begin{array}{ccc}
0&0&0\\
0&0&1\\
0&1&0\\
\end{array}
\right)
\vspace{2mm}
\\
\lambda_7 =
\left(
\begin{array}{ccc}
0&0&0\\
0&0&-i\\
0&i&0\\
\end{array}
\right)
&
\lambda_8 =
\frac{1}{\sqrt{3}}
\left(
\begin{array}{ccc}
1&0&0\\
0&1&0\\
0&0&-2\\
\end{array}
\right)
&
\end{array}
\end{eqnarray}
\begin{eqnarray}
%\hspace{15mm}
\varphi_a \in \yng(2,1):
\hspace{3mm}
G_i = T_i\,,
\hspace{3mm}
(T_a)_{bc} = -i f_{abc}\mbox{(structure const.)}
%\hspace{25mm}
\end{eqnarray}
\begin{eqnarray}
\begin{array}{c}
T_\pm = G_1 \pm iG_2\,,
\hspace{3mm}
V_\pm = G_4 \pm iG_5\,,
\hspace{3mm}
U_\pm = G_6 \pm iG_7\,,
\\
T_3 = G_3\,,
\hspace{3mm}
Y = \frac{2}{\sqrt{3}} G_8
\end{array}
\end{eqnarray}
%===================================================================
%===================================================================
%===================================================================

\subsection{Interacting Fields (1/2)}
\subsubsection{The Interaction Picture}
●Schr\"odinger Picture\\
For a given Hamiltonian $H$,
\begin{eqnarray}
\left\{
\begin{array}{l}
i \partial_t 
\ketend \Psi \ket_S
=
H \ketend \Psi \ket_S
\vspace{2mm}
\\
i \partial_t 
{\cal O}_S = 0
\end{array}
\right.
\end{eqnarray}
No explicit $t$ dependence in ${\cal O}_S$ is assumed.
Formal solution to the Schr\"odinger equation is written as
\begin{eqnarray}
\ketend \Psi \ket_S = e^{-iHt} \ketend \Psi_0 \ket_S
\label{eqn:SchrPicFormalSol}
\end{eqnarray}
Expectation value of ${\cal O}_S$ in a state $\ketend \Psi \ket_S$ is given as
\begin{equation}
{}_S\!\bra \Psi \braend {\cal O}_S \ketend \Psi \ket_S
\end{equation}

\bigskip

\noindent
●Heisenberg Picture\\
Related to the Schr\"odinger picture by
\begin{eqnarray}
\left\{
\begin{array}{l}
\ketend \Psi \ket_H
=
e^{iHt}
\ketend \Psi \ket_S
\vspace{2mm}
\\
{\cal O}_H = 
e^{iHt} {\cal O}_S e^{-iHt}
\end{array}
\right.
\end{eqnarray}
so that
\begin{equation}
{}_H\!\bra \Psi \braend {\cal O}_H \ketend \Psi \ket_H
=
{}_S\!\bra \Psi \braend {\cal O}_S \ketend \Psi \ket_S
\end{equation}
They evolve in time as
\begin{eqnarray}
\left\{
\begin{array}{l}
i \partial_t 
\ketend \Psi \ket_H
=
0
\vspace{2mm}
\\
i \partial_t 
{\cal O}_H = 
[ {\cal O}_H, H]
\end{array}
\right.
\end{eqnarray}

\bigskip

\noindent
●The Interaction Picture
\begin{eqnarray}
H = H_0 + H_{int}
\label{eqn:HamiltonianDecomposed}
\end{eqnarray}
Related to the Schr\"odinger picture by
\begin{eqnarray}
\left\{
\begin{array}{l}
\ketend \Psi \ket_I
=
e^{iH_0 t}
\ketend \Psi \ket_S
\vspace{2mm}
\\
{\cal O}_I = 
e^{iH_0 t} {\cal O}_S e^{-iH_0 t}
\end{array}
\right.
\end{eqnarray}
so that
\begin{equation}
{}_I\!\bra \Psi \braend {\cal O}_I \ketend \Psi \ket_I
=
{}_S\!\bra \Psi \braend {\cal O}_S \ketend \Psi \ket_S
\end{equation}
The interaction Hamiltonian in this picture is time dependent;
\begin{eqnarray}
H_I \leftdef (H_{int})_I = 
e^{iH_0 t} H_{int} e^{-iH_0 t}
\label{eqn:defHintI}
\end{eqnarray}
$\ketend \Psi \ket_I$ and ${\cal O}_I$ 
evolve in time as
\begin{eqnarray}
\left\{
\begin{array}{l}
i \partial_t 
\ketend \Psi \ket_I
=
e^{i H_0 t} (-H_0 + H) \ketend \Psi \ket_S
\vspace{2mm}
\\
\hspace{15mm}
=
e^{i H_0 t} H_{int} e^{-i H_0 t}e^{i H_0 t} \ketend \Psi \ket_S
\vspace{2mm}
\\
\hspace{15mm}
=
H_I \ketend \Psi \ket_I
\vspace{2mm}
\\
i \partial_t 
{\cal O}_I = 
[ {\cal O}_I, H_0]
\end{array}
\right.
\label{eqn:InteractionPicTime}
\end{eqnarray}

%=====================================================
\subsubsection{Dyson's Formula}
Formal solution of Eq. (\ref{eqn:InteractionPicTime}) 
can not be written in the form like one in Eq. (\ref{eqn:SchrPicFormalSol})
since $H_I$ is time dependent.
Writing the solution as
\begin{eqnarray}
\begin{array}{l}
\ketend \Psi(t) \ket_I
=
U(t,t_0)
\ketend \Psi(t_0) \ket_I
\vspace{2mm}
\\
U(t,t) = 1
\hspace{3mm} \mbox{and}
\hspace{3mm}
 U(t_3,t_2) U(t_2,t_1) = U(t_3,t_1) \,,
\end{array}
\label{eqn:timeEvolOpDef}
\end{eqnarray}
the time evolution unitary operator $U(t,t_0)$ is given as
\begin{eqnarray}
U(t, t_0) =
T exp \left(
-i \int_{t_0}^t H_I (t') dt'
\right)\,,
\label{eqn:DysonFormula}
\end{eqnarray}
where $T$ stands for time ordered product 
\cite{ref:Peskin-Schroeder, ref:Itzykson-Zuber, ref:NIsh.1-2, ref:Hioki, ref:Tong}
\begin{eqnarray}
T[ {\cal O}_2(t')  {\cal O}_1 (t) ]=
\theta(t' - t) {\cal O}_2(t')  {\cal O}_1 (t) +
\theta(t - t'){\cal O}_1(t)  {\cal O}_2 (t') \,.
\label{eqn:TProdDef}
\end{eqnarray}
In fact, for $t > t_0$
\begin{eqnarray}
i\partial_t U(t, t_0)
&=&
T \left[
H_I(t) 
 exp \left(
-i \int_{t_0}^t H_I (t') dt'
\right)
\right]
\nonumber\\
&=&
H_I(t) 
T exp \left(
-i \int_{t_0}^t H_I (t') dt'
\right)
\nonumber\\
&=&
H_I(t) 
U(t, t_0)
\end{eqnarray}
and the conditions for $U(t, t_0)$ in Eq. (\ref{eqn:timeEvolOpDef})
is obviously satisfied.
We may write Eq. (\ref{eqn:DysonFormula}) in a form of series as
\begin{eqnarray}
U(t, t_0) &=&
1 -i \int_{t_0}^t H_I (t') dt'
\nonumber\\
&&+
\frac{(-i)^2}{2}
\int_{t_0}^t dt'
\int_{t_0}^t dt''
\;T[ H_I (t') H_I (t'')]
+ \cdots
\label{eqn:TimeDevPerturbSerTprod}
\end{eqnarray}
However,
\begin{eqnarray*}
\int_{t_0}^t dt'
\int_{t_0}^t dt''
\;T[ {\cal O} (t') {\cal O} (t'')]
&=&
\int_{t_0}^t dt'
\int_{t_0}^{{t'}} dt''
\; {\cal O} (t') {\cal O} (t'')
+
\int_{t_0}^t dt'
\int_{t'}^t dt''
\; {\cal O} (t'') {\cal O} (t')
\nonumber\\
&=&
2
\int_{t_0}^t dt'
\int_{t_0}^{{t'}} dt''
\; {\cal O} (t') {\cal O} (t'')
\end{eqnarray*}
and
\begin{eqnarray}
U(t, t_0) &=&
1 -i \int_{t_0}^t H_I (t') dt'
\nonumber\\
&&+
(-i)^2
\int_{t_0}^t dt'
\int_{t_0}^{t'} dt''
 H_I (t') H_I (t'')
+ \cdots
\label{eqn:TimeDevOpPerturbSer}
\end{eqnarray}

%=====================================================
\subsubsection{The S-matrix}
Dealing with the scattering, we {\bf \textit{assume}} that initial and final states are
eigenstates of $H_0$. To be more concrete, let ${\cal O}_0$ be a complete set of
observables that includes $H_0$ and no one of them depends on $t$ in an explicit manner.
In the interaction picture, observables in ${\cal O}_0$ are constant as operators. [See Eq.
(\ref{eqn:InteractionPicTime}).] What we have assumed is that the initial and final states
are eigenstates of ${\cal O}_0$.

The scattering process takes place as follows. At $t_1 \to -\infty$, the system is in the initial
state $\ketend i \ket$, the system evolves in time by $U(t_2, t_1)$ under the effect of
$H_I$, then, at $t_2 \to \infty$, the system turns in to the scattered state.
The amplitude to find a final state $\ketend f \ket$ in the scattered state is 
\begin{eqnarray}
\lim_{\stackrel{\scriptstyle t_1 \to -\infty}{t_2 \to \infty}}
\bra f \braend U(t_2, t_1) \ketend i \ket
\rightdef
\bra f \braend S \ketend i \ket
\label{eqn:SmatrixDef}
\end{eqnarray}
This is the definition of the S-matrix. Obviously, $S$ is unitary.
Substituting Eq. (\ref{eqn:TimeDevPerturbSerTprod}) or (\ref{eqn:TimeDevOpPerturbSer}) 
in Eq. (\ref{eqn:SmatrixDef}), we 
may consider the operator $S$ as given in the form of a perturbation series.
When the expression of Eq.  (\ref{eqn:TimeDevPerturbSerTprod}) is used, it reads
\begin{eqnarray}
 S -1 
&=&
 -i \int_{-\infty}^\infty H_I (t') dt'
\nonumber\\
&&+
\frac{(-i)^2}{2}
\int_{-\infty}^\infty dt'
 dt''
\;T[ H_I (t') H_I (t'')]
+ \cdots
\label{eqn:SmatrixPertSer}
\end{eqnarray}

%========================================  section 3.5
\subsection{Scalar Fields}
Irreducible representations of the Loerentz group solely 
determines forms of free field equations.
Irreducible representations are classified by
spins of particles.
\subsubsection{Real Scalar Free Field}
Spinless and neutral.
\begin{eqnarray}
{\cal L} &=& \frac{1}{2} 
\left( \underline{\partial} \varphi(\underline{x}) \right)^2
-\frac{1}{2} m^2 \varphi^2(\underline{x})
\nonumber\\
&=& \frac{1}{2} 
\partial_\mu \varphi(\underline{x})\cdot \partial^\mu \varphi(\underline{x})
-\frac{1}{2} m^2 \varphi^2(\underline{x})
\label{eqn:RSCLagrangiandensity}
\end{eqnarray}
where a dot in the last equation indicates the former derivative acting only on the first $\varphi$.
Under a Lorentz transformation $\underline{x} \mapsto \underline{x}' = L\underline{x}$,
the field $\varphi(\underline{x})$ transforms as
\begin{equation}
\varphi(x) \mapsto \varphi'(x') = \varphi(x)\,.
\end{equation}
Here and hereafter, we omitt underlines on Lorentz vectors.
\begin{equation}
\begin{array}{c}
\displaystyle
\frac{\partial {\cal L}}{\partial(\partial_\mu \varphi)} = \partial^\mu \varphi
\\
\displaystyle
\pi(x) = \frac{ \partial {\cal L}}{ \partial \dot{\varphi}(x) } = \dot{\varphi}
%\pi_{}(x) = \partial_0 \varphi_{cl}(x)
%=
%\int \frac{d^3 \bld{k}}{\sqrt{(2\pi)^3}} (- i k^0 ) \left[
%a(\bld{k}) e^{-i k \cdot x} - b^\dagger(\bld{k})e^{i k \cdot x} \right]
\end{array}
\end{equation}
Euler-Lagrange equation
\begin{equation}
\left( \Box + m^2 \right) \varphi_{}(x) = 0
\hspace{10mm}
\mbox{Klein-Gordon}
\label{eqn:Klein-Gordon}
\end{equation}
where $\Box \leftdef \partial^2 = \partial_\mu \partial^\mu = \partial_0^2 - \bld{\partial}^2$.\\
Classical solution of the field equation (\ref{eqn:Klein-Gordon}) is written as
\footnote{%------------------------------- footnote >>
The following consideration lays behind:
\begin{eqnarray*}
\varphi(x)
&=&
\frac{1}{\sqrt{(2\pi)^3}}
\int 
d^4 k
\delta(k^2 - m^2)
\left[
\theta(k^0) + \theta(-k^0)
\right]
\tilde{\varphi} (k)
e^{-i k \cdot x}
\nonumber\\
&=&
\frac{1}{\sqrt{(2\pi)^3}}
\int 
d^4 k
\delta(k^2 - m^2)
\left[
\theta(k^0)
\tilde{\varphi} (k)
e^{-i k \cdot x}
 + \theta(k^0)
\tilde{\varphi} (-k)
e^{i k \cdot x}
\right]
\nonumber\\
&=&
\int 
\frac{d^3 \bld{k}}{\sqrt{(2\pi)^3}2 k^0}
\left[
\tilde{\varphi} (k)
e^{-i k \cdot x}
 + 
\tilde{\varphi} (-k)
e^{i k \cdot x}
\right]
\nonumber\\
&\rightdef&
\int 
\frac{d^3 \bld{k}}{\sqrt{(2\pi)^3}2 k^0}
\left[
a (\bld{k})
e^{-i k \cdot x}
 +
 a^*(\bld{k}) 
e^{i k \cdot x}
\right]
\end{eqnarray*}
}%------------------------------- footnote //
\begin{equation}
\varphi_{cl}(x) = \int \frac{d^3 \bld{k}}{\sqrt{(2\pi)^3}2k^0} \left[
a(\bld{k}) e^{-i k \cdot x} + a^*(\bld{k})e^{i k \cdot x} \right]
\label{eqn:KGclassicalsol}
\end{equation}
where $k^0 = + \sqrt{\bld{k}^2 + m^2}$. 
Canonical conjugate field reads,
\begin{equation}
\pi_{cl}(x) = \dot{\varphi}_{cl}(x)
=
\int \frac{d^3 \bld{k}}{\sqrt{(2\pi)^3}2k^0}  \left[
- i k^0 a(\bld{k}) e^{-i k \cdot x} + i k^0 a^*(\bld{k})e^{i k \cdot x} \right]
\label{eqn:KGclassicalpi}
\end{equation}
Hamiltonian density in the classical level is evaluated from Eq. (\ref{eqn:Hamiltoniandensity}) as
\begin{eqnarray}
{\cal H} &=& \pi(x) \dot{\varphi}(x) - \frac{1}{2} \left\{
\left( \dot{\varphi}(x) \right)^2 - 
\left( \bld{\partial} \varphi(x) \right)^2 \right\}
+ \frac{1}{2} m^2 \varphi^2(x)
\nonumber\\
&=&
\frac{1}{2} \left\{
\pi^2 (x) + \left( \bld{\partial} \varphi(x) \right)^2 + m^2 \varphi^2(x)
\right\}
\label{eqn:KGHamiltoniandens}
\end{eqnarray}
Canonical quantization
\begin{equation}
\begin{array}{c}
[ \varphi(t,\bld{x}), \pi(t, \bld{y})] = i \delta^3 (\bld{x} - \bld{y} )\,,
\vspace{2mm}
\\
\left[ \varphi(t,\bld{x}), \varphi(t,\bld{y}) \right] = 0\,,
\;\;\;\;\left[ \pi(t,\bld{x}), \pi(t,\bld{y}) \right] = 0\,.
\end{array}
\label{eqn:cancomm_eqtime_RS}
\end{equation}
The fields $\varphi(x)$ and $\pi(x)$ are settled as operators at a time $t$.
As Heisenberg operators,
they obey the Heisenberg equation (\ref{eqn:HeisenbergEOM}) with quantum Hamiltonian
%$\hat{H} = \int d^3 \bld{x} \hat{{\cal H}}(x)$ with $\hat{{\cal H}}(x)$ given as
%the normal product of the $r.h.s$ of Eq. (\ref{eqn:KGHamiltoniandens})
% fields replaced by 
given by
\begin{equation}
\hat{H} = \int d^3 \bld{x} \hat{{\cal H}}(x)\,,
\hspace{3mm}
\hat{{\cal H}}(x) = 
\normord{ \frac{1}{2} \left\{
\pi^2 (x) + \left( \bld{\partial} \varphi(x) \right)^2 + m^2 \varphi^2(x)
\right\}
}\,,
\label{eqn:KG_Hamiltonian}
\end{equation}
where $\normord{\dots}$ denotes the normal product.
Since we have the Hamiltonian,
the Heisenberg equation (\ref{eqn:HeisenbergEOM}) is equivalent to the
quantum level Euler equation (\ref{eqn:Klein-Gordon}) and its solution 
and the canonical conjugate field are written as
\begin{equation}
\varphi_{}(x) = \int \frac{d^3 \bld{k}}{\sqrt{(2\pi)^3}2k^0} \left[
a(\bld{k}) e^{-i k \cdot x} + a^\dagger(\bld{k})e^{i k \cdot x} \right]\,,
\label{eqn:RS_fieldexpansion}
\end{equation}
\begin{equation}
\pi_{}(x) 
=
\frac{-i}{2}
\int \frac{d^3 \bld{k}}{\sqrt{(2\pi)^3}}  \left[
a(\bld{k}) e^{-i k \cdot x} - a^\dagger(\bld{k})e^{i k \cdot x} \right]\,,
\label{eqn:RS_pifieldexpansion}
\end{equation}
They have the same form as Eqs. (\ref{eqn:KGclassicalsol}) and (\ref{eqn:KGclassicalpi})
but now the coefficients $a(\bld{k})$ and $a^\dagger(\bld{k})$ are
quantum operators defined by the equal-time commutation relations (\ref{eqn:cancomm_eqtime_RS})
at time $x^0 = t$.
The requirements of Eq. (\ref{eqn:cancomm_eqtime_RS}) is equivalent with
requiring
\begin{equation}
\begin{array}{c}
[ a(\bld{k}), a^\dagger(\bld{k}')] = 2k^0 \delta^3 (\bld{k} - \bld{k'} )\;,
\vspace{2mm}
\\
\left[ a(\bld{k}), a(\bld{k}') \right] = 0\;,\;\;\;\;\left[ a^\dagger(\bld{k}), a^\dagger(\bld{k}') \right] = 0
\end{array}
\label{eqn:cancomm_RS}
\end{equation}

\verb/-----------.-----------.-----------.-----------.-----------/\\
\vspace{-3mm}
{\small
\begin{center}
Addendum: Proof of (\ref{eqn:cancomm_eqtime_RS}) $\Leftrightarrow$ (\ref{eqn:cancomm_RS})
\end{center}
Proof of the necessity of Eq. (\ref{eqn:cancomm_RS}) is straightforward,
We show the sufficiency
of Eq. (\ref{eqn:cancomm_eqtime_RS}) in the following.
We may write Eqs. (\ref{eqn:RS_fieldexpansion}) and (\ref{eqn:RS_pifieldexpansion})
as
\begin{equation}
\begin{array}{l}
\displaystyle
\varphi(x) = \int \frac{d^3\bld{k}}{\sqrt{(2\pi)^3}} Q_{\bld{k}} (t) e^{i \bld{k} \cdot \bld{x}}\,,
\\
\displaystyle
\pi_{}(x)  = \int \frac{d^3\bld{k}}{\sqrt{(2\pi)^3}} P_{\bld{k}} (t) e^{- i \bld{k} \cdot \bld{x}}\,,
\end{array}
\end{equation}
with
\begin{equation}
Q_{\bld{k}}(t)
=
\frac{1}{2k^0} \left[
a(\bld{k}) e^{-i k^0 t} + a^\dagger(-\bld{k}) e^{i k^0 t} \right]
\label{eqn:RSC_Qk_def}
\end{equation}
\begin{equation}
P_{\bld{k}}(t)
=
\frac{i}{2} \left[
a^\dagger(\bld{k}) e^{i k^0 t} - a(-\bld{k}) e^{- i k^0 t} \right]
\label{eqn:RSC_Pk_def}
\end{equation}
Relationships $Q^\dagger_{\bld{k}} = Q_{-\bld{k}}$ and
$P^\dagger_{\bld{k}} = P_{-\bld{k}}$ ensure 
that $\varphi$ and $\pi$ are real.
From the linear independence of Fourier components, we have
\begin{equation}
\begin{array}{l}
0 = [\varphi(t, \bld{x}), \varphi(t, \bld{y})]
\;\Longleftrightarrow\;
[Q_{\bld{k}}(t), Q_{\bld{k'}}(t)] = 0\,,
\vspace{2mm}
\\
0 = [\pi(t, \bld{x}), \pi(t, \bld{y})]
\;\Longleftrightarrow\;
[P_{\bld{k}}(t), P_{\bld{k'}}(t)] = 0\,,
\end{array}
\label{eqn:RSC_QQcomm}
\end{equation}
and
\begin{eqnarray}
i\delta^3(\bld{x} - \bld{y})
&=&
\int \frac{d^3 \bld{k} d^3 \bld{k}'}{(2\pi)^3}
i \delta^3 (\bld{k} - \bld{k}')
e^{i \bld{k} \cdot \bld{x} - i \bld{k}' \bld{y}}
\nonumber\\
&=&
[\varphi(t, \bld{x}), \pi(t, \bld{y}) ]
\nonumber\\
&=&
\int \frac{d^3 \bld{k} d^3 \bld{k}'}{(2\pi)^3} [Q_{\bld{k}}, P_{\bld{k}'}]
e^{i \bld{k} \cdot \bld{x} - i \bld{k}' \cdot \bld{y}}
\nonumber\\
&\Longleftrightarrow&    
\nonumber\\
\left[ Q_{\bld{k}}, P_{\bld{k}'} \right] &=& i \delta^3 (\bld{k} - \bld{k}')
\label{eqn:RSC_QPcomm}
\end{eqnarray}
Eqs. (\ref{eqn:RSC_Qk_def}) and (\ref{eqn:RSC_Pk_def})
reads,
\begin{equation}
a(\bld{k}) = \left( k^0 Q_{\bld{k}} (t) + i P_{\bld{k}}^\dagger (t) \right)
e^{i k^0 t}\,,
\label{eqn:RSC_a_by_QP} 
\end{equation}
\begin{equation}
a^\dagger(\bld{k}) = \left( k^0 Q_{\bld{k}}^\dagger (t) - i P_{\bld{k}} (t) \right)
e^{- i k^0 t}\,,
\label{eqn:RSC_a_by_QP} 
\end{equation}
and Eq. (\ref{eqn:cancomm_RS}) follows
from Eqs. (\ref{eqn:RSC_QQcomm}) and (\ref{eqn:RSC_QPcomm}).
}\\
\verb/-----------.-----------.-----------.-----------.-----------/\\


%=================================

Notice that $dim [\varphi] = E^1$ as can be seen from Eq. (\ref{eqn:RSCLagrangiandensity})
and $dim [\pi] = E^2$. Thus they are physical quantities  quite different from
those in the case of the Schr\"odinger field theory.\\

We are now making use of a fact that operators in Eq. (\ref{eqn:cancomm_RS}) satisfiy
the condition of the bosonic creation-annihilation operators, (\ref{eqn:Nbodycreanncomm}), 
for the case of continuous eigenvalues.
The total number operator can be defined as
\begin{equation}
\hat{N} = \int \frac{d^3 \bld{k}}{2k^0} a^\dagger(\bld{k}) a(\bld{k})
\label{eqn:KG_totalNumberOp}
\end{equation}
It holds that
\begin{equation}
\begin{array}{l}
\displaystyle
\hat{N} a(\bld{k}) = 
\int \frac{d^3 \bld{k}'}{2k^{0'}} \left\{
a(\bld{k}) a^\dagger(\bld{k}') - 2 k^0 \delta^3(\bld{k} - \bld{k}') \right\} a(\bld{k}')\\
\hspace{12mm} 
= a(\bld{k}) (\hat{N} - 1)\,,
\\
\hat{N} a^\dagger(\bld{k}) 
= a^\dagger(\bld{k}) (\hat{N} + 1)
\end{array}
\end{equation}
The Hamiltonian (\ref{eqn:KG_Hamiltonian}) reads from 
Eqs. (\ref{eqn:RS_fieldexpansion}) and (\ref{eqn:RS_pifieldexpansion}) as,
\begin{equation}
\hat{H} = \int \frac{d^3 \bld{k}}{2k^0} k^0 a^\dagger(\bld{k}) a(\bld{k})
\label{eqn:KG_Hamiltonian_aadagg}
\end{equation}
□Total momentum comes here
%-------------------------------------------
\footnote{
Applying  Noether's theorem to the invariance under space-time translations,
we have an expression for the conserved energy-momentum vector as
\begin{equation*}
P^\mu = \int T^{0 \mu} (x) d^3 \bld{x}
\end{equation*}
where the conserved Noether current is given as
\begin{equation*}
T^{\mu}_{\;\nu} (x) =
\frac{\partial {\cal L}}{\partial (\partial_\mu \varphi(x))}
\partial_\nu \varphi(x) 
- g^\mu_\nu \,{\cal L}
\end{equation*}
\begin{equation*}
\hat{P}^\mu =  \int \frac{d^3 \bld{k}}{2k^0} k^\mu a^\dagger(\bld{k}) a(\bld{k})
\end{equation*}
} % footnote end
\\
We have
\begin{equation}
[ \hat{P}^\mu, \hat{N}] = 0
\label{eqn:commutingHandN}
\end{equation}
and this relationship establishes particle interpretation.
Namely a state $a^\dagger({\bld{k}})\ketend 0 \ket$ is interpreted as
one particle eigenstate of the momentum associated with an eigenvalue $\bld{k}$.
\begin{equation}
\hat{\bld{P}} \hat{a}^\dagger(\bld{k}) \ketend 0 \ket
=
\bld{k} \hat{a}^\dagger(\bld{k}) \ketend 0 \ket
\end{equation}
so that we may write
\begin{equation}
\hat{a}^\dagger(\bld{k}) \ketend 0 \ket = \ketend \bld{k} \ket
\end{equation}
State normalization
\[
\bra \bld{p} \braend \bld{p}' \ket
= \bra 0 \braend [a(\bld{p}), a^\dagger(\bld{p}')] \ketend 0 \ket
= 2 k^0 \delta^3 (\bld{p} - \bld{p}')
\]
The Hamiltonian (\ref{eqn:KG_Hamiltonian_aadagg}) is diagonalized
by creation-annihilation operators:
\begin{equation}
[ \hat{H}, \hat{a}^\dagger(\bld{k}) ] = k^0 \hat{a}^\dagger(\bld{k})
\,, \hspace{3mm}
[ \hat{H}, \hat{a}(\bld{k}) ] = - k^0 \hat{a}(\bld{k})
\label{eqn:KGcommHanda}
\end{equation}
so that
\begin{equation}
\hat{H} \hat{a}^\dagger(\bld{k}) \ketend 0 \ket
=
k^0 \hat{a}^\dagger(\bld{k}) \ketend 0 \ket
\end{equation}
Wave function:\\
For
\begin{equation*}
\ketend \Psi^{(1)} \ket
\leftdef
\int \frac{d^3 \bld{k}}{2k^0}  \Psi^{(1)} (\bld{k}) a^\dagger(\bld{k}) \ketend 0 \ket\,,
\end{equation*}
\begin{equation*}
\hat{N} \ketend \Psi^{(1)} \ket = \ketend \Psi^{(1)} \ket\,,
\end{equation*}
\begin{equation*}
\bra \bld{k} \braend \Psi^{(1)} \ket
=  \Psi^{(1)} (\bld{k})
\end{equation*}
The state is normalized through a relationship
\begin{eqnarray*}
\bra \Psi^{(1)} \braend \Psi^{(1)} \ket
=
\int \frac{d^3 \bld{k}}{2k^0}  |\Psi^{(1)}(\bld{k})|^2
\end{eqnarray*}

A state of $n$ particles at momenta $\bld{k}_1$, $\bld{k}_2, \dots \bld{k}_n$,  
is written as
\begin{eqnarray*}
\ketend \bld{k}_1 \cdots \bld{k}_n \ket
=
\frac{1}{\sqrt{n!}}
a^{\dagger}(\bld{k}_1) \cdots a^{\dagger}(\bld{k}_n)
\ketend 0 \ket
\end{eqnarray*}
Since $a^{\dagger}(\bld{k}_1), \dots a^{\dagger}(\bld{k}_n)$
commute among themselves, the state is symmetric under
change of orders of momenta.
We have
\footnote{%----------------------------------------------footnote >>
Useful relationships
\begin{eqnarray*}
\left[
a, a_1^\dagger \cdots a_n^\dagger
\right]
=
\sum_{i=1}^n
a_1^\dagger \cdots [a, a_i^\dagger] \cdots a_n^\dagger
\end{eqnarray*}
\begin{eqnarray*}
\left[
a_1 \cdots a_n, a^\dagger
\right]
=
\sum_{i=1}^n
a_1 \cdots [a_i, a^\dagger] \cdots a_n
\end{eqnarray*}

}%----------------------------------------------footnote //
\begin{eqnarray*}
\hat{N} \ketend \bld{k}_1 \cdots \bld{k}_n \ket
&=&
\frac{1}{\sqrt{n!}}
 \int \frac{d^3 \bld{k}}{2k^0} a^\dagger(\bld{k}) 
 \left[ a(\bld{k}), 
a^{\dagger}(\bld{k}_1) \cdots a^{\dagger}(\bld{k}_n)
\right]
\ketend 0 \ket
 \\
&=&
\frac{1}{\sqrt{n!}}
 \int \frac{d^3 \bld{k}}{2k^0} a^\dagger(\bld{k}) 
 \left(
 \left[ a(\bld{k}), 
a^{\dagger}(\bld{k}_1)
\right]
a^{\dagger}(\bld{k}_2) \cdots a^{\dagger}(\bld{k}_n)
\right.
\\
&&+
\left.
a^{\dagger}(\bld{k}_1)
 \left[ a(\bld{k}), 
a^{\dagger}(\bld{k}_2) \cdots a^{\dagger}(\bld{k}_n)
\right]
\right)
\ketend 0 \ket
 \\
&=&
\cdots
 \\
&=&
n \ketend \bld{k}_1 \cdots \bld{k}_n \ket
\end{eqnarray*}
It is normalized as
\begin{eqnarray}
&&\bra \bld{k}_1 \cdots \bld{k}_n 
\ketend \bld{k}'_1 \cdots \bld{k}'_n \ket
\nonumber\\
&=&
\frac{1}{n!}
\sum_{i'_1 = 1}^n
\bra 0 \braend
a_2 \cdots a_{n} a_{1'} \cdots [a_1, a_{i'_1}^\dagger ] \cdots a_{n'}^\dagger
\ketend 0 \ket
\nonumber\\
&=&
\frac{1}{n!}
\sum_{i'_1 = 1}^n
\sum_{i'_2 \neq i'_i}^n
\bra 0 \braend
a_3 \cdots a_{n} a_{1'} \cdots \cancel{a_{i'_1}} \cdots \cancel{a_{i'_2}} \cdots a_{n'}^\dagger
\ketend 0 \ket 
\left[ a_1, a_{i'_1}^\dagger \right]
\left[ a_2, a_{i'_2}^\dagger \right]
\nonumber\\
&=&
\cdots
\nonumber\\
&=&
\frac{1}{n!}
\sum_{i'_1 \dots i'_n = perm(1\dots n)}^{n! \mbox{ terms}}
\prod_{l}^n
2 k_l^0 \delta^3 (\bld{k}_l - \bld{k}_{i'_l})
\label{eqn:nscalarnorm}
\end{eqnarray}
We may construct a state of $n$ scalar particles described by a wave function
$\Psi^{(N)}(\bld{k}_1, \dots, \bld{k}_n)$ as
\begin{eqnarray}
\ketend \Psi^{(N)} \ket
=
\int \prod_i^N 
\frac{d^3 \bld{k}_i}{2 k_i^0}
 \Psi^{(N)} (\bld{k}_1,\dots,\bld{k}_N) 
 \ketend
 \bld{k}_1,\cdots,\bld{k}_N
\ket
\label{eqn:NscalarState}
\end{eqnarray}
Since $n$ particle momentum satate is symmetric under exchange of momenta, 
we may presuppose the function $\Psi^{(N)}(\bld{k}_1, \dots, \bld{k}_n)$ is also
symmetric. Then we have
\begin{eqnarray*}
\bra  \bld{k}_1 \cdots \bld{k}_n \braend
\Psi^{(N)} \ket
= 
 \Psi^{(N)} (\bld{k}_1,\dots,\bld{k}_N) 
 \end{eqnarray*}
When the number of particles is fixed, normalization of the state  is written as
\begin{eqnarray*}
\| \ketend \Psi^{(N)} \ket \|^2
=
\int \prod_i^N 
\frac{d^3 \bld{k}_i}{2 k_i^0}
\| \Psi^{(N)} (\bld{k}_1,\dots,\bld{k}_N) \|^2
=
1
\end{eqnarray*}
The most general state 
in the Fock space
%>>>>>>>>>>>>>>>>>>>>>>>>>comment
%\begin{comment}
may be written as
\begin{equation*}
\ketend \Psi^{} \ket
=
\sum_N
\ketend \Psi^{(N)} \ket
\end{equation*}
In this case, $\| \ketend \Psi^{(N)} \ket \|^2$
gives the probability to find the system 
with $n$ particles.
%\end{comment}
%<<<<<<<<<<<<<<<<<<<<<<<<<< comment //

Particle states composed like this are these in the Heisenberg picture.
If we choose $e^{-i k^0 t_0} \hat{a}(\bld{k})$ and $e^{i k^0 t_0} \hat{a}^\dagger(\bld{k})$
as initial values of Heisenberg operators $\hat{a}_H(\bld{k}, t)$ and $\hat{a}_H^\dagger(\bld{k}, t)$
respectively at $t = t_0$, we have from Eq. (\ref{eqn:Heisenbergformalsol}) that
\footnote{
From Eq. (\ref{eqn:KGcommHanda}), we have
\begin{equation*}
H a^\dagger (\bld{k}) = a^\dagger (H + k^0)
\,,\hspace{3mm}
H^2 a^\dagger (\bld{k}) = a^\dagger (H + k^0)^2\,,\dots
\end{equation*}
so that
\begin{eqnarray*}
e^{i H (t-t_0)} a^\dagger (\bld{k})
&=&
\sum_n^\infty \frac{i^n}{n!} (t-t_0)^n H^n a^\dagger (\bld{k})
\\
&=&
a^\dagger (\bld{k}) \sum_n^\infty \frac{i^n}{n!} (t-t_0)^n (H + k^0)^n 
\\
&=&
a^\dagger (\bld{k})  e^{i (H + k^0) (t-t_0)}
\end{eqnarray*}
} % footnote end
\begin{equation}
a_H(\bld{k}, t) = e^{-i k^0 t} a (\bld{k})\,,
\hspace{3mm}
a_H^\dagger(\bld{k}, t) = e^{i k^0 t} a^\dagger (\bld{k})
\end{equation}
We introduce state vectors in the Schr\"odinger picture by
\begin{equation}
{}_S\bra \Psi(t) \braend {\cal O} \ketend \Psi(t) \ket_S
=
\bra \Psi \braend {\cal O}_H(t) \ketend \Psi \ket
\end{equation}
State vector in the Schr\"odinger picture obeys 
\begin{equation}
i \partial_t \ketend \Psi(t) \ket_S = \hat{H} \ketend \Psi(t) \ket_S
\end{equation}

\bigskip

%<<<<<<<<<<<<<<<<<<<<<<<<<<<<<<<<<<<<<<
\noindent
%\begin{equation}
%\left( \Box + m^2 \right) \varphi(x) = 0
%\end{equation}
Propagator
\begin{equation}
\left( \Box + m^2 \right) \Delta_F(x) = \delta^4(x) 
\end{equation}
\begin{eqnarray}
\Delta_F(q) 
&=&
i \int d^4x e^{iqx} 
\bra 0 \braend \mbox{T}\varphi(x) \varphi(0) 
\ketend 0 \ket
\nonumber\\
&=&
\frac{-1}{q^2 - m^2 + i\epsilon}\,,
\end{eqnarray}
where $\epsilon$ is infinitesimal positive number.
T-product
\begin{equation}
T[\varphi(x) \varphi(y)] =
\theta(x^0 - y^0)\varphi(x) \varphi(y)
+
\theta(y^0 - x^0)\varphi(y) \varphi(x)
\end{equation}
One will find
\[
\left( \Box + m^2 \right) T[\varphi(x) \varphi(0)] = -i \delta^4(x)
\]
\begin{equation}
\Delta_F(x) = \int \frac{d^4 q}{(2\pi)^4} e^{-iqx} \Delta_F(q)
\end{equation}
\subsubsection{Complex Scalar Free Field}
Spinless and charged.
\begin{equation}
{\cal L} = \partial_\mu \varphi^\dagger(x)\cdot \partial^\mu \varphi(x)
- m^2 \varphi^\dagger(x)\varphi(x)
\label{eqn:cmplscLagdens}
\end{equation}
\begin{equation}
\frac{\partial {\cal L}}{\partial(\partial_\mu \varphi)} = \partial^\mu \varphi^\dagger
\,,
\hspace{5mm}
\frac{\partial {\cal L}}{\partial(\partial_\mu \varphi^\dagger)} = \partial^\mu \varphi
\end{equation}
\begin{equation}
\left( \Box + m^2 \right) \varphi(x) = 0\,,
\hspace{5mm}
\left( \Box + m^2 \right) \varphi^\dagger(x) = 0\,,
%\hspace{5mm}
%\varphi(x)^* = \varphi(x)
\end{equation}
\begin{equation}
\pi^\dagger(x) = \frac{ \partial {\cal L}}{ \partial \dot{\varphi}(x) } = \dot{\varphi}^\dagger
%\pi_{}(x) = \partial_0 \varphi_{cl}(x)
%=
%\int \frac{d^3 \bld{k}}{\sqrt{(2\pi)^3}} (- i k^0 ) \left[
%a(\bld{k}) e^{-i k \cdot x} - b^\dagger(\bld{k})e^{i k \cdot x} \right]
\,,
\hspace{5mm}
\pi(x) = \frac{ \partial {\cal L}}{ \partial \dot{\varphi}^\dagger(x) } = \dot{\varphi}
\end{equation}
\begin{eqnarray}
\mathcal{H} 
&=&
\pi^\dagger \dot{\varphi} + \pi \dot{\varphi}^\dagger - \mathcal{L}
\nonumber\\
&=&
\pi^\dagger \pi + (\partial_i \varphi^\dagger) (\partial_i \varphi) + m \varphi^\dagger \varphi
\end{eqnarray}
%Classical solution:
%\begin{equation}
%\varphi_{cl}(x) = \int \frac{d^3 \bld{k}}{\sqrt{(2\pi)^3}} \left[
%a(\bld{k}) e^{-i k \cdot x} + b^*(\bld{k})e^{i k \cdot x} \right]
%\end{equation}
%where $k^0 = + \sqrt{\bld{k}^2 + m^2}$. 
%Solution
\begin{equation}
\begin{array}{l}
\displaystyle
\varphi_{}(x) = \int \frac{d^3 \bld{k}}{\sqrt{(2\pi)^3}2k^0} \left[
a(\bld{k}) e^{-i k \cdot x} + b^\dagger(\bld{k})e^{i k \cdot x} \right]
\\
\displaystyle
\varphi^\dagger_{}(x) = \int \frac{d^3 \bld{k}}{\sqrt{(2\pi)^3}2k^0} \left[
b(\bld{k})e^{- i k \cdot x}  + a^\dagger (\bld{k}) e^{i k \cdot x} \right]
\end{array}
\label{eqn:CompScFourier}
\end{equation}
Canonical quantization
\begin{equation}
\begin{array}{c}
[ a(\bld{k}), a^\dagger(\bld{k}')] = 
[ b(\bld{k}), b^\dagger(\bld{k}')] = 
2k^0\delta^3 (\bld{k} - \bld{k'} )\;,
\vspace{3mm}
\\
\mbox{other commutators} = 0
\label{eqn:complscalarcancomm}
\end{array}
\end{equation}
Four momentum
\begin{equation}
\hat{P}^\mu = \int \frac{d^3 \bld{k}}{2k^0} k^\mu \left(
a^\dagger(\bld{k}) a(\bld{k}) + b^\dagger(\bld{k}) b(\bld{k})
\right)
\label{eqn:KGcomplex_FourMomentum}
\end{equation}
$\hat{P}^0$ is the Hamiltonian.
Vacuum state is defined as $a(\bld{p}) \ketend 0 \ket = b(\bld{p}) \ketend 0 \ket =0$.
There are two kinds of single particle states $a^\dagger(\bld{p}) \ketend 0 \ket$ and
$b^\dagger(\bld{p}) \ketend 0 \ket$ both an eigenstate of $\hat{P}^\mu$.
So far, $a$ and $b$ are just particles independent of each other and having
the common mass $m$. Let us examine the electric charge given from
Eq. (\ref{eqn:NoetherElectricCharge}). We find
\begin{equation}
\hat{Q} = e \int \frac{d^3 \bld{k}}{2k^0}  \left(
a^\dagger(\bld{k}) a(\bld{k}) - b^\dagger(\bld{k}) b(\bld{k})
\right)
\label{eqn:KGcomplex_Charge}
\end{equation}
Thus they carry opposite electric charges.

We consider now inversion symmetries $U = P, C, T$. 
The vacuum is invariant under these inversions.
\begin{itemize}
\item
Space inversion:
$(\bld{x}, t) \mapsto (-\bld{x}, t)$ 
\begin{eqnarray}
P a(\bld{k}) P^{-1} = \pm a(-\bld{k})\,,
\hspace{3mm}
P b(\bld{k}) P^{-1} = \pm b(-\bld{k})\,,
\end{eqnarray}
$+$ for scalar and $-$ for pseudoscalar.
\item
Charge conjugation
\begin{eqnarray}
C a(\bld{k}) C^{-1} = \pm b(\bld{k})\,,
\hspace{3mm}
C b(\bld{k}) C^{-1} = \pm a(\bld{k})\,,
\end{eqnarray}
\item
Time reversal: $(\bld{x}, t) \mapsto (\bld{x}, -t)$ 
\begin{eqnarray}
T a(\bld{k}) T^{-1} = \pm a(-\bld{k})\,,
\hspace{3mm}
C b(\bld{k}) C^{-1} = \pm b(-\bld{k})\,,
\end{eqnarray}
and $T$ is antilinear so that
\begin{eqnarray}
T \varphi(\bld{x}, t) T^{-1} = \pm \varphi(\bld{x}, -t)
\end{eqnarray}
The sign is fixed through the invariance of interactions 
with other kind of fields which have definite signatures
under the $T$ transformation. Usually, $+$ for scalar and
$-$ for pseudoscalar.
\end{itemize}

\bigskip




%Propagator $ = \Delta_F(q)$


\subsubsection{Internal Symmetry}
Complex scalar field with internal symmetry
\begin{equation}
{\cal L} = \sum_a \left[
\partial_\mu \varphi_a^\dagger(x)\cdot \partial^\mu \varphi_a(x)
- m^2 \varphi_a^\dagger(x)\varphi_a(x)
\right]
\end{equation}
We consider scalar fields $\varphi_a(x)$ transforms under a global $SU(n)$ 
as in Eq. (\ref{eqn:LieGrTransfFields}).
We assume $\{\varphi_a(x) \}$ composes an irreducible representation $\nu$
and write them in a combined form
%\begin{eqnarray}
$\varphi = (\varphi_1, \dots, \varphi_s)$
%\end{eqnarray}
where $s$ is the multiplicity of the representation $\nu$.
We may write the ${\cal L}$agrangian exactly in the same form as
Eq. (\ref{eqn:cmplscLagdens})
understanding our $\varphi$ is now multicomponent.
We may then write the Noether current and charge in Eqs (\ref{eqn:NoetherCurrGeneral}) 
and (\ref{eqn:NoetherChrgGeneral}) interms creation and annihilation operators.
States $\{ a^\dagger_j \ketend 0 \ket \}$ composes a multiplet of $\nu$.
( To be continued.)
\subsubsection{Scalar Yukawa Theory - A Toy Model -}
Let us consider a toy model composed of scalar fields:
\begin{eqnarray}
{\cal L} 
&=&
\partial_\mu \psi^* \partial^\mu \psi
+
\frac{1}{2} (\partial_\mu \phi)^2 
-
M^2 \psi^*  \psi
-
\frac{1}{2} m^2 \phi^2
-
g \psi^*  \psi \phi
\label{eqn:sclYkwLagDens}
\end{eqnarray}
$\psi$ is a mock nucleon complex scalar field with the mass $M$ 
and $\phi$ is a mock pion real scalar field with the mass $m$.
The first parts of $\cal L$ except for the last term compose free $\cal L$agrangians
${\cal L}_N$ and ${\cal L}_\pi$ for nucleon and pion fields, respectively.
The last term is the interaction $\cal L$agrangian in which
nucleons and pions interact each other with a coupling constant $g$.
The coupling $g$ has the dimension of energy and the dimensionless
parameter is $g / E$, where E is the energy scale of the process of interest.
This means that the interaction term ${\cal L}_{int} = - g \psi^* \psi \phi$ is
relevant at low energies. The relativistic nature gets important at $E \gg M, m$
and we may choose $g \ll M, m$ so that the perturbation series (\ref{eqn:TimeDevPerturbSerTprod})
converges.

Each fields are quantized through equal-time commutation relations at time,
say, 0. One may write down field equations but they are not solvable due to
the presence of ${\cal L}_{int}$. Remember they describe time evolution of
field operators in the Heisenberg picture. In the interaction picture, however,
we may let fields obey free field equations. 
%-------------------------------------------------------------------- comment block starts
\begin{comment}
Field equations read
\begin{eqnarray}
\begin{array}{l}
(\Box + m^2) \phi = - g\psi^* \psi\,,
\vspace{2mm}
\\
(\Box + M^2) \psi = -g\psi \phi\,,
\vspace{2mm}
\\
(\Box + M^2) \psi^* = -g\psi^* \phi
\end{array}
\end{eqnarray}
\end{comment}
%-------------------------------------------------------------------- comment block ends
In practice, 
conjugate fields are given by
\begin{eqnarray}
%\begin{array}{l}
\pi_\phi = \frac{\partial {\cal L}}{\partial(\dot{\phi})} = \dot{\phi}
\,,\hspace{3mm}
%\\
\pi_\psi = \frac{\partial {\cal L}}{\partial(\dot{\psi})} = \dot{\psi^*}
\,,\hspace{3mm}
%\\
\pi_{\psi^*} = \frac{\partial {\cal L}}{\partial(\dot{\psi^*})} = \dot{\psi}
\,,
%\end{array}
\end{eqnarray}
and classical Hamiltonian density is written as
\begin{eqnarray}
{\cal H} 
&=&
\pi_\phi \dot{\phi} +\pi_\psi \dot{\psi} + \pi_{\psi^*} \dot{\psi^*} - {\cal L}
\nonumber\\
&=&
\frac{1}{2} \left\{
\pi_\phi^2 + (\bld{\partial}\phi)^2 + m^2 \phi^2
\right\}
+
\left\{
\pi_\psi \pi_{\psi^*}
+ \bld{\partial} \psi^* \cdot \bld{\partial} \psi
+ M^2 \psi^* \psi \right\}
\nonumber\\
&&+ g \psi^* \psi \phi
\label{eqn:sclYkwHamDens}
\end{eqnarray} 
The first two terms are ${\cal H}_\pi$ and ${\cal H}_N$ corresponding to
${\cal L}_\pi$ and ${\cal L}_N$, respectively, and we assign them  as
the free ${\cal H}$amiltonian ${\cal H}_0 = {\cal H}_\pi + {\cal H}_N$.
Writing ${\cal H}_{int} = g\psi^*\psi \phi$, 
we have a decomposition ${\cal H} = {\cal H}_{0}  + {\cal H}_{int}$,
which corresponds to Eq. (\ref{eqn:HamiltonianDecomposed}).
After substituting quantized fields (at the fixed time) into these expressions
(and taking the normal ordering), we obtain ${\cal H}$amiltonian operators.
The time evolutions of fields in the interaction picture are given from
Eq. (\ref{eqn:InteractionPicTime}) as
\begin{eqnarray}
\begin{array}{l}
i \partial_t \phi_I = [\phi_I, H_0] = [\phi_I, H_\pi]\,,
\vspace{2mm}
\\
i \partial_t \psi_I = [\psi_I, H_0] = [\psi_I, H_N]\,,
\vspace{2mm}
\\
i \partial_t \psi^\dagger_I = [\psi^\dagger_I, H_0] = [\psi^\dagger_I, H_N]\,,
\end{array}
\label{eqn:sYfieldTimeEvo}
\end{eqnarray}
where we added an suffix $I$ to indicate quantities in the interaction picture.
We already know that Eqs. (\ref{eqn:sYfieldTimeEvo}) are equivalent to
free field equations all given as the Klein-Gordon equation with masses of
each fields. Therefore, the fields $\phi_I$, $\psi_I$ and $\psi^\dagger_I$ have
Fourier expanded forms as in Eqs. (\ref{eqn:RS_fieldexpansion}) and (\ref{eqn:CompScFourier}).
To be specific, we write
\begin{eqnarray}
\begin{array}{l}
\displaystyle
\phi_{I}(x) = \int \frac{d^3 \bld{k}}{\sqrt{(2\pi)^3}2k^0} \left[
a(\bld{k}) e^{-i k \cdot x} + a^\dagger(\bld{k})e^{i k \cdot x} \right]
\vspace{2mm}
\\
\displaystyle
\psi_{I}(x) = \int \frac{d^3 \bld{p}}{\sqrt{(2\pi)^3}2p^0} \left[
b(\bld{p}) e^{-i p \cdot x} + c^\dagger(\bld{p})e^{i p \cdot x} \right]
\vspace{2mm}
\\
\displaystyle
\psi^\dagger_{I}(x) = \int \frac{d^3 \bld{p}}{\sqrt{(2\pi)^3}2p^0} \left[
c(\bld{p})e^{- i p \cdot x}  + b^\dagger (\bld{p}) e^{i p \cdot x} \right]
\end{array}
\label{eqn:scYfields}
\end{eqnarray}
Operators $a$, $b$ and $c$ and their conjugates satisfy
commutation relations given in Eqs. (\ref{eqn:cancomm_RS}) and 
(\ref{eqn:complscalarcancomm}) and they establishes the particle interpretations.
Parts of the free ${\cal H}$amiltonian ${\cal H}_\pi$ and ${\cal H}_N$ are now
described by these operators as in Eq. (\ref{eqn:KG_Hamiltonian_aadagg}) and 
the time component of Eq. (\ref{eqn:KGcomplex_FourMomentum}).
We may define number operator $\hat{N}_\pi$ as in Eq. (\ref{eqn:KG_totalNumberOp})
and ones for nucleons as
\begin{eqnarray}
\hat{N}_N = \int \frac{d^3 \bld{p}}{2p^0} b^\dagger(\bld{p}) b(\bld{p})\,,
\hspace{3mm}
\hat{N}_{\bar{N}} = \int \frac{d^3 \bld{p}}{2p^0} c^\dagger(\bld{p}) c(\bld{p})\,,
\end{eqnarray}
where we have assigned $b^\dagger$ and $c^\dagger$ as creation operators
for a nucleon ($N$) and an anti-nucleon ($\bar{N})$, respectively.
In the interaction picture, ${\cal L}_{int}$ term in Eq.(\ref{eqn:sclYkwLagDens})
[or ${\cal H}_{int}$ term in Eq. (\ref{eqn:sclYkwHamDens})] contains creation and
annihilation operators in each field and they may change number of
particles. In practice, the number operators we have just defined do not commute 
with $H_I$, therefore neither with $H$ and do not conserve.
Though numbers of particles are not conserved, the total electric charge would 
be conserved since the ${\cal L}$agrangian (\ref{eqn:sclYkwLagDens}) is invariant
under constant phase change of the field $\psi$ and its conjugate for $\psi^*$.
According to the Noether's theorem and Eq. (\ref{eqn:KGcomplex_Charge}), 
the charge
\begin{eqnarray}
Q = e \int \frac{d^3 \bld{p}}{2p^0}
\left(
b^\dagger({\bld{p}}) b({\bld{p}}) - c^\dagger({\bld{p}}) c({\bld{p}})
\right)
\label{eqn:scYelchrg}
\end{eqnarray}
commutes with $H$ and conserved.

Keeping the formula (\ref{eqn:TimeDevPerturbSerTprod}) in mind,
let us examine particular expression of $H_I$:
\begin{eqnarray}
H_I(t) 
&=&
g \int d^3 \bld{x} 
\normord{ \psi_I^\dagger(x) \psi_I(x) \phi_I(x)}
\nonumber\\
&=&
\frac{g}{\sqrt{(2\pi)^9}} 
\int
\frac{d^3 \bld{p}}{2p^0}
\frac{d^3 \bld{p}'}{2{p^0}'}
\frac{d^3 \bld{k}}{2k^0}
\int 
d^3 \bld{x} 
:
[ e^{-ipx} c(\bld{p}) + e^{ipx} b^\dagger(\bld{p}) ]
\nonumber\\
&&
\hspace{7mm}
[ e^{-ip'x} b(\bld{p}') + e^{ip'x} c^\dagger(\bld{p}') ]
[ e^{-ikx} a(\bld{k}) + e^{ikx} a^\dagger(\bld{k}) ]
:
\nonumber\\
&=&
\frac{g}{\sqrt{(2\pi)^3}} 
\int
\frac{d^3 \bld{p}}{2p^0}
\frac{d^3 \bld{p}'}{2{p^0}'}
\frac{d^3 \bld{k}}{2k^0}
\nonumber\\
&&
[c(\bld{p})b(\bld{p}')a(\bld{k})\delta^3(\bld{p}+\bld{p}'+\bld{k})
e^{-i (p + p' + k_+)^0 t}
\nonumber\\
&&
+b^\dagger(\bld{p}) c^\dagger(\bld{p}') a^\dagger(\bld{k}) \delta^3(\bld{p}+\bld{p}'+\bld{k})
e^{i (p + p' + k_+)^0 t}
\nonumber\\
&&
+b^\dagger(\bld{p}) c^\dagger (\bld{p}')a(\bld{k})\delta^3(\bld{p}+\bld{p}'-\bld{k})
e^{i (p + p' - k_+)^0 t}
\nonumber\\
&&
+c^\dagger(\bld{p})c(\bld{p}')a(\bld{k})\delta^3(\bld{p}-\bld{p}'-\bld{k})
e^{i (p - p' - k_-)^0 t}
\nonumber\\
&&
+b^\dagger(\bld{p})b(\bld{p}')a(\bld{k})\delta^3(\bld{p}-\bld{p}'-\bld{k})
e^{i (p - p' - k_-)^0 t}
\nonumber\\
&&
+a^\dagger(\bld{k})c(\bld{p})b(\bld{p}')\delta^3(\bld{p}+\bld{p}'-\bld{k})
e^{-i (p + p' - k_-)^0 t}
\nonumber\\
&&
+a^\dagger(\bld{k})c^\dagger(\bld{p}')c(\bld{p})\delta^3(\bld{p}-\bld{p}'-\bld{k})
e^{-i (p - p' - k_-)^0 t}
\nonumber\\
&&
+a^\dagger(\bld{k})b^\dagger(\bld{p})b(\bld{p}')\delta^3(-\bld{p}+\bld{p}'-\bld{k})
e^{i (p - p' + k_-)^0 t}]
\label{eqn:scYHIterms}
\end{eqnarray}
where $k_\pm^0 = \sqrt{(\bld{p} \pm \bld{p}')^2 + m^2}$.
At this stage, we may already have some insights about the interaction.
Our ${\cal L}_{int}$ is a product of three field operators and each of them
involve two terms with annihilation and creation operators. This is why there
are $2^3 = 8$ terms in Eq. (\ref{eqn:scYHIterms}). Among them, the last
6 terms show possible processes. For instance, the third term corresponds
a process in which a particle $a$ (pion) disappears and particles $b$ and $c$
(nucleon and anti-nuclen) emerges. So this term corresponds to the pair creation
of $N\bar{N}$ by $\pi$. One can confirm the momentum is conserved in the process.
The energy is not conserved yet and there are exponential factors instead.
Later, we will see these exponentials turn into delta functions corresponding to
the energy conservation for each processes in the evaluation the scattering matrix.
The first two terms in Eq. (\ref{eqn:scYHIterms}) will not contribute to scattering matrices
since they violate the energy conservation.
The second and higher order terms in Eq. (\ref{eqn:TimeDevPerturbSerTprod}) will be
involved in the later discussion with introducing some techniques to expand time ordered
products.
\\

\noindent
●Meson Decay\\
Consider a process $\pi \rightarrow N\bar{N}$. This process is involved in the lowest
order term in Eq. (\ref{eqn:SmatrixPertSer}) through the third term in
Eq. (\ref{eqn:scYHIterms}). We write Initial and final states as
\begin{eqnarray}
\begin{array}{l}
\ketend i \ket = a^\dagger (\bld{k}) \ketend 0 \ket
\rightdef \ketend \pi(\bld{k}) \ket\,,
\vspace{2mm}
\\
\ketend f \ket = b^\dagger (\bld{p}_N) c^\dagger (\bld{p}_{\bar{N}})\ketend 0 \ket
\rightdef \ketend N(\bld{p}_N) \bar{N}(\bld{p}_{\bar{N}}) \ket
\end{array}
\end{eqnarray}
From Eq. (\ref{eqn:SmatrixPertSer}), we read to the leading (first) order in $g$ that
\begin{eqnarray}
\bra f \braend S^{(1)} \ketend i \ket
&=&
-ig \int d^4 x  
\bra N(\bld{p}_N) \bar{N}(\bld{p}_{\bar{N}}) \braend
\normalprod{ \psi^\dagger(x) \psi(x) \phi(x)}
\ketend \pi(\bld{k}) \ket
\nonumber\\
&=&
-ig \int d^4 x  
\bra 0 \braend b(\bld{p}_N) c(\bld{p}_{\bar{N}})
\normalprod{ \psi^\dagger(x) \psi(x) \phi(x)}
 a^\dagger(\bld{k}) \ketend 0 \ket
 \label{eqn:scYmesonDcyamp}
\end{eqnarray}
Here we omitted indeces $I$ on fields under understanding that we are in the interaction picture.
In expanding fields as $\psi^\dagger \sim c + b^\dagger$, 
$\psi \sim b + c^\dagger$ and $\phi \sim a + a^\dagger$,
we find only a term $\sim b^\dagger c^\dagger a$
contributes in Eq. (\ref{eqn:scYmesonDcyamp}).
This corresponds to the third term in $H_I$ in Eq. (\ref{eqn:scYHIterms}).
We proceed from Eq. (\ref{eqn:scYmesonDcyamp}):
\begin{eqnarray}
\bra f \braend S^{(1)} \ketend i \ket
&=&
\frac{-ig}{\sqrt{(2\pi)^9}} 
\int
\frac{d^3 \bld{p}}{2p^0}
\frac{d^3 \bld{p}'}{2{p^0}'}
\frac{d^3 \bld{k}'}{2{k^0}'}
\int 
d^4 x\;
e^{i(p + p' - k') x}
\nonumber\\
&&
\hspace{10mm}
\bra 0 \braend b(\bld{p}_N) c(\bld{p}_{\bar{N}})
[b^\dagger(\bld{p}) c^\dagger (\bld{p}')a(\bld{k}')]
 a^\dagger(\bld{k}) \ketend 0 \ket
\nonumber\\
&=&
\frac{-ig}{\sqrt{(2\pi)^9}} 
\int
\frac{d^3 \bld{p}}{2p^0}
\frac{d^3 \bld{p}'}{2{p^0}'}
\frac{d^3 \bld{k}'}{2{k^0}'}
(2\pi)^4 \delta^4(p + p' - k')
\nonumber\\
&&
\hspace{10mm}
\bra 0 \braend b(\bld{p}_N) c(\bld{p}_{\bar{N}})
[b^\dagger(\bld{p}) c^\dagger (\bld{p}')a(\bld{k}')]
 a^\dagger(\bld{k}) \ketend 0 \ket
\nonumber\\
&=&
\frac{-ig}{\sqrt{(2\pi)^9}} 
\int
\frac{d^3 \bld{p}}{2p^0}
\frac{d^3 \bld{p}'}{2{p^0}'}
(2\pi)^4 \delta^4(p + p' - k)
\nonumber\\
&&
\hspace{20mm}
\bra 0 \braend b(\bld{p}_N) c(\bld{p}_{\bar{N}})
b^\dagger(\bld{p}) c^\dagger (\bld{p}')
 \ketend 0 \ket
\nonumber\\
&=&
\frac{-ig}{\sqrt{(2\pi)^9}} 
(2\pi)^4 \delta^4(p_N + p_{\bar{N}} - k)
 \label{eqn:scYmesonDcySmtrx}
\end{eqnarray}

Reaction rate is defined in the same way as that for scatterting as
\begin{eqnarray}
R_{fi} = \int d\Phi_f | \bra f \braend T \ketend i \ket |^2\,,
\end{eqnarray}
where $\Phi_f$ and $T$ are defined in Eqs. (\ref{eqn:Phi_nDef}) and (\ref{eqn:Smatrix_and_T})
respectively
\footnote{
A note on dimensions: $dim[ d\Phi_f (\bra f \braend )^2] = 1/E^4$,
$dim [T] = E^4$ and $dim [\ketend i \ket]^2 = 1/E^{2k}$ for an initial state
composed of $k$ particles. In total, $dim[R] = E^{4 - 2k}$. 
$k = 2$ and $dim[R] = E^0$ for scatterings. $k = 1$ and $dim[R] = E^2$
for particle decays.
} % end of footnote
.
For decays of a particle with the mass $M$ into $n$ particles, differential decay width 
is defined as
\begin{eqnarray}
d \Gamma = \frac{(2\pi)^3}{2M} dR_{fi}
=
\frac{(2\pi)^3}{2M} 
\prod_i^n \frac{d^3\bld{p}_i}{2p_i^0} (2\pi)^4 \delta^4(P_f - P_i) 
 | \bra f \braend T \ketend i \ket |^2
\end{eqnarray}
In particular, for two particle decays,
\begin{eqnarray}
d \Gamma_2 
&=&
\frac{(2\pi)^3}{2M} 
\frac{d^3\bld{p}_1}{2p_1^0} 
d^4 p_2 \delta(p_2^2 - m_2^2)
(2\pi)^4 \delta^4(P_f - P_i) 
 | \bra f \braend T \ketend i \ket |^2
\nonumber\\
&=&
\frac{(2\pi)^7}{2M} 
\frac{P_1^{*2} dP_1^* d \Omega_1^*}{2p_1^{*0}} 
\delta(M^2 + m_1^2  - m_2^2 - 2 M p_1^{*0})
|T_{fi}|^2
\nonumber\\
&=&
\frac{(2\pi)^9}{32\pi^2M^2} 
P_1^{*}  d \Omega_1^*
|T_{fi}|^2
\label{eqn:decaywidth2bdy}
\end{eqnarray}
We note that $dim |T_{fi}|^2 = E^2$.
The way of writing the coefficient in Eq. (\ref{eqn:decaywidth2bdy}) is chosen
so that it is clear that a factor $(2\pi)^9$ is absorbed in the state normalizations
in notations of some authors including \cite{ref:Donnachie}, \cite{ref:Collins} and
\cite{ref:Itzykson-Zuber}.

Let's come back to our toy model. From Eq. (\ref{eqn:scYmesonDcySmtrx}),
we have $T_{fi} = g / \sqrt{(2\pi)^9}$. Substituting this in Eq. (\ref{eqn:decaywidth2bdy}),
we obtain
\begin{eqnarray}
\Gamma_2^{(1)} = 
\frac{g^2}{16\pi m^2} \lambda^{1/2}(m^2, M^2, M^2)
\end{eqnarray}
where a superscript $(1)$ indicates the order of perturbation expansion in Eq.  (\ref{eqn:SmatrixPertSer}).




%========================================  section 3.6
\subsection{Interacting Fields (2/2)}
\begin{comment}%================================================== comment <<
Provided with ${\cal H}_{int}$ in terms of quantized fields in the interaction picture, 
we write 
%we write Eq. (\ref{eqn:SmatrixPertSer}) as
\begin{eqnarray}
\bra f \braend S -1 \ketend i \ket
&=&
 -i \int d^4 x' \, \bra f \braend {\cal H}_{int}(x') \ketend i \ket
\nonumber\\
&&+
\frac{(-i)^2}{2}
\int d^4 x_1' d^4 x_2'
\;\bra f \braend T[ {\cal H}_{int}(x_1') {\cal H}_{int}(x_2')] \ketend i \ket
+ \cdots
\nonumber\\
\label{eqn:SmatrixPertSerCov}
\end{eqnarray}
To proceed beyond the second term requires some further preparations
that we describe in the following.
\end{comment}
%================================================== comment //
\subsection{Propagators of scalar fields}
Let's take a real scalar field in Eqs (\ref{eqn:cancomm_RS}, \ref{eqn:RS_fieldexpansion}).
We have
\begin{eqnarray}
\bra 0 \braend \phi(x) \phi(y) \ketend 0 \ket
&=&
\int \frac{d^3 \bld{k} d^3 \bld{k}'}{(2\pi)^3 4 k^0 {k^0}'} 
\bra 0 \braend a(\bld{k}) a^\dagger(\bld{k}') \ketend 0 \ket
e^{-i(kx - k'y)}
\nonumber\\
&=&
\int \frac{d^3 \bld{k} }{(2\pi)^3 2 k^0} 
e^{-ik(x - y)}
\rightdef
D(x-y)
\end{eqnarray}
\begin{eqnarray}
[\phi(x), \phi(y)]
&=&
\int \frac{d^3 \bld{k} d^3 \bld{k}'}{(2\pi)^3 4 k^0 {k^0}'} 
\left(
[a(\bld{k}), a^\dagger(\bld{k}')]e^{-i(kx - k'y)}
\right.
\nonumber\\
&&
\left.
\hspace{35mm}
+
[a^\dagger(\bld{k}), a(\bld{k}')]e^{i(kx - k'y)}
\right)
\nonumber\\
&=&
\int \frac{d^3 \bld{k} }{(2\pi)^3 2 k^0} 
\left(
e^{-ik(x - y)} - e^{ik(x - y)}
\right)
\nonumber\\
&=&
D(x-y) - D(y-x)
\label{eqn:scfieldcommrel}
\end{eqnarray}
When $x-y$ is spacelike, there exists a Lorentz frame where $x^0 = y^0$.
Then the $r.h.s.$ of Eq. (\ref{eqn:scfieldcommrel}) vanishes. 
The whole expression is Lorentz invariant and it must vanish for all $(x-y)^2 < 0$.
Nevertheless, $D(x-y)$ itself does not vanish even $x-y$ is spacelike.

The Feynman propagator is defined as
\begin{eqnarray}
\bra 0 \braend T[\phi(x) \phi(y)] \ketend 0 \ket
&=&
\theta(x^0 - y^0) D(x-y)
+
\theta(y^0 - x^0) D(y-x)
\label{eqn:scFeynPropbyD}
\\
&=&
i \int \frac{d^4 p}{(2\pi)^4}
\frac{e^{-ip (x-y)}}{p^2 - m^2 + i\epsilon}
\label{eqn:scFeynmanProp}
\label{eqn:FeynmanPropDefSc}
\\
&\rightdef&
\Delta_F(x-y)
=
\Delta_F(y-x)
\nonumber
\end{eqnarray}
{\it Proof of Eq.(\ref{eqn:scFeynmanProp}) $\Leftrightarrow$ Eq. (\ref{eqn:scFeynPropbyD})}\\
%\begin{proof}
\begin{eqnarray}
\frac{1}{p^2 - m^2 + i\epsilon}
=
\frac{1}{2 E_{\bld{p}}} \left(
\frac{1}{p^0 - E_{\bld{p}} + i\epsilon}
-
\frac{1}{p^0 + E_{\bld{p}} - i\epsilon}
\right)
\,,
\end{eqnarray}
\begin{eqnarray}
\Delta_F(x-y)
&=&
i \int \frac{d^3 \bld{p}}{(2\pi)^3 2 E_{\bld{p}}} e^{i\bld{p}\cdot (\bld{x} - \bld{y})}
\nonumber\\
&&
\times \int \frac{d p^0}{2\pi} e^{- ip^0 (x^0 - y^0)}
\left(
\frac{1}{p^0 - E_{\bld{p}} + i\epsilon}
-
\frac{1}{p^0 + E_{\bld{p}} - i\epsilon}
\right)
\nonumber\\
&=&
\int \frac{d^3 \bld{p}}{(2\pi)^3 2 E_{\bld{p}}} e^{i\bld{p}\cdot (\bld{x} - \bld{y})}
\frac{-1}{2\pi i} \left(
\theta(x^0 - y^0) (-2\pi i e^{-i E_{\bld{p}}(x^0 - y^0)})
\right.
\nonumber\\
&&
\left.
-
\theta(y^0 - x^0) (+2\pi i e^{i E_{\bld{p}}(x^0 - y^0)}
\right)
\nonumber\\
&=&
\int \frac{d^3 \bld{p}}{(2\pi)^3 2 E_{\bld{p}}} 
\left(
\theta(x^0 - y^0) e^{-ip(x-y)}
+
\theta(y^0 - x^0) e^{i p(x-y)}
\right)
\nonumber\\
&=&
\theta(x^0 - y^0) D(x-y)
+
\theta(y^0 - x^0) D(y-x)
\nonumber\\
&& 
\hspace{65mm}
\blacksquare
\end{eqnarray}
%\end{proof}
From the first line to the second, we have performed integrations in a complex $p^0$ plane.
Contour of the integration should go through the real axis from $- \infty$ to $\infty$ and
turn lower (upper) half plane (anti) clockwise when $x^0 - y^0 > (<)\, 0$. A factor
$\pm 2\pi i$ appears as a result of the residue calculus.

For the complex scalar field prescribed in Eqs. (\ref{eqn:CompScFourier}, \ref{eqn:complscalarcancomm}),
we find 
\begin{eqnarray}
\begin{array}{l}
\bra 0 \braend \varphi(x) \varphi(y) \ketend 0 \ket = 
\bra 0 \braend \varphi^\dagger(x) \varphi^\dagger(y) \ketend 0 \ket = 0\,,
\vspace{2mm}
\\
\bra 0 \braend T[\varphi^\dagger(x) \varphi(y)] \ketend 0 \ket 
= 
\bra 0 \braend T[\varphi(x) \varphi^\dagger(y)] \ketend 0 \ket 
= 
\Delta_F(x-y)
\end{array}
\label{eqn:CplxScProp}
\end{eqnarray}

\subsection{Prescription for time ordered products}
Since the second and higher order terms in Eq. (\ref{eqn:SmatrixPertSer})
are written in terms of time ordered products of $H_I$'s and
each $H_I$ is given in terms of a normal product of fields like
one in the first line of Eq. (\ref{eqn:scYHIterms}), 
we need a way to deal with time ordered products 
of fields. A technique to do this is provided by Wick's theorem, 
which we have described in Appendix \ref{sec:App_Wick}.
With the Feynman propagator defined in Eq. (\ref{eqn:FeynmanPropDefSc}),
Wick's theorem (\ref{eqn:WickTheorem}) for the real scalar field reads
\begin{eqnarray}
T[\phi_1 \phi_2 \cdots ]
&=&
\normalprod{\phi_1 \phi_2 \cdots }
\nonumber\\
&+&
\sum_{i < j}  
\normalprod{
\acontraction[1ex]{\cdots}{\phi}{{}_i\cdots}{\phi}
\cdots \phi_i \cdots \phi_j \cdots
}%end normalprod
\nonumber\\
&+&
\sum_{i < j, k<l}  
\normalprod{
\acontraction[1ex]{\cdots}{\phi}{{}_i\cdots \phi_k \cdots}{\Phi}
\acontraction[2ex]{\cdots \phi_i \cdots }{\phi}{{}_k\cdots \Phi_j \cdots}{\phi}
\cdots \phi_i \cdots \phi_k \cdots  \phi_j \cdots \phi_l \cdots
}%end normalprod
\nonumber\\
&+&
\cdots  \mbox{ (all possible contracts)}
\nonumber\\
&=&
\normalprod{\phi_1 \phi_2 \cdots }
\nonumber\\
&+&
\sum_{i < j}  
\Delta_F(x_i - x_j)
\normalprod{
\cdots \xout{\phi_i} \cdots \xout{\phi_j} \cdots
}%end normalprod
\nonumber\\
&+&
\sum_{i < j, k<l}  
\Delta_F(x_i - x_j)
\Delta_F(x_k - x_l)
%\hspace{50mm}
\normalprod{
\cdots \xout{\phi_i} \cdots \xout{\phi_k} \cdots  \xout{\phi_j} \cdots \xout{\phi_l} \cdots
}%end normalprod
\nonumber\\
&+&
\cdots
\nonumber\\
\label{eqn:WickTheoremRealSc}
\end{eqnarray}
where $\phi_i$ stands for $\phi(x_i)$.
For the complex scalar field, we have a similar formula as the above but
contracts are taken only among $\varphi$ and $\varphi^\dagger$ since
ones among the same kind disappears as it can be read from Eq.  (\ref{eqn:CplxScProp}).
Let us examine an example in which normal products of complex scalar fields 
are involved inside a time ordered product:
\begin{eqnarray}
T[\normalprod{\varphi_1^\dagger \varphi_2} \normalprod{\varphi_3^\dagger \varphi_4}]
&=&
\normalprod{\varphi_1^\dagger \varphi_2 \varphi_3^\dagger \varphi_4}
+
\acontraction[1ex]{:\!}{\varphi}{_1^\dagger \varphi_2 \!:\, : \!\varphi_3^\dagger}{\varphi}
:\!\varphi_1^\dagger \varphi_2 \!:\, : \!\varphi_3^\dagger \varphi_4 \!:
+
\acontraction[1ex]{:\!\varphi_1^\dagger}{\varphi}{_2 \!:\, : \!}{\varphi}
:\!\varphi_1^\dagger \varphi_2 \!:\, : \!\varphi_3^\dagger \varphi_4 \!:
\nonumber\\
&& +
\acontraction[1ex]{:\!\varphi_1^\dagger}{\varphi}{_2 \!:\, : \!}{\varphi}
\acontraction[2ex]{:\!}{\varphi}{_1^\dagger \varphi_2 \!:\, : \!\varphi_3^\dagger}{\varphi}
:\!\varphi_1^\dagger \varphi_2 \!:\, : \!\varphi_3^\dagger \varphi_4 \!:
\nonumber\\
&=&
\normalprod{\varphi_1^\dagger \varphi_2 \varphi_3^\dagger \varphi_4}
+
\Delta_F(x_1 - x_4) \normalprod{\varphi_2 \varphi_3^\dagger}
+
\Delta_F(x_2 - x_3) \normalprod{\varphi_1^\dagger \varphi_4}
\nonumber\\
&& +
\Delta_F(x_1 - x_4) \Delta_F(x_2 - x_3)
\end{eqnarray}






%<<<<<<<<<<<<<<<<<<<<<<<<<<<<<<<<<
\newpage
\subsection{Dirac Fields}
Massive Spin 1/2.
\subsubsection{Classical Free Field}
\begin{equation}
{\cal L} = i \bar{\psi}(x) \slashed{\partial} \psi(x) - m \bar{\psi}(x) \psi(x)
\end{equation}
where $\slashed{\partial} = \gamma^\mu \partial_\mu$ and
$\bar{\psi}(x) = \psi^\dagger (x) \gamma^0$.
\begin{equation}
%\slashed{\partial} = 0
\left(i \slashed{\partial} + m \right) \psi(x) = 0\,,
\hspace{3mm}
\bar{\psi}(x) \left(i \stackrel{\leftarrow}{\slashed{\partial}} + m \right)  = 0
\end{equation}
\subsubsection{Quantized Free Field}
\begin{equation}
\psi(x) 
=
\int \frac{d^3 \bld{p}}{\sqrt{(2\pi)^3} 2 p^0}
\sum_{s = \pm 1}
\left[
c(\bld{p}, s) u(\bld{p}, s) e^{-ipx}
+
d^\dagger(\bld{p}, s) v(\bld{p}, s) e^{ipx}
\right]
\end{equation}
\begin{equation}
\{ c(\bld{p}, s), c^\dagger(\bld{p}', s') \}
=
\{ d(\bld{p}, s), d^\dagger(\bld{p}', s') \}
=
2p^0 \delta_{ss'}
\delta^3(\bld{p} - \bld{p}')
\end{equation}
\begin{equation}
\begin{array}{l}
(\slashed{p} - m) u(\bld{p}, s) = 0\,,
\hspace{3mm}
\bar{u}(\bld{p}, s)(\slashed{p} - m)  = 0\,,
\\
(\slashed{p} + m) v(\bld{p}, s) = 0\,,
\hspace{3mm}
\bar{v}(\bld{p}, s)(\slashed{p} + m)  = 0\,,
\end{array}
\end{equation}

Propagetor
\begin{eqnarray}
S_F(q) &=& i \int d^4x e^{iqx}
\bra 0 \braend T[\psi(x) \bar{\psi}(0)]
\ketend 0 \ket
\nonumber\\
&=&
\frac{-1}{\slashed{q} - m + i \epsilon}
=
- \frac{\slashed{q} + m}{ q^2 - m^2 + i \epsilon}
\end{eqnarray}

\begin{comment}
\subsubsection{Majorana Field}
Massless, spin1/2, selfconjugate.
\begin{equation}
i \slashed{\partial} \psi(x) = 0\,,
\hspace{3mm}
\bar{\psi}(x) i \stackrel{\leftarrow}{\slashed{\partial}}   = 0
\end{equation}
\end{comment}
\subsubsection{Dirac Yukawa Theory}

\newpage
\subsection{Massive Vector  Fields}
Spin 1.  
\begin{equation}
\left( \Box + m^2 \right) \varphi^\mu(x) = 0
\end{equation}
\subsection{Local Gauge Symmetries}
U(1), color SU(3), ...
\subsubsection{Photon Field}
Gauge bosons with unviolated symmetries. Electromagnetic field, gluons.
\begin{equation}
\Box A^\mu(x) = 0
\end{equation}
\subsection{spin~3/2 Field}
$\Delta$
\subsection{Renormalization}

%-------------------------------------------------------- comment block start
\begin{comment}
%========================================  section x
\newpage
\setcounter{section}{10}
\section{interference terms}
\section{$\gamma p \rightarrow \phi X$}
\section{Sasha PRC72}
\section{$\gamma p \rightarrow K \Lambda$}
%-------------------------------------------------------- comment block end
\end{comment}

\newpage
%===============================<<<<<<<<<<<<<<<<<<<<<
\begin{thebibliography}{9}
\bibitem{ref:Donnachie} S. Donnachie, G. Dosch, P. Landshoff and O. Nachtmann, 
	{\it Pomeron Physics and QCD}, Cambridge Monographs, 2002.
\bibitem{ref:Collins} P. D. B. Collins,
	{\it Regge theory and high energy physics}, Cambridge Monographs, 1977.
\bibitem{ref:Itzykson-Zuber} Claude Itzykson and Jean-Bernard Zuber,
	{\it Qantum Field Theory}, McGraw-Hill, 1980.
\bibitem{ref:NIsh.1-2} K. Nishijima, {\it "Fields and Particles}, Benjamin, 1969, p.9-14.
\end{thebibliography}

\end{document}

