\begin{eqnarray}
{\cal L} &=& \frac{1}{2} 
\left( \underline{\partial} \varphi(\underline{x}) \right)^2
-\frac{1}{2} m^2 \varphi^2(\underline{x})
\nonumber\\
&=& \frac{1}{2} 
\partial_\mu \varphi(\underline{x})\cdot \partial^\mu \varphi(\underline{x})
-\frac{1}{2} m^2 \varphi^2(\underline{x})
\label{eqn:RSCLagrangiandensity}
\end{eqnarray}
where a dot in the last equation indicates the former derivative acting only on the first $\varphi$.
We frequently omit this dot.
Under a Lorentz transformation $\underline{x} \mapsto \underline{x}' = L\underline{x}$,
the field $\varphi(\underline{x})$ transforms as
\begin{equation}
\varphi(x) \mapsto \varphi'(x') = \varphi(x)\,.
\end{equation}
Here and hereafter, we omitt underlines on Lorentz vectors.
\begin{equation}
\begin{array}{c}
\displaystyle
\frac{\partial {\cal L}}{\partial(\partial_\mu \varphi)} = \partial^\mu \varphi
\\
\displaystyle
\pi(x) = \frac{ \partial {\cal L}}{ \partial \dot{\varphi}(x) } = \dot{\varphi}
%\pi_{}(x) = \partial_0 \varphi_{cl}(x)
%=
%\int \frac{d^3 \bld{k}}{\sqrt{(2\pi)^3}} (- i k^0 ) \left[
%a(\bld{k}) e^{-i k \cdot x} - b^\dagger(\bld{k})e^{i k \cdot x} \right]
\end{array}
\end{equation}
Euler-Lagrange equation
\begin{equation}
\left( \Box + m^2 \right) \varphi_{}(x) = 0
\hspace{10mm}
\mbox{Klein-Gordon}
\label{eqn:Klein-Gordon}
\end{equation}
where $\Box \leftdef \partial^2 = \partial_\mu \partial^\mu = \partial_0^2 - \bld{\partial}^2$.
Classical solution of the field equation (\ref{eqn:Klein-Gordon}) is written as
\footnote{%------------------------------- footnote >>
The following consideration lays behind:
\begin{eqnarray*}
\varphi(x)
&=&
\frac{1}{\sqrt{(2\pi)^3}}
\int 
d^4 k
\delta(k^2 - m^2)
\left[
\theta(k^0) + \theta(-k^0)
\right]
\tilde{\varphi} (k)
e^{-i k \cdot x}
\nonumber\\
&=&
\frac{1}{\sqrt{(2\pi)^3}}
\int 
d^4 k
\delta(k^2 - m^2)
\left[
\theta(k^0)
\tilde{\varphi} (k)
e^{-i k \cdot x}
 + \theta(k^0)
\tilde{\varphi} (-k)
e^{i k \cdot x}
\right]
\nonumber\\
&=&
\int 
\frac{d^3 \bld{k}}{\sqrt{(2\pi)^3}2 k^0}
\left[
\tilde{\varphi} (k)
e^{-i k \cdot x}
 + 
\tilde{\varphi} (-k)
e^{i k \cdot x}
\right]
\nonumber\\
&\rightdef&
\int 
\frac{d^3 \bld{k}}{\sqrt{(2\pi)^3}2 k^0}
\left[
a (\bld{k})
e^{-i k \cdot x}
 +
 a^*(\bld{k}) 
e^{i k \cdot x}
\right]
\end{eqnarray*}
}%------------------------------- footnote //
\begin{equation}
\varphi_{cl}(x) = \int \frac{d^3 \bld{k}}{\sqrt{(2\pi)^3}2k^0} \left[
a(\bld{k}) e^{-i k \cdot x} + a^*(\bld{k})e^{i k \cdot x} \right]
\label{eqn:KGclassicalsol}
\end{equation}
where $k^0 = + \sqrt{\bld{k}^2 + m^2}$. 
Canonical momentum field reads,
\begin{equation}
\pi_{cl}(x) = \partial_0 \varphi_{cl}(x)
=
\int \frac{d^3 \bld{k}}{\sqrt{(2\pi)^3}2k^0}  \left[
- i k^0 a(\bld{k}) e^{-i k \cdot x} + i k^0 a^*(\bld{k})e^{i k \cdot x} \right]
\label{eqn:KGclassicalpi}
\end{equation}


%<<<<<<<<<<<<<<<<<<<<<<<<<<<<<<<<<<<
Canonical quantization
\begin{equation}
\begin{array}{c}
[ \varphi(t,\bld{x}), \pi(t, \bld{y})] = i \delta^3 (\bld{x} - \bld{y} )\,,
\vspace{2mm}
\\
\left[ \varphi(t,\bld{x}), \varphi(t,\bld{y}) \right] = 0\,,
\;\;\;\;\left[ \pi(t,\bld{x}), \pi(t,\bld{y}) \right] = 0\,.
\end{array}
\label{eqn:cancomm_eqtime_RS}
\end{equation}
This is equivalent with requiring 
\begin{equation}
\begin{array}{c}
[ a(\bld{k}), a^\dagger(\bld{k}')] = 2k^0 \delta^3 (\bld{k} - \bld{k'} )\;,
\vspace{2mm}
\\
\left[ a(\bld{k}), a(\bld{k}') \right] = 0\;,\;\;\;\;\left[ a^\dagger(\bld{k}), a^\dagger(\bld{k}') \right] = 0
\end{array}
\label{eqn:cancomm_RS}
\end{equation}
and writing
\begin{equation}
\varphi_{}(x) = \int \frac{d^3 \bld{k}}{\sqrt{(2\pi)^3}2k^0} \left[
a(\bld{k}) e^{-i k \cdot x} + a^\dagger(\bld{k})e^{i k \cdot x} \right]
\label{eqn:RS_fieldexpansion}
\end{equation}
\begin{equation}
\pi_{}(x) 
=
\frac{-i}{2}
\int \frac{d^3 \bld{k}}{\sqrt{(2\pi)^3}}  \left[
a(\bld{k}) e^{-i k \cdot x} - a^\dagger(\bld{k})e^{i k \cdot x} \right]\,,
\label{eqn:RS_pifieldexpansion}
\end{equation}
They have the same form as Eqs. (\ref{eqn:KGclassicalsol}) and (\ref{eqn:KGclassicalpi})
but now the coefficients $a(\bld{k})$ and $a^\dagger(\bld{k})$ are
quantum operators.

%\bigskip
%
%\begin{equation}
%\varphi_{}(x) = \int \frac{d^3 \bld{k}}{\sqrt{(2\pi)^3}2k^0} \left[
%a(\bld{k}) e^{-i k \cdot x} + a^\dagger(\bld{k})e^{i k \cdot x} \right]\,,
%\label{eqn:RS_fieldexpansion}
%\end{equation}
%\begin{equation}
%\pi_{}(x) 
%=
%\frac{-i}{2}
%\int \frac{d^3 \bld{k}}{\sqrt{(2\pi)^3}}  \left[
%a(\bld{k}) e^{-i k \cdot x} - a^\dagger(\bld{k})e^{i k \cdot x} \right]\,,
%\label{eqn:RS_pifieldexpansion}
%\end{equation}
%\begin{eqnarray}
%&&\;\;\;\;\;\;[ a(\bld{k}), a^\dagger(\bld{k}')] = 2k^0 \delta^3 (\bld{k} - \bld{k'} )\;,
%\label{eqn:cancomm_RS}
%\\
%&&\left[ a(\bld{k}), a(\bld{k}') \right] = 0\;,\;\;\;\;\left[ a^\dagger(\bld{k}), a^\dagger(\bld{k}') \right] = 0
%\nonumber
%\label{eqn:realSccreannCommRel}
%\end{eqnarray}
%
\verb/-----------.-----------.-----------.-----------.-----------/\\
\vspace{-3mm}
{\small
\begin{center}
Addendum: Proof of (\ref{eqn:cancomm_eqtime_RS}) $\Leftrightarrow$ (\ref{eqn:cancomm_RS})
\end{center}
Proof of the necessity of Eq. (\ref{eqn:cancomm_RS}) is straightforward,
We show the sufficiency
of Eq. (\ref{eqn:cancomm_eqtime_RS}) in the following.
We may write Eqs. (\ref{eqn:RS_fieldexpansion}) and (\ref{eqn:RS_pifieldexpansion})
as
\begin{equation}
\begin{array}{l}
\displaystyle
\varphi(x) = \int \frac{d^3\bld{k}}{\sqrt{(2\pi)^3}} Q_{\bld{k}} (t) e^{i \bld{k} \cdot \bld{x}}\,,
\\
\displaystyle
\pi_{}(x)  = \int \frac{d^3\bld{k}}{\sqrt{(2\pi)^3}} P_{\bld{k}} (t) e^{- i \bld{k} \cdot \bld{x}}\,,
\end{array}
\end{equation}
with
\begin{equation}
Q_{\bld{k}}(t)
=
\frac{1}{2k^0} \left[
a(\bld{k}) e^{-i k^0 t} + a^\dagger(-\bld{k}) e^{i k^0 t} \right]
\label{eqn:RSC_Qk_def}
\end{equation}
\begin{equation}
P_{\bld{k}}(t)
=
\frac{i}{2} \left[
a^\dagger(\bld{k}) e^{i k^0 t} - a(-\bld{k}) e^{- i k^0 t} \right]
\label{eqn:RSC_Pk_def}
\end{equation}
Relationships $Q^\dagger_{\bld{k}} = Q_{-\bld{k}}$ and
$P^\dagger_{\bld{k}} = P_{-\bld{k}}$ ensure 
that $\varphi$ and $\pi$ are real.
From the linear independence of Fourier components, we have
\begin{equation}
\begin{array}{l}
0 = [\varphi(t, \bld{x}), \varphi(t, \bld{y})]
\;\Longleftrightarrow\;
[Q_{\bld{k}}(t), Q_{\bld{k'}}(t)] = 0\,,
\vspace{2mm}
\\
0 = [\pi(t, \bld{x}), \pi(t, \bld{y})]
\;\Longleftrightarrow\;
[P_{\bld{k}}(t), P_{\bld{k'}}(t)] = 0\,,
\end{array}
\label{eqn:RSC_QQcomm}
\end{equation}
and
\begin{eqnarray}
i\delta^3(\bld{x} - \bld{y})
&=&
\int \frac{d^3 \bld{k} d^3 \bld{k}'}{(2\pi)^3}
i \delta^3 (\bld{k} - \bld{k}')
e^{i \bld{k} \cdot \bld{x} - i \bld{k}' \bld{y}}
\nonumber\\
&=&
[\varphi(t, \bld{x}), \pi(t, \bld{y}) ]
\nonumber\\
&=&
\int \frac{d^3 \bld{k} d^3 \bld{k}'}{(2\pi)^3} [Q_{\bld{k}}, P_{\bld{k}'}]
e^{i \bld{k} \cdot \bld{x} - i \bld{k}' \cdot \bld{y}}
\nonumber\\
&\Longleftrightarrow&    
\nonumber\\
\left[ Q_{\bld{k}}, P_{\bld{k}'} \right] &=& i \delta^3 (\bld{k} - \bld{k}')
\label{eqn:RSC_QPcomm}
\end{eqnarray}
Eqs. (\ref{eqn:RSC_Qk_def}) and (\ref{eqn:RSC_Pk_def})
reads,
\begin{equation}
a(\bld{k}) = \left( k^0 Q_{\bld{k}} (t) + i P_{\bld{k}}^\dagger (t) \right)
e^{i k^0 t}\,,
\label{eqn:RSC_a_by_QP} 
\end{equation}
\begin{equation}
a^\dagger(\bld{k}) = \left( k^0 Q_{\bld{k}}^\dagger (t) - i P_{\bld{k}} (t) \right)
e^{- i k^0 t}\,,
\label{eqn:RSC_a_by_QP} 
\end{equation}
and Eq. (\ref{eqn:cancomm_RS}) follows
from Eqs. (\ref{eqn:RSC_QQcomm}) and (\ref{eqn:RSC_QPcomm}).
}\\
\verb/-----------.-----------.-----------.-----------.-----------/\\


%=================================
\noindent
Notice that $dim [\varphi] = E^1$ as can be seen from Eq. (\ref{eqn:RSCLagrangiandensity})
and $dim [\pi] = E^2$. Thus they are physical quantities  quite different from
those in the case of the Schr\"odinger field theory.
Nevertheless, operators $a$ and $a^\dagger$ satisfy conditions to be 
annihilation and creation operators. We define the vacuum state by
\begin{equation}
a(\bld{k}) \ketend 0 \ket = 0
\end{equation}
Hamiltonian density in the classical level is obtained from Eq. (\ref{eqn:Hamiltoniandensity}) as
\begin{eqnarray}
{\cal H} &=& \pi(x) \dot{\varphi}(x) - \frac{1}{2} \left\{
\left( \dot{\varphi}(x) \right)^2 - 
\left( \bld{\partial} \varphi(x) \right)^2 \right\}
+ \frac{1}{2} m^2 \varphi^2(x)
\nonumber\\
&=&
\frac{1}{2} \left\{
\pi^2 (x) + \left( \bld{\partial} \varphi(x) \right)^2 + m^2 \varphi^2(x)
\right\}
\label{eqn:KGHamiltoniandens}
\end{eqnarray}
To get the quantum level Hamiltonian,
we substitute Eqs (\ref{eqn:RS_fieldexpansion}) and (\ref{eqn:RS_pifieldexpansion})
into Eq. (\ref{eqn:KGHamiltoniandens}). However, since ${\cal H}$ is quadratic
in operators, we need to specify the ordering of them and write
\begin{equation}
\hat{H} = \int d^3 \bld{x} \hat{{\cal H}}(x)\,,
\hspace{3mm}
\hat{{\cal H}}(x) = 
\normord{ \frac{1}{2} \left\{
\pi^2 (x) + \left( \bld{\partial} \varphi(x) \right)^2 + m^2 \varphi^2(x)
\right\}
}\,,
\label{eqn:KG_Hamiltonian}
\end{equation}
where $\normord{\dots}$ denotes the normal order product \cite{ref:Peskin-Schroeder}-\cite{ref:Tong}.
As the result, we obtain an expression
\begin{equation}
\hat{H} = \int \frac{d^3 \bld{k}}{2k^0} k^0 a^\dagger(\bld{k}) a(\bld{k})
\label{eqn:KG_Hamiltonian_aadagg}
\end{equation}
This form indicates that our Hamiltonian is diagonalized.
To see this let us define a single particle state 
$\ketend \bld{p} \ket = a^\dagger(\bld{p}) \ketend 0 \ket$.
\footnote{
This state is normalized as
\[
\bra p \braend p' \ket
= \bra 0 \braend [a(p), a^\dagger(p')] \ketend 0 \ket
= 2 k^0 \delta^3 (\bld{p} - \bld{p}')
\]
}
We see from Eq. (\ref{eqn:cancomm_RS}) that
$\hat{H} \ketend \bld{p} \ket = E \ketend \bld{p} \ket$ with
$E = \sqrt{\bld{p}^2 + m^2}$.
Thus, our field acquires the particle interpretation
through the canonical quantization.

The total number operator is defined as
\begin{equation}
\hat{N} = \int \frac{d^3 \bld{k}}{2k^0} a^\dagger(\bld{k}) a(\bld{k})
\label{eqn:KG_totalNumberOp}
\end{equation}
It holds that
\begin{equation}
\begin{array}{l}
\displaystyle
\hat{N} a(\bld{k}) = 
\int \frac{d^3 \bld{k}'}{2k^{0'}} \left\{
a(\bld{k}) a^\dagger(\bld{k}') - 2 k^0 \delta^3(\bld{k} - \bld{k}') \right\} a(\bld{k}')
= a(\bld{k}) (\hat{N} - 1)\,,
\\
\hat{N} a^\dagger(\bld{k}) 
= a^\dagger(\bld{k}) (\hat{N} + 1)
\end{array}
\end{equation}

Hamiltonian in Eq. (\ref{eqn:KG_Hamiltonian_aadagg}) coinsides with
one derived (with a care on the operator orderings) from Eq. (\ref{eqn:NoetrherEnergyMomentum})
as $\int d^3\bld{x} T^{00}(x)$.
Other components of $T^{0 \mu}$ give an expression for the momentum as
\begin{eqnarray}
\hat{\bld{P}} =
\int \frac{d^3 \bld{k}}{2k^0} \bld{k} a^\dagger(\bld{k}) a(\bld{k})
\end{eqnarray}
We certainly have $\hat{\bld{P}} \ketend \bld{p} \ket = \bld{p}\ketend \bld{p} \ket$.
We omit here to write expressions for angular momenta corresponding to
Eq. (\ref{eqn:NoetherAngularMomentum}).

%<<<<<<<<<<<<<<<<<<<<<<<<<<<<<<<<<<<<<< below is commented out
\begin{comment}
\noindent
%\begin{equation}
%\left( \Box + m^2 \right) \varphi(x) = 0
%\end{equation}
Propagator
\begin{equation}
\left( \Box + m^2 \right) \Delta_F(x) = \delta^4(x) 
\end{equation}
\begin{eqnarray}
\Delta_F(q) 
&=&
i \int d^4x e^{iqx} 
\bra 0 \braend \mbox{T}\varphi(x) \varphi(0) 
\ketend 0 \ket
\nonumber\\
&=&
\frac{-1}{q^2 - m^2 + i\epsilon}\,,
\end{eqnarray}
where $\epsilon$ is infinitesimal positive number.
T-product
\begin{equation}
T[\varphi(x) \varphi(y)] =
\theta(x^0 - y^0)\varphi(x) \varphi(y)
+
\theta(y^0 - x^0)\varphi(y) \varphi(x)
\end{equation}
One will find
\[
\left( \Box + m^2 \right) T[\varphi(x) \varphi(0)] = -i \delta^4(x)
\]
\begin{equation}
\Delta_F(x) = \int \frac{d^4 q}{(2\pi)^4} e^{-iqx} \Delta_F(q)
\end{equation}
\end{comment}
