For each degree of freedom of continuous transformation of fields against which
the action remains invariant, there exist a conserved current $J^\mu (x)$.
Particularly, let a infinitesimal transformation with $s$ parameters $\delta \alpha_\lambda$ be
written as
\begin{eqnarray}
\begin{array}{l}
x^\mu \mapsto x^{\mu '} = x^\mu + \delta x^\mu\,,
\hspace{3mm}
\delta x^\mu = \sum_{\lambda=1}^s X_\lambda^\mu \delta \alpha_\lambda
\vspace{2mm}
\\
\varphi (x) \mapsto \varphi' (x')
=
\varphi(x) + \delta \varphi(x)\,,
\hspace{3mm}
\delta \varphi(x) =
\sum_{\lambda=1}^s \Phi_{ \lambda} (x) \delta \alpha_\lambda
\end{array}
\label{eqn:Noeth_infsimtransf}
\end{eqnarray}
This transformation may be associated with a space-time transformation
specified by $X_\lambda^\mu$ as shown in the first line in Eq. (\ref{eqn:Noeth_infsimtransf}).
When the transformation is related only with  internal degree of freedom of the field , 
one may set $X_\lambda^\mu = 0$.
When the action ${\cal A}$ is invariant under this transformation,
$s$ currents
\begin{eqnarray}
J_\lambda^\mu (x)
&\leftdef&
- \frac{\partial {\cal L}}{\partial (\partial_\mu \varphi)}
\left(
\Phi_{\lambda} (x) - \partial_\nu \varphi(x) X_\lambda^\nu 
\right)
-
{\cal L}(x) 
X_\lambda^\mu 
\label{eqn:NoetherCurrentGeneral}
\end{eqnarray}
are all conserved ones. Namely,
\begin{eqnarray}
\partial_\mu J_\lambda^\mu (x) = 0\,,
\hspace{3mm}
\lambda = 1, \dots, s
\end{eqnarray}
hold and corresponding charges
\begin{eqnarray}
Q_\lambda = 
\int d^3 \bld{x}
J_\lambda^0 (x)
\label{eqn:NoetherChargeGeneral}
\end{eqnarray}
are conserved.
Eq. (\ref{eqn:NoetherCurrentGeneral}) defines the Noether currents
and quantities $Q_\lambda$ in Eq. (\ref{eqn:NoetherChargeGeneral})
are called Noether charges.

%=====================================
\subsubsection{Energy-Momentum tensor}
Our action must be invariant under space-time translations.
The infinitesimal transformation
\begin{equation}
x^\mu \mapsto x^{\mu '} = x^\mu + \delta a^\mu
\end{equation}
has 4 continuous parameters $\delta a^\mu$ and in our notation
$X_\nu^\mu = \delta_\nu^\mu$. 
The field is invariant
\begin{equation}
\varphi (x) 
\mapsto
\varphi' (x') 
=
\varphi (x) 
\end{equation}
and $\Phi_{\nu} = 0$.
The Noether current (\ref{eqn:NoetherCurrentGeneral}) reads
\begin{eqnarray}
T^\mu_{\;\;\nu}
=
 \frac{\partial {\cal L}}{\partial (\partial_\mu \varphi)}
\partial_\nu \varphi(x) 
-
{\cal L}(x) 
g_\nu^\mu 
\label{eqn:NoetrherEnergyMomentum}
\end{eqnarray}

\subsubsection{Angular Momentum tensor}
Consider an infinitesimal spatial rotation
\begin{equation}
x^\mu \mapsto x^{\mu '} = x^\mu + \delta \omega^\mu_{\;\nu} x^\nu
\end{equation}
This transformation has 6 continuous parameters $\delta \omega^{\mu \nu} = -\delta \omega^{\nu \mu}$.
$\delta x^\mu$ in Eq. (\ref{eqn:Noeth_infsimtransf}) is given as
\begin{eqnarray}
\delta x^\mu &=& \delta \omega^\mu_{\;\nu} x^\nu
\nonumber\\
&=&
g^\mu_\xi g_{\nu \eta} \delta \omega^{\xi \eta} x^\nu
\nonumber\\
&=&
\frac{1}{2}( g^\mu_\xi g_{\nu \eta} - g^\mu_\eta g_{\nu \xi} ) x^\nu \delta \omega^{\xi \eta}
\nonumber\\
&=&
\frac{1}{2} (a_{\xi \eta})^\mu_{\;\nu} x^\nu \delta \omega^{\xi \eta}
\end{eqnarray}
where
\begin{eqnarray}
(a_{\xi \eta})^\mu_{\; \nu}
=
g^\mu_\xi g_{\nu \eta} - g^\mu_\eta g_{\nu \xi}
\end{eqnarray}
We have 
\begin{eqnarray}
X^\mu_{\;\;\xi \eta}
=
\frac{1}{2} (a_{\xi \eta})^\mu_{\;\nu} x^\nu
\end{eqnarray}
The field is transformed as
\begin{eqnarray}
\varphi_\alpha (x) 
\mapsto
\varphi_\alpha' (x') 
=
\delta \varphi_\alpha (x) 
\,,
\hspace{3mm}
\delta \varphi_\alpha (x) 
= 
\frac{1}{2} (S_{\xi \eta})_\alpha^{\;\beta}
\varphi_\beta (x) 
\delta \omega^{\xi \eta}
\end{eqnarray}
where
\begin{eqnarray}
(S_{\mu \nu})_\alpha^{\;\beta}
=
\left\{
\begin{array}{l}
0
\hspace{3mm}
\cdots
\mbox{scalar}
\\
(a_{\mu \nu})_\alpha^{\;\;\beta}
\hspace{3mm}
\cdots
\mbox{vector}
\\
\frac{1}{4} [ \gamma_\mu, \gamma_\nu ]_{\alpha}^{\;\;\beta}
\hspace{3mm}
\cdots
\mbox{Dirac spinor}
\end{array}
\right.
\end{eqnarray}
We have
\begin{eqnarray}
\Phi_{\alpha \xi \eta} = 
\frac{1}{2} (S_{\xi \eta})_\alpha^{\;\beta}
\varphi_\beta (x) 
\end{eqnarray}
The Noether current
\begin{eqnarray}
M^\mu_{\xi \eta}
&\leftdef&
2 J^\mu_{\;\xi \eta}
\nonumber\\
&=&
-2 \frac{\partial {\cal L}}{\partial \varphi_{\alpha : \mu}}
\left(
\Phi_{\alpha \xi \eta} (x) - \partial_\nu \varphi_\alpha(x) X_{\;\xi \eta}^\nu 
\right)
-2
{\cal L}(x) 
X_{\;\xi \eta}^\mu 
\nonumber\\
&=&
\left(
 \frac{\partial {\cal L}}{\partial \varphi_{\alpha : \mu}}
\partial_\nu \varphi_\alpha(x) 
- 
{\cal L} g^\mu_\nu
\right)
\cdot 2X^\nu_{\;\;\xi \eta}
- 2  \frac{\partial {\cal L}}{\partial \varphi_{\alpha : \mu}}
\Phi_{\alpha \xi \eta} (x)
\nonumber\\
&=&
(a_{\xi \eta})^\nu_{\;\rho} x^\rho \cdot
T^\mu_{\;\nu}
-  \frac{\partial {\cal L}}{\partial \varphi_{\alpha : \mu}}
(S_{\xi \eta})_\alpha^{\;\beta}
\varphi_\beta (x) 
\\
&\rightdef&
L^\mu_{(\xi \eta)} + S^\mu_{(\xi \eta)}
\end{eqnarray}
The last equation defines orbital and spin parts of the current.
The total angular momentum vector is defined as
\begin{eqnarray}
J^{k} = \frac{1}{2} \epsilon^{kij} M^{ij}\,,
\hspace{3mm}
M_{\xi \eta} = \int d^3 \bld{x} M^0_{\xi \eta}
\label{eqn:NoetherAngularMomentum}
\end{eqnarray}

%===================================================================
\subsubsection{Electric Charge}
For a complex field, gauge transformation of the first kind is defined as
\begin{eqnarray}
\varphi(x) \mapsto \varphi'_\alpha(x) = e^{ie \theta} \varphi(x)\,,
\hspace{3mm}
\varphi^*(x) \mapsto {\varphi^{*}}'(x)= e^{-ie \theta} \varphi^*(x)
\end{eqnarray}
This transformation consists a U(1). Considering infinitesimal $\theta$, 
we read in Eq. (\ref{eqn:Noeth_infsimtransf}) that
\begin{eqnarray}
X^\mu = 0\,,
\hspace{3mm}
\Phi = i e \varphi\,,
\hspace{3mm}
\Phi^* = -i e \varphi^*\,,
\end{eqnarray}
Invariance of the ${\cal L}$agragian density leads
\begin{equation}
\partial_\mu J^\mu = 0
\end{equation}
for
\begin{eqnarray}
J^\mu = 
ie \left(
\varphi^*_{} \frac{\partial {\cal L}}{\partial (\partial_\mu \varphi^*_{})} 
-\frac{\partial {\cal L}}{\partial (\partial_\mu \varphi_{})} \varphi_{}
\right)
\label{eqn:NoetherElectricCharge}
\end{eqnarray}
The corresponding charge is $Q = \int d^3 \bld{x} J^0$.

%===================================================================
\subsubsection{Internal Global Symmetries}
We consider a internal global SU(n) symmetry. 
The field $\varphi_a(x)$ is subject to a transformation
\begin{eqnarray}
\varphi_a(x)
\mapsto
\varphi'_a(x)
=
(e^{i \alpha_i G_i})_a^{\;b}
\varphi_b(x)
\label{eqn:LieGrTransfFields}
\end{eqnarray}
where $\alpha_i, (i = 1, \dots, n^2-1)$ are continuous real parameters, $G_i$ are matrix representations of
generators. 
Considering infinitesimal $\alpha_i$, we have in Eq. (\ref{eqn:Noeth_infsimtransf}) that
\begin{equation}
\Phi_{a i} = i (G_i)_a^{\;b} \varphi_b\,,
\end{equation}
The Noether current is given by
\begin{eqnarray}
J^\mu_i
&=&
-
i
\frac{\partial {\cal L}}{\partial (\partial_\mu \varphi_{a })}
(G_i)_a^{\;b} \varphi_b
\label{eqn:NoetherCurrGeneral}
\end{eqnarray}
and the corresponding $n^2 - 1$ charges are
\begin{eqnarray}
C_i
=
-i
\int d^3 \bld{x}
\frac{\partial {\cal L}}{\partial \dot{\varphi_{a }}}
(G_i)_a^{\;b} \varphi_b
\label{eqn:NoetherChrgGeneral}
\end{eqnarray}
Among $n^2-1$ generators $G_i$, only $n-1$ commutes to each other.
Accordingly, $n-1$ charges among $n^2-1$ can be diagonalized at the same time.

\bigskip

\noindent
■SU(2)\\
The conserved (internal) vector is the isospin $\bld{I} = (C_1, C_2, C_3)$.
\begin{eqnarray}
\varphi_a \in
\yng(1)&:&
\hspace{3mm}
G_i = \frac{1}{2} \sigma_i\,
\hspace{3mm}
{\scriptstyle \tiny
\sigma_1 = \left[
\begin{array}{cc}
0&1 \\ 1&0
\end{array}
\right]\,,
\sigma_2 = \left[
\begin{array}{cc}
\scriptstyle
0&-i \\ i&0
\end{array}
\right]\,,
\sigma_3 = \left[
\begin{array}{cc}
\scriptstyle
1&0 \\ 0&-1
\end{array}
\right]
} %scriptstyle, tiny
\label{eqn:SU2GenFundamental}
\\
\varphi_a \in
\yng(2)&:&
\hspace{3mm}
G_i = t_i\,,
\hspace{3mm}
(t_i)_{jk} = -i \epsilon_{ijk}
\end{eqnarray}


\noindent
■SU(3)\\
\hspace{15mm}$\varphi_a \in \yng(1):$
\hspace{7mm}$G_i = \frac{1}{2} \lambda_i$
\begin{eqnarray}
\begin{array}{ccc}
\lambda_1 =
\left(
\begin{array}{ccc}
0&1&0\\
1&0&0\\
0&0&0\\
\end{array}
\right)
&
\lambda_2 =
\left(
\begin{array}{ccc}
0&-i&0\\
i&0&0\\
0&0&0\\
\end{array}
\right)
&
\lambda_3 =
\left(
\begin{array}{ccc}
1&0&0\\
0&-1&0\\
0&0&0\\
\end{array}
\right)
\vspace{2mm}
\\
\lambda_4 =
\left(
\begin{array}{ccc}
0&0&1\\
0&0&0\\
1&0&0\\
\end{array}
\right)
&
\lambda_5 =
\left(
\begin{array}{ccc}
0&0&-i\\
0&0&0\\
i&0&0\\
\end{array}
\right)
&
\lambda_6 =
\left(
\begin{array}{ccc}
0&0&0\\
0&0&1\\
0&1&0\\
\end{array}
\right)
\vspace{2mm}
\\
\lambda_7 =
\left(
\begin{array}{ccc}
0&0&0\\
0&0&-i\\
0&i&0\\
\end{array}
\right)
&
\lambda_8 =
\frac{1}{\sqrt{3}}
\left(
\begin{array}{ccc}
1&0&0\\
0&1&0\\
0&0&-2\\
\end{array}
\right)
&
\end{array}
\end{eqnarray}
\begin{eqnarray}
%\hspace{15mm}
\varphi_a \in \yng(2,1):
\hspace{3mm}
G_i = T_i\,,
\hspace{3mm}
(T_a)_{bc} = -i f_{abc}\mbox{(structure const.)}
%\hspace{25mm}
\end{eqnarray}
\begin{eqnarray}
\begin{array}{c}
T_\pm = G_1 \pm iG_2\,,
\hspace{3mm}
V_\pm = G_4 \pm iG_5\,,
\hspace{3mm}
U_\pm = G_6 \pm iG_7\,,
\\
T_3 = G_3\,,
\hspace{3mm}
Y = \frac{2}{\sqrt{3}} G_8
\end{array}
\end{eqnarray}
%===================================================================
%===================================================================
%===================================================================
