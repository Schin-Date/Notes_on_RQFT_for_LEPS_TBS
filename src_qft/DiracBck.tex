\noindent
massive, charged, spin 1/2\\
Spinors belong to the fundamental representation of the Poincar\'e Group SL(2,C).
\subsubsection{Classical Free Field}
\noindent
{\bf Dirac equation}

The Dirac equation is written as
\begin{eqnarray}
i \partial_t \psi (x) = 
\left(
\frac{1}{i} \bld{\alpha} \cdot \bld{\partial} + \beta m \right) \psi(x)
\end{eqnarray}
Solution $\psi(x)$ of this equation satisfies the Klein-Gordon equation
\footnote{
The Dirac equation is not the unique form which is
the first order in time derivative and the solution satisfies the K-G eq.
The Duffin-Kemmer equation satisfies these requirements
\cite{ref:Itzykson-Zuber}.
 } % end of footnote
if and only if
\begin{eqnarray}
\{ \alpha_i, \alpha_j \}_+ = 2\delta_{ij}\,,
\hspace{3mm}
\{ \alpha_i, \beta \}_+ = 0\,,
\hspace{3mm}
\beta^2 = 1
\label{eqn:Dialphbetacond}
\end{eqnarray}
They must be traceless:
\begin{eqnarray}
\tr \alpha_i &=& \tr \beta^2 \alpha_i = \tr \beta \alpha_i \beta = - \tr \alpha_i = 0
\nonumber\\
\tr \beta &=& \tr \alpha_i^2 \beta  = \tr \alpha_i \beta \alpha_i  = - \tr \beta = 0
\nonumber
\end{eqnarray}
The minimum dimension of $\alpha$'s and $\beta$ as matrices to satisfy the conditions 
(\ref{eqn:Dialphbetacond}) is 4. 
Introduce $\gamma$ matrices by
\begin{eqnarray}
\gamma^0 = \beta\,,
\hspace{3mm}
\gamma^i = \beta \alpha_i\
\end{eqnarray}
In terms of $\gamma$ matrices, Eq. (\ref{eqn:Dialphbetacond}) reads
\begin{equation}
\{ \gamma^\mu, \gamma^\nu \}_+ = 2 g^{\mu \nu}
\end{equation}
This is the Clifford algebra. This means $(\gamma^0)^2 = 1$ and $(\gamma^i)^2 = -1$ for $i = 1,2,3$.
$\gamma^0$ is obviously traceless and $\gamma^i$'s are also for
\begin{eqnarray}
\tr \gamma^i = \tr \beta \alpha_i = \tr \alpha_i \beta = -\tr \beta \alpha_i = 0\,.
\end{eqnarray}
The Dirac equation is then written as
\begin{eqnarray}
\left( i\slashed{\partial} - m \right) \psi(x) = 0
\label{eqn:DiracEqstandard}
\end{eqnarray}
where $\slashed{a} \leftdef \gamma_\mu a^\mu = \gamma \cdot a$ and
$\slashed{\partial} = \gamma^0 \partial_0 + \gamma^i \partial_i = \gamma^0 \partial_0 + \bld{\gamma}\cdot \bld{\partial}$.

We may choose $\alpha_i$'s and $\beta$ as hermitian matrices.
In that case we have
\begin{eqnarray}
\gamma^{0\dagger} = \gamma^0\,,
\hspace{3mm}
\gamma^{i\dagger} = - \gamma^i
\hspace{3mm}
\Longleftrightarrow
\hspace{3mm}
{\gamma^\mu}^\dagger = \gamma^0 \gamma^\mu \gamma^0
\label{eqn:hermitealphabeta}
\end{eqnarray}
Define the conjugate Dirac field by
\begin{equation}
\overline{\psi} = \psi^{*t} \gamma^0
\label{eqn:DefConjDiracField}
\end{equation}
Taking complex conjugate and transpose of Eq. (\ref{eqn:DiracEqstandard}), we obtain
\begin{eqnarray}
0 &=&  
\psi^{*t}(x) 
\left( -i \stackrel{\leftarrow}{\partial} \cdot \gamma^\dagger - m \right) 
\nonumber\\
&=&
\psi^{*t}(x) (\gamma^0)^2
\left( -i \stackrel{\leftarrow}{\partial} \cdot \gamma^\dagger - m \right) 
\nonumber\\
&=&
\overline{\psi}(x)
\left( -i \stackrel{\leftarrow}{\partial} \cdot \gamma^0\gamma^\dagger \gamma^0 - m \right) \gamma^0
\nonumber\\
&=&
- \overline{\psi}(x)
\left( i \stackrel{\leftarrow}{\partial} \cdot \gamma  + m \right) \gamma^0
\nonumber
\end{eqnarray}
Thus, the conjugate Dirac field satisfies
\begin{eqnarray}
\overline{\psi}(x)
\left( i \stackrel{\leftarrow}{\slashed{\partial}}  + m \right)
= 0
\end{eqnarray}
This equation is equivalent with Eq. (\ref{eqn:DiracEqstandard}) in the classical level
provided we choose $\gamma$'s in such a way that Eq. (\ref{eqn:hermitealphabeta}) holds.

\bigskip
%------------------------------------------------------------------------------------------------------------------------------------------------
\noindent
{\bf Dirac matrices}

We define
\begin{eqnarray}
\gamma_5 \equiv \gamma^5 \leftdef
i \gamma^0 \gamma^1 \gamma^2 \gamma^3
=
- \frac{i}{4!}
\epsilon_{\mu \nu \rho \sigma}
\gamma^\mu \gamma^\nu \gamma^\rho \gamma^\sigma
\end{eqnarray}
The $\gamma_5$ satisfies
\begin{eqnarray}
\{
\gamma_5, \gamma_\mu 
\}_+ = 0\,,
\hspace{3mm}
(\gamma_5)^2 = 1\,,
\hspace{3mm}
\gamma_5^\dagger = \gamma_5
\end{eqnarray}
We also define
\begin{eqnarray}
\sigma_{\mu \nu} \equiv
\frac{i}{2}
[ \gamma_\mu, \gamma_\nu]\,,
\end{eqnarray}
which has a property $\sigma_{\mu \nu}^\dagger = \gamma^0 \sigma_{\mu \nu} \gamma^0$.
The reason why we are defining these quantities lays in the follwing fact:
When $d$ is the rank of an irreducible matrix representation of an algebra composed of hypercomplex numbers,
the number of linearly independent elements of the algebra is $n = d^2$. Therefore, the minimal algebra composed
of $\gamma$'s has $4^2 = 16$ linearly independent elements. They are given as
\begin{eqnarray}
\left\{
\begin{array}{l}
\Gamma^S
=
1
\\
\Gamma^V_\mu
=
\gamma_\mu
\\
\Gamma^T_{\mu \nu}
=
\sigma_{\mu \nu} \leftdef
\frac{i}{2}
[ \gamma_\mu, \gamma_\nu]
\\
\Gamma^A_{\mu}
=
\gamma_5 \gamma_\mu
\\
\Gamma^P
=
\gamma_5
\end{array}
\right.
\end{eqnarray}

We are now presenting three unitary equivalent expressions of $\alpha_i$'s, $\beta$ and $\gamma$'s.
First we write down the Pauli matrices here again.
\footnote{
It was already given in Eq. (\ref{eqn:SU2GenFundamental}).
}
\begin{eqnarray}
\sigma_1 =
\left(
\begin{array}{cc}
0 & 1 \\ 1 & 0
\end{array}
\right)\,,
\hspace{3mm}
\sigma_2 =
\left(
\begin{array}{cc}
0 & -i \\ i & 0
\end{array}
\right)\,,
\hspace{3mm}
\sigma_3 =
\left(
\begin{array}{cc}
1 &  0\\ 0 & -1
\end{array}
\right)
\end{eqnarray}
They satisfy relationships
\begin{eqnarray}
\{
\sigma_i, \sigma_j \}_+ = 2 \delta_{ij}\,,
\hspace{3mm}
\sigma_i \sigma_j = i\epsilon_{ijk} \sigma_k + \delta_{ij}
\end{eqnarray}
%----------------------------------------------------------------------------
(1) Dirac representation\\
\begin{eqnarray}
\gamma^0
=
\left[
\begin{array}{cc}
1 & 0 \\ 0 & -1
\end{array}
\right]
= \sigma_3 \otimes 1
\,,
\hspace{5mm}
\gamma^i
=
\left[
\begin{array}{cc}
0 & \sigma_i \\ -\sigma_i & 0
\end{array}
\right]
=
i\sigma_2 \otimes \sigma_i
\end{eqnarray}
Elements of $\left[\;\ddots\;\right]$ are $2 \times 2$ matrices.
Expressions with the direct product makes calculations more transparent. 
For instance,
\begin{eqnarray}
\begin{array}{l}
\alpha_i = \beta \gamma^i = \gamma^0 \gamma^i
= (\sigma_3 \otimes 1)(i\sigma_2 \otimes \sigma_i)
= -i \sigma_2 \sigma_3 \otimes \sigma_i
= \sigma_1 \otimes \sigma_i
\\
\hspace{35mm}
=
\left[
\begin{array}{cc}
0 & \sigma_i
\\
\sigma_i & 0
\end{array}
\right]
\end{array}
\end{eqnarray}
Other matrices are similarly obtained as
\begin{eqnarray}
\gamma_5
=
\left[
\begin{array}{cc}
0 & 1 \\ 1 & 0
\end{array}
\right]
= \sigma_1 \otimes 1\,,
\hspace{3mm}
\sigma_{0 i} = -i  \sigma_1 \otimes \sigma_i\,,
\hspace{3mm}
\sigma_{i j} = \epsilon_{ijk} \otimes \sigma_k
\end{eqnarray}
When we write the Dirac field as composed of two component 
large and small components,
\begin{eqnarray}
\psi =
\left[
\begin{array}{c}
\varphi \\ \chi
\end{array}
\right]\,
\end{eqnarray}
the Dirac equation for these components reads
\begin{eqnarray}
\left\{
\begin{array}{l}
i \partial_t \varphi(x)
=
m \varphi(x) + \frac{1}{i} \bld{\sigma} \cdot \bld{\partial} \chi(x)
\vspace{2mm}
\\
i \partial_t \chi(x)
=
-m \chi(x) + \frac{1}{i} \bld{\sigma} \cdot \bld{\partial} \varphi(x)
\end{array}
\right.
\end{eqnarray}
\\

%----------------------------------------------------------
\noindent
(2) Majorna representation\\
\begin{eqnarray}
\begin{array}{l}
\gamma^0
=
\left[
\begin{array}{cc}
0 & \sigma_2 \\ \sigma_2 & 0
\end{array}
\right]
= \sigma_1 \otimes \sigma_2
\,,
\hspace{5mm}
\gamma^1
=
\left[
\begin{array}{cc}
i\sigma_3 & 0 \\  0 & i\sigma_3\,,
\end{array}
\right]
=
1 \otimes i\sigma_3
\vspace{2mm}
\\ %------------------------
\gamma^2
=
\left[
\begin{array}{cc}
0 & -\sigma_2 \\ \sigma_2 & 0
\end{array}
\right]
= -i\sigma_2 \otimes \sigma_2
\,,
\hspace{5mm}
\gamma^3
=
\left[
\begin{array}{cc}
-i\sigma_1 & 0 \\  0 & -i\sigma_1
\end{array}
\right]
=
1 \otimes -i\sigma_1\,,
\end{array}
\end{eqnarray}
and $\gamma_5 = \sigma_3 \otimes \sigma_2$.
In this representation, we have $(\gamma^\mu)^* = - \gamma^\mu$ and
the Dirac equation becomes real. The Dirac fields are given as linear combinations
of real sollutions. This representation is related with the Dirac one by
\begin{eqnarray}
\gamma^\mu_{\mbox{\tiny Majorana}}
= U \gamma^\mu_{\mbox{\tiny Dirac}} U^{\dagger}\,,
\hspace{5mm}
U = U^\dagger =
\frac{1}{\sqrt{2}}
\left[
\begin{array}{cc}
1 & \sigma_2 \\
\sigma_2 & -1
\end{array}
\right]
\end{eqnarray}\\

%----------------------------------------------------------
\noindent
(3) Chiral representation\\
\begin{eqnarray}
\begin{array}{l}
\gamma^0
=
\left[
\begin{array}{cc}
0 & -1 \\ -1 & 0
\end{array}
\right]
= \sigma_1 \otimes -1
\,,
\hspace{5mm}
\gamma^i
=
\left[
\begin{array}{cc}
0  & \sigma_i \\  -\sigma_i & 0\,,
\end{array}
\right]
=
i\sigma_2 \otimes \sigma_i
\vspace{2mm}
\\ %------------------------
\gamma_5
=
\left[
\begin{array}{cc}
1 & 0 \\ 0 & -1
\end{array}
\right]
= \sigma_3 \otimes 1
\,,
\hspace{5mm}
\sigma_{0i}
=
\left[
\begin{array}{cc}
-i\sigma_i & 0 \\  0 & i\sigma_i
\end{array}
\right]
=
-i\sigma_3 \otimes \sigma_i\,,
\vspace{2mm}
\\ %------------------------
\sigma_{ij}
=
\left[
\begin{array}{cc}
\epsilon_{ijk}\sigma_k & 0 \\  0 & \epsilon_{ijk}\sigma_k
\end{array}
\right]
=
\epsilon_{ijk} \otimes \sigma_k\,,
\end{array}
\end{eqnarray}
In this representation, spatial rotators and Lorentz boosters, namely, proper Lorentz transformations
take forms as diagonal in the $\left[\;\ddots\;\right]$ space so that $\varphi$ and $\chi$ fields
are transformed independently. This representation is related with the Dirac one by
\begin{eqnarray}
\gamma^\mu_{\mbox{\tiny Chiral}}
= U \gamma^\mu_{\mbox{\tiny Dirac}} U^{\dagger}\,,
\hspace{5mm}
U = 
\frac{1}{\sqrt{2}}
(1 - \gamma_5 \gamma^0)
=
\frac{1}{\sqrt{2}}
\left[
\begin{array}{cc}
1 & 1 \\
-1 & 1
\end{array}
\right]
\end{eqnarray}\\

\bigskip
%------------------------------------------------------------------------------------------------------------------------------------------------
\noindent
{\bf Lorentz transformation property}

Under a proper homogeneous Lorentz transformation
\footnote{%--------------------------------------------------------
\begin{eqnarray*}
L_\mu^{\;\;\rho} L^\mu_{\;\;\sigma} = g^\rho_\sigma\,,
\hspace{5mm}
\end{eqnarray*}
Denoting a matrix with its elements given by $L^\mu_{\;\;\nu}$ as $L$,
we have
\begin{eqnarray*}
L_\mu^{\;\;\rho} = (L^{-1})^\rho_{\;\;\mu}
\end{eqnarray*}
}%-------------------------------------------------------- end of footnote
\begin{equation}
x^\mu \mapsto x^{\mu'} = L(\bld{\beta}, \bld{\theta})^\mu_{\;\nu}\; x^\nu\,,
\end{equation}
the Dirac field transforms as
\begin{eqnarray}
\psi(x) \mapsto \psi'(x') = S(L) \psi(x)\,,
\hspace{5mm}
S(L) = \exp[ -\frac{i}{4}\sigma_{\mu \nu} \omega^{\mu \nu}]\,,
\end{eqnarray}
where $\omega^{\nu \mu} = -\omega^{\mu \nu}$ and
\begin{eqnarray}
\omega^{0i} &=& \xi_i\,,
\hspace{5mm}
\bld{\xi} = \xi \bld{\beta}/\beta\,,
\hspace{3mm}
\xi = \frac{1}{2} \ln \frac{1+\beta}{1-\beta}
\\
\omega^{ij}
&=&
-\epsilon_{ijk} \theta_k
\end{eqnarray}
We may examine the properties under boosts and rotations separately by considering
\begin{eqnarray}
S_{boost}(L(\bld{\beta},\bld{0})) &=& \exp[
-\frac{i}{2}\sigma_{0i}\xi_i ]
\\
&=&
\cosh \frac{\xi}{2} - \frac{\bld{\beta}\cdot \bld{\alpha}}{\beta} \sinh \frac{\xi}{2}
\end{eqnarray}
and
\begin{eqnarray}
S_{rotation}(L(\bld{0},\bld{\theta}))&=& \exp[
\frac{i}{2} \bld{\sigma}\cdot \bld{\theta} ]
\\
&=&
\cos \frac{\theta}{2}
+ i \frac{\bld{\theta}\cdot\bld{\sigma}}{\theta} \sin \frac{\theta}{2}
\end{eqnarray}
where three components of
\begin{eqnarray}
\bld{\sigma}  \leftdef 
\frac{1}{2} \epsilon^{\uparrow ij}\sigma_{ij}
=
\frac{i}{2} \epsilon^{\uparrow ij} \gamma^i \gamma^j
=
\gamma_5 \gamma^0 \bld{\gamma}
\label{eqn:DiGenRotation}
\end{eqnarray}
are the spinor (fundamental) reperesentation of generators of the spatial rotation.
They certainly satisfy $\{ \sigma^i, \sigma^j \}_+ = 2 \delta^{ij}$. Also,
$S^\dagger_{boost}(L) = S_{boost}(L)$,
$S^\dagger_{rotation}(L) = S^{-1}_{rotation}(L)$ and
$\gamma^0 S^\dagger_{}(L) \gamma^0 = S^{-1}_{}(L)$.
The covariance of the Dirac equation (\ref{eqn:DiracEqstandard}) is guaranteed by
a relationship
\begin{eqnarray}
S^{-1}(L) \gamma^\mu S(L) = L^\mu_{\;\;\nu} \gamma^\nu
\label{eqn:LorentzTransfgamma}
\end{eqnarray}
$S(L)$ has a property
\begin{eqnarray}
\gamma^0 S^\dagger(L) \gamma^0 = S^{-1}(L)\,,
\end{eqnarray}
so that
\begin{eqnarray}
\overline{\psi}(x) \mapsto {\overline{\psi}}'(x') = \psi^{*t} (x) S^\dagger(L) \gamma^0
= \overline{\psi}(x) S^{-1}(L)
\end{eqnarray}
and $\overline{\psi} \psi$ is a Lorentz scalar.

For later use, we write down a particular form of $S_{boost}$.
For a Lorentz transformation
\begin{eqnarray}
L(-\bld{\beta}) = \left(
\begin{array}{cc}
\gamma &
 \gamma \bld{\beta} \cdot
\\
\gamma  \bld{\beta} &
1 + \hat{\bld{\beta}} (\gamma - 1) \hat{\bld{\beta}} \cdot
\end{array}
\right)\,,
\end{eqnarray}
which transforms $p^{(0)}= (m, \bld{0}) \mapsto p = L p^{(0)} = (E, \bld{p})$,
we have
\begin{eqnarray}
\begin{array}{l}
S_{boost}(L)
=
\frac{1}{\sqrt{2m(E + m)}}
[ \slashed{p} \gamma^0 + m ]\,,
\\
S_{boost}^{-1}(L)
=
\frac{1}{\sqrt{2m(E + m)}}
[ \gamma^0 \slashed{p}  + m ]\,,
\end{array}
\end{eqnarray}
Since $L^{-1}$ corresponds to the change of the sign of the spacial momentum,
we have $S_{boost}^{-1}(L) = S_{boost}(L^{-1})$.


\bigskip
%------------------------------------------------------------------------------------------------------------------------------------------------
\noindent
{\bf The Dirac spinors}

We write solutions of the Dirac equation (\ref{eqn:DiracEqstandard}) as
\begin{eqnarray}
\psi(x) = \int \frac{d^3 \bld{p}}{\sqrt{(2\pi)^3} 2 p^0}
 \left[
\psi^{(+)}(\bld{p}) e^{-ipx} + \psi^{(-)}(\bld{p}) e^{ipx}
\right]\,,
\end{eqnarray}
where $p^0 = \sqrt{\bld{p}^2 + m^2}$.
The Dirac equation demands
\footnote{
%------------------------------------------------
As  a solution of the K-G eq., we can write
\begin{eqnarray*}
\psi(x) &=& \int \frac{d^4 p}{\sqrt{(2\pi)^3}}
\delta(p^2 - m^2) 
 \left[
 \theta(p^0) +  \theta(-p^0) \right]
 \psi(p) e^{-ipx}
\\
&=&
\int \frac{d^4 p}{\sqrt{(2\pi)^3}}
\delta(p^2 - m^2)  
{\large \mbox{[}}
\underbrace{\psi(p)\theta(p^0)}_{\psi^{+}}  e^{-ipx} + 
\underbrace{\psi(-p)\theta(p^0)}_{\psi^{-}}  e^{ipx}
{\large \mbox{]}}
\\
&=&
\int \frac{d^3 \bld{p}}{\sqrt{(2\pi)^3}2p^0}
\left[
\psi^{+}(\bld{p})  e^{-ipx} + \psi^{-}(\bld{p})  e^{ipx}
\right]
\end{eqnarray*}
The Dirac eq. (\ref{eqn:DiracEqstandard}) requires
\begin{eqnarray*}
\left. (\slashed{p} - m) \psi(p) \right|_{p^2 = m^2} = 0\,,
\end{eqnarray*}
from which Eq. (\ref{eqn:DiracEqMomentumamp}) follows.
}%---------------------------------- end of footnote
\begin{eqnarray}
(\slashed{p} - m) \psi^{(+)}(\bld{p}) = 0\,,
\hspace{5mm}
(\slashed{p} + m) \psi^{(-)}(\bld{p}) = 0
\label{eqn:DiracEqMomentumamp}
\end{eqnarray}
For a Lorentz transformation $L: p^{(0)} = (m, \bld{0}) \mapsto p$, we write
$\psi^{(\pm)}(\bld{p}) = S(L) \psi^{(\pm)}(\bld{0})$.
Eq. (\ref{eqn:LorentzTransfgamma}) reads
\begin{eqnarray*}
S^{-1}(L) \slashed{p} S(L)
=
p_\mu L^\mu_{\;\;\nu} \gamma^\nu
=
L_\mu^{\;\;\rho} L^\mu_{\;\;\nu} p^{(0)}_\rho \gamma^\nu
=
g^\rho_\nu p^{(0)}_\rho \gamma^\nu
=
m \gamma^0
\end{eqnarray*}
and we have
\begin{eqnarray*}
(\gamma^0 - 1) \psi^{(+)}(\bld{0}) = 0\,,
\hspace{5mm}
(\gamma^0 + 1) \psi^{(-)}(\bld{0}) = 0
\end{eqnarray*}
\bigskip
Each of these equations have two linearly independent solutions
for $\gamma_0^2 =1$, $\tr \gamma_0 = 0$ and rank$(\gamma_0) = 4$.
Denoting normalized solutions as $u^{(\alpha)}(\bld{p})$ and $v^{(\alpha)}(\bld{p})$,
$\alpha = 1, 2$ respectively for $\psi^{(+)}(\bld{p})$ and  $\psi^{(-)}(\bld{p})$,
we write
\begin{eqnarray}
\psi(x) = \int \frac{d^3 \bld{p}}{\sqrt{(2\pi)^3} 2 p^0}
\sum_{\alpha = 1, 2} \left[
b_\alpha(\bld{p}) u^{(\alpha)}(\bld{p}) e^{-ipx} + d_\alpha^*(\bld{p}) v^{(\alpha)}(\bld{p}) e^{ipx}
\right]
\end{eqnarray}
Eq. (\ref{eqn:DiracEqMomentumamp}) reads
\begin{eqnarray}
(\slashed{p} - m) u^{(\alpha)}(\bld{p}) = 0\,,
\hspace{5mm}
(\slashed{p} + m) v^{(\alpha)}(\bld{p}) = 0
\label{eqn:DiracEqDiracspinors}
\end{eqnarray}
For conjugate (adjoint) spinors defined in Eq. (\ref{eqn:DefConjDiracField}) we have
\begin{eqnarray}
\overline{u}^{(\alpha)}(\bld{p}) (\slashed{p} - m) = 0\,,
\hspace{5mm}
\overline{v}^{(\alpha)}(\bld{p}) (\slashed{p} + m) = 0
\label{eqn:DiracEqConjspinors}
\end{eqnarray}
We choose their normalization
\footnote{%-------------------------------------------------
Izykson: $\{ b(\bld{p}),b^\dagger(\bld{p}') \}_+ = \frac{2E}{2m}\delta^3(\bld{p}-\bld{p}')$,
$\bar{u}^\alpha u^\beta = \delta^{\alpha \beta}$,\\
\hspace{20mm}$\psi(x) \sim \int d^3\bld{p}\frac{2m}{2E}[ b u e^{-ipx} \dots]$\\
Hioki, Tong: $\{ b(\bld{p}),b^\dagger(\bld{p}') \}_+ = (2\pi)^3 2E\delta^3(\bld{p}-\bld{p}')$,
$\bar{u}^\alpha u^\beta = 2m \delta^{\alpha \beta}$,\\
\hspace{20mm}$\psi(x) = \int \frac{d^3\bld{p}}{(2\pi)^32E}[ b u e^{-ipx} \dots]$\\
dim $\psi = E^{3/2}$
}%-------------------------------------------end of footnote
 as
\begin{eqnarray}
\begin{array}{c}
\overline{u}^{(\alpha)}(\bld{p})
u^{(\beta)}(\bld{p})
= 2m \delta^{\alpha \beta}\,,
\hspace{5mm}
\overline{v}^{(\alpha)}(\bld{p})
v^{(\beta)}(\bld{p})
= -2m \delta^{\alpha \beta}\,,
\\
\overline{u}^{(\alpha)}(\bld{p})
v^{(\beta)}(\bld{p})
= 0\,,
\hspace{5mm}
\overline{v}^{(\alpha)}(\bld{p})
u^{(\beta)}(\bld{p})
= 0
\end{array}
\end{eqnarray}
%------------------------------------------------------------------------------
\underline{Energy state projection}
\begin{eqnarray}
\hat{\Omega}_+(\bld{p})
=
\frac{\slashed{p} + m}{2m}
\,, \hspace{5mm}
\hat{\Omega}_-(\bld{p})
=
\frac{- \slashed{p} + m}{2m}
\end{eqnarray}
%------------------------------------------------------------------------------
\underline{Spin state projection}\\
The spin states are defined in the rest frame of particle.
Consider the generator of spatial rotation given in Eq. (\ref{eqn:DiGenRotation}).
For $s^{(0)}= (0, \bld{s})$ with a spatial 3 vector $\bld{s}$, 
with helps of Eq. (\ref{eqn:LorentzTransfgamma}) and a relation
\begin{eqnarray}
S^{-1}(L) \gamma_5 S^{}(L) = \det (L) \gamma_5\,,
\end{eqnarray}
we have 
\begin{eqnarray}
S^{-1}(L) \bld{\sigma} \cdot \bld{s} S^{}(L) 
&=&
S^{-1}(L) \gamma_5 \gamma^0 \bld{\gamma} \cdot \bld{s} S^{}(L) 
\nonumber\\
&=&
\det (L) \gamma_5 L^0_{\;\;\mu} \gamma^\mu L^{i}_{\;\;\nu} \gamma^\nu s^i
\nonumber\\
&=&
\det (L) \gamma_5 (p^{(0)}_\rho / m) L^\rho_{\;\;\mu} \gamma^\mu (- s^{(0)}_\sigma) L^{\sigma}_{\;\;\nu} \gamma^\nu 
\nonumber\\
&=&
\det (L) \gamma_5  (L^{-1})^{\;\;\rho}_{\mu} (p^{(0)}_\rho / m) \gamma^\mu  
(L^{-1})^{\;\;\sigma}_{\nu} \gamma^\nu (- s^{(0)}_\sigma)\,,
\nonumber
\end{eqnarray}
where $p^{(0)}= (m, \bld{0})$.
Then, for $L= L(\bld{\beta}): p = (E, \bld{p}) \mapsto p^{(0)}$,
we have
\begin{eqnarray}
S^{-1}(L) \bld{\sigma} \cdot \bld{s} S^{}(L) 
&=&
- (1/m) \det (L) \gamma_5  p^{}_\mu  \gamma^\mu   \gamma^\nu  s^{}_\nu\,,
\nonumber\\
&=&
\det (L) \frac{\gamma_5 \slashed{s} \slashed{p}}{m}\,,
\end{eqnarray}
where we have used a fact $\slashed{p} \slashed{s} = - \slashed{s} \slashed{p} $
for $p\cdot s = 0$.
Thus, we have for a proper Lorentz transformation that
\begin{eqnarray}
\frac{\gamma_5 \slashed{s} \slashed{p}}{m}
u(\bld{p})
=
S^{-1}(L) \bld{\sigma} \cdot \bld{s} S^{}(L) S^{-1}(L) u(\bld{0})
=
S^{-1}(L) \bld{\sigma} \cdot \bld{s} u(\bld{0})
\end{eqnarray}
and similar relationship for $v$. We set two linearly independent spinors 
in the rest frame as eigenstates of $\bld{\sigma}\cdot \bld{s}$ 
for $\bld{s}^2 = 1$
and write
\begin{eqnarray}
\begin{array}{l}
\bld{\sigma}\cdot \bld{s}\; u(\bld{0}, \bld{s} ) = u(\bld{0}, \bld{s} )
\\
\bld{\sigma}\cdot \bld{s}\; u(\bld{0}, -\bld{s} ) = -u(\bld{0}, -\bld{s} )
\\
- \bld{\sigma}\cdot \bld{s}\; v(\bld{0}, \bld{s} ) =  v(\bld{0}, \bld{s} )
\\
- \bld{\sigma}\cdot \bld{s}\; v(\bld{0}, -\bld{s} ) =  - v(\bld{0}, -\bld{s} )
\end{array}
\end{eqnarray}
Since $S^{-1}(L) u(\bld{0},\bld{s}) = u(\bld{p},\bld{s})$ and so on, we have
a set of similar relations by replacing $\bld{0}$ by $\bld{p}$ and
$\bld{\sigma}\cdot \bld{s}$ by $\gamma_5 \slashed{s} \slashed{p} / m$.
Considering relationships $(\slashed{p}/m)u(\bld{p}) = u(\bld{p})$
and $(-\slashed{p}/m)v(\bld{p}) = v(\bld{p})$, an operator matirx
\begin{eqnarray}
\hat{\Sigma}(s)
 = 
\frac{1 + \gamma_5 \slashed{s}}{2}
%\,,\hspace{5mm}
%s = L s^{(0)}\,,
%\hspace{2mm}
%s^{(0)} = (0, \bld{s})\,,
 \end{eqnarray}
projects out 
spinstates along $\bld{s}$ as
\begin{eqnarray}
\begin{array}{l}
\hat{\Sigma}(s) u(\bld{p}, \bld{s}) = u(\bld{p}, \bld{s})
\\
\hat{\Sigma}(s) u(\bld{p}, -\bld{s}) = 0
\\
\hat{\Sigma}(s) v(\bld{p}, \bld{s}) = v(\bld{p}, \bld{s})
\\
\hat{\Sigma}(s) v(\bld{p}, -\bld{s}) = 0
\end{array}
\end{eqnarray}
In the Dirac representation,
\begin{eqnarray}
\bld{\sigma}
&=&
1 \otimes \bld{\sigma}_{Pauli}
=
\left[
\begin{array}{cc}
\bld{\sigma}_{Pauli} & 0 \\
0 & \bld{\sigma}_{Pauli}
\end{array}
\right]
\end{eqnarray}


\bigskip

For $p = Lp^{(0)}$, $p^{(0)} = (m, \bld{0})$,
\begin{eqnarray}
\hat{\Sigma}(s) u(\bld{p})
=
\hat{\Sigma}(s)
S_{boost}(L)
u(\bld{0})
=
\frac{1 + \gamma_5 \slashed{s}}{2}
\frac{\slashed{p}\gamma^0 + m}{\sqrt{2m(E+m)}}
\end{eqnarray}

\noindent
{\bf Lagrangian density}
\begin{equation}
{\cal L} = i \bar{\psi}(x) \slashed{\partial} \psi(x) - m \bar{\psi}(x) \psi(x)
\end{equation}
where $\slashed{\partial} = \gamma^\mu \partial_\mu$ and
$\bar{\psi}(x) = \psi^\dagger (x) \gamma^0$.
The Euler-Lagrange equation reads
\begin{equation}
%\slashed{\partial} = 0
\left(i \slashed{\partial} + m \right) \psi(x) = 0\,,
\hspace{3mm}
\bar{\psi}(x) \left(i \stackrel{\leftarrow}{\slashed{\partial}} + m \right)  = 0
\end{equation}

\subsubsection{Quantized Free Field}
%------------------------------------------------------------- comment out
\begin{comment}
\begin{equation}
\psi(x) 
=
\int \frac{d^3 \bld{p}}{\sqrt{(2\pi)^3} 2 p^0}
\sum_{s = \pm 1}
\left[
c(\bld{p}, s) u(\bld{p}, s) e^{-ipx}
+
d^\dagger(\bld{p}, s) v(\bld{p}, s) e^{ipx}
\right]
\end{equation}
\end{comment}
%------------------------------------------------------------- comment out end
\begin{eqnarray}
\psi(x) 
&=&
\int \frac{d^3 \bld{p}}{\sqrt{(2\pi)^3} 2 p^0}
\sum_{\alpha = 1,2}
\left[
c_\alpha(\bld{p}) u^{(\alpha)}(\bld{p}) e^{-ipx}
+
d_\alpha^\dagger(\bld{p}, s) v^{(\alpha)}(\bld{p}) e^{ipx}
\right]
\nonumber\\
\end{eqnarray}
\begin{eqnarray}
\begin{array}{l}
\{ c_\alpha(\bld{p}), c^\dagger_\beta(\bld{p}') \}_+
=
2 p^0 \delta_{\alpha \beta}
\delta^3(\bld{p}-\bld{p}')\,,
\hspace{3mm}
\{ d_\alpha(\bld{p}), d^\dagger_\beta(\bld{p}') \}_+
=
2 p^0 \delta_{\alpha \beta}
\delta^3(\bld{p}-\bld{p}')
\vspace{2mm}
\\
\{ c_\alpha(\bld{p}), c_\beta(\bld{p}') \}_+
=
\{ d_\alpha(\bld{p}), d_\beta(\bld{p}') \}_+
=
\{ c_\alpha(\bld{p}), d^\dagger_\beta(\bld{p}') \}_+
=
\dots
= 0
\end{array}
\end{eqnarray}
\begin{eqnarray}
\pi(x) =
\frac{\partial {\cal L}}{\partial \dot{\psi}}
=
i \overline{\psi} \gamma^0
= i \psi^\dagger
\end{eqnarray}
\begin{eqnarray}
\{
\psi_\xi (t, \bld{x}), \pi_\eta(t, \bld{y}) \}
=
\{
\psi_\xi (t, \bld{x}), i\psi^\dagger_\eta(t, \bld{y}) \}
=
i \delta_{\xi \eta }
\delta^3(\bld{x} - \bld{y})
\end{eqnarray}
\begin{eqnarray}
\{ \psi_\xi(t, \bld{x}), \psi_\eta^\dagger(t, \bld{y}) \}_+
&=&
\int \frac{d^3 \bld{p} d^3 \bld{p}'}{(2\pi)^3 4 E E'}
\sum_{\alpha, \beta} \left(
\{ c_\alpha(\bld{p}), c^\dagger_\beta(\bld{p}') \}
u^{(\alpha)}_\xi (\bld{p}) u^{(\beta) \dagger}_\eta (\bld{p}')
\right.
\nonumber\\
&&
\left.
+
\{ d^\dagger_\alpha(\bld{p}), d_\beta(\bld{p}') \}
v^{(\alpha)}_\xi (\bld{p}) v^{(\beta) \dagger}_\eta (\bld{p}')
\right)
\nonumber\\
&=& \dots
\nonumber\\
&=& 
\delta_{\xi \eta} \delta^3(\bld{x} - \bld{y})
\end{eqnarray}
%------------------------------------------------------------- comment out
\begin{comment}
\begin{equation}
\{ c(\bld{p}, s), c^\dagger(\bld{p}', s') \}
=
\{ d(\bld{p}, s), d^\dagger(\bld{p}', s') \}
=
2p^0 \delta_{ss'}
\delta^3(\bld{p} - \bld{p}')
\end{equation}
\begin{equation}
\begin{array}{l}
(\slashed{p} - m) u(\bld{p}, s) = 0\,,
\hspace{3mm}
\bar{u}(\bld{p}, s)(\slashed{p} - m)  = 0\,,
\\
(\slashed{p} + m) v(\bld{p}, s) = 0\,,
\hspace{3mm}
\bar{v}(\bld{p}, s)(\slashed{p} + m)  = 0\,,
\end{array}
\end{equation}
\end{comment}
%------------------------------------------------------------- comment out end

\noindent
\underline{Propagator}\\
\begin{eqnarray}
i S_{\xi \eta} &\equiv&
\{
\psi_\xi (x) , \overline{\psi}_\eta (y) \}
\end{eqnarray}
\begin{eqnarray}
iS(x - y)
&=& 
\dots
=
(i \slashed{\partial}_x + m)
\left[
D(x-y) - D(y - x)
\right]
\end{eqnarray}

\begin{eqnarray}
S_F(x - y) &\leftdef&
\bra 0 \braend T[ \psi(x) \overline{\psi}(y) \ketend 0 \ket
\\
&=&
i \int \frac{d^4 p}{(2\pi)^4}
\frac{e^{-ip(x-y)}}{\slashed{p} - m + i\epsilon}
\end{eqnarray}
%------------------------------------------------------------- comment out
\begin{comment}
\begin{eqnarray}
S_F(q) &=& i \int d^4x e^{iqx}
\bra 0 \braend T[\psi(x) \bar{\psi}(0)]
\ketend 0 \ket
\nonumber\\
&=&
\frac{-1}{\slashed{q} - m + i \epsilon}
=
- \frac{\slashed{q} + m}{ q^2 - m^2 + i \epsilon}
\end{eqnarray}
\end{comment}
%------------------------------------------------------------- comment out end

%------------------------------------------------------------- comment out
\begin{comment}
\subsubsection{Majorana Field}
Massless, spin1/2, selfconjugate.
\begin{equation}
i \slashed{\partial} \psi(x) = 0\,,
\hspace{3mm}
\bar{\psi}(x) i \stackrel{\leftarrow}{\slashed{\partial}}   = 0
\end{equation}
\end{comment}
%------------------------------------------------------------- comment out end

