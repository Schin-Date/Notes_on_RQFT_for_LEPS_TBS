Let us consider a toy model composed of scalar fields:
\begin{eqnarray}
{\cal L} 
&=&
\partial_\mu \psi^* \partial^\mu \psi
+
\frac{1}{2} (\partial_\mu \phi)^2 
-
M^2 \psi^*  \psi
-
\frac{1}{2} m^2 \phi^2
-
g \psi^*  \psi \phi
\label{eqn:sclYkwLagDens}
\end{eqnarray}
$\psi$ is a mock nucleon complex scalar field with the mass $M$ 
and $\phi$ is a mock pion real scalar field with the mass $m$.
The first parts of $\cal L$ except for the last term compose free $\cal L$agrangians
${\cal L}_N$ and ${\cal L}_\pi$ for nucleon and pion fields, respectively.
The last term is the interaction $\cal L$agrangian in which
nucleons and pions interact each other with a coupling constant $g$.
The coupling $g$ has the dimension of energy and the dimensionless
parameter is $g / E$, where E is the energy scale of the process of interest.
This means that the interaction term ${\cal L}_{int} = - g \psi^* \psi \phi$ is
relevant at low energies. The relativistic nature gets important at $E \gg M, m$
and we may choose $g \ll M, m$ so that the perturbation series (\ref{eqn:TimeDevPerturbSerTprod})
converges.

Each fields are quantized through equal-time commutation relations at time,
say, 0. One may write down field equations but they are not solvable due to
the presence of ${\cal L}_{int}$. Remember they describe time evolution of
field operators in the Heisenberg picture. In the interaction picture, however,
we may let fields obey free field equations. 
%-------------------------------------------------------------------- comment block starts
\begin{comment}
Field equations read
\begin{eqnarray}
\begin{array}{l}
(\Box + m^2) \phi = - g\psi^* \psi\,,
\vspace{2mm}
\\
(\Box + M^2) \psi = -g\psi \phi\,,
\vspace{2mm}
\\
(\Box + M^2) \psi^* = -g\psi^* \phi
\end{array}
\end{eqnarray}
\end{comment}
%-------------------------------------------------------------------- comment block ends
In practice, 
conjugate fields are given by
\begin{eqnarray}
%\begin{array}{l}
\pi_\phi = \frac{\partial {\cal L}}{\partial(\dot{\phi})} = \dot{\phi}
\,,\hspace{3mm}
%\\
\pi_\psi = \frac{\partial {\cal L}}{\partial(\dot{\psi})} = \dot{\psi^*}
\,,\hspace{3mm}
%\\
\pi_{\psi^*} = \frac{\partial {\cal L}}{\partial(\dot{\psi^*})} = \dot{\psi}
\,,
%\end{array}
\end{eqnarray}
and classical Hamiltonian density is written as
\begin{eqnarray}
{\cal H} 
&=&
\pi_\phi \dot{\phi} +\pi_\psi \dot{\psi} + \pi_{\psi^*} \dot{\psi^*} - {\cal L}
\nonumber\\
&=&
\frac{1}{2} \left\{
\pi_\phi^2 + (\bld{\partial}\phi)^2 + m^2 \phi^2
\right\}
+
\left\{
\pi_\psi \pi_{\psi^*}
+ \bld{\partial} \psi^* \cdot \bld{\partial} \psi
+ M^2 \psi^* \psi \right\}
\nonumber\\
&&+ g \psi^* \psi \phi
\label{eqn:sclYkwHamDens}
\end{eqnarray} 
The first two terms are ${\cal H}_\pi$ and ${\cal H}_N$ corresponding to
${\cal L}_\pi$ and ${\cal L}_N$, respectively, and we assign them  as
the free ${\cal H}$amiltonian ${\cal H}_0 = {\cal H}_\pi + {\cal H}_N$.
Writing ${\cal H}_{int} = g\psi^*\psi \phi$, 
we have a decomposition ${\cal H} = {\cal H}_{0}  + {\cal H}_{int}$,
which corresponds to Eq. (\ref{eqn:HamiltonianDecomposed}).
After substituting quantized fields (at the fixed time) into these expressions
(and taking the normal ordering), we obtain ${\cal H}$amiltonian operators.
The time evolutions of fields in the interaction picture are given from
Eq. (\ref{eqn:InteractionPicTime}) as
\begin{eqnarray}
\begin{array}{l}
i \partial_t \phi_I = [\phi_I, H_0] = [\phi_I, H_\pi]\,,
\vspace{2mm}
\\
i \partial_t \psi_I = [\psi_I, H_0] = [\psi_I, H_N]\,,
\vspace{2mm}
\\
i \partial_t \psi^\dagger_I = [\psi^\dagger_I, H_0] = [\psi^\dagger_I, H_N]\,,
\end{array}
\label{eqn:sYfieldTimeEvo}
\end{eqnarray}
where we added an suffix $I$ to indicate quantities in the interaction picture.
We already know that Eqs. (\ref{eqn:sYfieldTimeEvo}) are equivalent to
free field equations all given as the Klein-Gordon equation with masses of
each fields. Therefore, the fields $\phi_I$, $\psi_I$ and $\psi^\dagger_I$ have
Fourier expanded forms as in Eqs. (\ref{eqn:RS_fieldexpansion}) and (\ref{eqn:CompScFourier}).
To be specific, we write
\begin{eqnarray}
\begin{array}{l}
\displaystyle
\phi_{I}(x) = \int \frac{d^3 \bld{k}}{\sqrt{(2\pi)^3}2k^0} \left[
a(\bld{k}) e^{-i k \cdot x} + a^\dagger(\bld{k})e^{i k \cdot x} \right]
\vspace{2mm}
\\
\displaystyle
\psi_{I}(x) = \int \frac{d^3 \bld{p}}{\sqrt{(2\pi)^3}2p^0} \left[
b(\bld{p}) e^{-i p \cdot x} + c^\dagger(\bld{p})e^{i p \cdot x} \right]
\vspace{2mm}
\\
\displaystyle
\psi^\dagger_{I}(x) = \int \frac{d^3 \bld{p}}{\sqrt{(2\pi)^3}2p^0} \left[
c(\bld{p})e^{- i p \cdot x}  + b^\dagger (\bld{p}) e^{i p \cdot x} \right]
\end{array}
\label{eqn:scYfields}
\end{eqnarray}
Operators $a$, $b$ and $c$ and their conjugates satisfy
commutation relations given in Eqs. (\ref{eqn:cancomm_RS}) and 
(\ref{eqn:complscalarcancomm}) and they establishes the particle interpretations.
Parts of the free ${\cal H}$amiltonian ${\cal H}_\pi$ and ${\cal H}_N$ are now
described by these operators as in Eq. (\ref{eqn:KG_Hamiltonian_aadagg}) and 
the time component of Eq. (\ref{eqn:KGcomplex_FourMomentum}).
We may define number operator $\hat{N}_\pi$ as in Eq. (\ref{eqn:KG_totalNumberOp})
and ones for nucleons as
\begin{eqnarray}
\hat{N}_N = \int \frac{d^3 \bld{p}}{2p^0} b^\dagger(\bld{p}) b(\bld{p})\,,
\hspace{3mm}
\hat{N}_{\bar{N}} = \int \frac{d^3 \bld{p}}{2p^0} c^\dagger(\bld{p}) c(\bld{p})\,,
\end{eqnarray}
where we have assigned $b^\dagger$ and $c^\dagger$ as creation operators
for a nucleon ($N$) and an anti-nucleon ($\bar{N})$, respectively.
In the interaction picture, ${\cal L}_{int}$ term in Eq.(\ref{eqn:sclYkwLagDens})
[or ${\cal H}_{int}$ term in Eq. (\ref{eqn:sclYkwHamDens})] contains creation and
annihilation operators in each field and they may change number of
particles. In practice, the number operators we have just defined do not commute 
with $H_I$, therefore neither with $H$ and do not conserve.
Though numbers of particles are not conserved, the total electric charge would 
be conserved since the ${\cal L}$agrangian (\ref{eqn:sclYkwLagDens}) is invariant
under constant phase change of the field $\psi$ and its conjugate for $\psi^*$.
According to the Noether's theorem and Eq. (\ref{eqn:KGcomplex_Charge}), 
the charge
\begin{eqnarray}
Q = e \int \frac{d^3 \bld{p}}{2p^0}
\left(
b^\dagger({\bld{p}}) b({\bld{p}}) - c^\dagger({\bld{p}}) c({\bld{p}})
\right)
\label{eqn:scYelchrg}
\end{eqnarray}
commutes with $H$ and conserved.

Keeping the formula (\ref{eqn:TimeDevPerturbSerTprod}) in mind,
let us examine particular expression of $H_I$:
\begin{eqnarray}
H_I(t) 
&=&
g \int d^3 \bld{x} 
\normord{ \psi_I^\dagger(x) \psi_I(x) \phi_I(x)}
\nonumber\\
&=&
\frac{g}{\sqrt{(2\pi)^9}} 
\int
\frac{d^3 \bld{p}}{2p^0}
\frac{d^3 \bld{p}'}{2{p^0}'}
\frac{d^3 \bld{k}}{2k^0}
\int 
d^3 \bld{x} 
:
[ e^{-ipx} c(\bld{p}) + e^{ipx} b^\dagger(\bld{p}) ]
\nonumber\\
&&
\hspace{7mm}
[ e^{-ip'x} b(\bld{p}') + e^{ip'x} c^\dagger(\bld{p}') ]
[ e^{-ikx} a(\bld{k}) + e^{ikx} a^\dagger(\bld{k}) ]
:
\nonumber\\
&=&
\frac{g}{\sqrt{(2\pi)^3}} 
\int
\frac{d^3 \bld{p}}{2p^0}
\frac{d^3 \bld{p}'}{2{p^0}'}
\frac{d^3 \bld{k}}{2k^0}
\nonumber\\
&&
[c(\bld{p})b(\bld{p}')a(\bld{k})\delta^3(\bld{p}+\bld{p}'+\bld{k})
e^{-i (p + p' + k_+)^0 t}
\nonumber\\
&&
+b^\dagger(\bld{p}) c^\dagger(\bld{p}') a^\dagger(\bld{k}) \delta^3(\bld{p}+\bld{p}'+\bld{k})
e^{i (p + p' + k_+)^0 t}
\nonumber\\
&&
+b^\dagger(\bld{p}) c^\dagger (\bld{p}')a(\bld{k})\delta^3(\bld{p}+\bld{p}'-\bld{k})
e^{i (p + p' - k_+)^0 t}
\nonumber\\
&&
+c^\dagger(\bld{p})c(\bld{p}')a(\bld{k})\delta^3(\bld{p}-\bld{p}'-\bld{k})
e^{i (p - p' - k_-)^0 t}
\nonumber\\
&&
+b^\dagger(\bld{p})b(\bld{p}')a(\bld{k})\delta^3(\bld{p}-\bld{p}'-\bld{k})
e^{i (p - p' - k_-)^0 t}
\nonumber\\
&&
+a^\dagger(\bld{k})c(\bld{p})b(\bld{p}')\delta^3(\bld{p}+\bld{p}'-\bld{k})
e^{-i (p + p' - k_-)^0 t}
\nonumber\\
&&
+a^\dagger(\bld{k})c^\dagger(\bld{p}')c(\bld{p})\delta^3(\bld{p}-\bld{p}'-\bld{k})
e^{-i (p - p' - k_-)^0 t}
\nonumber\\
&&
+a^\dagger(\bld{k})b^\dagger(\bld{p})b(\bld{p}')\delta^3(-\bld{p}+\bld{p}'-\bld{k})
e^{i (p - p' + k_-)^0 t}]
\label{eqn:scYHIterms}
\end{eqnarray}
where $k_\pm^0 = \sqrt{(\bld{p} \pm \bld{p}')^2 + m^2}$.
At this stage, we may already have some insights about the interaction.
Our ${\cal L}_{int}$ is a product of three field operators and each of them
involve two terms with annihilation and creation operators. This is why there
are $2^3 = 8$ terms in Eq. (\ref{eqn:scYHIterms}). Among them, the last
6 terms show possible processes. For instance, the third term corresponds
a process in which a particle $a$ (pion) disappears and particles $b$ and $c$
(nucleon and anti-nuclen) emerges. So this term corresponds to the pair creation
of $N\bar{N}$ by $\pi$. One can confirm the momentum is conserved in the process.
The energy is not conserved yet and there are exponential factors instead.
Later, we will see these exponentials turn into delta functions corresponding to
the energy conservation for each processes in the evaluation the scattering matrix.
The first two terms in Eq. (\ref{eqn:scYHIterms}) will not contribute to scattering matrices
since they violate the energy conservation.
The second and higher order terms in Eq. (\ref{eqn:TimeDevPerturbSerTprod}) will be
involved in the later discussion with introducing some techniques to expand time ordered
products.
\\

\noindent
●Meson Decay\\
Consider a process $\pi \rightarrow N\bar{N}$. This process is involved in the lowest
order term in Eq. (\ref{eqn:SmatrixPertSer}) through the third term in
Eq. (\ref{eqn:scYHIterms}). We write Initial and final states as
\begin{eqnarray}
\begin{array}{l}
\ketend i \ket = a^\dagger (\bld{k}) \ketend 0 \ket
\rightdef \ketend \pi(\bld{k}) \ket\,,
\vspace{2mm}
\\
\ketend f \ket = b^\dagger (\bld{p}_N) c^\dagger (\bld{p}_{\bar{N}})\ketend 0 \ket
\rightdef \ketend N(\bld{p}_N) \bar{N}(\bld{p}_{\bar{N}}) \ket
\end{array}
\end{eqnarray}
From Eq. (\ref{eqn:SmatrixPertSer}), we read to the leading (first) order in $g$ that
\begin{eqnarray}
\bra f \braend S^{(1)} \ketend i \ket
&=&
-ig \int d^4 x  
\bra N(\bld{p}_N) \bar{N}(\bld{p}_{\bar{N}}) \braend
\normalprod{ \psi^\dagger(x) \psi(x) \phi(x)}
\ketend \pi(\bld{k}) \ket
\nonumber\\
&=&
-ig \int d^4 x  
\bra 0 \braend b(\bld{p}_N) c(\bld{p}_{\bar{N}})
\normalprod{ \psi^\dagger(x) \psi(x) \phi(x)}
 a^\dagger(\bld{k}) \ketend 0 \ket
 \label{eqn:scYmesonDcyamp}
\end{eqnarray}
Here we omitted indeces $I$ on fields under understanding that we are in the interaction picture.
In expanding fields as $\psi^\dagger \sim c + b^\dagger$, 
$\psi \sim b + c^\dagger$ and $\phi \sim a + a^\dagger$,
we find only a term $\sim b^\dagger c^\dagger a$
contributes in Eq. (\ref{eqn:scYmesonDcyamp}).
This corresponds to the third term in $H_I$ in Eq. (\ref{eqn:scYHIterms}).
We proceed from Eq. (\ref{eqn:scYmesonDcyamp}):
\begin{eqnarray}
\bra f \braend S^{(1)} \ketend i \ket
&=&
\frac{-ig}{\sqrt{(2\pi)^9}} 
\int
\frac{d^3 \bld{p}}{2p^0}
\frac{d^3 \bld{p}'}{2{p^0}'}
\frac{d^3 \bld{k}'}{2{k^0}'}
\int 
d^4 x\;
e^{i(p + p' - k') x}
\nonumber\\
&&
\hspace{10mm}
\bra 0 \braend b(\bld{p}_N) c(\bld{p}_{\bar{N}})
[b^\dagger(\bld{p}) c^\dagger (\bld{p}')a(\bld{k}')]
 a^\dagger(\bld{k}) \ketend 0 \ket
\nonumber\\
&=&
\frac{-ig}{\sqrt{(2\pi)^9}} 
\int
\frac{d^3 \bld{p}}{2p^0}
\frac{d^3 \bld{p}'}{2{p^0}'}
\frac{d^3 \bld{k}'}{2{k^0}'}
(2\pi)^4 \delta^4(p + p' - k')
\nonumber\\
&&
\hspace{10mm}
\bra 0 \braend b(\bld{p}_N) c(\bld{p}_{\bar{N}})
[b^\dagger(\bld{p}) c^\dagger (\bld{p}')a(\bld{k}')]
 a^\dagger(\bld{k}) \ketend 0 \ket
\nonumber\\
&=&
\frac{-ig}{\sqrt{(2\pi)^9}} 
\int
\frac{d^3 \bld{p}}{2p^0}
\frac{d^3 \bld{p}'}{2{p^0}'}
(2\pi)^4 \delta^4(p + p' - k)
\nonumber\\
&&
\hspace{20mm}
\bra 0 \braend b(\bld{p}_N) c(\bld{p}_{\bar{N}})
b^\dagger(\bld{p}) c^\dagger (\bld{p}')
 \ketend 0 \ket
\nonumber\\
&=&
\frac{-ig}{\sqrt{(2\pi)^9}} 
(2\pi)^4 \delta^4(p_N + p_{\bar{N}} - k)
 \label{eqn:scYmesonDcySmtrx}
\end{eqnarray}

Reaction rate is defined in the same way as that for scatterting as
\begin{eqnarray}
R_{fi} = \int d\Phi_f | \bra f \braend T \ketend i \ket |^2\,,
\end{eqnarray}
where $\Phi_f$ and $T$ are defined in Eqs. (\ref{eqn:Phi_nDef}) and (\ref{eqn:Smatrix_and_T})
respectively
\footnote{
A note on dimensions: $dim[ d\Phi_f (\bra f \braend )^2] = 1/E^4$,
$dim [T] = E^4$ and $dim [\ketend i \ket]^2 = 1/E^{2k}$ for an initial state
composed of $k$ particles. In total, $dim[R] = E^{4 - 2k}$. 
$k = 2$ and $dim[R] = E^0$ for scatterings. $k = 1$ and $dim[R] = E^2$
for particle decays.
} % end of footnote
.
For decays of a particle with the mass $M$ into $n$ particles, differential decay width 
is defined as
\begin{eqnarray}
d \Gamma = \frac{(2\pi)^3}{2M} dR_{fi}
=
\frac{(2\pi)^3}{2M} 
\prod_i^n \frac{d^3\bld{p}_i}{2p_i^0} (2\pi)^4 \delta^4(P_f - P_i) 
 | \bra f \braend T \ketend i \ket |^2
\end{eqnarray}
In particular, for two particle decays,
\begin{eqnarray}
d \Gamma_2 
&=&
\frac{(2\pi)^3}{2M} 
\frac{d^3\bld{p}_1}{2p_1^0} 
d^4 p_2 \delta(p_2^2 - m_2^2)
(2\pi)^4 \delta^4(P_f - P_i) 
 | \bra f \braend T \ketend i \ket |^2
\nonumber\\
&=&
\frac{(2\pi)^7}{2M} 
\frac{P_1^{*2} dP_1^* d \Omega_1^*}{2p_1^{*0}} 
\delta(M^2 + m_1^2  - m_2^2 - 2 M p_1^{*0})
|T_{fi}|^2
\nonumber\\
&=&
\frac{(2\pi)^9}{32\pi^2M^2} 
P_1^{*}  d \Omega_1^*
|T_{fi}|^2
\label{eqn:decaywidth2bdy}
\end{eqnarray}
We note that $dim |T_{fi}|^2 = E^2$.
The way of writing the coefficient in Eq. (\ref{eqn:decaywidth2bdy}) is chosen
so that it is clear that a factor $(2\pi)^9$ is absorbed in the state normalizations
in notations of some authors including \cite{ref:Donnachie}, \cite{ref:Collins} and
\cite{ref:Itzykson-Zuber}.

Let's come back to our toy model. From Eq. (\ref{eqn:scYmesonDcySmtrx}),
we have $T_{fi} = g / \sqrt{(2\pi)^9}$. Substituting this in Eq. (\ref{eqn:decaywidth2bdy}),
we obtain
\begin{eqnarray}
\Gamma_2^{(1)} = 
\frac{g^2}{16\pi m^2} \lambda^{1/2}(m^2, M^2, M^2)
\end{eqnarray}
where a superscript $(1)$ indicates the order of perturbation expansion in Eq.  (\ref{eqn:SmatrixPertSer}).


