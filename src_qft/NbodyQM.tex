Schr\"odinger eq.
\begin{equation}
\begin{array}{l}
i \partial_t
\Psi^{(N)}(t; \bld{x}_1, \dots \bld{x}_N) = H^{(N)} \Psi^{(N)}(t; \bld{x}_1, \dots \bld{x}_N)\,,
\vspace{2mm}
\\
\displaystyle
H^{(N)} = \sum_i^N H_i\,,
\hspace{3mm}
H_i = - \frac{\bld{\partial}^2}{2m} + V(\bld{x}_i)
\hspace{5mm}
\mbox{(for simplicity)}
\end{array}
\end{equation}
\begin{equation}
H_i \varphi_l^{(i)} (\bld{x} ) = \epsilon_l \varphi_l^{(i)} (\bld{x} )\,,
\hspace{5mm}
i = 1, \dots, N
\end{equation}
\begin{equation}
\begin{array}{l}
\Psi^{(N)}(t; \bld{x}_1, \dots \bld{x}_N)
=
\hspace{50mm}
\\
\hspace{10mm}
%{\stackrel{\displaystyle \sum}{\scriptstyle \epsilon_l^{(1)} + \dots + \epsilon_l^{(n)} = E^{(n)} } }
{\mathop{\displaystyle \sum}_{\scriptstyle  l_1, \dots, l_N} }
\Psi^{(N)}(l_1,\dots, l_N)
e^{-i E^{(N)} t}
\left\{
\varphi_{l_1}^{(1)} (\bld{x}_1 ) \cdots \varphi_{l_N}^{(N)} (\bld{x}_N )
\right\}_P\,,
\end{array}
\end{equation}
where $E^{(N)} = \epsilon_{l_1} + \dots + \epsilon_{l_N}$ and
$\{\cdots \}_P$ denotes symmetrization  for systems of identical bosons
and antisymmetrization for identical fermions.
Thus $\Psi^{(N)}$ is defined in the direct product of $N$ Hilbert space.

\bigskip
\noindent
■Theorem of bosonic creation and annihilation operators\\
If an operator $\hat{a}$ and its hermite conjugate $\hat{a}^\dagger$ satisfy
\begin{equation}
[\hat{a}, \hat{a}^\dagger] = 1\,,
\end{equation}
then
\begin{enumerate}
\item Eigenvalues of an operator $\hat{N} \equiv \hat{a}^\dagger \hat{a}$ is
nonnegative integers $\{0,1,\dots,\infty\}$ and we can call it number operator.
\item Vaccum state $\ketend 0 \ket$ with respect to the dynamical freedom described by $\hat{a}$
and $\hat{a}^\dagger$ can be defined as the eigenstate of
$\hat{N}$ belonging to its eigenvalue 0.
\item
If we normalize the vacuum state by $\bra 0 \braend 0 \ket = 1$, then the eigenstate
of $\hat{N}$ belonging to the eigenvalue $n$ is given by
\begin{equation}
\ketend n \ket = \frac{1}{\sqrt{n!}} (\hat{a}^\dagger )^n \ketend  0 \ket \,,
\hspace{3mm}
\bra n \braend m \ket = \delta_{nm}
\end{equation}
\end{enumerate}

\bigskip
\noindent
{\bf ■Theorem of fermionic creation and annihilation operators}\\\
If an operator $\hat{c}$ and its hermite conjugate $\hat{c}^\dagger$ satisfy
\begin{equation}
\{\hat{c}, \hat{c}^\dagger\} = 1
\hspace{3mm}
\mbox{and}
\hspace{3mm}
\{\hat{c}, \hat{c} \} = 0\,,
\end{equation}
where $\{\dots\}$ is anti-commutator, then
\begin{enumerate}
\item Eigenvalues of an operator $\hat{N} \equiv \hat{c}^\dagger \hat{c}$ is
0 or 1 and we can call it number operator.
\item Vaccum state $\ketend 0 \ket$ with respect to the dynamical freedom described by $\hat{c}$
and $\hat{c}^\dagger$ can be defined as the eigenstate of
$\hat{N}$ belonging to its eigenvalue 0.
\item
If we normalize the vacuum state by $\bra 0 \braend 0 \ket = 1$, then the eigenstates
of $\hat{N}$ are $\ketend 0 \ket$ and $\ketend 1 \ket = c^\dagger \ketend 0 \ket$.
\end{enumerate}






\bigskip
Suppose now we have a set of $\hat{a}_l$  and $\hat{a}_l^\dagger$ for $l = 1, 2, \dots, \infty$
corresponding to energy eigenvalues $\epsilon_1, \epsilon_2,\dots$.
We assume that we can have a set of operators such that each pair of $\hat{a}_l$  and $\hat{a}_l^\dagger$
satisfies the condition of bosonic creation-annihilation operators mentioned above and they are independent
for different suffices:
\begin{equation}
\begin{array}{l}
[\hat{a}_l, \hat{a}^\dagger_m] = \delta_{lm}
\\
\left[ \hat{a}_l, \hat{a}_m \right] 
= [\hat{a}^\dagger_l, \hat{a}^\dagger_m] = 0
\end{array}
\label{eqn:Nbodycreanncomm}
\end{equation}
Having with these operators, we define
\begin{equation}
\hat{\varphi}(\bld{x}) = \sum_l \hat{a}_l \varphi_l (\bld{x})\,,
\label{eqn:SchrFieldprimitive}
\end{equation}
where $\varphi_l (\bld{x})$ is eigenvector of $H = -\bld{\partial}^2/2m + V(\bld{x})$ belonging
to the $l$th eigenvalue $\epsilon_l$.
Assign $\hat{a}_l^\dagger$ as operator to create a particle in the $l$th energy eigenstate:
\begin{equation}
\ketend \epsilon_l \ket = \hat{a}_l^\dagger \ketend 0 \ket
\end{equation}
Then we find
\begin{eqnarray}
\bra 0 \braend \hat{\varphi}(\bld{x}) \ketend \epsilon_l \ket
&=&
\sum_{l'} \varphi_{l'} (\bld{x}) 
\bra 0 \braend \hat{a}_{l'} \hat{a}_l^\dagger \ketend 0 \ket
\nonumber\\
&=&
\sum_{l'} \varphi_{l'} (\bld{x}) 
\bra 0 \braend [\hat{a}_{l'}, \hat{a}_l^\dagger] \ketend 0 \ket
\nonumber\\
&=&
\varphi_{l} (\bld{x}) \,,
\end{eqnarray}
where we have used relationships $\hat{a} \ketend 0 \ket = \bra 0 \braend \hat{a}^\dagger= 0$.
Comparing this result with the second equation in Eq. (\ref{eqn:coordrepbraket}), we may write
\begin{equation}
\bra \bld{x} \braend = \bra 0 \braend \hat{\varphi}(\bld{x})
\end{equation}
If we denote by $\hat{a}^\dagger_{\bld{x}}$ the creation operator 
that creates a a particle at a position $\bld{x}$, we can write
\begin{equation}
\hat{a}^\dagger_{\bld{x}} = \hat{\varphi}^\dagger(\bld{x})
\end{equation}
The operator $\hat{\varphi}(\bld{x})$ defined in Eq. (\ref{eqn:SchrFieldprimitive})
is a primitive form of field operators discussed later.
 
 N particle states can be constructed as
 \begin{equation}
 \ketend \epsilon_{l_1}, \dots, \epsilon_{l_N} \ket
 =
 \hat{a}_{l_1}^\dagger \dots  \hat{a}_{l_N}^\dagger \ketend 0 \ket
 \label{eqn:NpartEnergystate}
 \end{equation}
 and
 \begin{equation}
\bra \bld{x}_1, \dots  \bld{x}_N \braend
= \bra 0 \braend
\hat{\varphi}(\bld{x}_1) \dots \hat{\varphi}(\bld{x}_N)
\label{eqn:NpartCoordstate}
\end{equation}
We read
\begin{equation}
\Psi^{(N)}(t; \bld{x}_1, \dots, \bld{x}_N)
= \bra \bld{x}_1, \dots  \bld{x}_N \braend \Psi^{(N)}(t) \ket
\end{equation}
and 
\begin{equation}
\ketend \Psi^{(N)}(t) \ket
=
{\mathop{\displaystyle \sum}_{\scriptstyle {l_1},  \dots, {l_N} } }
\Psi^{(N)}(l_1,\dots, l_N)
e^{-i E^{(N)} t}
\ketend
\epsilon_{l_1}, \dots, \epsilon_{l_N} >
\end{equation}
When $N$ particles are identical bosons,
these expressions in Eq. (\ref{eqn:NpartEnergystate}) and (\ref{eqn:NpartCoordstate})
are redundant because all operators in them are commuting with each other
and the order of variables in these bra and ket have no meaning.
Suppose we have $n_1$ particle in state of energy $\epsilon_1$, $n_2$ in $\epsilon_2$ 
and so on, we may write the $l.h.s.$ of Eq. (\ref{eqn:NpartEnergystate}) as
$\ketend n_1, n_2, \dots \ket$. This notation is commonly used in condensed matter
and nuclear physics.

Basis $\ketend \bld{x}_1, \dots  \bld{x}_N \ket$ or 
$\ketend \epsilon_{l_1}, \dots, \epsilon_{l_N} \ket$ 
span N-particle Hilbert space. Set of all N-particle
Hilbert space spanned by $\ketend \bld{x}_1, \bld{x}_2, \dots   \ket$
is called the Fock space.
A state vector $\ketend \Psi \ket$ of the Fock space can be
expanded as
\begin{equation}
\ketend \Psi \ket =
\sum_N \int \prod_i^N d^3 \bld{x}_i \ketend \bld{x}_1, \dots,  \bld{x}_N \ket
\bra \bld{x}_1, \dots,  \bld{x}_N \braend \Psi \ket
\end{equation}


%<<<<<<<<<<<<<<<<<<<<<<<<<<<<<<<<<
\begin{comment}
\bigskip

\bigskip

\bigskip

\begin{equation}
\ketend p_1, p_2, \dots, p_n \ket
= \{a^\dagger(p_1) a^\dagger(p_2) \dots a^\dagger(p_n)\}
\ketend 0 \ket
\end{equation}
\end{comment}