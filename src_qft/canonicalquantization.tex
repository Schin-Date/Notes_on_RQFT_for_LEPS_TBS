${\cal L}$agrangian density ($dim = E^4$) is given as a functional of field $\varphi(x)$
and its space time derivatives
\begin{equation}
	{\cal L} = {\cal L}[\varphi(x), \partial^\mu\varphi(x)]
\end{equation}
Euler-Lagrange Eq.
\begin{equation}
 \frac{ \partial {\cal L}}{ \partial {\varphi}(x) } - 
\partial_\mu \frac{ \partial {\cal L}}{ \partial (\partial_\mu {\varphi}(x) )} = 0
 \;\;
 \label{eqn:EulerLagrange}
\end{equation}
Canonical momentum field
\begin{equation}
\pi(x) = \frac{ \partial {\cal L}}{ \partial \dot{\varphi}(x) }
\label{eqn:conjmomentum}
\end{equation}
where $\dot{\varphi}(x) = \partial_0 \varphi(x) = \partial^0 \varphi(x)$.\\
${\cal H}$amiltonian density
\begin{equation}
{\cal H} = \pi(x) \dot{\varphi}(x) -  {\cal L}\;\;.
\label{eqn:Hamiltoniandensity}
\end{equation}
By solving Eq. (\ref{eqn:conjmomentum}) for $\dot\varphi(x)$,
${\cal H}$ is a function solely of $\pi(x), \varphi(x)$ and $ \partial_i \varphi(x)$.
In the classical field theory, temporal developments of $\varphi(x)$ and $\pi(x)$
are given by Hamiltonian $H = \int \mbox{d}^3\bld{x} \; {\cal H}$
through the canonical equation of motion
\begin{equation}
\dot{\varphi}(x) = -i [\varphi(x), H ]\;, \; \; \; \; \; \dot{\pi}(x) = -i [\pi(x), H ]
\label{eqn:canonicalEOM}
\end{equation}
where $-i[\cdots]$ is the Poisson braket.
The Euler equation (\ref{eqn:EulerLagrange}) and 
the canonical equation (\ref{eqn:canonicalEOM})
are equivalent.

\bigskip

Following the canonical quantization method, 
fields $\varphi(x)$ and $\pi(x')$ at a time 
$x^0 = x'^0 = t_0$ (which is called the time of quantization)
are postulated to satisfy an equal time commutation relation
\begin{eqnarray}
&&\;\;\;\;\;\;[ \varphi(x), \pi(x')] = i \delta^3 (\bld{x} - \bld{x'} )\;,
\label{eqn:cancomm}
\\
&&\left[ \varphi(x), \varphi(x') \right] = 0\;,\;\;\;\;\left[ \pi(x), \pi(x') \right] = 0
\nonumber
\end{eqnarray}
Assuming the existence of the Hamiltonian and using this equal time
commutation relation, it is shown that
the Euler equation (\ref{eqn:EulerLagrange}) and
the Heisenberg equation (\ref{eqn:canonicalEOM})
are equivalent in the level of quantum theory.
\cite{ref:NIsh.1-2}
Here, we shold read $[\cdots]$ in Eq. (\ref{eqn:canonicalEOM}) as
commutation relation.
In that case, using EOM of fields, it is shown that
the commutation relation (\ref{eqn:cancomm}) holds at
arbitrary time once it is set.\cite{ref:NIsh.1-2}
In this sense, the quantization time is arbitral when
the canonical quantization method works in usual manner.
Formal solution of $\varphi(x)$ for the Heisenberg Eq. (\ref{eqn:canonicalEOM})
is written as 
\begin{equation}
\varphi(x) = e^{iH(t-t_0)} \varphi(\bld{x}, t_0) e^{-iH(t-t_0)} 
\label{eqn:timeevophi}
\end{equation}
